\documentclass{article}
\usepackage[utf8]{inputenc}
\usepackage[margin=1in]{geometry}

%%%%%% To use hyperlinks, including the formula ones
\usepackage{hyperref}
\hypersetup{
    colorlinks=true,
    linkcolor=blue,
    filecolor=magenta,      
    urlcolor=cyan,
}

%%%%%% Make paragraphs start with no indentation and leave spaces between paragraphs
\setlength{\parindent}{0em}
\setlength{\parskip}{1em}

%%%%%% Math stuff
\usepackage{amsmath, amssymb}
\usepackage{amsthm}

%%%%%% Mis Codigos

% TODO notes package
\usepackage{xargs}                  % Use more than one optional parameter in a new command
\usepackage[pdftex,dvipsnames]{xcolor}
%\usepackage{xargs}                      % Use more than one optional parameter in a new
%\usepackage[pdftex,dvipsnames]{xcolor}  % Coloured text etc.

%
%\usepackage[colorinlistoftodos,prependcaption,textsize=tiny]{todonotes}
\usepackage[disable]{todonotes}

\newcommandx{\unsure}[2][1=]{\todo[linecolor=red,backgroundcolor=red!25,bordercolor=red,#1]{#2}}
\newcommandx{\change}[2][1=]{\todo[linecolor=blue,backgroundcolor=blue!25,bordercolor=blue,#1]{#2}}
\newcommandx{\complete}[2][1=]{\todo[linecolor=pink,backgroundcolor=pink!25,bordercolor=blue,#1]{#2}}
\newcommandx{\info}[2][1=]{\todo[linecolor=OliveGreen,backgroundcolor=OliveGreen!25,bordercolor=OliveGreen,#1]{#2}}
\newcommandx{\improvement}[2][1=]{\todo[linecolor=Plum,backgroundcolor=Plum!25,bordercolor=Plum,#1]{#2}}
\newcommandx{\thiswillnotshow}[2][1=]{\todo[disable,#1]{#2}}

% Colored text and boxes with my color conventions for highlighting
\usepackage[dvipsnames]{xcolor}
%\usepackage[dvipsnames]{xcolor}

%
% \newcommand{\ytext}[1]{\textcolor{yellow}{#1}}
% \newcommand{\otext}[1]{\textcolor{orange}{#1}}
% \newcommand{\rtext}[1]{\textcolor{red}{#1}}
% \newcommand{\lbtext}[1]{\textcolor{cyan}{#1}}
% \newcommand{\dbtext}[1]{\textcolor{blue}{#1}}
% \newcommand{\ptext}[1]{\textcolor{Plum}{#1}}
% \newcommand{\lgtext}[1]{\textcolor{LimeGreen}{#1}}
% \newcommand{\dgtext}[1]{\textcolor{OliveGreen}{#1}}

\newcommand{\ytext}[1]{\textcolor{black}{#1}}
\newcommand{\otext}[1]{\textcolor{black}{#1}}
\newcommand{\rtext}[1]{\textit{#1}}
\newcommand{\lbtext}[1]{\textcolor{black}{#1}}
\newcommand{\dbtext}[1]{\textcolor{black}{#1}}
\newcommand{\ptext}[1]{\textcolor{black}{#1}}
\newcommand{\lgtext}[1]{\textcolor{black}{#1}}
\newcommand{\dgtext}[1]{\textcolor{black}{#1}}



\newcommand{\ybox}[1]{\colorbox{yellow}{#1}}
\newcommand{\obox}[1]{\colorbox{orange}{#1}}
\newcommand{\rbox}[1]{\colorbox{Salmon}{#1}}
\newcommand{\lbbox}[1]{\colorbox{SkyBlue}{#1}}
\newcommand{\dbbox}[1]{\colorbox{NavyBlue}{#1}}
\newcommand{\pbox}[1]{\colorbox{Plum}{#1}}
\newcommand{\lgbox}[1]{\colorbox{LimeGreen}{#1}}
\newcommand{\dgbox}[1]{\colorbox{OliveGreen}{#1}}



% Math symbols
\usepackage{xparse}

%\usepackage{amssymb,amsmath,amsthm}
%\usepackage{xparse}

%%% Common symbols redifined
\let\oldepsilon\epsilon
\renewcommand{\epsilon}{\varepsilon}
\renewcommand{\varepsilon}{\oldepsilon}

\let\oldphi\phi
\renewcommand{\phi}{\varphi}
\renewcommand{\varphi}{\oldphi}

\newcommand{\emty}{\varnothing}
\newcommand{\varempty}{\nothing}

%%% Common Sets
\newcommand{\bb}[1]{\ensuremath{\mathbb{#1}} }
\newcommand{\ZZ}{\ensuremath{\mathbb{Z}} }
\newcommand{\NN}{\ensuremath{\mathbb{N}} }
\newcommand{\QQ}{\ensuremath{\mathbb{Q}} }
\newcommand{\RR}{\ensuremath{\mathbb{R}} }
\newcommand{\CC}{\ensuremath{\mathbb{C}} }


\newcommand{\iss}{\cong}  % Isomorphism symbol, here it is the one with a tilde

%%%%%%%%%%%%%% Sets

\newcommand{\set}[1]{\ensuremath{\left\{ #1 \right\}}} % Set function, simply puts nice left and right braces
\newcommand{\st}{\ensuremath{\ |\ }} % Such that symbol, TODO improve

\newcommand{\psubset}{\ensuremath{\subset}}
\renewcommand{\subset}{\ensuremath{\subseteq}} % Subset symbol, in this case it is the subset or equal to symbol

\newcommand{\bunion}{\bigcup}
%\newcommand{\biun}{\bun^{\infty}}
%\newcommand{\bfun}[3][n]{\bun_{#1 = #2}^{#3}}
\newcommand{\union}{\cup}
%\newcommand{\siun}{\sun^{\infty}}
%\newcommand{\sfun}[3][n]{\sun_{#1 = #2}^{#3}}

\newcommand{\binter}{\bigcap}
%\newcommand{\biinter}{\binter^{\infty}}
%\newcommand{\bfifter}[3][n]{\binter_{#1 = #2}^{#3}}
\newcommand{\inter}{\cap}
%\newcommand{\sininter}{\sinter^{\infty}}
%\newcommand{\sfinter}[3][n]{\sinter_{#1 = #2}^{#3}}

%%%%%% Calculus

% Integrals

%% Single Integrals
\NewDocumentCommand \integ {s O{} O{} m o}
{
	\IfBooleanTF{#1}{\oint}{\int}_{#2}^{#3} #4 %
	\IfNoValueF {#5} {\mathrm{d} #5}
}

%% Derivatives

% Normal derivative
% Example: \der[n]{f}{x}[x_0]
\NewDocumentCommand \der {O{} m m o}
{
	\frac{\mathrm{d}^{#1} #2}{\mathrm{d} {#3}^{#1}}%
	\IfNoValueF{#4} {\biggr|_{#4}}
}

%% Partial Derivatives

% With respect to one variable
% Example: \pder[n]{f}{y}[\pthvars[x_0][y_0][z_0]][(x, z)
\NewDocumentCommand \pder {O{} m m O{} o}
{
    \ensuremath{
	\IfNoValueTF {#5}
	{
		\frac{\partial^{#1} #2}{\partial {#3}^{#1}} #4
	}
	{
		\left(%
		\frac{\partial^{#1} #2}{\partial {#3}^{#1}}%
		\right)_{#5}  #4
	}
	}
}

% With respecto to two variables
% Example: \twpder{g}{y}{x}[(x_0, y_0)]
\NewDocumentCommand \twpder {m m m O{}}
{
	\frac{\partial^2 #1}{\partial #2 \partial #3} #4
}

\newcommand{\abs}[1]{\left\lvert #1 \right\rvert}
\newcommand{\norm}[1]{\left\lVert #1 \right\rVert}

% Physics symbols (vectors, units)
\usepackage{tikz}
% Version of December 26 2016

%%%%%%%%%%%%%%%%%%%%%%%%%%%%%%%%%%%%%%%%%%% Vectors

%%%%%%%% For SI units %%%%%%%
%\usepackage{siunitx}

%%%%%%%% Vectors %%%%%%%%%%%% (Just see last lines)
%\usepackage{tikz}         % For arrow and dots in \xvec

% --- Macro \xvec
\makeatletter
\newlength\xvec@height%
\newlength\xvec@depth%
\newlength\xvec@width%
\newcommand{\xvec}[2][]{%
  \ifmmode%
    \settoheight{\xvec@height}{$#2$}%
    \settodepth{\xvec@depth}{$#2$}%
    \settowidth{\xvec@width}{$#2$}%
  \else%
    \settoheight{\xvec@height}{#2}%
    \settodepth{\xvec@depth}{#2}%
    \settowidth{\xvec@width}{#2}%
  \fi%
  \def\xvec@arg{#1}%
  \def\xvec@dd{:}%
  \def\xvec@d{.}%
  \raisebox{.2ex}{\raisebox{\xvec@height}{\rlap{%
    \kern.05em%  (Because left edge of drawing is at .05em)
    \begin{tikzpicture}[scale=1]
    \pgfsetroundcap
    \draw (.05em,0)--(\xvec@width-.05em,0);
    \draw (\xvec@width-.05em,0)--(\xvec@width-.15em, .075em);
    \draw (\xvec@width-.05em,0)--(\xvec@width-.15em,-.075em);
    \ifx\xvec@arg\xvec@d%
      \fill(\xvec@width*.45,.5ex) circle (.5pt);%
    \else\ifx\xvec@arg\xvec@dd%
      \fill(\xvec@width*.30,.5ex) circle (.5pt);%
      \fill(\xvec@width*.65,.5ex) circle (.5pt);%
    \fi\fi%
    \end{tikzpicture}%
  }}}%
  #2%
}
\makeatother

% --- Override \vec with an invocation of \xvec.
\let\stdvec\vec
\renewcommand{\vec}[1]{\xvec[]{#1}}                             % Vector
% --- Define \dvec and \ddvec for dotted and double-dotted vectors.
\newcommand{\tvec}[1]{\xvec[.]{#1}}                             % Vector derived wrt time
\newcommand{\ttvec}[1]{\xvec[:]{#1}}                            % Vector derived twice wrt time


% Theorem environments
%Version of October 8, 2016

%\usepackage{amsthm}

\theoremstyle{definition} %To avoid the annoying italics all the time, and to not sloppily redefine all of them 

\newtheorem{theo}{Theorem}[section]  %numbered according to section environment, so in section to it restarts as 2.1 
\newtheorem{prop}{Proposition}[section]  %numbered according to section environment, so in section to it restarts as 2.1 
\newtheorem{lemma}[theo]{Lemma}     %numbering shared with theorem 
\newtheorem{defn}{Definition}[section]   
\newtheorem{coro}{Corollary}[theo]


\theoremstyle{remark} 
\newtheorem*{remark}{Remark} 
 
%\let\oldtheo\theo 
%\renewcommand{\theo}{\oldtheo\normalfont}  
%  
%\let\olddefn\defn  
%\renewcommand{\defn}{\olddefn\normalfont}  
%  
%\let\oldlemma\lemma  
%\renewcommand{\lemma}{\oldlemma\normalfont}  
%  
%\let\oldcoro\coro  
%\renewcommand{\coro}{\oldcoro\normalfont}

%%%%%%%% ``Example'' environment, very basic, doesnt work with itemize
\theoremstyle{definition}

\newtheorem*{exmp}{Example}

%%%%%%
\title{Patil - Thesis Summary\\Connes Spectral Distance on NCG spaces}
\author{Sebastian Camilo Puerto}
\date{July 2020}

%%%%%%%%%%%%%%%%%%%%%%%%%%%%%%%%%%%%%%%%%%%%%%%%%%%%%%%%%%%%%%%%%%%%%%%%%%%%%
%%%%%%%%%%%%%%%%%%%%%%%%%%%%%%%%%%%%%%%%%%%%%%%%%%%%%%%%%%%%%%%%%%%%%%%%%%%%%
%%%%%%%%%%%%%%%%%%%%%%%%%%%%%%%%%%%%%%%%%%%%%%%%%%%%%%%%%%%%%%%%%%%%%%%%%%%%%
%%%%%%%%%%%%%%%%%%%%%%%%%%%%%%%%%%%%%%%%%%%%%%%%%%%%%%%%%%%%%%%%%%%%%%%%%%%%%
\begin{document}

\maketitle

\tableofcontents
TODO: Landau problem -> NCG

TODO: noncommutative quantum mechanical modesl for SGI, Coulomb problem, spherical potential well

NCQM: can e given a (weak) position measurement

Best example of NCSpace in physics: phase space in quantum mechanics -> von Neumann studied it -> von Neumann algebras

Connes generalized to setting of NC C* algebras, PROVIDING A DIFFERENTIAL STRUCTURE ON THEM.
\textbf{The main inspirate for NCG was QM}/

\textbf{In NG, the focus is focused from the points (of phase space) to the algebra of observables}. Pure states of the algebra are close to the notion of points (TODO: these are be vectors in a Hilbert state, right?... I guess that, because in the C setting pure states are points, I THINK, the equivalent notion is transported to NC, but it so happens that there we have a theory of the states: GNS). SO, the notion that is closer to that of point is that of algebra state.

TODO; are pure states of a C* algebra always vectors of the GNS space?

What has been done:

Connes distance between pure states in moyal and FUzzy have been calculated.

\textbf{In 15, a general algorithm was developed to calculatte Connes distance btween (infinitesimally separated / perp?) states offfff a NC space IN THE SETUP of hilbert-Schmidt operator formulatioooooon ofNCQMM} Key: \rtext{Hilbert=Schmidt operator formulation OF NCQM}. This algorithm was later modified/correted.

Subsequenntly, tge Connes distance between infinitesimally separated coherent states.

Using the modified algorithm, the calculation was hard.

But, in 11 an alternative approach is developed to calculate distanec between coherent states of the Moyal plane.

So, using this alternative approach adopted from 11, \textbf{Xonnes distance between finitely separated coherent and discrete states in Moyal and FUzzy were calculated}.

\textbf{A corrected general method to find Connees distance is presented in the Conclusions} 

%%%%%%%%%%%%%%%%%%%%%%%%%%%%%%%%%%%%%%%%%%%%%%%%%%%%%%%%%%%%%%%%%%%%%%%%%%%%%
\subsection{High Level Summary}

    \begin{itemize}

    \item Noncommutative Quantum Mechanics
    
        \begin{itemize}
            
        \item It is a model in which an intermediate Noncommutative ``configuration'' space appears: a space of representation of some desired ``coordinate functions''(observables) $x_i$ conmutation relations "$\mathcal H_c$.
        
            \begin{itemize}
                
            \item This will be close, but not equal, to the Hilbert space considered for the studied spectral triples.
                
            \end{itemize}
        
        \item \textbf{(Generalized) Coherent vectors of $\mathcal H_c$}, whose density matrices $\rho_z = |z\langle \rangle z| : \mathcal H_c \to \mathcal H_c$ can be reformulated as \emph{states} of the algebra $\mathcal A = \mathcal H_q \subset \mathcal B(\mathcal H_c) \cong \mathcal H \otimes \mathcal H^*$ of Hilbert-Schmidt operators, \textbf{allow us to think again of ``points of the noncommutative space'': the states of $\mathcal A = \mathcal H_q$ that they induce, called \lbtext{coherent states}}.
        
        \item It is a model in which we can make sense to ``position measurements'' in a weak sense, because we have something similar to the Spectral Theorem.
        
        \item Constructions for Moyal plane
        
            \begin{itemize}
            
            \item Hilbert spaces, 
            
            \item Representation of Heisenbert algebra in $\mathcal H_q$. 
            
            \item Statement: \rtext{\lbtext{Physical states} of a system are represented by normalized vectors in $\mathcal H_q$}. (instead of $\mathcal H_c$).
            
            \item \lbtext{Coherent states}: elements of $\mathcal H_q$, $|z)$; density matrices $|z) = |z\rangle \langle z|$, where $|z\langle \in \mathcal H_c$ are \textbf{coherent vectors in $\mathcal H_c$} (eigenvector of destruction operator such that $\delta x_1 \delta x_2 = \frac{\hbar}{2}$) $|z\langle = e^{-\overline{z} b + z b^\dagger}|0\rangle$
            
            \item Positive operators $\pi_z: \mathcal H_q \to \mathcal H_q$ ``provide a POVM'' for position measurement
            
            \item \rtext{The probability of finding a particle in a ``state'' represented by the density matrix $\rho: \mathcal H_q \to \mathcal H_q$ (e.g. the ``pure state'' $|\psi)(\psi|$), at $z = (x_1, x_2)$} is $p(x_1, x_2) = tr_q(\pi_z \rho)$, e.g. $p(x_1, x_2) = (\psi|\pi_z|\psi)$.
                
            \end{itemize}
        
        \item Constructions for Fuzzy Spheres
        
            \begin{itemize}
                
            \item $\mathcal H_c = span\set{|n, n_3\rangle}$, $\mathcal H_q = span\set{|n, n_3\rangle \langle n, n_3|}$. Decomposition where components labeled by $n$, where $\mathcal H_c^n$ is the $n$th irrep of $su(2)$.
            
            \item Coherent states: \lbtext{Perelomov coherent states (maybe I will call them ``vectors'' instead of states)}: elements of the action of group $X = SU(2)/U(1)$ on the state $|n, n_3\langle$. Given by $|z\langle = \cdots \in H^n_c$, as discussed in [21]. The label $z = -tan(\frac{\theta}{2})e^{-i \phi}$
                
            \end{itemize}
        
        \item We will see how \rtext{there is a one-to-one correspondence between the points in the NC space and the coherent state labeled by $z = (x_1, x_2) \in \CC$}. Therefore \rtext{defining a distance on the set of coherent states will mean investigating the geometry of the underlying noncommutative space}. (\dbtext{TODO}: in what general setting do these coherent states exist, i.e. in which ``noncommutative spaces'' do they exist and when can I say they represent the points of the NC space?)
        
        \item To do the spectral distance calculations and even to apply the proposed algorithm:
        
            \begin{itemize}
                
            \item The coherent states are important, at least in the sense that they are $1$ of the $2$ families of examples that we have,
            
            \item And this interpretation is usefull to \textbf{say that we calculated the distance between ``points of the noncommutative configuration spaces''}, because, due to the POVM that they induce, we can think of (weakly) measuring position... something like that
            
            \item But, to do the calculations per se they are not necessary, as the first 2 elements of the proposed Spectral Triples $(\mathcal A = \mathcal H_q, \mathcal H = \mathcal H_c \otimes \CC^2, D)$ coincide with the usual spectral triples of the noncommutative spaces behind, at least in the case of the Moyal Plane and the Fuzzy Sphere.
            
            \item The representation of the Hilbert algebra on $\mathcal H_q$ does NOT seem to be used anywhere... in the calculations never, but perhaps in finding the POVM?
            
            \item The POVM is also never used to calculate the spectral distance, only to interpret the distance between coherent states as distance between ``points'' of the NC space. 
            
            \end{itemize}
            
        \end{itemize}
    
    \item Noncommutative Geometry
    
        \begin{itemize}
        
        \item Basic definitions like $C^*$ algebra, state
        
        \item Equivalenec between (the set of characters of) commutivative $C^*$ algebras and locally compact T2 spaces.
        
        \item Algebraic definition of a connection: A connection on a module
        
        \item Serre-swan theorem: equivalence between finitely generated projective $C^\infty(M)$-modules AND vector bundles on $M$.
        
        \item Universal first-order calculus over $\mathcal A$: apparently a way to talk about vector fields on the algebra and to do calculus on it.
        
        \item Canonical spectral triple: $M$ compact, spin manifold. Proof of the validity of Connes' distance formula.
        
        \item \rtext{``Construct spectral triples in order to provide a geometric structure to the configuration space'', i.e. classical Hilbert space: calculate Connes distance between states on $\mathcal A = \mathcal H_q$}
        
        \item A spectral triple on the Moyal plane
        
            \begin{itemize}
                
            \item Constructed in [15] (them)
            
            \item Related to spectral triple of [28]: isospectral deformation of the canonical spectral triple on Euclidean space $\RR^2$. In [11,10] this one is used and the distance between coherent states is calculated.
            
            \item The 1st property of the Dirac op proven, the 2nd one is not done, and instead we are refeerd to the prove for the other spectral triple in [28]
                
            \end{itemize}
            
        \item Spectral triple on the Fuzzy spheres
        
            \begin{itemize}
                
            \item Constructed in [25] and reviewed in [16], deformed from the canonical triple of $S^2$
            
            \item In [27] a similar spectral triple proposed, and distance between coherent states calculated and its relation to commutative $S^2$ analyzed.
            
            \item The 2 properties necessary for the operator to be Dirac are carefully proven (short).
            
            \end{itemize}
        
        \item To do the spectral distance calculations and even to apply the proposed algorithm
        
            \begin{itemize}
            
            \item Basic definitions like $C^*$ algebra, state, operator norm... perhaps missing the definition of normal state
            
            \item The definition of a Spectral Triple and of the Connes Distance Function, is used.
            
            \item The proofs that the proposed Dirac operators on the Moyal plane and the Fuzzy spheres are indeed Dirac operators.
            
            \item Translation invariance of Connes formula?
            
            \item Arguably: the quickly mentioned properties of a Dirac operator due to its defining properties: 
            
                \begin{itemize}
                    
                \item Its spectrum is real and discrete.
                
                \item The eigenspaces are finite dimensional
                
                \item The eigenvalues follow a growth property that imply that there is no accumulation point other than at infinity, i.e. $lim_{n \to \infty} \lambda_n = \infty$.
                
                \item [19] (Don't know if it is a general result) The supremum element belongs to the hermitian elements of $\mathcal A$
                    
                \end{itemize}
            
            \end{itemize}
            
            \item NEVER do we use the differential calculus here defined $(\Omega^1 \mathcal A \subset \mathcal A \otimes \mathcal A, d)$. NEVER used the, supposedly important, property of projective modules of admiting ``universal'' connections.
        
        \end{itemize}
    
    \item Connes Distance on Moyal Plane and Fuzzy Sphere
    
        \begin{itemize}
            
        \item We study the distance between some sets of states of the algebra $\mathcal A = \mathcal H_q$ that correspond to ``physical states''(as defined above: normalized elements of $\mathcal A = \mathcal H_q$) since they are normal states: they come from density matrices $\rho: \mathcal H_c \to \mathcal H_c$, but this density matrices ARE normalized elements of $\mathcal H_q = \mathcal A$.
        
        \item General strategy:
        
            \begin{itemize}
                
            \item Find tight upperbound for the distance (recall that the Connes distance is a supremum over $a \in B$)
            
            \item Construct/find/show it works: supremum element $a_s \in \mathcal A$ for which the upper bound is achieved / Find a sequence of elements $a_n \in B \subset \mathcal A$ such that their inner part of Connes formula accumulates in the limit. And prove that it is in $B$.
            
            \item Use translation invariance of Connes formula.
            
            \item For the discrete set of states: first find using this algorithm the \lbtext{infinitesimal distances} i.e. distance between states with adjacent labels $n$ and $n+1$. Then apply again the algorithm for other states/ ``finitely'' separated states.
                
            \end{itemize}
            
        \end{itemize}
    
    \end{itemize}


%\section{Introduction}

%%%%%%%%%%%%%%%%%%%%%%%%%%%%%%%%%%%%%%%%%%%%%%%%%%%%%%%%%%%%%%%%%%%%%%%%%%%%%
\subsection{High Level Summary}

    \begin{itemize}

    \item 
    
    \end{itemize}

%%%%%%%%%%%%%%%%%%%%%%%%%%%%%%%%%%%%%%%%%%%%%%%%%%%%%%%%%%%%%%%%%%%%%%%%%%%%%
\subsection{Very Important Facts}

    \begin{itemize}

    \item 
    
    \end{itemize}

%%%%%%%%%%%%%%%%%%%%%%%%%%%%%%%%%%%%%%%%%%%%%%%%%%%%%%%%%%%%%%%%%%%%%%%%%%%%%
\subsection{Important Facts}

    \begin{itemize}

    \item 
    
    \end{itemize}

%%%%%%%%%%%%%%%%%%%%%%%%%%%%%%%%%%%%%%%%%%%%%%%%%%%%%%%%%%%%%%%%%%%%%%%%%%%%%
\subsection{Memorize}

    \begin{itemize}

    \item 
    
    \end{itemize}

%%%%%%%%%%%%%%%%%%%%%%%%%%%%%%%%%%%%%%%%%%%%%%%%%%%%%%%%%%%%%%%%%%%%%%%%%%%%%
\subsection{Doubts}

    \begin{itemize}

    \item 
    
    \end{itemize}

%%%%%%%%%%%%%%%%%%%%%%%%%%%%%%%%%%%%%%%%%%%%%%%%%%%%%%%%%%%%%%%%%%%%%%%%%%%%%
\subsection{Detailed summary}

    \begin{itemize}

    \item 
    
    \end{itemize}

%%%%%%%%%%%%%%%%%%%%%%%%%%%%%%%%%%%%%%%%%%%%%%%%%%%%%%%%%%%%%%%%%%%%%%%%%%%%%
\subsection{Notice}

    \begin{itemize}

    \item 
    
    \end{itemize}

%%%%%%%%%%%%%%%%%%%%%%%%%%%%%%%%%%%%%%%%%%%%%%%%%%%%%%%%%%%%%%%%%%%%%%%%%%%%% 
\subsection{Yet to understand}

    \begin{itemize}

    \item 
    
    \end{itemize}
%%%%%%%%%%%%%%%%%%%%%%%%%%%%%%%%%%%%%%%%%%%%%%%%%%%%%%%%%%%%%%%%%%%%%%%%%%%%%
%%%%%%%%%%%%%%%%%%%%%%%%%%%%%%%%%%%%%%%%%%%%%%%%%%%%%%%%%%%%%%%%%%%%%%%%%%%%%

%\section{Noncommutative Quantum Mechanics}

%%%%%%%%%%%%%%%%%%%%%%%%%%%%%%%%%%%%%%%%%%%%%%%%%%%%%%%%%%%%%%%%%%%%%%%%%%%%%
\subsection{High Level Summary}

    \begin{itemize}

    \item 
    
    \end{itemize}

%%%%%%%%%%%%%%%%%%%%%%%%%%%%%%%%%%%%%%%%%%%%%%%%%%%%%%%%%%%%%%%%%%%%%%%%%%%%%
\subsection{Very Important Facts}

    \begin{itemize}

    \item 
    
    \end{itemize}

%%%%%%%%%%%%%%%%%%%%%%%%%%%%%%%%%%%%%%%%%%%%%%%%%%%%%%%%%%%%%%%%%%%%%%%%%%%%%
\subsection{Important Facts}

    \begin{itemize}

    \item 
    
    \end{itemize}

%%%%%%%%%%%%%%%%%%%%%%%%%%%%%%%%%%%%%%%%%%%%%%%%%%%%%%%%%%%%%%%%%%%%%%%%%%%%%
\subsection{Memorize}

    \begin{itemize}

    \item 
    
    \end{itemize}

%%%%%%%%%%%%%%%%%%%%%%%%%%%%%%%%%%%%%%%%%%%%%%%%%%%%%%%%%%%%%%%%%%%%%%%%%%%%%
\subsection{Doubts}

    \begin{itemize}

    \item 
    
    \end{itemize}

%%%%%%%%%%%%%%%%%%%%%%%%%%%%%%%%%%%%%%%%%%%%%%%%%%%%%%%%%%%%%%%%%%%%%%%%%%%%%
\subsection{Detailed summary}

    \begin{itemize}

    \item 
    
    \end{itemize}

%%%%%%%%%%%%%%%%%%%%%%%%%%%%%%%%%%%%%%%%%%%%%%%%%%%%%%%%%%%%%%%%%%%%%%%%%%%%%
\subsection{Notice}

    \begin{itemize}

    \item 
    
    \end{itemize}

%%%%%%%%%%%%%%%%%%%%%%%%%%%%%%%%%%%%%%%%%%%%%%%%%%%%%%%%%%%%%%%%%%%%%%%%%%%%% 
\subsection{Yet to understand}

    \begin{itemize}

    \item 
    
    \end{itemize}
%%%%%%%%%%%%%%%%%%%%%%%%%%%%%%%%%%%%%%%%%%%%%%%%%%%%%%%%%%%%%%%%%%%%%%%%%%%%%
%%%%%%%%%%%%%%%%%%%%%%%%%%%%%%%%%%%%%%%%%%%%%%%%%%%%%%%%%%%%%%%%%%%%%%%%%%%%%

%\section{Noncommutative Geometry}

%%%%%%%%%%%%%%%%%%%%%%%%%%%%%%%%%%%%%%%%%%%%%%%%%%%%%%%%%%%%%%%%%%%%%%%%%%%%%
\subsection{High Level Summary}

    \begin{itemize}

    \item 
    
    \end{itemize}

%%%%%%%%%%%%%%%%%%%%%%%%%%%%%%%%%%%%%%%%%%%%%%%%%%%%%%%%%%%%%%%%%%%%%%%%%%%%%
\subsection{Very Important Facts}

    \begin{itemize}

    \item 
    
    \end{itemize}

%%%%%%%%%%%%%%%%%%%%%%%%%%%%%%%%%%%%%%%%%%%%%%%%%%%%%%%%%%%%%%%%%%%%%%%%%%%%%
\subsection{Important Facts}

    \begin{itemize}

    \item 
    
    \end{itemize}

%%%%%%%%%%%%%%%%%%%%%%%%%%%%%%%%%%%%%%%%%%%%%%%%%%%%%%%%%%%%%%%%%%%%%%%%%%%%%
\subsection{Memorize}

    \begin{itemize}

    \item 
    
    \end{itemize}

%%%%%%%%%%%%%%%%%%%%%%%%%%%%%%%%%%%%%%%%%%%%%%%%%%%%%%%%%%%%%%%%%%%%%%%%%%%%%
\subsection{Doubts}

    \begin{itemize}

    \item 
    
    \end{itemize}

%%%%%%%%%%%%%%%%%%%%%%%%%%%%%%%%%%%%%%%%%%%%%%%%%%%%%%%%%%%%%%%%%%%%%%%%%%%%%
\subsection{Detailed summary}

    \begin{itemize}

    \item 
    
    \end{itemize}

%%%%%%%%%%%%%%%%%%%%%%%%%%%%%%%%%%%%%%%%%%%%%%%%%%%%%%%%%%%%%%%%%%%%%%%%%%%%%
\subsection{Notice}

    \begin{itemize}

    \item 
    
    \end{itemize}

%%%%%%%%%%%%%%%%%%%%%%%%%%%%%%%%%%%%%%%%%%%%%%%%%%%%%%%%%%%%%%%%%%%%%%%%%%%%% 
\subsection{Yet to understand}

    \begin{itemize}

    \item 
    
    \end{itemize}
%%%%%%%%%%%%%%%%%%%%%%%%%%%%%%%%%%%%%%%%%%%%%%%%%%%%%%%%%%%%%%%%%%%%%%%%%%%%%
%%%%%%%%%%%%%%%%%%%%%%%%%%%%%%%%%%%%%%%%%%%%%%%%%%%%%%%%%%%%%%%%%%%%%%%%%%%%%

%\section{Connes Distance on Moyal Plane and Fuzzy Sphere}

%%%%%%%%%%%%%%%%%%%%%%%%%%%%%%%%%%%%%%%%%%%%%%%%%%%%%%%%%%%%%%%%%%%%%%%%%%%%%
\subsection{High Level Summary}

    \begin{itemize}

    \item 
    
    \end{itemize}

%%%%%%%%%%%%%%%%%%%%%%%%%%%%%%%%%%%%%%%%%%%%%%%%%%%%%%%%%%%%%%%%%%%%%%%%%%%%%
\subsection{Very Important Facts}

    \begin{itemize}

    \item 
    
    \end{itemize}

%%%%%%%%%%%%%%%%%%%%%%%%%%%%%%%%%%%%%%%%%%%%%%%%%%%%%%%%%%%%%%%%%%%%%%%%%%%%%
\subsection{Important Facts}

    \begin{itemize}

    \item 
    
    \end{itemize}

%%%%%%%%%%%%%%%%%%%%%%%%%%%%%%%%%%%%%%%%%%%%%%%%%%%%%%%%%%%%%%%%%%%%%%%%%%%%%
\subsection{Memorize}

    \begin{itemize}

    \item 
    
    \end{itemize}

%%%%%%%%%%%%%%%%%%%%%%%%%%%%%%%%%%%%%%%%%%%%%%%%%%%%%%%%%%%%%%%%%%%%%%%%%%%%%
\subsection{Doubts}

    \begin{itemize}

    \item 
    
    \end{itemize}

%%%%%%%%%%%%%%%%%%%%%%%%%%%%%%%%%%%%%%%%%%%%%%%%%%%%%%%%%%%%%%%%%%%%%%%%%%%%%
\subsection{Detailed summary}

    \begin{itemize}

    \item 
    
    \end{itemize}

%%%%%%%%%%%%%%%%%%%%%%%%%%%%%%%%%%%%%%%%%%%%%%%%%%%%%%%%%%%%%%%%%%%%%%%%%%%%%
\subsection{Notice}

    \begin{itemize}

    \item 
    
    \end{itemize}

%%%%%%%%%%%%%%%%%%%%%%%%%%%%%%%%%%%%%%%%%%%%%%%%%%%%%%%%%%%%%%%%%%%%%%%%%%%%% 
\subsection{Yet to understand}

    \begin{itemize}

    \item 
    
    \end{itemize}
%%%%%%%%%%%%%%%%%%%%%%%%%%%%%%%%%%%%%%%%%%%%%%%%%%%%%%%%%%%%%%%%%%%%%%%%%%%%%
%%%%%%%%%%%%%%%%%%%%%%%%%%%%%%%%%%%%%%%%%%%%%%%%%%%%%%%%%%%%%%%%%%%%%%%%%%%%%

%\input{Patil/5-Conclusions}
%%%%%%%%%%%%%%%%%%%%%%%%%%%%%%%%%%%%%%%%%%%%%%%%%%%%%%%%%%%%%%%%%%%%%%%%%%%%%
%%%%%%%%%%%%%%%%%%%%%%%%%%%%%%%%%%%%%%%%%%%%%%%%%%%%%%%%%%%%%%%%%%%%%%%%%%%%%

\end{document}
