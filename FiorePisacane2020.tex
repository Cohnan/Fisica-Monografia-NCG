\documentclass{article}
\usepackage[utf8]{inputenc}
\usepackage[margin=1in]{geometry}

%%%%%% To use hyperlinks, including the formula ones
\usepackage{hyperref}
\hypersetup{
    colorlinks=true,
    linkcolor=blue,
    filecolor=magenta,      
    urlcolor=cyan,
}

%%%%%% Make paragraphs start with no indentation and leave spaces between paragraphs
\setlength{\parindent}{0em}
\setlength{\parskip}{1em}

%%%%%% Math stuff
\usepackage{amsmath, amssymb}
\usepackage{amsthm}

%%%%%% Mis Codigos

% TODO notes package
\usepackage{xargs}                  % Use more than one optional parameter in a new command
\usepackage[pdftex,dvipsnames]{xcolor}
%\usepackage{xargs}                      % Use more than one optional parameter in a new
%\usepackage[pdftex,dvipsnames]{xcolor}  % Coloured text etc.

%
%\usepackage[colorinlistoftodos,prependcaption,textsize=tiny]{todonotes}
\usepackage[disable]{todonotes}

\newcommandx{\unsure}[2][1=]{\todo[linecolor=red,backgroundcolor=red!25,bordercolor=red,#1]{#2}}
\newcommandx{\change}[2][1=]{\todo[linecolor=blue,backgroundcolor=blue!25,bordercolor=blue,#1]{#2}}
\newcommandx{\complete}[2][1=]{\todo[linecolor=pink,backgroundcolor=pink!25,bordercolor=blue,#1]{#2}}
\newcommandx{\info}[2][1=]{\todo[linecolor=OliveGreen,backgroundcolor=OliveGreen!25,bordercolor=OliveGreen,#1]{#2}}
\newcommandx{\improvement}[2][1=]{\todo[linecolor=Plum,backgroundcolor=Plum!25,bordercolor=Plum,#1]{#2}}
\newcommandx{\thiswillnotshow}[2][1=]{\todo[disable,#1]{#2}}

% Colored text and boxes with my color conventions for highlighting
\usepackage[dvipsnames]{xcolor}
%\usepackage[dvipsnames]{xcolor}

%
% \newcommand{\ytext}[1]{\textcolor{yellow}{#1}}
% \newcommand{\otext}[1]{\textcolor{orange}{#1}}
% \newcommand{\rtext}[1]{\textcolor{red}{#1}}
% \newcommand{\lbtext}[1]{\textcolor{cyan}{#1}}
% \newcommand{\dbtext}[1]{\textcolor{blue}{#1}}
% \newcommand{\ptext}[1]{\textcolor{Plum}{#1}}
% \newcommand{\lgtext}[1]{\textcolor{LimeGreen}{#1}}
% \newcommand{\dgtext}[1]{\textcolor{OliveGreen}{#1}}

\newcommand{\ytext}[1]{\textcolor{black}{#1}}
\newcommand{\otext}[1]{\textcolor{black}{#1}}
\newcommand{\rtext}[1]{\textit{#1}}
\newcommand{\lbtext}[1]{\textcolor{black}{#1}}
\newcommand{\dbtext}[1]{\textcolor{black}{#1}}
\newcommand{\ptext}[1]{\textcolor{black}{#1}}
\newcommand{\lgtext}[1]{\textcolor{black}{#1}}
\newcommand{\dgtext}[1]{\textcolor{black}{#1}}



\newcommand{\ybox}[1]{\colorbox{yellow}{#1}}
\newcommand{\obox}[1]{\colorbox{orange}{#1}}
\newcommand{\rbox}[1]{\colorbox{Salmon}{#1}}
\newcommand{\lbbox}[1]{\colorbox{SkyBlue}{#1}}
\newcommand{\dbbox}[1]{\colorbox{NavyBlue}{#1}}
\newcommand{\pbox}[1]{\colorbox{Plum}{#1}}
\newcommand{\lgbox}[1]{\colorbox{LimeGreen}{#1}}
\newcommand{\dgbox}[1]{\colorbox{OliveGreen}{#1}}



% Math symbols
\usepackage{xparse}

%\usepackage{amssymb,amsmath,amsthm}
%\usepackage{xparse}

%%% Common symbols redifined
\let\oldepsilon\epsilon
\renewcommand{\epsilon}{\varepsilon}
\renewcommand{\varepsilon}{\oldepsilon}

\let\oldphi\phi
\renewcommand{\phi}{\varphi}
\renewcommand{\varphi}{\oldphi}

\newcommand{\emty}{\varnothing}
\newcommand{\varempty}{\nothing}

%%% Common Sets
\newcommand{\bb}[1]{\ensuremath{\mathbb{#1}} }
\newcommand{\ZZ}{\ensuremath{\mathbb{Z}} }
\newcommand{\NN}{\ensuremath{\mathbb{N}} }
\newcommand{\QQ}{\ensuremath{\mathbb{Q}} }
\newcommand{\RR}{\ensuremath{\mathbb{R}} }
\newcommand{\CC}{\ensuremath{\mathbb{C}} }


\newcommand{\iss}{\cong}  % Isomorphism symbol, here it is the one with a tilde

%%%%%%%%%%%%%% Sets

\newcommand{\set}[1]{\ensuremath{\left\{ #1 \right\}}} % Set function, simply puts nice left and right braces
\newcommand{\st}{\ensuremath{\ |\ }} % Such that symbol, TODO improve

\newcommand{\psubset}{\ensuremath{\subset}}
\renewcommand{\subset}{\ensuremath{\subseteq}} % Subset symbol, in this case it is the subset or equal to symbol

\newcommand{\bunion}{\bigcup}
%\newcommand{\biun}{\bun^{\infty}}
%\newcommand{\bfun}[3][n]{\bun_{#1 = #2}^{#3}}
\newcommand{\union}{\cup}
%\newcommand{\siun}{\sun^{\infty}}
%\newcommand{\sfun}[3][n]{\sun_{#1 = #2}^{#3}}

\newcommand{\binter}{\bigcap}
%\newcommand{\biinter}{\binter^{\infty}}
%\newcommand{\bfifter}[3][n]{\binter_{#1 = #2}^{#3}}
\newcommand{\inter}{\cap}
%\newcommand{\sininter}{\sinter^{\infty}}
%\newcommand{\sfinter}[3][n]{\sinter_{#1 = #2}^{#3}}

%%%%%% Calculus

% Integrals

%% Single Integrals
\NewDocumentCommand \integ {s O{} O{} m o}
{
	\IfBooleanTF{#1}{\oint}{\int}_{#2}^{#3} #4 %
	\IfNoValueF {#5} {\mathrm{d} #5}
}

%% Derivatives

% Normal derivative
% Example: \der[n]{f}{x}[x_0]
\NewDocumentCommand \der {O{} m m o}
{
	\frac{\mathrm{d}^{#1} #2}{\mathrm{d} {#3}^{#1}}%
	\IfNoValueF{#4} {\biggr|_{#4}}
}

%% Partial Derivatives

% With respect to one variable
% Example: \pder[n]{f}{y}[\pthvars[x_0][y_0][z_0]][(x, z)
\NewDocumentCommand \pder {O{} m m O{} o}
{
    \ensuremath{
	\IfNoValueTF {#5}
	{
		\frac{\partial^{#1} #2}{\partial {#3}^{#1}} #4
	}
	{
		\left(%
		\frac{\partial^{#1} #2}{\partial {#3}^{#1}}%
		\right)_{#5}  #4
	}
	}
}

% With respecto to two variables
% Example: \twpder{g}{y}{x}[(x_0, y_0)]
\NewDocumentCommand \twpder {m m m O{}}
{
	\frac{\partial^2 #1}{\partial #2 \partial #3} #4
}

\newcommand{\abs}[1]{\left\lvert #1 \right\rvert}
\newcommand{\norm}[1]{\left\lVert #1 \right\rVert}

% Physics symbols (vectors, units)
\usepackage{tikz}
% Version of December 26 2016

%%%%%%%%%%%%%%%%%%%%%%%%%%%%%%%%%%%%%%%%%%% Vectors

%%%%%%%% For SI units %%%%%%%
%\usepackage{siunitx}

%%%%%%%% Vectors %%%%%%%%%%%% (Just see last lines)
%\usepackage{tikz}         % For arrow and dots in \xvec

% --- Macro \xvec
\makeatletter
\newlength\xvec@height%
\newlength\xvec@depth%
\newlength\xvec@width%
\newcommand{\xvec}[2][]{%
  \ifmmode%
    \settoheight{\xvec@height}{$#2$}%
    \settodepth{\xvec@depth}{$#2$}%
    \settowidth{\xvec@width}{$#2$}%
  \else%
    \settoheight{\xvec@height}{#2}%
    \settodepth{\xvec@depth}{#2}%
    \settowidth{\xvec@width}{#2}%
  \fi%
  \def\xvec@arg{#1}%
  \def\xvec@dd{:}%
  \def\xvec@d{.}%
  \raisebox{.2ex}{\raisebox{\xvec@height}{\rlap{%
    \kern.05em%  (Because left edge of drawing is at .05em)
    \begin{tikzpicture}[scale=1]
    \pgfsetroundcap
    \draw (.05em,0)--(\xvec@width-.05em,0);
    \draw (\xvec@width-.05em,0)--(\xvec@width-.15em, .075em);
    \draw (\xvec@width-.05em,0)--(\xvec@width-.15em,-.075em);
    \ifx\xvec@arg\xvec@d%
      \fill(\xvec@width*.45,.5ex) circle (.5pt);%
    \else\ifx\xvec@arg\xvec@dd%
      \fill(\xvec@width*.30,.5ex) circle (.5pt);%
      \fill(\xvec@width*.65,.5ex) circle (.5pt);%
    \fi\fi%
    \end{tikzpicture}%
  }}}%
  #2%
}
\makeatother

% --- Override \vec with an invocation of \xvec.
\let\stdvec\vec
\renewcommand{\vec}[1]{\xvec[]{#1}}                             % Vector
% --- Define \dvec and \ddvec for dotted and double-dotted vectors.
\newcommand{\tvec}[1]{\xvec[.]{#1}}                             % Vector derived wrt time
\newcommand{\ttvec}[1]{\xvec[:]{#1}}                            % Vector derived twice wrt time


% Theorem environments
%Version of October 8, 2016

%\usepackage{amsthm}

\theoremstyle{definition} %To avoid the annoying italics all the time, and to not sloppily redefine all of them 

\newtheorem{theo}{Theorem}[section]  %numbered according to section environment, so in section to it restarts as 2.1 
\newtheorem{prop}{Proposition}[section]  %numbered according to section environment, so in section to it restarts as 2.1 
\newtheorem{lemma}[theo]{Lemma}     %numbering shared with theorem 
\newtheorem{defn}{Definition}[section]   
\newtheorem{coro}{Corollary}[theo]


\theoremstyle{remark} 
\newtheorem*{remark}{Remark} 
 
%\let\oldtheo\theo 
%\renewcommand{\theo}{\oldtheo\normalfont}  
%  
%\let\olddefn\defn  
%\renewcommand{\defn}{\olddefn\normalfont}  
%  
%\let\oldlemma\lemma  
%\renewcommand{\lemma}{\oldlemma\normalfont}  
%  
%\let\oldcoro\coro  
%\renewcommand{\coro}{\oldcoro\normalfont}

%%%%%%%% ``Example'' environment, very basic, doesnt work with itemize
\theoremstyle{definition}

\newtheorem*{exmp}{Example}

%%%%%%
\title{Fiore Pisacane 2020: On localized and coherent states on some new fuzzy spheres}
\author{Sebastian Camilo Puerto}
\date{July 2020}

%%%%%%%%%%%%%%%%%%%%%%%%%%%%%%%%%%%%%%%%%%%%%%%%%%%%%%%%%%%%%%%%%%%%%%%%%%%%%
%%%%%%%%%%%%%%%%%%%%%%%%%%%%%%%%%%%%%%%%%%%%%%%%%%%%%%%%%%%%%%%%%%%%%%%%%%%%%
%%%%%%%%%%%%%%%%%%%%%%%%%%%%%%%%%%%%%%%%%%%%%%%%%%%%%%%%%%%%%%%%%%%%%%%%%%%%%
%%%%%%%%%%%%%%%%%%%%%%%%%%%%%%%%%%%%%%%%%%%%%%%%%%%%%%%%%%%%%%%%%%%%%%%%%%%%%
\begin{document}

\maketitle

\tableofcontents


%%%%%%%%%%%%%%%%%%%%%%%%%%%%%%%%%%%%%%%%%%%%%%%%%%%%%%%%%%%%%%%%%%%%%%%%%%%%%
\subsection{High Level Summary}

    \begin{itemize}

    \item
    
    \end{itemize}

%%%%%%%%%%%%%%%%%%%%%%%%%%%%%%%%%%%%%%%%%%%%%%%%%%%%%%%%%%%%%%%%%%%%%%%%%%%%%
\subsection{Very Important Facts}

    \begin{itemize}

    \item 
    
    \end{itemize}

%%%%%%%%%%%%%%%%%%%%%%%%%%%%%%%%%%%%%%%%%%%%%%%%%%%%%%%%%%%%%%%%%%%%%%%%%%%%%
\subsection{Important Facts}

    \begin{itemize}

    \item The construction of various systems of coherent states on our fuzzy circle and fuzzy sphere will be very useful to study quantum mechanics and above all quantum field theory on these fuzzy spaces.
    
    \end{itemize}

%%%%%%%%%%%%%%%%%%%%%%%%%%%%%%%%%%%%%%%%%%%%%%%%%%%%%%%%%%%%%%%%%%%%%%%%%%%%%
\subsection{Memorize}

    \begin{itemize}

    \item 
    
    \end{itemize}

%%%%%%%%%%%%%%%%%%%%%%%%%%%%%%%%%%%%%%%%%%%%%%%%%%%%%%%%%%%%%%%%%%%%%%%%%%%%%
\subsection{Doubts}

    \begin{itemize}

    \item 
    
    \end{itemize}

%%%%%%%%%%%%%%%%%%%%%%%%%%%%%%%%%%%%%%%%%%%%%%%%%%%%%%%%%%%%%%%%%%%%%%%%%%%%%
\subsection{Detailed summary}

    \begin{itemize}

    \item 
    
    \end{itemize}

%%%%%%%%%%%%%%%%%%%%%%%%%%%%%%%%%%%%%%%%%%%%%%%%%%%%%%%%%%%%%%%%%%%%%%%%%%%%%
\subsection{Notice}

    \begin{itemize}

    \item 
    
    \end{itemize}

%%%%%%%%%%%%%%%%%%%%%%%%%%%%%%%%%%%%%%%%%%%%%%%%%%%%%%%%%%%%%%%%%%%%%%%%%%%%% 
\subsection{Yet to understand}

    \begin{itemize}

    \item 
    
    \end{itemize}

%%%%%%%%%%%%%%%%%%%%%%%%%%%%%%%%%%%%%%%%%%%%%%%%%%%%%%%%%%%%%%%%%%%%%%%%%%%%%
%%%%%%%%%%%%%%%%%%%%%%%%%%%%%%%%%%%%%%%%%%%%%%%%%%%%%%%%%%%%%%%%%%%%%%%%%%%%%
\section{Introduction (pg. 1)}

%%%%%%%%%%%%%%%%%%%%%%%%%%%%%%%%%%%%%%%%%%%%%%%%%%%%%%%%%%%%%%%%%%%%%%%%%%%%%
%%%%%%%%%%%%%%%%%%%%%%%%%%%%%%%%%%%%%%%%%%%%%%%%%%%%%%%%%%%%%%%%%%%%%%%%%%%%%
\section{Further Preliminaries (pg. 6)}

%%%%%%%%%%%%%%%%%%%%%%%%%%%%%%%%%%%%%%%%%%%%%%%%%%%%%%%%%%%%%%%%%%%%%%%%%%%%%
\subsection{Basics about Coherent States (pg. 6)}

%%%%%%%%%%%%%%%%%%%%%%%%%%%%%%%%%%%%%%%%%%%%%%%%%%%%%%%%%%%%%%%%%%%%%%%%%%%%%
\subsection{Uncertainty relations and Coherent states on Commutative $S^1$ (pg. 7)}

%%%%%%%%%%%%%%%%%%%%%%%%%%%%%%%%%%%%%%%%%%%%%%%%%%%%%%%%%%%%%%%%%%%%%%%%%%%%%
\subsection{Uncertainty relations and Coherent states on Commutative $S^2$ (pg. 9)}

%%%%%%%%%%%%%%%%%%%%%%%%%%%%%%%%%%%%%%%%%%%%%%%%%%%%%%%%%%%%%%%%%%%%%%%%%%%%%
\subsection{Diagonalization of Toeplitz tridiagonal matrices (pg. 10)}

%%%%%%%%%%%%%%%%%%%%%%%%%%%%%%%%%%%%%%%%%%%%%%%%%%%%%%%%%%%%%%%%%%%%%%%%%%%%%
%%%%%%%%%%%%%%%%%%%%%%%%%%%%%%%%%%%%%%%%%%%%%%%%%%%%%%%%%%%%%%%%%%%%%%%%%%%%%
\section{Coherent and Localized states on the fuzzy circle $S^1_{\Lambda}$ (pg. 10)}

%%%%%%%%%%%%%%%%%%%%%%%%%%%%%%%%%%%%%%%%%%%%%%%%%%%%%%%%%%%%%%%%%%%%%%%%%%%%%
\subsection{$O(2)$-invariant Uncertainty Relations (UR) and strong System of Coherent States (SCS) on $S^1_\Lambda$ (pg. 12)}

%%%%%%%%%%%%%%%%%%%%%%%%%%%%%%%%%%%%%%%%%%%%%%%%%%%%%%%%%%%%%%%%%%%%%%%%%%%%%
\subsection{$O(2)$-invariant weak SCS on $S^1_\Lambda$ minimizing $(\Delta x)^2$ (pg. 14)}


%%%%%%%%%%%%%%%%%%%%%%%%%%%%%%%%%%%%%%%%%%%%%%%%%%%%%%%%%%%%%%%%%%%%%%%%%%%%%
%%%%%%%%%%%%%%%%%%%%%%%%%%%%%%%%%%%%%%%%%%%%%%%%%%%%%%%%%%%%%%%%%%%%%%%%%%%%%
\section{Coherent and Localized states on the fuzzy sphere $S^2_\Lambda$ (pg. 15)}

%%%%%%%%%%%%%%%%%%%%%%%%%%%%%%%%%%%%%%%%%%%%%%%%%%%%%%%%%%%%%%%%%%%%%%%%%%%%%
\subsection{$O(3)$-invariant UR and strong SCS on $S^2_\Lambda$ (pg. 17)}

%%%%%%%%%%%%%%%%%%%%%%%%%%%%%%%%%%%%%%%%%%%%%%%%%%%%%%%%%%%%%%%%%%%%%%%%%%%%%
\subsection{$O(3)$-invariant weak SCS on $S^2_\Lambda$ minimizing $(\Delta x)^2$ (pg. 19)}

%%%%%%%%%%%%%%%%%%%%%%%%%%%%%%%%%%%%%%%%%%%%%%%%%%%%%%%%%%%%%%%%%%%%%%%%%%%%%
%%%%%%%%%%%%%%%%%%%%%%%%%%%%%%%%%%%%%%%%%%%%%%%%%%%%%%%%%%%%%%%%%%%%%%%%%%%%%
\section{Outlook, Final remarks, and Conclusions (pg. 21)}

%%%%%%%%%%%%%%%%%%%%%%%%%%%%%%%%%%%%%%%%%%%%%%%%%%%%%%%%%%%%%%%%%%%%%%%%%%%%%
%%%%%%%%%%%%%%%%%%%%%%%%%%%%%%%%%%%%%%%%%%%%%%%%%%%%%%%%%%%%%%%%%%%%%%%%%%%%%
\section{Appendices (pg. 23)}

%%%%%%%%%%%%%%%%%%%%%%%%%%%%%%%%%%%%%%%%%%%%%%%%%%%%%%%%%%%%%%%%%%%%%%%%%%%%%
\subsection{Proof of Theorems $2.1$ and $4.1$ (pg. 23)}

%%%%%%%%%%%%%%%%%%%%%%%%%%%%%%%%%%%%%%%%%%%%%%%%%%%%%%%%%%%%%%%%%%%%%%%%%%%%%
\subsection{Some useful summations (pg. 24)}

%%%%%%%%%%%%%%%%%%%%%%%%%%%%%%%%%%%%%%%%%%%%%%%%%%%%%%%%%%%%%%%%%%%%%%%%%%%%%
\subsection{Proofs of some results regarding $S^1_\Lambda$ (pg. 25)}

%%%%%%%%%%%%%%%%%%%%%%%%%%%%%%%%%%%%%%%%%%%%%%%%%%%%%%%%%%%%%%%%%%%%%%%%%%%%%
\subsection{States saturating the Heisenberg UR on $S^1$, $S^1_\Lambda$ (pg. 27)}

%%%%%%%%%%%%%%%%%%%%%%%%%%%%%%%%%%%%%%%%%%%%%%%%%%%%%%%%%%%%%%%%%%%%%%%%%%%%%
\subsection{Proof of Theorem 4.2 (pg. 29)}

%%%%%%%%%%%%%%%%%%%%%%%%%%%%%%%%%%%%%%%%%%%%%%%%%%%%%%%%%%%%%%%%%%%%%%%%%%%%%
\subsection{Proof of Lemma 6.2 (pg. 30)}

%%%%%%%%%%%%%%%%%%%%%%%%%%%%%%%%%%%%%%%%%%%%%%%%%%%%%%%%%%%%%%%%%%%%%%%%%%%%%
\subsection{Proofs of some results regarding $S^2_\Lambda$ (pg. 36-39)}

\end{document}
