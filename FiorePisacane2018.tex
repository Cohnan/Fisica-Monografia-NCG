\documentclass{article}
\usepackage[utf8]{inputenc}
\usepackage[margin=1in]{geometry}

%%%%%% To use hyperlinks, including the formula ones
\usepackage{hyperref}
\hypersetup{
    colorlinks=true,
    linkcolor=blue,
    filecolor=magenta,      
    urlcolor=cyan,
}

%%%%%% Make paragraphs start with no indentation and leave spaces between paragraphs
\setlength{\parindent}{0em}
\setlength{\parskip}{1em}

%%%%%% Math stuff
\usepackage{amsmath, amssymb}
\usepackage{amsthm}

%%%%%% Mis Codigos

% TODO notes package
\usepackage{xargs}                  % Use more than one optional parameter in a new command
\usepackage[pdftex,dvipsnames]{xcolor}
%\usepackage{xargs}                      % Use more than one optional parameter in a new
%\usepackage[pdftex,dvipsnames]{xcolor}  % Coloured text etc.

%
%\usepackage[colorinlistoftodos,prependcaption,textsize=tiny]{todonotes}
\usepackage[disable]{todonotes}

\newcommandx{\unsure}[2][1=]{\todo[linecolor=red,backgroundcolor=red!25,bordercolor=red,#1]{#2}}
\newcommandx{\change}[2][1=]{\todo[linecolor=blue,backgroundcolor=blue!25,bordercolor=blue,#1]{#2}}
\newcommandx{\complete}[2][1=]{\todo[linecolor=pink,backgroundcolor=pink!25,bordercolor=blue,#1]{#2}}
\newcommandx{\info}[2][1=]{\todo[linecolor=OliveGreen,backgroundcolor=OliveGreen!25,bordercolor=OliveGreen,#1]{#2}}
\newcommandx{\improvement}[2][1=]{\todo[linecolor=Plum,backgroundcolor=Plum!25,bordercolor=Plum,#1]{#2}}
\newcommandx{\thiswillnotshow}[2][1=]{\todo[disable,#1]{#2}}

% Colored text and boxes with my color conventions for highlighting
\usepackage[dvipsnames]{xcolor}
%\usepackage[dvipsnames]{xcolor}

%
% \newcommand{\ytext}[1]{\textcolor{yellow}{#1}}
% \newcommand{\otext}[1]{\textcolor{orange}{#1}}
% \newcommand{\rtext}[1]{\textcolor{red}{#1}}
% \newcommand{\lbtext}[1]{\textcolor{cyan}{#1}}
% \newcommand{\dbtext}[1]{\textcolor{blue}{#1}}
% \newcommand{\ptext}[1]{\textcolor{Plum}{#1}}
% \newcommand{\lgtext}[1]{\textcolor{LimeGreen}{#1}}
% \newcommand{\dgtext}[1]{\textcolor{OliveGreen}{#1}}

\newcommand{\ytext}[1]{\textcolor{black}{#1}}
\newcommand{\otext}[1]{\textcolor{black}{#1}}
\newcommand{\rtext}[1]{\textit{#1}}
\newcommand{\lbtext}[1]{\textcolor{black}{#1}}
\newcommand{\dbtext}[1]{\textcolor{black}{#1}}
\newcommand{\ptext}[1]{\textcolor{black}{#1}}
\newcommand{\lgtext}[1]{\textcolor{black}{#1}}
\newcommand{\dgtext}[1]{\textcolor{black}{#1}}



\newcommand{\ybox}[1]{\colorbox{yellow}{#1}}
\newcommand{\obox}[1]{\colorbox{orange}{#1}}
\newcommand{\rbox}[1]{\colorbox{Salmon}{#1}}
\newcommand{\lbbox}[1]{\colorbox{SkyBlue}{#1}}
\newcommand{\dbbox}[1]{\colorbox{NavyBlue}{#1}}
\newcommand{\pbox}[1]{\colorbox{Plum}{#1}}
\newcommand{\lgbox}[1]{\colorbox{LimeGreen}{#1}}
\newcommand{\dgbox}[1]{\colorbox{OliveGreen}{#1}}



% Math symbols
\usepackage{xparse}

%\usepackage{amssymb,amsmath,amsthm}
%\usepackage{xparse}

%%% Common symbols redifined
\let\oldepsilon\epsilon
\renewcommand{\epsilon}{\varepsilon}
\renewcommand{\varepsilon}{\oldepsilon}

\let\oldphi\phi
\renewcommand{\phi}{\varphi}
\renewcommand{\varphi}{\oldphi}

\newcommand{\emty}{\varnothing}
\newcommand{\varempty}{\nothing}

%%% Common Sets
\newcommand{\bb}[1]{\ensuremath{\mathbb{#1}} }
\newcommand{\ZZ}{\ensuremath{\mathbb{Z}} }
\newcommand{\NN}{\ensuremath{\mathbb{N}} }
\newcommand{\QQ}{\ensuremath{\mathbb{Q}} }
\newcommand{\RR}{\ensuremath{\mathbb{R}} }
\newcommand{\CC}{\ensuremath{\mathbb{C}} }


\newcommand{\iss}{\cong}  % Isomorphism symbol, here it is the one with a tilde

%%%%%%%%%%%%%% Sets

\newcommand{\set}[1]{\ensuremath{\left\{ #1 \right\}}} % Set function, simply puts nice left and right braces
\newcommand{\st}{\ensuremath{\ |\ }} % Such that symbol, TODO improve

\newcommand{\psubset}{\ensuremath{\subset}}
\renewcommand{\subset}{\ensuremath{\subseteq}} % Subset symbol, in this case it is the subset or equal to symbol

\newcommand{\bunion}{\bigcup}
%\newcommand{\biun}{\bun^{\infty}}
%\newcommand{\bfun}[3][n]{\bun_{#1 = #2}^{#3}}
\newcommand{\union}{\cup}
%\newcommand{\siun}{\sun^{\infty}}
%\newcommand{\sfun}[3][n]{\sun_{#1 = #2}^{#3}}

\newcommand{\binter}{\bigcap}
%\newcommand{\biinter}{\binter^{\infty}}
%\newcommand{\bfifter}[3][n]{\binter_{#1 = #2}^{#3}}
\newcommand{\inter}{\cap}
%\newcommand{\sininter}{\sinter^{\infty}}
%\newcommand{\sfinter}[3][n]{\sinter_{#1 = #2}^{#3}}

%%%%%% Calculus

% Integrals

%% Single Integrals
\NewDocumentCommand \integ {s O{} O{} m o}
{
	\IfBooleanTF{#1}{\oint}{\int}_{#2}^{#3} #4 %
	\IfNoValueF {#5} {\mathrm{d} #5}
}

%% Derivatives

% Normal derivative
% Example: \der[n]{f}{x}[x_0]
\NewDocumentCommand \der {O{} m m o}
{
	\frac{\mathrm{d}^{#1} #2}{\mathrm{d} {#3}^{#1}}%
	\IfNoValueF{#4} {\biggr|_{#4}}
}

%% Partial Derivatives

% With respect to one variable
% Example: \pder[n]{f}{y}[\pthvars[x_0][y_0][z_0]][(x, z)
\NewDocumentCommand \pder {O{} m m O{} o}
{
    \ensuremath{
	\IfNoValueTF {#5}
	{
		\frac{\partial^{#1} #2}{\partial {#3}^{#1}} #4
	}
	{
		\left(%
		\frac{\partial^{#1} #2}{\partial {#3}^{#1}}%
		\right)_{#5}  #4
	}
	}
}

% With respecto to two variables
% Example: \twpder{g}{y}{x}[(x_0, y_0)]
\NewDocumentCommand \twpder {m m m O{}}
{
	\frac{\partial^2 #1}{\partial #2 \partial #3} #4
}

\newcommand{\abs}[1]{\left\lvert #1 \right\rvert}
\newcommand{\norm}[1]{\left\lVert #1 \right\rVert}

% Physics symbols (vectors, units)
\usepackage{tikz}
% Version of December 26 2016

%%%%%%%%%%%%%%%%%%%%%%%%%%%%%%%%%%%%%%%%%%% Vectors

%%%%%%%% For SI units %%%%%%%
%\usepackage{siunitx}

%%%%%%%% Vectors %%%%%%%%%%%% (Just see last lines)
%\usepackage{tikz}         % For arrow and dots in \xvec

% --- Macro \xvec
\makeatletter
\newlength\xvec@height%
\newlength\xvec@depth%
\newlength\xvec@width%
\newcommand{\xvec}[2][]{%
  \ifmmode%
    \settoheight{\xvec@height}{$#2$}%
    \settodepth{\xvec@depth}{$#2$}%
    \settowidth{\xvec@width}{$#2$}%
  \else%
    \settoheight{\xvec@height}{#2}%
    \settodepth{\xvec@depth}{#2}%
    \settowidth{\xvec@width}{#2}%
  \fi%
  \def\xvec@arg{#1}%
  \def\xvec@dd{:}%
  \def\xvec@d{.}%
  \raisebox{.2ex}{\raisebox{\xvec@height}{\rlap{%
    \kern.05em%  (Because left edge of drawing is at .05em)
    \begin{tikzpicture}[scale=1]
    \pgfsetroundcap
    \draw (.05em,0)--(\xvec@width-.05em,0);
    \draw (\xvec@width-.05em,0)--(\xvec@width-.15em, .075em);
    \draw (\xvec@width-.05em,0)--(\xvec@width-.15em,-.075em);
    \ifx\xvec@arg\xvec@d%
      \fill(\xvec@width*.45,.5ex) circle (.5pt);%
    \else\ifx\xvec@arg\xvec@dd%
      \fill(\xvec@width*.30,.5ex) circle (.5pt);%
      \fill(\xvec@width*.65,.5ex) circle (.5pt);%
    \fi\fi%
    \end{tikzpicture}%
  }}}%
  #2%
}
\makeatother

% --- Override \vec with an invocation of \xvec.
\let\stdvec\vec
\renewcommand{\vec}[1]{\xvec[]{#1}}                             % Vector
% --- Define \dvec and \ddvec for dotted and double-dotted vectors.
\newcommand{\tvec}[1]{\xvec[.]{#1}}                             % Vector derived wrt time
\newcommand{\ttvec}[1]{\xvec[:]{#1}}                            % Vector derived twice wrt time


% Theorem environments
%Version of October 8, 2016

%\usepackage{amsthm}

\theoremstyle{definition} %To avoid the annoying italics all the time, and to not sloppily redefine all of them 

\newtheorem{theo}{Theorem}[section]  %numbered according to section environment, so in section to it restarts as 2.1 
\newtheorem{prop}{Proposition}[section]  %numbered according to section environment, so in section to it restarts as 2.1 
\newtheorem{lemma}[theo]{Lemma}     %numbering shared with theorem 
\newtheorem{defn}{Definition}[section]   
\newtheorem{coro}{Corollary}[theo]


\theoremstyle{remark} 
\newtheorem*{remark}{Remark} 
 
%\let\oldtheo\theo 
%\renewcommand{\theo}{\oldtheo\normalfont}  
%  
%\let\olddefn\defn  
%\renewcommand{\defn}{\olddefn\normalfont}  
%  
%\let\oldlemma\lemma  
%\renewcommand{\lemma}{\oldlemma\normalfont}  
%  
%\let\oldcoro\coro  
%\renewcommand{\coro}{\oldcoro\normalfont}

%%%%%%%% ``Example'' environment, very basic, doesnt work with itemize
\theoremstyle{definition}

\newtheorem*{exmp}{Example}

%%%%%%
\title{Fiore Pisacane 2018: Fuzzy circle and new fuzzy sphere through confining potentials and energy cutoffs}
\author{Sebastian Camilo Puerto}
\date{July 2020}

%%%%%%%%%%%%%%%%%%%%%%%%%%%%%%%%%%%%%%%%%%%%%%%%%%%%%%%%%%%%%%%%%%%%%%%%%%%%%
%%%%%%%%%%%%%%%%%%%%%%%%%%%%%%%%%%%%%%%%%%%%%%%%%%%%%%%%%%%%%%%%%%%%%%%%%%%%%
%%%%%%%%%%%%%%%%%%%%%%%%%%%%%%%%%%%%%%%%%%%%%%%%%%%%%%%%%%%%%%%%%%%%%%%%%%%%%
%%%%%%%%%%%%%%%%%%%%%%%%%%%%%%%%%%%%%%%%%%%%%%%%%%%%%%%%%%%%%%%%%%%%%%%%%%%%%
\begin{document}

\maketitle

\tableofcontents

The covariance here has a different meaning to the one for the Madore Fuzzy sphere, since or at least a bigger meaning, since we are thinking of the algebras NOT only as noncommutative version of the space, but also as approximation of the algebra of observables of a quantum system, which is the NC $\mathcal B(L^2(S^d))$.
    \begin{itemize}
        
    \item As a NC approximation of a space, the equivariance under a group means that: 
    
        \begin{enumerate}
        
        \item The group acts by ``diffeomorphisms'', i.e. by $*$-algebra morphisms % C^*? normed & *-algebra?
        that respect the derivations.
        
        \item The group acts by ``isometries'': the spectral triples are equivariant, i.e. the group action on the Hilbert space commutes with the actions of both the Dirac operator and the Algebra of the spectral triples.
        
        \item The ``NC approximations of the points'' of $S^2$ are equivariant: this interpretation appears since there are states of the approximating algebras that can be thought of as approximations of the points of the commutative space, and it turns out that these identifications are also $SU(2)$-equivariant.
        
        \end{enumerate}
    
    \item However, as ``approximations of QM is $S^d$'', i.e. as algebras of operators acting on wavefunctions $\in \mathcal B(L^2(S^d))$, they are equivariant in the sense of symmetries of a dynamical system: the group acts on the new observables, i.e. the algebras $\mathcal A_\Lambda$ as operators on $\mathcal B(\mathcal H_\Lambda)$, and the action of the group on $\mathcal H_\Lambda$ commutes with the action of the observables (specially $L^2$ and $L_z$). 
    
        \begin{itemize}
            
        \item Recall that in the Hamiltonian formalism of classical mechanics, a symmetry of the system is a group action that leaves invariant the observables $C^P(mathcal P) = \mathcal A$. Also recall that an observable is a constant of motion if its bracket with the Hamiltonian $H$ is 0. {\tiny In the lagrangian formalism a symmetry of the system is a group action that doesn't change the lagrangian}.
        
        \item When the observables are operators, a group is a symmetry of the system if the group action and the observables commute: once the group action $A \mapsto A^g = g \circ A \circ g^{-1}$ on the observables is induced, this means that $A^g$ is ``the same'' as $A$, but on the ``rotated'' space $g\mathcal H$: \lbtext{the algebra of observables is $G$-equivariant/covariant} CREO.
            
        \end{itemize}
    
    \end{itemize}

%%%%%%%%%%%%%%%%%%%%%%%%%%%%%%%%%%%%%%%%%%%%%%%%%%%%%%%%%%%%%%%%%%%%%%%%%%%%%
\subsection{High Level Summary}

    \begin{itemize}

    \item 
    
    \end{itemize}

%%%%%%%%%%%%%%%%%%%%%%%%%%%%%%%%%%%%%%%%%%%%%%%%%%%%%%%%%%%%%%%%%%%%%%%%%%%%%
\subsection{Very Important Facts}

    \begin{itemize}

    \item  We are thinking of the algebras NOT really as noncommutative version of the space, of $L^2(S^2)$, but also as approximation of the algebra of observables of a quantum system, which is the NC $\mathcal B(L^2(S^d))$.
    
        \begin{itemize}
            
        \item $\mathcal A_\Lambda \to \mathcal B(L^2(S^d))$
        
        \item $\mathcal C_\Lambda \subset \mathcal A_N \to \text{ and } \hookrightarrow C(S^d)$ space of polynomials on the coordinates $x^i$.
        
        \item Also, $\mathcal H_\Lambda \to \to \text{ and } \hookrightarrow L^2(S^d)$
            
        \end{itemize}
    
    \item The fuzzy sphere gives us a sequence of algebras which can all be though of as approximations of $\mathcal B(L^2(S^2))$. Each of these approximations can be seen to come from: if only energies below a certain cutoff value $\overline{E}_\Lambda$ are accessible for the wavefunctions, and for the accessible energies the potential is nearly harmonic near $r = 1$ and ``sufficiently'' steep THEN the accessible state space $\mathcal H_{\overline{E}}$ can be studied with the projected observables: $A \mapsto \overline{A} := P_{\overline E} A P_{\overline E}$; this, in particular, means that new commutation relations appear for the coordinate functions $\hat x^i$.
    
    \item The new dynamical system has $O(d)$ as a symmetry: at least it is ``true'' that the Hamiltonian is $O(d)$-covariant (which I'm defining to mean that the group action and $H$ commute) and that acting with this group on a vector state $\psi \in \mathcal H_\Lambda$ doesn't  take it outside $\mathcal H_\lambda$ nor changes its eigenvalues under $L^2$ and $L_z$ (wait... are there actual eigenvalues of these observables or just approximate ones?)
    
    \item The coordinates generate the whole algebra $\mathcal A_{\overline{E}}$.
    
    \item $\mathcal A_{\overline{E}}$ can be realized as the (\dbtext{sub?})algebra of operators of an irrep. $\pi_{\overline{E}}$ of $so(d+2)$. This means that $\mathcal H_{\overline{E}}$ can be though of as an irrep. of $so(d+2)$, which, in particular, means that $\mathcal H_{\overline{E}}$ is a reducible representation of $so(d+1)$, namely, $\mathcal H_{\overline{E}} \cong \bigoplus_{E \leq \overline{E}} V_l$ as representation space of $so(d+1)$.
    
    \item Embeddings and limits: a subalgebra $\mathcal C_\Lambda$ of $\mathcal A_{\Lambda}$ which has the $so(d+1)$-module decomposition $\bigoplus_{E \leq \overline{E}} V_l$ does approximate the commutative space.
    
    \item Basis of $\mathcal A_{\overline{E}}$
    
    \item $L_{ij}$ and $x^i$ tend to the angular momentum and coordinate functions of $S^d$. (Elements of $\mathcal C_\Lambda$?)
    
    \end{itemize}

%%%%%%%%%%%%%%%%%%%%%%%%%%%%%%%%%%%%%%%%%%%%%%%%%%%%%%%%%%%%%%%%%%%%%%%%%%%%%
\subsection{Important Facts}

    \begin{itemize}

    \item 
    
    \end{itemize}

%%%%%%%%%%%%%%%%%%%%%%%%%%%%%%%%%%%%%%%%%%%%%%%%%%%%%%%%%%%%%%%%%%%%%%%%%%%%%
\subsection{Memorize}

    \begin{itemize}

    \item 
    
    \end{itemize}

%%%%%%%%%%%%%%%%%%%%%%%%%%%%%%%%%%%%%%%%%%%%%%%%%%%%%%%%%%%%%%%%%%%%%%%%%%%%%
\subsection{Doubts}

    \begin{itemize}

    \item 
    
    \end{itemize}

%%%%%%%%%%%%%%%%%%%%%%%%%%%%%%%%%%%%%%%%%%%%%%%%%%%%%%%%%%%%%%%%%%%%%%%%%%%%%
\subsection{Detailed summary}

    \begin{itemize}

    \item 
    
    \end{itemize}

%%%%%%%%%%%%%%%%%%%%%%%%%%%%%%%%%%%%%%%%%%%%%%%%%%%%%%%%%%%%%%%%%%%%%%%%%%%%%
\subsection{Notice}

    \begin{itemize}

    \item 
    
    \end{itemize}

%%%%%%%%%%%%%%%%%%%%%%%%%%%%%%%%%%%%%%%%%%%%%%%%%%%%%%%%%%%%%%%%%%%%%%%%%%%%% 
\subsection{Yet to understand}

    \begin{itemize}

    \item 
    
    \end{itemize}

%%%%%%%%%%%%%%%%%%%%%%%%%%%%%%%%%%%%%%%%%%%%%%%%%%%%%%%%%%%%%%%%%%%%%%%%%%%%%
%%%%%%%%%%%%%%%%%%%%%%%%%%%%%%%%%%%%%%%%%%%%%%%%%%%%%%%%%%%%%%%%%%%%%%%%%%%%%
\section{Introduction (pg. 1)}

%%%%%%%%%%%%%%%%%%%%%%%%%%%%%%%%%%%%%%%%%%%%%%%%%%%%%%%%%%%%%%%%%%%%%%%%%%%%%
%%%%%%%%%%%%%%%%%%%%%%%%%%%%%%%%%%%%%%%%%%%%%%%%%%%%%%%%%%%%%%%%%%%%%%%%%%%%%
\section{General setting (pg. 5)}

%%%%%%%%%%%%%%%%%%%%%%%%%%%%%%%%%%%%%%%%%%%%%%%%%%%%%%%%%%%%%%%%%%%%%%%%%%%%%
%%%%%%%%%%%%%%%%%%%%%%%%%%%%%%%%%%%%%%%%%%%%%%%%%%%%%%%%%%%%%%%%%%%%%%%%%%%%%
\section{$D=2$: $O(2)$-equivariant fuzzy circle (pg. 5.75)}

%%%%%%%%%%%%%%%%%%%%%%%%%%%%%%%%%%%%%%%%%%%%%%%%%%%%%%%%%%%%%%%%%%%%%%%%%%%%%
\subsection{Realization of the Algebra of Observables throuth $Uso(3)$ (pg. 11.5)}

%%%%%%%%%%%%%%%%%%%%%%%%%%%%%%%%%%%%%%%%%%%%%%%%%%%%%%%%%%%%%%%%%%%%%%%%%%%%%
\subsection{Convergence to $O(2)$-equivariant quantum mechanics on $S^1$ (pg. 13.5)}

%%%%%%%%%%%%%%%%%%%%%%%%%%%%%%%%%%%%%%%%%%%%%%%%%%%%%%%%%%%%%%%%%%%%%%%%%%%%%
%%%%%%%%%%%%%%%%%%%%%%%%%%%%%%%%%%%%%%%%%%%%%%%%%%%%%%%%%%%%%%%%%%%%%%%%%%%%%
\section{$D=3$: $O(3)$-equivariant fuzzy sphere (pg. 15)}

%%%%%%%%%%%%%%%%%%%%%%%%%%%%%%%%%%%%%%%%%%%%%%%%%%%%%%%%%%%%%%%%%%%%%%%%%%%%%
\subsection{Realization of the Algebra of Observables throuth $Uso(4)$ (pg. 17.75)}

%%%%%%%%%%%%%%%%%%%%%%%%%%%%%%%%%%%%%%%%%%%%%%%%%%%%%%%%%%%%%%%%%%%%%%%%%%%%%
\subsection{Convergence to $O(3)$-equivariant quantum mechanics on $S^2$ (pg. 19.75)}

%%%%%%%%%%%%%%%%%%%%%%%%%%%%%%%%%%%%%%%%%%%%%%%%%%%%%%%%%%%%%%%%%%%%%%%%%%%%%
%%%%%%%%%%%%%%%%%%%%%%%%%%%%%%%%%%%%%%%%%%%%%%%%%%%%%%%%%%%%%%%%%%%%%%%%%%%%%
\section{Final remarks, Outlook and Conclusions (pg. 21.8)}

%%%%%%%%%%%%%%%%%%%%%%%%%%%%%%%%%%%%%%%%%%%%%%%%%%%%%%%%%%%%%%%%%%%%%%%%%%%%%
%%%%%%%%%%%%%%%%%%%%%%%%%%%%%%%%%%%%%%%%%%%%%%%%%%%%%%%%%%%%%%%%%%%%%%%%%%%%%
\section{Appendix (pg. 23)}

%%%%%%%%%%%%%%%%%%%%%%%%%%%%%%%%%%%%%%%%%%%%%%%%%%%%%%%%%%%%%%%%%%%%%%%%%%%%%
\subsection{Calculation of a rather general scalar product in $D=2$ (pg. 23)}

%%%%%%%%%%%%%%%%%%%%%%%%%%%%%%%%%%%%%%%%%%%%%%%%%%%%%%%%%%%%%%%%%%%%%%%%%%%%%
\subsection{Calculation of the action of operators in $D=2$ (pg. 26)}

%%%%%%%%%%%%%%%%%%%%%%%%%%%%%%%%%%%%%%%%%%%%%%%%%%%%%%%%%%%%%%%%%%%%%%%%%%%%%
\subsection{Proof of proposition [??] (pg. 28)}

%%%%%%%%%%%%%%%%%%%%%%%%%%%%%%%%%%%%%%%%%%%%%%%%%%%%%%%%%%%%%%%%%%%%%%%%%%%%%
\subsection{Spherical Harmonics (pg. 29)}

%%%%%%%%%%%%%%%%%%%%%%%%%%%%%%%%%%%%%%%%%%%%%%%%%%%%%%%%%%%%%%%%%%%%%%%%%%%%%
\subsection{Calculation of $|N_l|$ in $D = 3$ (pg. 30)}

%%%%%%%%%%%%%%%%%%%%%%%%%%%%%%%%%%%%%%%%%%%%%%%%%%%%%%%%%%%%%%%%%%%%%%%%%%%%%
\subsection{Calculation of a rather general scalar product in $D = 3$ (pg. 30)}

%%%%%%%%%%%%%%%%%%%%%%%%%%%%%%%%%%%%%%%%%%%%%%%%%%%%%%%%%%%%%%%%%%%%%%%%%%%%%
\subsection{Proof or (40) and or proposition 4.1 (pg. 32)}

%%%%%%%%%%%%%%%%%%%%%%%%%%%%%%%%%%%%%%%%%%%%%%%%%%%%%%%%%%%%%%%%%%%%%%%%%%%%%
\subsection{Action of commutators of the $\overline{\partial_a}$ (pg. 34)}

%%%%%%%%%%%%%%%%%%%%%%%%%%%%%%%%%%%%%%%%%%%%%%%%%%%%%%%%%%%%%%%%%%%%%%%%%%%%%
\subsection{Proof of proposition 4.2 and other results of subsection 4.1 (pg. 35)}

%%%%%%%%%%%%%%%%%%%%%%%%%%%%%%%%%%%%%%%%%%%%%%%%%%%%%%%%%%%%%%%%%%%%%%%%%%%%%
\subsection{Shifting the lower extreme of integration over $r$ (pg. 37)}

%%%%%%%%%%%%%%%%%%%%%%%%%%%%%%%%%%%%%%%%%%%%%%%%%%%%%%%%%%%%%%%%%%%%%%%%%%%%%
\subsection{Proof of proposition 4.3 and other results of section 4.2 (pg. 38-41)}

\end{document}
