\documentclass{article}
\usepackage[utf8]{inputenc}
\usepackage[margin=1in]{geometry}

%%%%%% To use hyperlinks, including the formula ones
\usepackage{hyperref}
\hypersetup{
    colorlinks=true,
    linkcolor=blue,
    filecolor=magenta,      
    urlcolor=cyan,
}

%%%%%% Make paragraphs start with no indentation and leave spaces between paragraphs
\setlength{\parindent}{0em}
\setlength{\parskip}{1em}

%%%%%% Math stuff
\usepackage{amsmath, amssymb}
\usepackage{amsthm}

%%%%%% Mis Codigos

% TODO notes package
\usepackage{xargs}                  % Use more than one optional parameter in a new command
\usepackage[pdftex,dvipsnames]{xcolor}
%\usepackage{xargs}                      % Use more than one optional parameter in a new
%\usepackage[pdftex,dvipsnames]{xcolor}  % Coloured text etc.

%
%\usepackage[colorinlistoftodos,prependcaption,textsize=tiny]{todonotes}
\usepackage[disable]{todonotes}

\newcommandx{\unsure}[2][1=]{\todo[linecolor=red,backgroundcolor=red!25,bordercolor=red,#1]{#2}}
\newcommandx{\change}[2][1=]{\todo[linecolor=blue,backgroundcolor=blue!25,bordercolor=blue,#1]{#2}}
\newcommandx{\complete}[2][1=]{\todo[linecolor=pink,backgroundcolor=pink!25,bordercolor=blue,#1]{#2}}
\newcommandx{\info}[2][1=]{\todo[linecolor=OliveGreen,backgroundcolor=OliveGreen!25,bordercolor=OliveGreen,#1]{#2}}
\newcommandx{\improvement}[2][1=]{\todo[linecolor=Plum,backgroundcolor=Plum!25,bordercolor=Plum,#1]{#2}}
\newcommandx{\thiswillnotshow}[2][1=]{\todo[disable,#1]{#2}}

% Colored text and boxes with my color conventions for highlighting
\usepackage[dvipsnames]{xcolor}
%\usepackage[dvipsnames]{xcolor}

%
% \newcommand{\ytext}[1]{\textcolor{yellow}{#1}}
% \newcommand{\otext}[1]{\textcolor{orange}{#1}}
% \newcommand{\rtext}[1]{\textcolor{red}{#1}}
% \newcommand{\lbtext}[1]{\textcolor{cyan}{#1}}
% \newcommand{\dbtext}[1]{\textcolor{blue}{#1}}
% \newcommand{\ptext}[1]{\textcolor{Plum}{#1}}
% \newcommand{\lgtext}[1]{\textcolor{LimeGreen}{#1}}
% \newcommand{\dgtext}[1]{\textcolor{OliveGreen}{#1}}

\newcommand{\ytext}[1]{\textcolor{black}{#1}}
\newcommand{\otext}[1]{\textcolor{black}{#1}}
\newcommand{\rtext}[1]{\textit{#1}}
\newcommand{\lbtext}[1]{\textcolor{black}{#1}}
\newcommand{\dbtext}[1]{\textcolor{black}{#1}}
\newcommand{\ptext}[1]{\textcolor{black}{#1}}
\newcommand{\lgtext}[1]{\textcolor{black}{#1}}
\newcommand{\dgtext}[1]{\textcolor{black}{#1}}



\newcommand{\ybox}[1]{\colorbox{yellow}{#1}}
\newcommand{\obox}[1]{\colorbox{orange}{#1}}
\newcommand{\rbox}[1]{\colorbox{Salmon}{#1}}
\newcommand{\lbbox}[1]{\colorbox{SkyBlue}{#1}}
\newcommand{\dbbox}[1]{\colorbox{NavyBlue}{#1}}
\newcommand{\pbox}[1]{\colorbox{Plum}{#1}}
\newcommand{\lgbox}[1]{\colorbox{LimeGreen}{#1}}
\newcommand{\dgbox}[1]{\colorbox{OliveGreen}{#1}}



% Math symbols
\usepackage{xparse}

%\usepackage{amssymb,amsmath,amsthm}
%\usepackage{xparse}

%%% Common symbols redifined
\let\oldepsilon\epsilon
\renewcommand{\epsilon}{\varepsilon}
\renewcommand{\varepsilon}{\oldepsilon}

\let\oldphi\phi
\renewcommand{\phi}{\varphi}
\renewcommand{\varphi}{\oldphi}

\newcommand{\emty}{\varnothing}
\newcommand{\varempty}{\nothing}

%%% Common Sets
\newcommand{\bb}[1]{\ensuremath{\mathbb{#1}} }
\newcommand{\ZZ}{\ensuremath{\mathbb{Z}} }
\newcommand{\NN}{\ensuremath{\mathbb{N}} }
\newcommand{\QQ}{\ensuremath{\mathbb{Q}} }
\newcommand{\RR}{\ensuremath{\mathbb{R}} }
\newcommand{\CC}{\ensuremath{\mathbb{C}} }


\newcommand{\iss}{\cong}  % Isomorphism symbol, here it is the one with a tilde

%%%%%%%%%%%%%% Sets

\newcommand{\set}[1]{\ensuremath{\left\{ #1 \right\}}} % Set function, simply puts nice left and right braces
\newcommand{\st}{\ensuremath{\ |\ }} % Such that symbol, TODO improve

\newcommand{\psubset}{\ensuremath{\subset}}
\renewcommand{\subset}{\ensuremath{\subseteq}} % Subset symbol, in this case it is the subset or equal to symbol

\newcommand{\bunion}{\bigcup}
%\newcommand{\biun}{\bun^{\infty}}
%\newcommand{\bfun}[3][n]{\bun_{#1 = #2}^{#3}}
\newcommand{\union}{\cup}
%\newcommand{\siun}{\sun^{\infty}}
%\newcommand{\sfun}[3][n]{\sun_{#1 = #2}^{#3}}

\newcommand{\binter}{\bigcap}
%\newcommand{\biinter}{\binter^{\infty}}
%\newcommand{\bfifter}[3][n]{\binter_{#1 = #2}^{#3}}
\newcommand{\inter}{\cap}
%\newcommand{\sininter}{\sinter^{\infty}}
%\newcommand{\sfinter}[3][n]{\sinter_{#1 = #2}^{#3}}

%%%%%% Calculus

% Integrals

%% Single Integrals
\NewDocumentCommand \integ {s O{} O{} m o}
{
	\IfBooleanTF{#1}{\oint}{\int}_{#2}^{#3} #4 %
	\IfNoValueF {#5} {\mathrm{d} #5}
}

%% Derivatives

% Normal derivative
% Example: \der[n]{f}{x}[x_0]
\NewDocumentCommand \der {O{} m m o}
{
	\frac{\mathrm{d}^{#1} #2}{\mathrm{d} {#3}^{#1}}%
	\IfNoValueF{#4} {\biggr|_{#4}}
}

%% Partial Derivatives

% With respect to one variable
% Example: \pder[n]{f}{y}[\pthvars[x_0][y_0][z_0]][(x, z)
\NewDocumentCommand \pder {O{} m m O{} o}
{
    \ensuremath{
	\IfNoValueTF {#5}
	{
		\frac{\partial^{#1} #2}{\partial {#3}^{#1}} #4
	}
	{
		\left(%
		\frac{\partial^{#1} #2}{\partial {#3}^{#1}}%
		\right)_{#5}  #4
	}
	}
}

% With respecto to two variables
% Example: \twpder{g}{y}{x}[(x_0, y_0)]
\NewDocumentCommand \twpder {m m m O{}}
{
	\frac{\partial^2 #1}{\partial #2 \partial #3} #4
}

\newcommand{\abs}[1]{\left\lvert #1 \right\rvert}
\newcommand{\norm}[1]{\left\lVert #1 \right\rVert}

% Physics symbols (vectors, units)
\usepackage{tikz}
% Version of December 26 2016

%%%%%%%%%%%%%%%%%%%%%%%%%%%%%%%%%%%%%%%%%%% Vectors

%%%%%%%% For SI units %%%%%%%
%\usepackage{siunitx}

%%%%%%%% Vectors %%%%%%%%%%%% (Just see last lines)
%\usepackage{tikz}         % For arrow and dots in \xvec

% --- Macro \xvec
\makeatletter
\newlength\xvec@height%
\newlength\xvec@depth%
\newlength\xvec@width%
\newcommand{\xvec}[2][]{%
  \ifmmode%
    \settoheight{\xvec@height}{$#2$}%
    \settodepth{\xvec@depth}{$#2$}%
    \settowidth{\xvec@width}{$#2$}%
  \else%
    \settoheight{\xvec@height}{#2}%
    \settodepth{\xvec@depth}{#2}%
    \settowidth{\xvec@width}{#2}%
  \fi%
  \def\xvec@arg{#1}%
  \def\xvec@dd{:}%
  \def\xvec@d{.}%
  \raisebox{.2ex}{\raisebox{\xvec@height}{\rlap{%
    \kern.05em%  (Because left edge of drawing is at .05em)
    \begin{tikzpicture}[scale=1]
    \pgfsetroundcap
    \draw (.05em,0)--(\xvec@width-.05em,0);
    \draw (\xvec@width-.05em,0)--(\xvec@width-.15em, .075em);
    \draw (\xvec@width-.05em,0)--(\xvec@width-.15em,-.075em);
    \ifx\xvec@arg\xvec@d%
      \fill(\xvec@width*.45,.5ex) circle (.5pt);%
    \else\ifx\xvec@arg\xvec@dd%
      \fill(\xvec@width*.30,.5ex) circle (.5pt);%
      \fill(\xvec@width*.65,.5ex) circle (.5pt);%
    \fi\fi%
    \end{tikzpicture}%
  }}}%
  #2%
}
\makeatother

% --- Override \vec with an invocation of \xvec.
\let\stdvec\vec
\renewcommand{\vec}[1]{\xvec[]{#1}}                             % Vector
% --- Define \dvec and \ddvec for dotted and double-dotted vectors.
\newcommand{\tvec}[1]{\xvec[.]{#1}}                             % Vector derived wrt time
\newcommand{\ttvec}[1]{\xvec[:]{#1}}                            % Vector derived twice wrt time


% Theorem environments
%Version of October 8, 2016

%\usepackage{amsthm}

\theoremstyle{definition} %To avoid the annoying italics all the time, and to not sloppily redefine all of them 

\newtheorem{theo}{Theorem}[section]  %numbered according to section environment, so in section to it restarts as 2.1 
\newtheorem{prop}{Proposition}[section]  %numbered according to section environment, so in section to it restarts as 2.1 
\newtheorem{lemma}[theo]{Lemma}     %numbering shared with theorem 
\newtheorem{defn}{Definition}[section]   
\newtheorem{coro}{Corollary}[theo]


\theoremstyle{remark} 
\newtheorem*{remark}{Remark} 
 
%\let\oldtheo\theo 
%\renewcommand{\theo}{\oldtheo\normalfont}  
%  
%\let\olddefn\defn  
%\renewcommand{\defn}{\olddefn\normalfont}  
%  
%\let\oldlemma\lemma  
%\renewcommand{\lemma}{\oldlemma\normalfont}  
%  
%\let\oldcoro\coro  
%\renewcommand{\coro}{\oldcoro\normalfont}

%%%%%%%% ``Example'' environment, very basic, doesnt work with itemize
\theoremstyle{definition}

\newtheorem*{exmp}{Example}


%%%%%%%% Comandos adicionales
\newcommand{\cut}[1]{\overline{#1}}

%%%%%%
\title{Fiore Pisacane 2018: Fuzzy circle and new fuzzy sphere through confining potentials and energy cutoffs}
\author{Sebastian Camilo Puerto}
\date{July 2020}

%%%%%%%%%%%%%%%%%%%%%%%%%%%%%%%%%%%%%%%%%%%%%%%%%%%%%%%%%%%%%%%%%%%%%%%%%%%%%
%%%%%%%%%%%%%%%%%%%%%%%%%%%%%%%%%%%%%%%%%%%%%%%%%%%%%%%%%%%%%%%%%%%%%%%%%%%%%
%%%%%%%%%%%%%%%%%%%%%%%%%%%%%%%%%%%%%%%%%%%%%%%%%%%%%%%%%%%%%%%%%%%%%%%%%%%%%
%%%%%%%%%%%%%%%%%%%%%%%%%%%%%%%%%%%%%%%%%%%%%%%%%%%%%%%%%%%%%%%%%%%%%%%%%%%%%
\begin{document}

\maketitle

\tableofcontents

The covariance here has a different meaning to the one for the Madore Fuzzy sphere, since or at least a bigger meaning, since we are thinking of the algebras NOT only as noncommutative version of the space, but also as approximation of the algebra of observables of a quantum system, which is the NC $\mathcal B(L^2(S^d))$.
    \begin{itemize}
        
    \item As a NC approximation of a space, the equivariance under a group means that: 
    
        \begin{enumerate}
        
        \item The group acts by ``diffeomorphisms'', i.e. by $*$-algebra morphisms % C^*? normed & *-algebra?
        that respect the derivations.
        
        \item The group acts by ``isometries'': the spectral triples are equivariant, i.e. the group action on the Hilbert space commutes with the actions of both the Dirac operator and the Algebra of the spectral triples.
        
        \item The ``NC approximations of the points'' of $S^2$ are equivariant: this interpretation appears since there are states of the approximating algebras that can be thought of as approximations of the points of the commutative space, and it turns out that these identifications are also $SU(2)$-equivariant.
        
            \begin{itemize}
                
            \item Recall: in NC QM coherent states are fuzzy approximations of points since they have associated weak positivion measuremetns (positive-valued measures).
                
            \end{itemize}
        
        \end{enumerate}
    
    \item However, as ``approximations of QM is $S^d$'', i.e. as algebras of operators acting on wavefunctions $\in \mathcal B(L^2(S^d))$, they are equivariant in the sense of \rtext{symmetries of a dynamical system}: \dbtext{a) the group action and the Hamiltonian / time evolution operator commute AND(?) b) the measurement of the observables isn't affected by the group action}.
    %the group acts on the new observables, i.e. the algebras $\mathcal A_\Lambda$ as operators on $\mathcal B(\mathcal H_\Lambda)$, and the action of the group on $\mathcal H_\Lambda$ commutes with the action of the observables (specially $L^2$ and $L_z$). 
    \tiny{This , I THINK, is analogous to saying that the Lagrangian is invariant under $G$, which then implies that there are $|G|$ conserved quantities (e.g. If under rotations, then angular momentums; if spatial translation, then parity; if spacetime translations, then energy-momentum)}
    
        \begin{itemize}
            
        %\item Recall that in the Hamiltonian formalism of classical mechanics, a symmetry of the system is a group action that leaves invariant the observables $C^\infty (\mathcal P) = \mathcal A$ {\tiny A possible }. Also recall that an observable is a constant of motion if its bracket with the Hamiltonian $H$ is 0. {\tiny In the lagrangian formalism a symmetry of the system is a group action that doesn't change the lagrangian}.
        
        %\item Lagrangian, Noether theorem gives observable from group action that leaves invariant the Lagrangian.
        
        \item When the observables are operators, a group is a symmetry of the system if the group action and the observables commute: once the group action $A \mapsto A^g = g \circ A \circ g^{-1}$ on the observables is induced, this means that $A^g$ is ``the same'' as $A$, but on the ``rotated'' space $g\mathcal H$: \lbtext{the algebra of observables is $G$-equivariant/covariant} CREO.
            
        \end{itemize}
    
    \end{itemize}

%%%%%%%%%%%%%%%%%%%%%%%%%%%%%%%%%%%%%%%%%%%%%%%%%%%%%%%%%%%%%%%%%%%%%%%%%%%%%
\subsection{High Level Summary}

    \begin{itemize}

    \item 
    
    \end{itemize}

%%%%%%%%%%%%%%%%%%%%%%%%%%%%%%%%%%%%%%%%%%%%%%%%%%%%%%%%%%%%%%%%%%%%%%%%%%%%%
\subsection{Very Important Facts}

    \begin{itemize}

    \item  We are thinking of the algebras NOT really as noncommutative version of the space, of $L^2(S^2)$, but also as approximation of the algebra of observables of a quantum system, which is the NC $\mathcal B(L^2(S^d))$.
    
        \begin{itemize}
            
        \item $\mathcal A_\Lambda \to \mathcal B(L^2(S^d))$
        
        \item $\mathcal C_\Lambda \subset \mathcal A_N \to \text{ and } \hookrightarrow C(S^d)$ space of polynomials on the coordinates $x^i$.
        
        \item Also, $\mathcal H_\Lambda \to \to \text{ and } \hookrightarrow L^2(S^d)$
            
        \end{itemize}
    
    \item The fuzzy sphere gives us a sequence of algebras which can all be though of as approximations of $\mathcal B(L^2(S^2))$. Each of these approximations can be seen to come from: if only energies below a certain cutoff value $\cut{E}_\Lambda$ are accessible for the wavefunctions, and for the accessible energies the potential is nearly harmonic near $r = 1$ and ``sufficiently'' steep THEN the accessible state space $\mathcal H_{\cut{E}}$ can be studied with the projected observables: $A \mapsto \cut{A} := P_{\overline E} A P_{\overline E}$; this, in particular, means that new commutation relations appear for the coordinate functions $\hat x^i$.
    
    \item The new dynamical system has $O(d)$ as a symmetry: at least it is ``true'' that the Hamiltonian is $O(d)$-covariant (which I'm defining to mean that the group action and $H$ commute) and that acting with this group on a vector state $\psi \in \mathcal H_\Lambda$ doesn't  take it outside $\mathcal H_\lambda$ nor changes its eigenvalues under $L^2$ and $L_z$ (wait... are there actual eigenvalues of these observables or just approximate ones?)
    
    \item The coordinates generate the whole algebra $\mathcal A_{\cut{E}}$.
    
    \item $\mathcal A_{\cut{E}}$ can be realized as the algebra of operators of an irrep. $\pi_{\cut{E}}$ of $so(d+2)$. This means that $\mathcal H_{\cut{E}}$ can be though of as an irrep. of $so(d+2)$, which, in particular, means that $\mathcal H_{\cut{E}}$ is a reducible representation of $so(d+1)$, namely, $\mathcal H_{\cut{E}} \cong \bigoplus_{E \leq \cut{E}} V_l$ as representation space of $so(d+1)$.
    
    \item Embeddings and limits: a subalgebra $\mathcal C_\Lambda$ of $\mathcal A_{\Lambda}$ which has the $so(d+1)$-module decomposition $\bigoplus_{E \leq \cut{E}} V_l$ does approximate the commutative space.
    
    \item Basis of $\mathcal A_{\cut{E}}$
    
    \item $L_{ij}$ and $x^i$ tend to the angular momentum and coordinate functions of $S^d$. (Elements of $\mathcal C_\Lambda$?)
    
    \end{itemize}

%%%%%%%%%%%%%%%%%%%%%%%%%%%%%%%%%%%%%%%%%%%%%%%%%%%%%%%%%%%%%%%%%%%%%%%%%%%%%
\subsection{Important Facts}

    \begin{itemize}

    \item 
    
    \end{itemize}

%%%%%%%%%%%%%%%%%%%%%%%%%%%%%%%%%%%%%%%%%%%%%%%%%%%%%%%%%%%%%%%%%%%%%%%%%%%%%
\subsection{Memorize}

    \begin{itemize}

    \item 
    
    \end{itemize}

%%%%%%%%%%%%%%%%%%%%%%%%%%%%%%%%%%%%%%%%%%%%%%%%%%%%%%%%%%%%%%%%%%%%%%%%%%%%%
\subsection{Doubts}

    \begin{itemize}

    \item 
    
    \end{itemize}

%%%%%%%%%%%%%%%%%%%%%%%%%%%%%%%%%%%%%%%%%%%%%%%%%%%%%%%%%%%%%%%%%%%%%%%%%%%%%
\subsection{Detailed summary}

    \begin{itemize}

    \item 
    
    \end{itemize}

%%%%%%%%%%%%%%%%%%%%%%%%%%%%%%%%%%%%%%%%%%%%%%%%%%%%%%%%%%%%%%%%%%%%%%%%%%%%%
\subsection{Notice}

    \begin{itemize}

    \item 
    
    \end{itemize}

%%%%%%%%%%%%%%%%%%%%%%%%%%%%%%%%%%%%%%%%%%%%%%%%%%%%%%%%%%%%%%%%%%%%%%%%%%%%% 
\subsection{Yet to understand}

    \begin{itemize}

    \item In what sense is these "fuzzy approximation of quantum mechanics" $O(d)$-covariant?
    
        \begin{itemize}
            
        \item Do the algebra generators ($x^\pm$) commute with the action of $O(2)$? If so, these would imply that the complete algebra of observables commutes with $O(2)$. RTA: NO, for example I know (29.5) that the $x$'s do change under the action of a rotation.
        
        \item \rtext{My choice:} It is true that the Hamiltonian and the group action commute, which implies that
        
            \begin{itemize}
                
            \item The action of $G$ makes things fall again in $\mathcal H_{\cut E}$, since it even makes an eigenvalue of $H$ fall again into an eigenvalue of $H$ with the same energy.
$\Rightarrow$ $O(2)$ acts on $\mathcal H_{\cut E}$ $\Rightarrow$ $O(2)$ acts on $\mathcal A_{\cut E}$ by inner automorphisms, i.e. \rtext{$O(2)$ acts by diffeomorphisms on the NC space}.         
            \item The time evolution of a ``rotated'' vector is simply given at any time as the rotation of the evolution of the orignal vector. Alternatively, $H^g = H$, so the time evolution of the a rotated vector is given by the same Hamiltonian.
            
            \item The two above things imply that the evolution of a vector in $\mathcal H_{\cut E}$ is given by the same $H$, and that \rtext{this evolution is invariant under $O(2)$}
            
                
            \end{itemize}
            
        \item Why make so much emphasis on ``the commutation relations (satisfied by the algebra generators) are $G$-invariant?'' And what does it even mean? {\tiny It can't be simply that $[A, B]^g = [A^g, B^g]$ since that is trivial to see given that $g$ acts by inner isomorphisms, and so it wouldn't be necessary to say that this is true since the commutation relations and $H$ (and hence) $P_{\cut E}$ are so... \rtext{but I think that's what makes sense since it would mean that the generated algebra is ``the same''}. It can't mean that $[A, B]^g = [A, B]$, since not even for the generators $\cut{x^\pm}$ that is true, since under reflections $\cut L$ does change.
        }  
        
        \item Fiore and Pisacane try to give an explanation about what this covariance means in one of their last papers, perhaps even the last one, but \dbtext{I haven't read it}.
        
        \end{itemize}
    
    \item Why are the algebras generated only by $x^\pm$ in the $2$D case, at least, meaning that we can ignore $\partial_\pm$? {\tiny Supposedly by $(23)$, or something of the sort is said right before equation $(16)$}
    
    
    \end{itemize}

%%%%%%%%%%%%%%%%%%%%%%%%%%%%%%%%%%%%%%%%%%%%%%%%%%%%%%%%%%%%%%%%%%%%%%%%%%%%%
%%%%%%%%%%%%%%%%%%%%%%%%%%%%%%%%%%%%%%%%%%%%%%%%%%%%%%%%%%%%%%%%%%%%%%%%%%%%%
\section{Introduction (pg. 1)}

%%%%%%%%%%%%%%%%%%%%%%%%%%%%%%%%%%%%%%%%%%%%%%%%%%%%%%%%%%%%%%%%%%%%%%%%%%%%%
\subsection*{High Level Summary}

    \begin{itemize}

    \item 
    
    \end{itemize}

%%%%%%%%%%%%%%%%%%%%%%%%%%%%%%%%%%%%%%%%%%%%%%%%%%%%%%%%%%%%%%%%%%%%%%%%%%%%%
\subsection*{Very Important Facts}

    \begin{itemize}

    \item Why does it make sense to have an energy cutoff? At least $2$ reasons:
    
        \begin{enumerate}
            
        \item We might add $\cut E$ as a point where higher energy physics is unkwnown.
        
        \item Where neither we nor the environment can bring a state to higher energies. This gives an effective description of the system. It \dbtext{leas to a lower (distance? ~)} bound in the accuracy with which our apparatus can measure observables, \dbtext{coming from a maximum transferable energy}.
            
        \end{enumerate}
    
    \end{itemize}

%%%%%%%%%%%%%%%%%%%%%%%%%%%%%%%%%%%%%%%%%%%%%%%%%%%%%%%%%%%%%%%%%%%%%%%%%%%%%
\subsection*{Important Facts}

    \begin{itemize}

    \item 
    
    \end{itemize}

%%%%%%%%%%%%%%%%%%%%%%%%%%%%%%%%%%%%%%%%%%%%%%%%%%%%%%%%%%%%%%%%%%%%%%%%%%%%%
\subsection*{Memorize}

    \begin{itemize}

    \item 
    
    \end{itemize}

%%%%%%%%%%%%%%%%%%%%%%%%%%%%%%%%%%%%%%%%%%%%%%%%%%%%%%%%%%%%%%%%%%%%%%%%%%%%%
\subsection*{Directly Relevant Doubts}

    \begin{itemize}

    \item\dbtext{ Do we need high energies to measure high energies / small distances?}
    
    \item The odd commutation relations of the coordinates appear for energies \textbf{below} the energy cutoff... but isn't that the opposite of what we want (i.e. that below some energy space works as we understand it, but "above" it the odd noncommutativity appear)?... Perhaps understand this as how NC is natural, that it is actually there since the beginning?
    
    \item Is this example actually a fuzzy space? Can we formally say that our algebras approximate a commutative space? (I need to know the precise meaning of a fuzzy space)
    
    \end{itemize}
    
%%%%%%%%%%%%%%%%%%%%%%%%%%%%%%%%%%%%%%%%%%%%%%%%%%%%%%%%%%%%%%%%%%%%%%%%%%%%%
\subsection*{Small Doubts}

    \begin{itemize}

    \item 
    
    \end{itemize}

%%%%%%%%%%%%%%%%%%%%%%%%%%%%%%%%%%%%%%%%%%%%%%%%%%%%%%%%%%%%%%%%%%%%%%%%%%%%%
\subsection*{Careful}

    \begin{itemize}

    \item 
    
    \end{itemize}


%%%%%%%%%%%%%%%%%%%%%%%%%%%%%%%%%%%%%%%%%%%%%%%%%%%%%%%%%%%%%%%%%%%%%%%%%%%%%
\subsection*{Detailed summary}

    \begin{itemize}

    \item 
    
    \end{itemize}

%%%%%%%%%%%%%%%%%%%%%%%%%%%%%%%%%%%%%%%%%%%%%%%%%%%%%%%%%%%%%%%%%%%%%%%%%%%%%
%%%%%%%%%%%%%%%%%%%%%%%%%%%%%%%%%%%%%%%%%%%%%%%%%%%%%%%%%%%%%%%%%%%%%%%%%%%%%
\section{General setting (pg. 5)}

%%%%%%%%%%%%%%%%%%%%%%%%%%%%%%%%%%%%%%%%%%%%%%%%%%%%%%%%%%%%%%%%%%%%%%%%%%%%%
\subsection*{High Level Summary}

    \begin{itemize}

    \item $H = - \frac{1}{2} \Delta + V(r)$ invariant under $O(D)$
    
    \item Introducing the cutoff, as the energy under which the potential has a harmonic behavior in the region $\nu_{\cut E} = \{r \st V(r) \leq \cut E\}$. This means that we ``have'' to change/update the observables, and in particular means we get a new Schrodinger equation. This new observables will have new commutation relations.
    
    \item Alternative complete set of commutating operators $B = \{\partial_r, L_{ij}\}$. Globally defined outside $r = 0$, and redundant for $D \geq 3$.
    
    \item Eigenfunctions of $H$ as product of a spherical harmonic (eigenfunction of $L^2$) and an eigenfunction of radial equation $9$. This last equation, under the assumption of a nice potential, can be approximated by a harmonic oscillator equation, since outside the region $\nu_{\cut E}$ $\psi$ is negligibly small.
    
    \end{itemize}

%%%%%%%%%%%%%%%%%%%%%%%%%%%%%%%%%%%%%%%%%%%%%%%%%%%%%%%%%%%%%%%%%%%%%%%%%%%%%
\subsection*{Very Important Facts}

    \begin{itemize}

    \item $[P_{\cut E}, H] = 0$. This is equivalent to saying that a state in $\mathcal H_{\cut E}$ doesn't evolve out of it.
    
    \item $[A, B] = 0$ means, between other things, that, if $v$ is an eigenvector of $A$ for the eigenvalue $\lambda$, then so is $B(v)$. Examples: $A, B = H, L_{ij}, P_{\cut E}, Q ``\in" O(D)$.
    
    \item $\cut H = H$
    
    \item We replace any Schrodinger equation $i \partial_t \psi = (H + H')\psi$ with a finite dimensional one within $\mathcal H_{\cut E}$, $i \partial_t \psi = \cut{H + H'}$... but I don't understand well what this means
    
    \item At leading order in \dbtext{what} the lowest eigenvalues of $H$ be considered those of the SHO approximation of equation $9$:
    \begin{align*}
        \left[-\partial_r^2 - (D-1) \frac{1}{r} \partial_r + \frac{1}{r^2} j(j+D-2) + V(r)\right] \tilde f(r) = E \tilde f(r)
    \end{align*} multiplied by the $Y$ (eigenvectors of $L^2$ and a Cartan subalgebra elements of $\mathfrak{so}(D)$).
    
    \item Eigenvalues of $L^2$ are $\{j(j+D-2)\}$ for $j \in \ZZ$.
    
    \item The energy of a quantum system can't increase. The energy of an energy eigenvector can't change.
    
    \item Standard quantum mechanics on the sphere $S^d$ are obtained by making $k$ and $\cut E$ diverge while having the region $\nu_{\cut E} = \{r \st V(r) \leq \cut E\}$ go to $\{r = 1\}$.
    
    \end{itemize}

%%%%%%%%%%%%%%%%%%%%%%%%%%%%%%%%%%%%%%%%%%%%%%%%%%%%%%%%%%%%%%%%%%%%%%%%%%%%%
\subsection*{Important Facts}

    \begin{itemize}

    \item There are two conditions on the potential. They are what allow us to approximate the time independent Schrodinger equation $9$ as a $1$-dimensional harmonic oscillator (2D: equation $10$, 3D: equation $36$)
    
        \begin{enumerate}
        
        \item For $r \to 0$, $V(r)$ stays bounded or grows at most as $\beta/r^2$ ($+c/r + d + er +\cdots$) ($\beta \geq 0$). {\tiny This allows equation $9$ to be solved}. Notice that the first condition, if valid in $r = 0$, means that this second condition is also satisfied.
        
        \item Equation $5$: \begin{align*}
            V(r) \approx V_0 + 2k(r-1)^2 && \text{for $r$ such that $V(r) \leq  \cut E$}.
        \end{align*} This condition, if the first one is already satisfied, means that \rtext{$\psi \longleftarrow \tilde f$ are neglibly small outside the shell $V(r) \leq \cut E$}, meaning that \rtext{at leading order \dbtext{in what?} the lowest eigenvalues $E$ are those of the $1$-dimensional harmonic oscillator approximation of $9$.} 
        
        \end{enumerate}
    
    \item $[H, L_{ij}] = 0$... if this is important, it may have to do with having a complete commuting set of observables for our systems, allowing a description of the vectors as lineara combination s of eigenvalues of these observables.
    
    \item $[P_{\cut E}, L_{ij}] = 0$ and so $[P_{\cut E}, L^2] = 0$
    
    \item $\Delta = \partial_r^2 + (D-1)\frac{1}{r}\partial_r - \frac{1}{r^2}L^2$, where $L^2 = L_{ij} L_{ij}/2$
    
    \item For ``\lbtext{covariance}'': $G$ is a symmetry of the theory:
        
        \begin{itemize}
            
        \item $[g\cdot , H] = 0$ implies \otext{$[g\cdot , P_{\cut E}] = 0$}
        
        \item That, in turn, implies that \otext{$\cut{A}^g = \cut{A^g}$}
            
        \end{itemize}
    
    \end{itemize}

%%%%%%%%%%%%%%%%%%%%%%%%%%%%%%%%%%%%%%%%%%%%%%%%%%%%%%%%%%%%%%%%%%%%%%%%%%%%%
\subsection*{Memorize}

    \begin{itemize}

    \item 
    
    \end{itemize}

%%%%%%%%%%%%%%%%%%%%%%%%%%%%%%%%%%%%%%%%%%%%%%%%%%%%%%%%%%%%%%%%%%%%%%%%%%%%%
\subsection*{Directly Relevant Doubts}

    \begin{itemize}

    \item $\cut A \in ??$
    
    \item Are the eigenvectors of the harmonic oscillator approximation of $9$ also eigenvectors of $L^2$? At least approximately? RTA: Yes, in an exact way; that is how equation $9$ is derived from equation $8$: by sayint that $\psi = \tilde f(r) Y(\phi, \dots)$
    
    \item What governs the time evolution of a state in $\mathcal H_{\cut E}$? Do we computationally need to change the Schrodinger equation, or is that replacement only a mathematical one (making explicit where the operators live)? I'm not sure if when the projection is done we are somehow systematically getting rid of some type of interaction terms; or if we are simply being mathematically rigorous and indicating that the new operators appearing in the Schrodinger equation are operators acting on $\mathcal H_{\cut E}$?
    
    \item Why is that complete set of commuting operators $B$ introduced?
    
        \begin{itemize}
            
        \item Is the redundancy of the $L_{ij}$ for higher dimensions relevant to us?
            
        \end{itemize}
    
    \item At leading order in what can the lowest eigenvalues of $H$ be considered rhose of the SHO approximation of equation $9$?
    
    \end{itemize}

%%%%%%%%%%%%%%%%%%%%%%%%%%%%%%%%%%%%%%%%%%%%%%%%%%%%%%%%%%%%%%%%%%%%%%%%%%%%%
\subsection*{Small Doubts}

    \begin{itemize}

    \item $[H, L_{ij}] = 0$
    
    \item Are the tangent vectors of $\RR^D$/ $S^d$ \textit{bounded} operators acting on $L^2(\RR^D)$ / $L^2(S^d)$?
    
    \item Are $x^i$, $\partial_i$, $H$ bounded operators on $L^2(\RR^D)$ / $L^2(S^d)$?
    
    \item The complete set of operators $B$ is introduced as an alternative to which one? $\{H, L^2, L_z\}$ in 3D?
    
    \item Eigenvectors of $L^2$, denoted by $Y$, are also eigenvectors of the elements of a Cartan subalgebra of $so(D)$.
    
    \end{itemize}

%%%%%%%%%%%%%%%%%%%%%%%%%%%%%%%%%%%%%%%%%%%%%%%%%%%%%%%%%%%%%%%%%%%%%%%%%%%%%
\subsection*{Careful}

    \begin{itemize}

    \item 
    
    \end{itemize}

%%%%%%%%%%%%%%%%%%%%%%%%%%%%%%%%%%%%%%%%%%%%%%%%%%%%%%%%%%%%%%%%%%%%%%%%%%%%%
\subsection*{Detailed summary}

    \begin{itemize}

    \item 
    
    \end{itemize}

%%%%%%%%%%%%%%%%%%%%%%%%%%%%%%%%%%%%%%%%%%%%%%%%%%%%%%%%%%%%%%%%%%%%%%%%%%%%%
%%%%%%%%%%%%%%%%%%%%%%%%%%%%%%%%%%%%%%%%%%%%%%%%%%%%%%%%%%%%%%%%%%%%%%%%%%%%%
\section{$D=2$: $O(2)$-equivariant fuzzy circle (pg. 5.75)}

%%%%%%%%%%%%%%%%%%%%%%%%%%%%%%%%%%%%%%%%%%%%%%%%%%%%%%%%%%%%%%%%%%%%%%%%%%%%%
\subsection*{High Level Summary (Small)}

    \begin{itemize}

    \item The $O(2)$-covariant fuzzy circle: the Hilbert space $\mathcal H_{\cut E}$ of spanned by the approximate solutions to the Schrodinger equation $\longrightarrow$ the algebras $\mathcal A_\Lambda$ of the fuzzy sphere as $\mathcal B(\mathcal H_{\cut E})$ $\longrightarrow$ important observables, their commutation relations (inluding the coordinate functions) and their corrections generators of the algebras.
    
        \begin{itemize}
            
        \item The Hilbert space $\mathcal H = L^2(\RR^2)$ and its algebra of observables are $\ZZ$-graded by the respective actions of $L = -i\partial_\phi$.
        
        \item (Approximate) solutions of equation $9$:
        
            \begin{itemize}
            
            \item $SHO$ approximation of equation $9$
            
            \item Finding $e_{m, n}(k)$
            
            \item Finding $E'_{m,n}(k, V_0)$ from $e_{m,n}(k)$
            
            \item (Gauge) fixing $V_0 = V_0(k)$ so that $E_{0, 0}(V_0) = 0$ (creo que esa es la dependencia de $E_{n,m}$)
            
            \item $E_{n,m}(k) = m^2 + 2n \sqrt{2k} - 2n + O(\frac{1}{\sqrt{k}})$ and so $\cut E < 2 \sqrt{2k} - 2$
            
            \item Resulting (approximate) solutions to the (harmonic approximation) of Schrodinger's equation $\psi_m$ each with energy $E_m = m^2 + O(\frac{1}{\sqrt{k}})$ which are exact eigenvectors of $L$ for the eigenvalue $m$.
            
            \end{itemize}
        
        \item Action and commutation relations of important observables: $\cut L, \cut x^+, \cut x^-, \cut H, \mathcal R$: power expansion on $k$. In particular (ignoring powers of $1/k^2$ and higher):
        \begin{align*}
            [\xi^+, \xi^-] &= - \frac{\cut L}{k} + \left[ 1 + \frac{\Lambda(\Lambda + 1)}{k} \right] \frac{\tilde P_\Lambda - \tilde P_{-\Lambda}}{2} &
            [\cut L, \xi^\pm] &= \pm \xi^\pm
        \end{align*}
        
        \item Corrected observables (\dbtext{which ones exactly?}) (their name is abuse of notation, except for $\cut L$) generate the $*$-algebra, defined by equation $17$:
        \begin{align}
            \cut L \psi_m &= m \psi_m; & 
            \cut H &= \cut L^2; & 
            \xi^\pm \psi_m = \frac{\cut x^\pm}{a} \psi_m = 
                \begin{cases}
                    \frac{1}{\sqrt{2}} \sqrt{ 1 + \frac{m(m \pm 1)}{k} } \psi_{m \pm 1} & \text{if } -\Lambda \leq \pm m \leq \Lambda - 1 \\
                    0 & \text{otherwise}
                \end{cases}
        \end{align}
        
        \item Summary and analysis: $O(2)$-covariance \dbtext{of the some commutation relations of the algebra generators, which somehow imply what I want?}
                
        \end{itemize}
        
    \item Isomorphic (as algebras and as representations of the group, they should be) version of the fuzzy space algebras and their Hilbert spaces.
    
    \item 
    
    \end{itemize}

%%%%%%%%%%%%%%%%%%%%%%%%%%%%%%%%%%%%%%%%%%%%%%%%%%%%%%%%%%%%%%%%%%%%%%%%%%%%%
\subsection*{Very Important Facts}

    \begin{itemize}

    \item The basis vectors of $\mathcal H_{\cut E}$ are exact eigenvectors of $L$, since equation $9$ comes from separating variables and stating that $\psi_m (r, \phi) - \tilde f(r) e^{im\phi}$.
    
    \item When finding $f(r)$ two approximations are made: one to change equation $9$ by its harmonic oscillator approximation; another approximating the Hermite polynomials to $1$. From these two approximations we get the approximate solutions of equation $9$, below the cutoff energy:
        \begin{equation}
            \lbtext{\psi_m(\rho, \phi)} := N_m e^{im \phi} \exp{\left[ -\frac{(\rho - \tilde \rho_m)^2 \sqrt{k_m}}{2} \right]}
        \end{equation}
        ($\rho = \ln r$) with eigenvalues
        \begin{align}
            L \psi_m &= m \psi_m & H \psi_m &= \lbtext{E_m} \psi_m \equiv (m^2 + O(1/\sqrt{k})) \psi_m
        \end{align}
        where $k_m$ and $\tilde \rho_m$ are functions of $k$ and $m$,
    
    \item The approximations are made taking the expansion of ``things'' w.r.t. $\rho = \ln r$ and $k$.
    
    \end{itemize}



%%%%%%%%%%%%%%%%%%%%%%%%%%%%%%%%%%%%%%%%%%%%%%%%%%%%%%%%%%%%%%%%%%%%%%%%%%%%%
\subsection*{Memorize}

    \begin{itemize}

    \item 
    
    \end{itemize}

%%%%%%%%%%%%%%%%%%%%%%%%%%%%%%%%%%%%%%%%%%%%%%%%%%%%%%%%%%%%%%%%%%%%%%%%%%%%%
\subsection*{Directly Relevant Doubts}

    \begin{itemize}

    \item How is the expansion on $k$ handled? I mean, is it actually consistent? Is perhaps a term above which we are ignoring... perhaps one in the Hilbert space, one in the operators(?)
    
    \item Is it consistent to divide by $a = a(k)$ the operators $x^\pm$? 
    
    \item Why does it make sense to define the distance operator in terms of $\xi^\pm$, meaning with a factor of $1/a^2$ with respect to the more natural one?
    
    \end{itemize}
    
%%%%%%%%%%%%%%%%%%%%%%%%%%%%%%%%%%%%%%%%%%%%%%%%%%%%%%%%%%%%%%%%%%%%%%%%%%%%%
\subsection*{Important Facts}

    \begin{itemize}

    \item $L = - i \partial_\phi = x^+ \partial_+ - x^- \partial_-$
    
    \item \textbf{Heisenberg algebra: of observables of QM in $\RR^2$}. Generated by $x^\pm$ and $\partial_\pm$
    
    \item The dependence of $E_{n, m} = E_{n, m}(k)$ comes from two sources, mainly the first one: $e_{n, m} = e_{n, m}(k)$ and $V_0 = V_0(k)$.
    
    \item We are "gauge fixing" $E_{0,0} = 0$, and that fixes $V_0$, as a function of $k$.
    
    \item The approximation w.r.t expansions w.r.t $\rho$ is only used to find the harmonic approximation of equation $9$.
    
    \item $sqrt{ 1 + \frac{m(m+1)}{k} }$ falls like $1 + \frac{1}{k}$ when $k \to \infty$.
    
    \end{itemize}

%%%%%%%%%%%%%%%%%%%%%%%%%%%%%%%%%%%%%%%%%%%%%%%%%%%%%%%%%%%%%%%%%%%%%%%%%%%%%
\subsection*{Iteresting Facts}

    \begin{itemize}

    \item $[x^\pm, \partial_\pm] = i \hbar$, I'm pretty sure.
    
    \item The algebra of observables can be graded according to their eigenvalue w.r.t. to the action $[L, \cdot] = m \cdot$ for $m \in \ZZ$
    
    \item $x^\pm = \frac{r}{\sqrt{2}} e^{\pm i \phi}$
    
    \item The generators $x^\pm ~ e^{\pm i \phi}$ and $\partial_\pm$ of the algebra of observables $\mathcal O$ are eigenvectors under the adjoint action of $L$ on $\mathcal O$ with eigenvalues $\pm 1$. Since they generate, this implies that the algebra of observables of $\RR^2$ is $\ZZ$-graded by the eigenvalues of $L$, and this grading is compatible with the similar grading of $\mathcal H$.
    
    \end{itemize}

%%%%%%%%%%%%%%%%%%%%%%%%%%%%%%%%%%%%%%%%%%%%%%%%%%%%%%%%%%%%%%%%%%%%%%%%%%%%%
\subsection*{Small Doubts}

    \begin{itemize}

    \item Heisenberg algebra in 2 dimensions generated by $x^\pm$ and $\partial_\pm$
    
    \item What changes if we leave, instead, $V_0$ as a fixed value w.r.t. $k$? Say, $V_0 = 0$? Which expansions w.r.t. $k$ still appear?
    
    \end{itemize}

%%%%%%%%%%%%%%%%%%%%%%%%%%%%%%%%%%%%%%%%%%%%%%%%%%%%%%%%%%%%%%%%%%%%%%%%%%%%%
\subsection*{Careful}

    \begin{itemize}

    \item 
    
    \end{itemize}

%%%%%%%%%%%%%%%%%%%%%%%%%%%%%%%%%%%%%%%%%%%%%%%%%%%%%%%%%%%%%%%%%%%%%%%%%%%%%
\subsection*{Details}

    \begin{itemize}

    \item When we come to equation $9$, we are already thinking of $E$ as a constant (and $m$, of course), and solving the equation will give restrictions on what can that constant be. Similarly, equation $10$ is to be understood as a linear differential equation with CONSTANT coefficients since: everything depends on $k$, of course, but we are working for a fixed $k$; hence, $V_0 = V_0(k)$ is also a constant; inherited from equation $9$, \textbf{$E$ is the constant on which we want to find restrictions}, and it is already something like a $E^m$; 
    
    \end{itemize}

%%%%%%%%%%%%%%%%%%%%%%%%%%%%%%%%%%%%%%%%%%%%%%%%%%%%%%%%%%%%%%%%%%%%%%%%%%%%%
\subsection{Realization of the Algebra of Observables throuth $Uso(3)$ (pg. 11.5)}

%%%%%%%%%%%%%%%%%%%%%%%%%%%%%%%%%%%%%%%%%%%%%%%%%%%%%%%%%%%%%%%%%%%%%%%%%%%%%
\subsection{Convergence to $O(2)$-equivariant quantum mechanics on $S^1$ (pg. 13.5)}

%%%%%%%%%%%%%%%%%%%%%%%%%%%%%%%%%%%%%%%%%%%%%%%%%%%%%%%%%%%%%%%%%%%%%%%%%%%%%
%%%%%%%%%%%%%%%%%%%%%%%%%%%%%%%%%%%%%%%%%%%%%%%%%%%%%%%%%%%%%%%%%%%%%%%%%%%%%
\section{$D=3$: $O(3)$-equivariant fuzzy sphere (pg. 15)}

%%%%%%%%%%%%%%%%%%%%%%%%%%%%%%%%%%%%%%%%%%%%%%%%%%%%%%%%%%%%%%%%%%%%%%%%%%%%%
\subsection{Realization of the Algebra of Observables throuth $Uso(4)$ (pg. 17.75)}

%%%%%%%%%%%%%%%%%%%%%%%%%%%%%%%%%%%%%%%%%%%%%%%%%%%%%%%%%%%%%%%%%%%%%%%%%%%%%
\subsection{Convergence to $O(3)$-equivariant quantum mechanics on $S^2$ (pg. 19.75)}

%%%%%%%%%%%%%%%%%%%%%%%%%%%%%%%%%%%%%%%%%%%%%%%%%%%%%%%%%%%%%%%%%%%%%%%%%%%%%
%%%%%%%%%%%%%%%%%%%%%%%%%%%%%%%%%%%%%%%%%%%%%%%%%%%%%%%%%%%%%%%%%%%%%%%%%%%%%
\section{Final remarks, Outlook and Conclusions (pg. 21.8)}

%%%%%%%%%%%%%%%%%%%%%%%%%%%%%%%%%%%%%%%%%%%%%%%%%%%%%%%%%%%%%%%%%%%%%%%%%%%%%
%%%%%%%%%%%%%%%%%%%%%%%%%%%%%%%%%%%%%%%%%%%%%%%%%%%%%%%%%%%%%%%%%%%%%%%%%%%%%
\section{Appendix (pg. 23)}

%%%%%%%%%%%%%%%%%%%%%%%%%%%%%%%%%%%%%%%%%%%%%%%%%%%%%%%%%%%%%%%%%%%%%%%%%%%%%
\subsection{Calculation of a rather general scalar product in $D=2$ (pg. 23)}

%%%%%%%%%%%%%%%%%%%%%%%%%%%%%%%%%%%%%%%%%%%%%%%%%%%%%%%%%%%%%%%%%%%%%%%%%%%%%
\subsection{Calculation of the action of operators in $D=2$ (pg. 26)}

%%%%%%%%%%%%%%%%%%%%%%%%%%%%%%%%%%%%%%%%%%%%%%%%%%%%%%%%%%%%%%%%%%%%%%%%%%%%%
\subsection{Proof of proposition [??] (pg. 28)}

%%%%%%%%%%%%%%%%%%%%%%%%%%%%%%%%%%%%%%%%%%%%%%%%%%%%%%%%%%%%%%%%%%%%%%%%%%%%%
\subsection{Spherical Harmonics (pg. 29)}

%%%%%%%%%%%%%%%%%%%%%%%%%%%%%%%%%%%%%%%%%%%%%%%%%%%%%%%%%%%%%%%%%%%%%%%%%%%%%
\subsection{Calculation of $|N_l|$ in $D = 3$ (pg. 30)}

%%%%%%%%%%%%%%%%%%%%%%%%%%%%%%%%%%%%%%%%%%%%%%%%%%%%%%%%%%%%%%%%%%%%%%%%%%%%%
\subsection{Calculation of a rather general scalar product in $D = 3$ (pg. 30)}

%%%%%%%%%%%%%%%%%%%%%%%%%%%%%%%%%%%%%%%%%%%%%%%%%%%%%%%%%%%%%%%%%%%%%%%%%%%%%
\subsection{Proof or (40) and or proposition 4.1 (pg. 32)}

%%%%%%%%%%%%%%%%%%%%%%%%%%%%%%%%%%%%%%%%%%%%%%%%%%%%%%%%%%%%%%%%%%%%%%%%%%%%%
\subsection{Action of commutators of the $\overline{\partial_a}$ (pg. 34)}

%%%%%%%%%%%%%%%%%%%%%%%%%%%%%%%%%%%%%%%%%%%%%%%%%%%%%%%%%%%%%%%%%%%%%%%%%%%%%
\subsection{Proof of proposition 4.2 and other results of subsection 4.1 (pg. 35)}

%%%%%%%%%%%%%%%%%%%%%%%%%%%%%%%%%%%%%%%%%%%%%%%%%%%%%%%%%%%%%%%%%%%%%%%%%%%%%
\subsection{Shifting the lower extreme of integration over $r$ (pg. 37)}

%%%%%%%%%%%%%%%%%%%%%%%%%%%%%%%%%%%%%%%%%%%%%%%%%%%%%%%%%%%%%%%%%%%%%%%%%%%%%
\subsection{Proof of proposition 4.3 and other results of section 4.2 (pg. 38-41)}

\end{document}
