\documentclass{article}
\usepackage[utf8]{inputenc}
\usepackage[margin=1in]{geometry}

%%%%%% To use hyperlinks, including the formula ones
\usepackage{hyperref}
\hypersetup{
    colorlinks=true,
    linkcolor=blue,
    filecolor=magenta,      
    urlcolor=cyan,
}

%%%%%% Make paragraphs start with no indentation and leave spaces between paragraphs
\setlength{\parindent}{0em}
\setlength{\parskip}{1em}

%%%%%% Math stuff
\usepackage{amsmath, amssymb}
\usepackage{amsthm}

%%%%%% Mis Codigos

% TODO notes package
\usepackage{xargs}                  % Use more than one optional parameter in a new command
\usepackage[pdftex,dvipsnames]{xcolor}
\input{tools/todoCode}

% Colored text and boxes with my color conventions for highlighting
\usepackage[dvipsnames]{xcolor}
\input{tools/colorCode}

% Math symbols
\usepackage{xparse}
\input{tools/common_math_symbols}

% Physics symbols (vectors, units)
\usepackage{tikz}
\input{tools/physics_macros}

% Theorem environments
\input{tools/theorem_definitions}


%%%%%%%% Comandos adicionales
\newcommand{\cut}[1]{\overline{#1}}

%%%%%%
\title{Fiore Pisacane 2018: Fuzzy circle and new fuzzy sphere through confining potentials and energy cutoffs}
\author{Sebastian Camilo Puerto}
\date{July 2020}

%%%%%%%%%%%%%%%%%%%%%%%%%%%%%%%%%%%%%%%%%%%%%%%%%%%%%%%%%%%%%%%%%%%%%%%%%%%%%
%%%%%%%%%%%%%%%%%%%%%%%%%%%%%%%%%%%%%%%%%%%%%%%%%%%%%%%%%%%%%%%%%%%%%%%%%%%%%
%%%%%%%%%%%%%%%%%%%%%%%%%%%%%%%%%%%%%%%%%%%%%%%%%%%%%%%%%%%%%%%%%%%%%%%%%%%%%
%%%%%%%%%%%%%%%%%%%%%%%%%%%%%%%%%%%%%%%%%%%%%%%%%%%%%%%%%%%%%%%%%%%%%%%%%%%%%
\begin{document}

\maketitle

\tableofcontents

The covariance here has a different meaning to the one for the Madore Fuzzy sphere, since or at least a bigger meaning, since we are thinking of the algebras NOT only as noncommutative version of the space, but also as approximation of the algebra of observables of a quantum system, which is the NC $\mathcal B(L^2(S^d))$.
    \begin{itemize}
        
    \item As a NC approximation of a space, the equivariance under a group means that: 
    
        \begin{enumerate}
        
        \item The group acts by ``diffeomorphisms'', i.e. by $*$-algebra morphisms % C^*? normed & *-algebra?
        that respect the derivations.
        
        \item The group acts by ``isometries'': the spectral triples are equivariant, i.e. the group action on the Hilbert space commutes with the actions of both the Dirac operator and the Algebra of the spectral triples.
        
        \item The ``NC approximations of the points'' of $S^2$ are equivariant: this interpretation appears since there are states of the approximating algebras that can be thought of as approximations of the points of the commutative space, and it turns out that these identifications are also $SU(2)$-equivariant.
        
        \end{enumerate}
    
    \item However, as ``approximations of QM is $S^d$'', i.e. as algebras of operators acting on wavefunctions $\in \mathcal B(L^2(S^d))$, they are equivariant in the sense of symmetries of a dynamical system: \rtext{the group action and the Hamiltonian / time evolution operator commute}.
    %the group acts on the new observables, i.e. the algebras $\mathcal A_\Lambda$ as operators on $\mathcal B(\mathcal H_\Lambda)$, and the action of the group on $\mathcal H_\Lambda$ commutes with the action of the observables (specially $L^2$ and $L_z$). 
    
        \begin{itemize}
            
        %\item Recall that in the Hamiltonian formalism of classical mechanics, a symmetry of the system is a group action that leaves invariant the observables $C^\infty (\mathcal P) = \mathcal A$ {\tiny A possible }. Also recall that an observable is a constant of motion if its bracket with the Hamiltonian $H$ is 0. {\tiny In the lagrangian formalism a symmetry of the system is a group action that doesn't change the lagrangian}.
        
        %\item Lagrangian, Noether theorem gives observable from group action that leaves invariant the Lagrangian.
        
        \item When the observables are operators, a group is a symmetry of the system if the group action and the observables commute: once the group action $A \mapsto A^g = g \circ A \circ g^{-1}$ on the observables is induced, this means that $A^g$ is ``the same'' as $A$, but on the ``rotated'' space $g\mathcal H$: \lbtext{the algebra of observables is $G$-equivariant/covariant} CREO.
            
        \end{itemize}
    
    \end{itemize}

%%%%%%%%%%%%%%%%%%%%%%%%%%%%%%%%%%%%%%%%%%%%%%%%%%%%%%%%%%%%%%%%%%%%%%%%%%%%%
\subsection{High Level Summary}

    \begin{itemize}

    \item 
    
    \end{itemize}

%%%%%%%%%%%%%%%%%%%%%%%%%%%%%%%%%%%%%%%%%%%%%%%%%%%%%%%%%%%%%%%%%%%%%%%%%%%%%
\subsection{Very Important Facts}

    \begin{itemize}

    \item  We are thinking of the algebras NOT really as noncommutative version of the space, of $L^2(S^2)$, but also as approximation of the algebra of observables of a quantum system, which is the NC $\mathcal B(L^2(S^d))$.
    
        \begin{itemize}
            
        \item $\mathcal A_\Lambda \to \mathcal B(L^2(S^d))$
        
        \item $\mathcal C_\Lambda \subset \mathcal A_N \to \text{ and } \hookrightarrow C(S^d)$ space of polynomials on the coordinates $x^i$.
        
        \item Also, $\mathcal H_\Lambda \to \to \text{ and } \hookrightarrow L^2(S^d)$
            
        \end{itemize}
    
    \item The fuzzy sphere gives us a sequence of algebras which can all be though of as approximations of $\mathcal B(L^2(S^2))$. Each of these approximations can be seen to come from: if only energies below a certain cutoff value $\cut{E}_\Lambda$ are accessible for the wavefunctions, and for the accessible energies the potential is nearly harmonic near $r = 1$ and ``sufficiently'' steep THEN the accessible state space $\mathcal H_{\cut{E}}$ can be studied with the projected observables: $A \mapsto \cut{A} := P_{\overline E} A P_{\overline E}$; this, in particular, means that new commutation relations appear for the coordinate functions $\hat x^i$.
    
    \item The new dynamical system has $O(d)$ as a symmetry: at least it is ``true'' that the Hamiltonian is $O(d)$-covariant (which I'm defining to mean that the group action and $H$ commute) and that acting with this group on a vector state $\psi \in \mathcal H_\Lambda$ doesn't  take it outside $\mathcal H_\lambda$ nor changes its eigenvalues under $L^2$ and $L_z$ (wait... are there actual eigenvalues of these observables or just approximate ones?)
    
    \item The coordinates generate the whole algebra $\mathcal A_{\cut{E}}$.
    
    \item $\mathcal A_{\cut{E}}$ can be realized as the (\dbtext{sub?})algebra of operators of an irrep. $\pi_{\cut{E}}$ of $so(d+2)$. This means that $\mathcal H_{\cut{E}}$ can be though of as an irrep. of $so(d+2)$, which, in particular, means that $\mathcal H_{\cut{E}}$ is a reducible representation of $so(d+1)$, namely, $\mathcal H_{\cut{E}} \cong \bigoplus_{E \leq \cut{E}} V_l$ as representation space of $so(d+1)$.
    
    \item Embeddings and limits: a subalgebra $\mathcal C_\Lambda$ of $\mathcal A_{\Lambda}$ which has the $so(d+1)$-module decomposition $\bigoplus_{E \leq \cut{E}} V_l$ does approximate the commutative space.
    
    \item Basis of $\mathcal A_{\cut{E}}$
    
    \item $L_{ij}$ and $x^i$ tend to the angular momentum and coordinate functions of $S^d$. (Elements of $\mathcal C_\Lambda$?)
    
    \end{itemize}

%%%%%%%%%%%%%%%%%%%%%%%%%%%%%%%%%%%%%%%%%%%%%%%%%%%%%%%%%%%%%%%%%%%%%%%%%%%%%
\subsection{Important Facts}

    \begin{itemize}

    \item 
    
    \end{itemize}

%%%%%%%%%%%%%%%%%%%%%%%%%%%%%%%%%%%%%%%%%%%%%%%%%%%%%%%%%%%%%%%%%%%%%%%%%%%%%
\subsection{Memorize}

    \begin{itemize}

    \item 
    
    \end{itemize}

%%%%%%%%%%%%%%%%%%%%%%%%%%%%%%%%%%%%%%%%%%%%%%%%%%%%%%%%%%%%%%%%%%%%%%%%%%%%%
\subsection{Doubts}

    \begin{itemize}

    \item 
    
    \end{itemize}

%%%%%%%%%%%%%%%%%%%%%%%%%%%%%%%%%%%%%%%%%%%%%%%%%%%%%%%%%%%%%%%%%%%%%%%%%%%%%
\subsection{Detailed summary}

    \begin{itemize}

    \item 
    
    \end{itemize}

%%%%%%%%%%%%%%%%%%%%%%%%%%%%%%%%%%%%%%%%%%%%%%%%%%%%%%%%%%%%%%%%%%%%%%%%%%%%%
\subsection{Notice}

    \begin{itemize}

    \item 
    
    \end{itemize}

%%%%%%%%%%%%%%%%%%%%%%%%%%%%%%%%%%%%%%%%%%%%%%%%%%%%%%%%%%%%%%%%%%%%%%%%%%%%% 
\subsection{Yet to understand}

    \begin{itemize}

    \item 
    
    \end{itemize}

%%%%%%%%%%%%%%%%%%%%%%%%%%%%%%%%%%%%%%%%%%%%%%%%%%%%%%%%%%%%%%%%%%%%%%%%%%%%%
%%%%%%%%%%%%%%%%%%%%%%%%%%%%%%%%%%%%%%%%%%%%%%%%%%%%%%%%%%%%%%%%%%%%%%%%%%%%%
\section{Introduction (pg. 1)}

%%%%%%%%%%%%%%%%%%%%%%%%%%%%%%%%%%%%%%%%%%%%%%%%%%%%%%%%%%%%%%%%%%%%%%%%%%%%%
\subsection*{High Level Summary}

    \begin{itemize}

    \item 
    
    \end{itemize}

%%%%%%%%%%%%%%%%%%%%%%%%%%%%%%%%%%%%%%%%%%%%%%%%%%%%%%%%%%%%%%%%%%%%%%%%%%%%%
\subsection*{Very Important Facts}

    \begin{itemize}

    \item Why does it make sense to have an energy cutoff? At least $2$ reasons:
    
        \begin{enumerate}
            
        \item We might add $\cut E$ as a point where higher energy physics is unkwnown.
        
        \item Where neither we nor the environment can bring a state to higher energies. This gives an effective description of the system. It \dbtext{leas to a lower (distance? ~)} bound in the accuracy with which our apparatus can measure observables, \dbtext{coming from a maximum transferable energy}.
            
        \end{enumerate}
    
    \end{itemize}

%%%%%%%%%%%%%%%%%%%%%%%%%%%%%%%%%%%%%%%%%%%%%%%%%%%%%%%%%%%%%%%%%%%%%%%%%%%%%
\subsection*{Important Facts}

    \begin{itemize}

    \item 
    
    \end{itemize}

%%%%%%%%%%%%%%%%%%%%%%%%%%%%%%%%%%%%%%%%%%%%%%%%%%%%%%%%%%%%%%%%%%%%%%%%%%%%%
\subsection*{Memorize}

    \begin{itemize}

    \item 
    
    \end{itemize}

%%%%%%%%%%%%%%%%%%%%%%%%%%%%%%%%%%%%%%%%%%%%%%%%%%%%%%%%%%%%%%%%%%%%%%%%%%%%%
\subsection*{Directly Relevant Doubts}

    \begin{itemize}

    \item\dbtext{ Do we need high energies to measure high energies / small distances?}
    
    \end{itemize}
    
%%%%%%%%%%%%%%%%%%%%%%%%%%%%%%%%%%%%%%%%%%%%%%%%%%%%%%%%%%%%%%%%%%%%%%%%%%%%%
\subsection*{Small Doubts}

    \begin{itemize}

    \item 
    
    \end{itemize}

%%%%%%%%%%%%%%%%%%%%%%%%%%%%%%%%%%%%%%%%%%%%%%%%%%%%%%%%%%%%%%%%%%%%%%%%%%%%%
\subsection*{Careful}

    \begin{itemize}

    \item 
    
    \end{itemize}


%%%%%%%%%%%%%%%%%%%%%%%%%%%%%%%%%%%%%%%%%%%%%%%%%%%%%%%%%%%%%%%%%%%%%%%%%%%%%
\subsection*{Detailed summary}

    \begin{itemize}

    \item 
    
    \end{itemize}

%%%%%%%%%%%%%%%%%%%%%%%%%%%%%%%%%%%%%%%%%%%%%%%%%%%%%%%%%%%%%%%%%%%%%%%%%%%%%
%%%%%%%%%%%%%%%%%%%%%%%%%%%%%%%%%%%%%%%%%%%%%%%%%%%%%%%%%%%%%%%%%%%%%%%%%%%%%
\section{General setting (pg. 5)}

%%%%%%%%%%%%%%%%%%%%%%%%%%%%%%%%%%%%%%%%%%%%%%%%%%%%%%%%%%%%%%%%%%%%%%%%%%%%%
\subsection*{High Level Summary}

    \begin{itemize}

    \item $H = - \frac{1}{2} \Delta + V(r)$ invariant under $O(D)$
    
    \item Introducing the cutoff, as the energy under which the potential has a harmonic behavior in the region $\nu_{\cut E} = \{r \st V(r) \leq \cut E\}$. This means that we ``have'' to change/update the observables, and in particular means we get a new Schrodinger equation. This new observables will have new commutation relations.
    
    \item Alternative complete set of commutating operators $B = \{\partial_r, L_{ij}\}$. Globally defined outside $r = 0$, and redundant for $D \geq 3$.
    
    \item Eigenfunctions of $H$ as product of a spherical harmonic (eigenfunction of $L^2$) and an eigenfunction of radial equation $9$. This last equation, under the assumption of a nice potential, can be approximated by a harmonic oscillator equation, since outside the region $\nu_{\cut E}$ $\psi$ is negligibly small.
    
    \end{itemize}

%%%%%%%%%%%%%%%%%%%%%%%%%%%%%%%%%%%%%%%%%%%%%%%%%%%%%%%%%%%%%%%%%%%%%%%%%%%%%
\subsection*{Very Important Facts}

    \begin{itemize}

    \item $[P_{\cut E}, H] = 0$. This is equivalent to saying that a state in $\mathcal H_{\cut E}$ doesn't evolve out of it.
    
    \item $[A, B] = 0$ means, between other things, that, if $v$ is an eigenvector of $A$ for the eigenvalue $\lambda$, then so is $B(v)$. Examples: $A, B = H, L_{ij}, P_{\cut E}, Q ``\in" O(D)$.
    
    \item $\cut H = H$
    
    \item We replace any Schrodinger equation $i \partial_t \psi = (H + H')\psi$ with a finite dimensional one within $\mathcal H_{\cut E}$, $i \partial_t \psi = \cut{H + H'}$... but I don't understand well what this means
    
    \item At leading order in \dbtext{what} the lowest eigenvalues of $H$ be considered those of the SHO approximation of equation $9$, multiplied by the $Y$ (eigenvectors of $L^2$ and a Cartan subalgebra elements).
    
    \item Eigenvalues of $L^2$ are $\{j(j+D-2)\}$ for $j \in \ZZ$.
    
    \item The energy of a quantum system can't increase. The energy of an energy eigenvector can't change.
    
    \item Standard quantum mechanics on the sphere $S^d$ are obtained by making $k$ and $\cut E$ diverge while having the region $\nu_{\cut E} = \{r \st V(r) \leq \cut E\}$ go to $\{r = 1\}$.
    
    \end{itemize}

%%%%%%%%%%%%%%%%%%%%%%%%%%%%%%%%%%%%%%%%%%%%%%%%%%%%%%%%%%%%%%%%%%%%%%%%%%%%%
\subsection*{Important Facts}

    \begin{itemize}

    \item There are two conditions on the potential. They are what allow us to approximate the time independent Schrodinger equation $9$ as a $1$-dimensional harmonic oscillator (2D: equation $10$, 3D: equation $36$)
    
        \begin{enumerate}
        
        \item For $r \to 0$, $V(r)$ stays bounded or grows at most as $\beta/r^2$ ($+c/r + d + er +\cdots$) ($\beta \geq 0$). {\tiny This allows equation $9$ to be solved}. Notice that the first condition, if valid in $r = 0$, means that this second condition is also satisfied.
        
        \item Equation $5$: \begin{align*}
            V(r) \approx V_0 + 2k(r-1)^2 && \text{for $r$ such that $V(r) \leq  \cut E$}.
        \end{align*} This condition, if the first one is already satisfied, means that \rtext{$\psi \longleftarrow \tilde f$ are neglibly small outside the shell $V(r) \leq \cut E$}, meaning that \rtext{at leading order \dbtext{in what?} the lowest eigenvalues $E$ are those of the $1$-dimensional harmonic oscillator approximation of $9$.} 
        
        \end{enumerate}
    
    \item $[H, L_{ij}] = 0$... if this is important, it may have to do with having a complete commuting set of observables for our systems, allowing a description of the vectors as lineara combination s of eigenvalues of these observables.
    
    \item $[P_{\cut E}, L_{ij}] = 0$ and so $[P_{\cut E}, L^2] = 0$
    
    \item $\Delta = \partial_r^2 + (D-1)\frac{1}{r}\partial_r - \frac{1}{r^2}L^2$, where $L^2 = L_{ij} L_{ij}/2$
    
    
    \end{itemize}

%%%%%%%%%%%%%%%%%%%%%%%%%%%%%%%%%%%%%%%%%%%%%%%%%%%%%%%%%%%%%%%%%%%%%%%%%%%%%
\subsection*{Memorize}

    \begin{itemize}

    \item 
    
    \end{itemize}

%%%%%%%%%%%%%%%%%%%%%%%%%%%%%%%%%%%%%%%%%%%%%%%%%%%%%%%%%%%%%%%%%%%%%%%%%%%%%
\subsection*{Directly Relevant Doubts}

    \begin{itemize}

    \item $\cut A \in ??$
    
    \item Are the eigenvectors of the harmonic oscillator approximation of $9$ also eigenvectors of $L^2$? At least approximately? RTA: Yes, in an exact way; that is how equation $9$ is derived from equation $8$: by sayint that $\psi = \tilde f(r) Y(\phi, \dots)$
    
    \item What governs the time evolution of a state in $\mathcal H_{\cut E}$? Do we computationally need to change the Schrodinger equation, or is that replacement only a mathematical one (making explicit where the operators live)? I'm not sure if when the projection is done we are somehow systematically getting rid of some type of interaction terms; or if we are simply being mathematically rigorous and indicating that the new operators appearing in the Schrodinger equation are operators acting on $\mathcal H_{\cut E}$?
    
    \item Why is that complete set of commuting operators $B$ introduced?
    
        \begin{itemize}
            
        \item Is the redundancy of the $L_{ij}$ for higher dimensions relevant to us?
            
        \end{itemize}
    
    \item At leading order in what can the lowest eigenvalues of $H$ be considered rhose of the SHO approximation of equation $9$?
    
    \end{itemize}

%%%%%%%%%%%%%%%%%%%%%%%%%%%%%%%%%%%%%%%%%%%%%%%%%%%%%%%%%%%%%%%%%%%%%%%%%%%%%
\subsection*{Small Doubts}

    \begin{itemize}

    \item $[H, L_{ij}] = 0$
    
    \item Are the tangent vectors of $\RR^D$/ $S^d$ \textit{bounded} operators acting on $L^2(\RR^D)$ / $L^2(S^d)$?
    
    \item Are $x^i$, $\partial_i$, $H$ bounded operators on $L^2(\RR^D)$ / $L^2(S^d)$?
    
    \item The complete set of operators $B$ is introduced as an alternative to which one? $\{H, L^2, L_z\}$ in 3D?
    
    \item Eigenvectors of $L^2$, denoted by $Y$, are also eigenvectors of the elements of a Cartan subalgebra of $so(D)$.
    
    \end{itemize}

%%%%%%%%%%%%%%%%%%%%%%%%%%%%%%%%%%%%%%%%%%%%%%%%%%%%%%%%%%%%%%%%%%%%%%%%%%%%%
\subsection*{Careful}

    \begin{itemize}

    \item 
    
    \end{itemize}

%%%%%%%%%%%%%%%%%%%%%%%%%%%%%%%%%%%%%%%%%%%%%%%%%%%%%%%%%%%%%%%%%%%%%%%%%%%%%
\subsection*{Detailed summary}

    \begin{itemize}

    \item 
    
    \end{itemize}

%%%%%%%%%%%%%%%%%%%%%%%%%%%%%%%%%%%%%%%%%%%%%%%%%%%%%%%%%%%%%%%%%%%%%%%%%%%%%
%%%%%%%%%%%%%%%%%%%%%%%%%%%%%%%%%%%%%%%%%%%%%%%%%%%%%%%%%%%%%%%%%%%%%%%%%%%%%
\section{$D=2$: $O(2)$-equivariant fuzzy circle (pg. 5.75)}

%%%%%%%%%%%%%%%%%%%%%%%%%%%%%%%%%%%%%%%%%%%%%%%%%%%%%%%%%%%%%%%%%%%%%%%%%%%%%
\subsection{Realization of the Algebra of Observables throuth $Uso(3)$ (pg. 11.5)}

%%%%%%%%%%%%%%%%%%%%%%%%%%%%%%%%%%%%%%%%%%%%%%%%%%%%%%%%%%%%%%%%%%%%%%%%%%%%%
\subsection{Convergence to $O(2)$-equivariant quantum mechanics on $S^1$ (pg. 13.5)}

%%%%%%%%%%%%%%%%%%%%%%%%%%%%%%%%%%%%%%%%%%%%%%%%%%%%%%%%%%%%%%%%%%%%%%%%%%%%%
%%%%%%%%%%%%%%%%%%%%%%%%%%%%%%%%%%%%%%%%%%%%%%%%%%%%%%%%%%%%%%%%%%%%%%%%%%%%%
\section{$D=3$: $O(3)$-equivariant fuzzy sphere (pg. 15)}

%%%%%%%%%%%%%%%%%%%%%%%%%%%%%%%%%%%%%%%%%%%%%%%%%%%%%%%%%%%%%%%%%%%%%%%%%%%%%
\subsection{Realization of the Algebra of Observables throuth $Uso(4)$ (pg. 17.75)}

%%%%%%%%%%%%%%%%%%%%%%%%%%%%%%%%%%%%%%%%%%%%%%%%%%%%%%%%%%%%%%%%%%%%%%%%%%%%%
\subsection{Convergence to $O(3)$-equivariant quantum mechanics on $S^2$ (pg. 19.75)}

%%%%%%%%%%%%%%%%%%%%%%%%%%%%%%%%%%%%%%%%%%%%%%%%%%%%%%%%%%%%%%%%%%%%%%%%%%%%%
%%%%%%%%%%%%%%%%%%%%%%%%%%%%%%%%%%%%%%%%%%%%%%%%%%%%%%%%%%%%%%%%%%%%%%%%%%%%%
\section{Final remarks, Outlook and Conclusions (pg. 21.8)}

%%%%%%%%%%%%%%%%%%%%%%%%%%%%%%%%%%%%%%%%%%%%%%%%%%%%%%%%%%%%%%%%%%%%%%%%%%%%%
%%%%%%%%%%%%%%%%%%%%%%%%%%%%%%%%%%%%%%%%%%%%%%%%%%%%%%%%%%%%%%%%%%%%%%%%%%%%%
\section{Appendix (pg. 23)}

%%%%%%%%%%%%%%%%%%%%%%%%%%%%%%%%%%%%%%%%%%%%%%%%%%%%%%%%%%%%%%%%%%%%%%%%%%%%%
\subsection{Calculation of a rather general scalar product in $D=2$ (pg. 23)}

%%%%%%%%%%%%%%%%%%%%%%%%%%%%%%%%%%%%%%%%%%%%%%%%%%%%%%%%%%%%%%%%%%%%%%%%%%%%%
\subsection{Calculation of the action of operators in $D=2$ (pg. 26)}

%%%%%%%%%%%%%%%%%%%%%%%%%%%%%%%%%%%%%%%%%%%%%%%%%%%%%%%%%%%%%%%%%%%%%%%%%%%%%
\subsection{Proof of proposition [??] (pg. 28)}

%%%%%%%%%%%%%%%%%%%%%%%%%%%%%%%%%%%%%%%%%%%%%%%%%%%%%%%%%%%%%%%%%%%%%%%%%%%%%
\subsection{Spherical Harmonics (pg. 29)}

%%%%%%%%%%%%%%%%%%%%%%%%%%%%%%%%%%%%%%%%%%%%%%%%%%%%%%%%%%%%%%%%%%%%%%%%%%%%%
\subsection{Calculation of $|N_l|$ in $D = 3$ (pg. 30)}

%%%%%%%%%%%%%%%%%%%%%%%%%%%%%%%%%%%%%%%%%%%%%%%%%%%%%%%%%%%%%%%%%%%%%%%%%%%%%
\subsection{Calculation of a rather general scalar product in $D = 3$ (pg. 30)}

%%%%%%%%%%%%%%%%%%%%%%%%%%%%%%%%%%%%%%%%%%%%%%%%%%%%%%%%%%%%%%%%%%%%%%%%%%%%%
\subsection{Proof or (40) and or proposition 4.1 (pg. 32)}

%%%%%%%%%%%%%%%%%%%%%%%%%%%%%%%%%%%%%%%%%%%%%%%%%%%%%%%%%%%%%%%%%%%%%%%%%%%%%
\subsection{Action of commutators of the $\overline{\partial_a}$ (pg. 34)}

%%%%%%%%%%%%%%%%%%%%%%%%%%%%%%%%%%%%%%%%%%%%%%%%%%%%%%%%%%%%%%%%%%%%%%%%%%%%%
\subsection{Proof of proposition 4.2 and other results of subsection 4.1 (pg. 35)}

%%%%%%%%%%%%%%%%%%%%%%%%%%%%%%%%%%%%%%%%%%%%%%%%%%%%%%%%%%%%%%%%%%%%%%%%%%%%%
\subsection{Shifting the lower extreme of integration over $r$ (pg. 37)}

%%%%%%%%%%%%%%%%%%%%%%%%%%%%%%%%%%%%%%%%%%%%%%%%%%%%%%%%%%%%%%%%%%%%%%%%%%%%%
\subsection{Proof of proposition 4.3 and other results of section 4.2 (pg. 38-41)}

\end{document}
