\documentclass{article}
\usepackage[utf8]{inputenc}
\usepackage[margin=1in]{geometry}

%%%%%% To use hyperlinks, including the formula ones
\usepackage{hyperref}
\hypersetup{
    colorlinks=true,
    linkcolor=blue,
    filecolor=magenta,      
    urlcolor=cyan,
}

%%%%%% Make paragraphs start with no indentation and leave spaces between paragraphs
\setlength{\parindent}{0em}
\setlength{\parskip}{1em}

%%%%%% Math stuff
\usepackage{amsmath, amssymb}
\usepackage{amsthm}

%%%%%% Mis Codigos

% TODO notes package
\usepackage{xargs}                  % Use more than one optional parameter in a new command
\usepackage[pdftex,dvipsnames]{xcolor}
\input{tools/todoCode}

% Colored text and boxes with my color conventions for highlighting
\usepackage[dvipsnames]{xcolor}
\input{tools/colorCode}

% Math symbols
\usepackage{xparse}
\input{tools/common_math_symbols}

% Physics symbols (vectors, units)
\usepackage{tikz}
\input{tools/physics_macros}

% Theorem environments
\input{tools/theorem_definitions}

%%%%%% Specific New Commands
\newcommand{\alg}[1]{\ensuremath{\mathfrak{#1}}}
\newcommand{\sut}{\ensuremath{\mathfrak{su}(2)}}

%%%%%%
\title{Metric Properties of the Fuzzy Sphere - D'Andrea, Lizzi, V\'arilly}
\author{Sebastian Camilo Puerto}
\date{July 2020}

%%%%%%%%%%%%%%%%%%%%%%%%%%%%%%%%%%%%%%%%%%%%%%%%%%%%%%%%%%%%%%%%%%%%%%%%%%%%%
%%%%%%%%%%%%%%%%%%%%%%%%%%%%%%%%%%%%%%%%%%%%%%%%%%%%%%%%%%%%%%%%%%%%%%%%%%%%%
%%%%%%%%%%%%%%%%%%%%%%%%%%%%%%%%%%%%%%%%%%%%%%%%%%%%%%%%%%%%%%%%%%%%%%%%%%%%%
%%%%%%%%%%%%%%%%%%%%%%%%%%%%%%%%%%%%%%%%%%%%%%%%%%%%%%%%%%%%%%%%%%%%%%%%%%%%%
\begin{document}

\maketitle

\tableofcontents

%%%%%%%%%%%%%%%%%%%%%%%%%%%%%%%%%%%%%%%%%%%%%%%%%%%%%%%%%%%%%%%%%%%%%%%%%%%%%
\subsection{Perhaps Relevant to Build Dirac Operator in NEW Fuzzy Spheres}

    \begin{itemize}

    \item Other $3$ spectral triples on the Fuzzy Sphere that begin with a chirality operator and then find an anticommuting self-adjoint Dirac-like operator with a plausible spectrum.
    
    \item How is the Dirac operator proposed here?: start from the $SU(2)$-equivariance and arrive at a neat \textbf{truncation of the classical spectrum}, paying the price of spectral assymetry.
    
    \end{itemize}

%%%%%%%%%%%%%%%%%%%%%%%%%%%%%%%%%%%%%%%%%%%%%%%%%%%%%%%%%%%%%%%%%%%%%%%%%%%%%
\subsection{High Level Summary}

    \begin{itemize}

    \item Introduction: definition of the Fuzzy sphere and introduction of the $3$ notions that gives meaning to ``the Fuzzy Sphere converges to $S^2$'': as *-algebras, as representations of $SU(2)$ and as metric spaces.
    
    \item Background definitions: spectral triple definitions, distance, compact quantum metric space.
    
    \item Detailed construction and analysis of spectrum of the canonical Dirac operator on $S^2$ through the approach that will be replicated(?) for the Fuzzy Sphere.
    
    \item Construction of $2$ $SU(2)$ invariant spectral triples on $\mathcal A_N$, the eigenvalues of the Dirac operators, and proposition that shows that these $2$ triples generate the same metric on the state of states of $\mathcal A_N$.
    
    \item Quick overview of other $3$ spectral triples on the Fuzzy Sphere that begin with a chirality operator and then find an anticommuting self-adjoint Dirac-like operator with a plausible spectrum.
    
    \item Introduce a family of pairs of pure states in the state space $\mathcal S(\mathcal A_N)$: (Bloch) $SU(2)$-coherent states on $\mathcal A_N$: $\psi^N_{(\phi, \theta)}$
    
    \item $N = 1$ case: all pure state are coherent states and an explicit formula for the distance is found.
    
    \item Distance between basis vector states associated to $|j, m\rangle$ explicitly found for all $N$.
    
    \item An auxiliary distance $\rho_N$, which will be a lower bound for $d_N$ is introduced: distance along diagonal matrices of $\mathcal A_N$.
    
    \item The distance is $SU(2)$-invariant
    
    \item The distance between $2$ states associated to the same points $p, q \in S^2$ is non-decreasing with $N$.
    
    \item In general:
    \begin{equation}
        \rho_N(\theta - \theta') \leq d_N(\psi^N_{(\phi, \theta)}, \psi^N_{(\phi', \theta')}) \leq d_{geo}((\phi, \theta), (\phi', \theta'))
    \end{equation} and, finally,
    \begin{equation}
        \lim_{N \to \infty} d_N(\psi^N_{(\phi, \theta)}, \psi^N_{(\phi', \theta')}) = d_{geo}(\psi^N_{(\phi, \theta)}, \psi^N_{(\phi', \theta')})
    \end{equation}
    
    \end{itemize}

%%%%%%%%%%%%%%%%%%%%%%%%%%%%%%%%%%%%%%%%%%%%%%%%%%%%%%%%%%%%%%%%%%%%%%%%%%%%%
\subsection{Very Important Facts} % Everything written here is also written somewhere else

    \begin{itemize}

    \item Why Fuzzy Spaces?: To preserve the symmetries and keep the algebra finite dimensional.
    
    \item GOAL: The fuzzy sphere approximates the: \textbf{1. $U(\alg{su}(2))$-module (i.e. $\alg{su}(2)$ Lie algebra representation) 2. *-algebra 3. metric space} $\mathcal A(S^2) \cong L^2(S^2)$ of complex polynomials in $x_1, x_2, x_3$.
    
    \item The notion of convergence of this fuzzy space to $S^2$ must also involve metrics: $d_N$ in $\mathcal A_n$ and $d_{geo}$ in $S^2$: in this paper the points of $S^2$ are idetified with the corresponding (Block) coherent states of $\mathcal A_N$, and under this identification:
    \begin{equation}
        \lim_{N \to \infty} d_N(p, q) = d_{geo}(p, q)\quad \text{for all }p, q \in S^2
    \end{equation}
    
    \item How is the Dirac operator proposed here?: start from the $SU(2)$-equivariance and arrive at a neat \textbf{truncation of the classical spectrum}, paying the price of spectral assymetry.
    
    \item The Bloch $SU(2)$-coherent states are for the group $SU(2)$ what the usual harmonic oscillator coherent states are for the Heisenberg group. In particular, they are minimum uncertainty states.
    
    \item How are the spectral triples obtained... analogous procedure to the one in commutative $S^2$? TODO
    
    \end{itemize}

%%%%%%%%%%%%%%%%%%%%%%%%%%%%%%%%%%%%%%%%%%%%%%%%%%%%%%%%%%%%%%%%%%%%%%%%%%%%%
\subsection{Important Facts}

    \begin{itemize}

    \item 
    
    \end{itemize}

%%%%%%%%%%%%%%%%%%%%%%%%%%%%%%%%%%%%%%%%%%%%%%%%%%%%%%%%%%%%%%%%%%%%%%%%%%%%%
\subsection{Memorize}

    \begin{itemize}

    \item 
    
    \end{itemize}

%%%%%%%%%%%%%%%%%%%%%%%%%%%%%%%%%%%%%%%%%%%%%%%%%%%%%%%%%%%%%%%%%%%%%%%%%%%%%
\subsection{Doubts}

    \begin{itemize}

    \item 
    
    \end{itemize}

%%%%%%%%%%%%%%%%%%%%%%%%%%%%%%%%%%%%%%%%%%%%%%%%%%%%%%%%%%%%%%%%%%%%%%%%%%%%%
\subsection{Detailed summary}

    \begin{itemize}

    \item 
    
    \end{itemize}

%%%%%%%%%%%%%%%%%%%%%%%%%%%%%%%%%%%%%%%%%%%%%%%%%%%%%%%%%%%%%%%%%%%%%%%%%%%%%
\subsection{Notice}

    \begin{itemize}

    \item 
    
    \end{itemize}

%%%%%%%%%%%%%%%%%%%%%%%%%%%%%%%%%%%%%%%%%%%%%%%%%%%%%%%%%%%%%%%%%%%%%%%%%%%%% 
\subsection{Yet to understand}

    \begin{itemize}

    \item 
    
    \end{itemize}

%%%%%%%%%%%%%%%%%%%%%%%%%%%%%%%%%%%%%%%%%%%%%%%%%%%%%%%%%%%%%%%%%%%%%%%%%%%%%
%%%%%%%%%%%%%%%%%%%%%%%%%%%%%%%%%%%%%%%%%%%%%%%%%%%%%%%%%%%%%%%%%%%%%%%%%%%%%
\section{Introduction (pg. 1)}

    \begin{itemize}
        
    \item Why Fuzzy Spaces?: To preserve the symmetries and keep the algebra finite dimensional.    
    
    \item GOAL: The fuzzy sphere approximates the: \textbf{1. $U(\alg{su}(2))$-module (i.e. $\alg{su}(2)$ Lie algebra representation) 2. *-algebra 3. metric space} $\mathcal A(S^2) \cong L^2(S^2)$ of complex polynomials in $x_1, x_2, x_3$.
        
    \item Definition of the $*$-algebra and $\alg{su(2)}$-representation $\mathcal A_N$. With respect to these $2$ properties, $\lim_{n \to \infty}$ of these spaces is indeed $\mathcal A(S^2)$, but the third property for real convergence is still missing.
    
        \begin{itemize}
            
        \item Parting from the decomposition of in irreps of $\alg{su}(2)$ of $\mathcal{A}(S^2)$ as $ = \bigoplus_{l =  0, 1, \dots}^\infty V_l$
        
            \begin{itemize}
                
            \item where $V_l$ is homogeneous polynomials of degree $l$
                
            \end{itemize}
        
        \item Introduce cut-off in the energy spectrum: keep only the first $N =: 2j$ representations: it works as a $\sut(2)$ representation but not as an algebra.
        
        \item To make it an algebra, pullback the algebra structure from the right of the following vector space isomorphism... actually, in this article the RHS is taken to BE the $\mathcal A_N$:
        \begin{align}
            \bigoplus_{l =  0, 1, \dots}^{2j} V_l \subset \mathcal{A}(S^2) &\to \mathcal A_N := M_{N+1}(\CC)\\
            x_k &\mapsto \hat x_k := \frac{1}{\sqrt{j(j+1)}}\pi_j(J_k)
        \end{align}
        
            \begin{itemize}
                
            \item $[J_i, J_k] = i \epsilon_{ijk} J_k$ are the standard real generators of $\sut$
            
            \item $\pi_j:\sut\text{ o }SU(2) \to M_{N+1}(\CC)$ is the irrep of spin $j \in \frac{\NN}{2}$
                
            \end{itemize}
        
        \item The *-algebra is made into a $\sut$ representation by pushing forward the representation of the above map, i.e. making it an isomorphism of representations of $\sut$ / $SU(2)$ TODO
        
            \begin{itemize}
                
            \item This action of $SU(2)$ is equivalent to the action of $U(\sut)$ given by:
            
            \end{itemize}
        
        \item The above definitions imply that $g_{ij}\hat x_i \hat x_j = 1 = \hat x_1^2 + \hat x_2^2 + \hat x_3^2$. This means, in particular, that:
        
        \begin{equation}
            [\hat x_k, \hat x_l] = \frac{1}{\sqrt{j(j+1)}} i \epsilon_{klm}\hat x_m
        \end{equation}
        
        \end{itemize}
    
    \item The notion of convergence of this fuzzy space to $S^2$ must also involve metrics: $d_N$ in $\mathcal A_n$ and $d_{geo}$ in $S^2$: in this paper the points of $S^2$ are idetified with the corresponding (Block) coherent states of $\mathcal A_N$, and under this identification:
    \begin{equation}
        \lim_{N \to \infty} d_N(p, q) = d_{geo}(p, q)\quad \text{for all }p, q \in S^2
    \end{equation}    
    
    \end{itemize}
    
\subsection*{Definitions}

    \begin{itemize}

    \item 
    
    \end{itemize}

%%%%%%%%%%%%%%%%%%%%%%%%%%%%%%%%%%%%%%%%%%%%%%%%%%%%%%%%%%%%%%%%%%%%%%%%%%%%%
%%%%%%%%%%%%%%%%%%%%%%%%%%%%%%%%%%%%%%%%%%%%%%%%%%%%%%%%%%%%%%%%%%%%%%%%%%%%%
\section{Preliminaries on noncommutative manifolds (pg. 3)}

\begin{itemize}
        
    \item 
        
    \end{itemize}
    
\subsection*{Definitions}

    \begin{itemize}

    \item 
    
    \end{itemize}

%%%%%%%%%%%%%%%%%%%%%%%%%%%%%%%%%%%%%%%%%%%%%%%%%%%%%%%%%%%%%%%%%%%%%%%%%%%%%
%%%%%%%%%%%%%%%%%%%%%%%%%%%%%%%%%%%%%%%%%%%%%%%%%%%%%%%%%%%%%%%%%%%%%%%%%%%%%
\section{Dirac operators for the fuzzy sphere (pg. 4)}

\begin{itemize}
        
    \item 
        
    \end{itemize}
    
\subsection*{Definitions}

    \begin{itemize}

    \item 
    
    \end{itemize}


%%%%%%%%%%%%%%%%%%%%%%%%%%%%%%%%%%%%%%%%%%%%%%%%%%%%%%%%%%%%%%%%%%%%%%%%%%%%%
\subsection{An abstract Dirac operator (pg. 4.5)}

%%%%%%%%%%%%%%%%%%%%%%%%%%%%%%%%%%%%%%%%%%%%%%%%%%%%%%%%%%%%%%%%%%%%%%%%%%%%%
\subsection{The Dirac operator of $S^2$ (pg. 5.3)}

%%%%%%%%%%%%%%%%%%%%%%%%%%%%%%%%%%%%%%%%%%%%%%%%%%%%%%%%%%%%%%%%%%%%%%%%%%%%%
\subsection{Dirac operators on the fuzzy sphere (pg. 5.6)}

%%%%%%%%%%%%%%%%%%%%%%%%%%%%%%%%%%%%%%%%%%%%%%%%%%%%%%%%%%%%%%%%%%%%%%%%%%%%%
\subsection{Comparisons with the literature (pg. 9)}

%%%%%%%%%%%%%%%%%%%%%%%%%%%%%%%%%%%%%%%%%%%%%%%%%%%%%%%%%%%%%%%%%%%%%%%%%%%%%
%%%%%%%%%%%%%%%%%%%%%%%%%%%%%%%%%%%%%%%%%%%%%%%%%%%%%%%%%%%%%%%%%%%%%%%%%%%%%
\section{Spectral distance between coherent states (pg. 11)}

%%%%%%%%%%%%%%%%%%%%%%%%%%%%%%%%%%%%%%%%%%%%%%%%%%%%%%%%%%%%%%%%%%%%%%%%%%%%%
\subsection{The $N = 1$ case (pg. 12)}

%%%%%%%%%%%%%%%%%%%%%%%%%%%%%%%%%%%%%%%%%%%%%%%%%%%%%%%%%%%%%%%%%%%%%%%%%%%%%
\subsection{Distances between basis vectors (pg. 13)}

%%%%%%%%%%%%%%%%%%%%%%%%%%%%%%%%%%%%%%%%%%%%%%%%%%%%%%%%%%%%%%%%%%%%%%%%%%%%%
\subsection{An auxiliary distance (pg. 14)}

%%%%%%%%%%%%%%%%%%%%%%%%%%%%%%%%%%%%%%%%%%%%%%%%%%%%%%%%%%%%%%%%%%%%%%%%%%%%%
\subsection{$SU(2)$-invariance of the distance (pg. 15)}

%%%%%%%%%%%%%%%%%%%%%%%%%%%%%%%%%%%%%%%%%%%%%%%%%%%%%%%%%%%%%%%%%%%%%%%%%%%%%
\subsection{Dependence on the dimension (pg. 15.5)}

%%%%%%%%%%%%%%%%%%%%%%%%%%%%%%%%%%%%%%%%%%%%%%%%%%%%%%%%%%%%%%%%%%%%%%%%%%%%%
\subsection{Upper and lower bounds and the large $N$ limit (pg. 17-18)}

\end{document}
