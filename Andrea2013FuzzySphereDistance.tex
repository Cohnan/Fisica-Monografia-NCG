\documentclass{article}
\usepackage[utf8]{inputenc}
\usepackage[margin=1in]{geometry}

%%%%%% To use hyperlinks, including the formula ones
\usepackage{hyperref}
\hypersetup{
    colorlinks=true,
    linkcolor=blue,
    filecolor=magenta,      
    urlcolor=cyan,
}

%%%%%% Make paragraphs start with no indentation and leave spaces between paragraphs
\setlength{\parindent}{0em}
\setlength{\parskip}{1em}

%%%%%% Math stuff
\usepackage{amsmath, amssymb}
\usepackage{amsthm}

%%%%%% Mis Codigos

% TODO notes package
\usepackage{xargs}                  % Use more than one optional parameter in a new command
\usepackage[pdftex,dvipsnames]{xcolor}
%\usepackage{xargs}                      % Use more than one optional parameter in a new
%\usepackage[pdftex,dvipsnames]{xcolor}  % Coloured text etc.

%
%\usepackage[colorinlistoftodos,prependcaption,textsize=tiny]{todonotes}
\usepackage[disable]{todonotes}

\newcommandx{\unsure}[2][1=]{\todo[linecolor=red,backgroundcolor=red!25,bordercolor=red,#1]{#2}}
\newcommandx{\change}[2][1=]{\todo[linecolor=blue,backgroundcolor=blue!25,bordercolor=blue,#1]{#2}}
\newcommandx{\complete}[2][1=]{\todo[linecolor=pink,backgroundcolor=pink!25,bordercolor=blue,#1]{#2}}
\newcommandx{\info}[2][1=]{\todo[linecolor=OliveGreen,backgroundcolor=OliveGreen!25,bordercolor=OliveGreen,#1]{#2}}
\newcommandx{\improvement}[2][1=]{\todo[linecolor=Plum,backgroundcolor=Plum!25,bordercolor=Plum,#1]{#2}}
\newcommandx{\thiswillnotshow}[2][1=]{\todo[disable,#1]{#2}}

% Colored text and boxes with my color conventions for highlighting
\usepackage[dvipsnames]{xcolor}
%\usepackage[dvipsnames]{xcolor}

%
% \newcommand{\ytext}[1]{\textcolor{yellow}{#1}}
% \newcommand{\otext}[1]{\textcolor{orange}{#1}}
% \newcommand{\rtext}[1]{\textcolor{red}{#1}}
% \newcommand{\lbtext}[1]{\textcolor{cyan}{#1}}
% \newcommand{\dbtext}[1]{\textcolor{blue}{#1}}
% \newcommand{\ptext}[1]{\textcolor{Plum}{#1}}
% \newcommand{\lgtext}[1]{\textcolor{LimeGreen}{#1}}
% \newcommand{\dgtext}[1]{\textcolor{OliveGreen}{#1}}

\newcommand{\ytext}[1]{\textcolor{black}{#1}}
\newcommand{\otext}[1]{\textcolor{black}{#1}}
\newcommand{\rtext}[1]{\textit{#1}}
\newcommand{\lbtext}[1]{\textcolor{black}{#1}}
\newcommand{\dbtext}[1]{\textcolor{black}{#1}}
\newcommand{\ptext}[1]{\textcolor{black}{#1}}
\newcommand{\lgtext}[1]{\textcolor{black}{#1}}
\newcommand{\dgtext}[1]{\textcolor{black}{#1}}



\newcommand{\ybox}[1]{\colorbox{yellow}{#1}}
\newcommand{\obox}[1]{\colorbox{orange}{#1}}
\newcommand{\rbox}[1]{\colorbox{Salmon}{#1}}
\newcommand{\lbbox}[1]{\colorbox{SkyBlue}{#1}}
\newcommand{\dbbox}[1]{\colorbox{NavyBlue}{#1}}
\newcommand{\pbox}[1]{\colorbox{Plum}{#1}}
\newcommand{\lgbox}[1]{\colorbox{LimeGreen}{#1}}
\newcommand{\dgbox}[1]{\colorbox{OliveGreen}{#1}}



% Math symbols
\usepackage{xparse}

%\usepackage{amssymb,amsmath,amsthm}
%\usepackage{xparse}

%%% Common symbols redifined
\let\oldepsilon\epsilon
\renewcommand{\epsilon}{\varepsilon}
\renewcommand{\varepsilon}{\oldepsilon}

\let\oldphi\phi
\renewcommand{\phi}{\varphi}
\renewcommand{\varphi}{\oldphi}

\newcommand{\emty}{\varnothing}
\newcommand{\varempty}{\nothing}

%%% Common Sets
\newcommand{\bb}[1]{\ensuremath{\mathbb{#1}} }
\newcommand{\ZZ}{\ensuremath{\mathbb{Z}} }
\newcommand{\NN}{\ensuremath{\mathbb{N}} }
\newcommand{\QQ}{\ensuremath{\mathbb{Q}} }
\newcommand{\RR}{\ensuremath{\mathbb{R}} }
\newcommand{\CC}{\ensuremath{\mathbb{C}} }


\newcommand{\iss}{\cong}  % Isomorphism symbol, here it is the one with a tilde

%%%%%%%%%%%%%% Sets

\newcommand{\set}[1]{\ensuremath{\left\{ #1 \right\}}} % Set function, simply puts nice left and right braces
\newcommand{\st}{\ensuremath{\ |\ }} % Such that symbol, TODO improve

\newcommand{\psubset}{\ensuremath{\subset}}
\renewcommand{\subset}{\ensuremath{\subseteq}} % Subset symbol, in this case it is the subset or equal to symbol

\newcommand{\bunion}{\bigcup}
%\newcommand{\biun}{\bun^{\infty}}
%\newcommand{\bfun}[3][n]{\bun_{#1 = #2}^{#3}}
\newcommand{\union}{\cup}
%\newcommand{\siun}{\sun^{\infty}}
%\newcommand{\sfun}[3][n]{\sun_{#1 = #2}^{#3}}

\newcommand{\binter}{\bigcap}
%\newcommand{\biinter}{\binter^{\infty}}
%\newcommand{\bfifter}[3][n]{\binter_{#1 = #2}^{#3}}
\newcommand{\inter}{\cap}
%\newcommand{\sininter}{\sinter^{\infty}}
%\newcommand{\sfinter}[3][n]{\sinter_{#1 = #2}^{#3}}

%%%%%% Calculus

% Integrals

%% Single Integrals
\NewDocumentCommand \integ {s O{} O{} m o}
{
	\IfBooleanTF{#1}{\oint}{\int}_{#2}^{#3} #4 %
	\IfNoValueF {#5} {\mathrm{d} #5}
}

%% Derivatives

% Normal derivative
% Example: \der[n]{f}{x}[x_0]
\NewDocumentCommand \der {O{} m m o}
{
	\frac{\mathrm{d}^{#1} #2}{\mathrm{d} {#3}^{#1}}%
	\IfNoValueF{#4} {\biggr|_{#4}}
}

%% Partial Derivatives

% With respect to one variable
% Example: \pder[n]{f}{y}[\pthvars[x_0][y_0][z_0]][(x, z)
\NewDocumentCommand \pder {O{} m m O{} o}
{
    \ensuremath{
	\IfNoValueTF {#5}
	{
		\frac{\partial^{#1} #2}{\partial {#3}^{#1}} #4
	}
	{
		\left(%
		\frac{\partial^{#1} #2}{\partial {#3}^{#1}}%
		\right)_{#5}  #4
	}
	}
}

% With respecto to two variables
% Example: \twpder{g}{y}{x}[(x_0, y_0)]
\NewDocumentCommand \twpder {m m m O{}}
{
	\frac{\partial^2 #1}{\partial #2 \partial #3} #4
}

\newcommand{\abs}[1]{\left\lvert #1 \right\rvert}
\newcommand{\norm}[1]{\left\lVert #1 \right\rVert}

% Physics symbols (vectors, units)
\usepackage{tikz}
% Version of December 26 2016

%%%%%%%%%%%%%%%%%%%%%%%%%%%%%%%%%%%%%%%%%%% Vectors

%%%%%%%% For SI units %%%%%%%
%\usepackage{siunitx}

%%%%%%%% Vectors %%%%%%%%%%%% (Just see last lines)
%\usepackage{tikz}         % For arrow and dots in \xvec

% --- Macro \xvec
\makeatletter
\newlength\xvec@height%
\newlength\xvec@depth%
\newlength\xvec@width%
\newcommand{\xvec}[2][]{%
  \ifmmode%
    \settoheight{\xvec@height}{$#2$}%
    \settodepth{\xvec@depth}{$#2$}%
    \settowidth{\xvec@width}{$#2$}%
  \else%
    \settoheight{\xvec@height}{#2}%
    \settodepth{\xvec@depth}{#2}%
    \settowidth{\xvec@width}{#2}%
  \fi%
  \def\xvec@arg{#1}%
  \def\xvec@dd{:}%
  \def\xvec@d{.}%
  \raisebox{.2ex}{\raisebox{\xvec@height}{\rlap{%
    \kern.05em%  (Because left edge of drawing is at .05em)
    \begin{tikzpicture}[scale=1]
    \pgfsetroundcap
    \draw (.05em,0)--(\xvec@width-.05em,0);
    \draw (\xvec@width-.05em,0)--(\xvec@width-.15em, .075em);
    \draw (\xvec@width-.05em,0)--(\xvec@width-.15em,-.075em);
    \ifx\xvec@arg\xvec@d%
      \fill(\xvec@width*.45,.5ex) circle (.5pt);%
    \else\ifx\xvec@arg\xvec@dd%
      \fill(\xvec@width*.30,.5ex) circle (.5pt);%
      \fill(\xvec@width*.65,.5ex) circle (.5pt);%
    \fi\fi%
    \end{tikzpicture}%
  }}}%
  #2%
}
\makeatother

% --- Override \vec with an invocation of \xvec.
\let\stdvec\vec
\renewcommand{\vec}[1]{\xvec[]{#1}}                             % Vector
% --- Define \dvec and \ddvec for dotted and double-dotted vectors.
\newcommand{\tvec}[1]{\xvec[.]{#1}}                             % Vector derived wrt time
\newcommand{\ttvec}[1]{\xvec[:]{#1}}                            % Vector derived twice wrt time


% Theorem environments
%Version of October 8, 2016

%\usepackage{amsthm}

\theoremstyle{definition} %To avoid the annoying italics all the time, and to not sloppily redefine all of them 

\newtheorem{theo}{Theorem}[section]  %numbered according to section environment, so in section to it restarts as 2.1 
\newtheorem{prop}{Proposition}[section]  %numbered according to section environment, so in section to it restarts as 2.1 
\newtheorem{lemma}[theo]{Lemma}     %numbering shared with theorem 
\newtheorem{defn}{Definition}[section]   
\newtheorem{coro}{Corollary}[theo]


\theoremstyle{remark} 
\newtheorem*{remark}{Remark} 
 
%\let\oldtheo\theo 
%\renewcommand{\theo}{\oldtheo\normalfont}  
%  
%\let\olddefn\defn  
%\renewcommand{\defn}{\olddefn\normalfont}  
%  
%\let\oldlemma\lemma  
%\renewcommand{\lemma}{\oldlemma\normalfont}  
%  
%\let\oldcoro\coro  
%\renewcommand{\coro}{\oldcoro\normalfont}

%%%%%%%% ``Example'' environment, very basic, doesnt work with itemize
\theoremstyle{definition}

\newtheorem*{exmp}{Example}

%%%%%% Small extra packages
\usepackage[normalem]{ulem} % To strike-through text with the command \sout{}
\usepackage{slashed} % To draw Dirac operators as \slashed{D}

%%%%%% Specific New Commands
\newcommand{\alg}[1]{\ensuremath{\mathfrak{#1}}}
\newcommand{\sut}{\ensuremath{\mathfrak{su}(2)}}

%%%%%%
\title{Metric Properties of the Fuzzy Sphere - D'Andrea, Lizzi, V\'arilly}
\author{Sebastian Camilo Puerto}
\date{July 2020}

%%%%%%%%%%%%%%%%%%%%%%%%%%%%%%%%%%%%%%%%%%%%%%%%%%%%%%%%%%%%%%%%%%%%%%%%%%%%%
%%%%%%%%%%%%%%%%%%%%%%%%%%%%%%%%%%%%%%%%%%%%%%%%%%%%%%%%%%%%%%%%%%%%%%%%%%%%%
%%%%%%%%%%%%%%%%%%%%%%%%%%%%%%%%%%%%%%%%%%%%%%%%%%%%%%%%%%%%%%%%%%%%%%%%%%%%%
%%%%%%%%%%%%%%%%%%%%%%%%%%%%%%%%%%%%%%%%%%%%%%%%%%%%%%%%%%%%%%%%%%%%%%%%%%%%%
\begin{document}

\maketitle

\tableofcontents

%%%%%%%%%%%%%%%%%%%%%%%%%%%%%%%%%%%%%%%%%%%%%%%%%%%%%%%%%%%%%%%%%%%%%%%%%%%%%
\subsection{Perhaps Relevant to Build Dirac Operator in NEW Fuzzy Spheres}

    \begin{itemize}

    \item Other $3$ spectral triples on the Fuzzy Sphere that begin with a chirality operator and then find an anticommuting self-adjoint Dirac-like operator with a plausible spectrum.
    
    \item How is the Dirac operator proposed here?: start from the $SU(2)$-equivariance and arrive at a neat \textbf{truncation of the classical spectrum} by replacing usual spherical harmonics by fuzzy spherical harmonics, paying the price of spectral assymetry in the spectrum.
    
    \item There is a full theory of equivariant Dirac operators in Riemannian geometry, where, for example there is the Schrodinger-Lichnerowics formula $D^2 = C_G + R/8$ where $C_G$ is the Casimir operator and $R$ is the scalar curvature.
    
    \end{itemize}

%%%%%%%%%%%%%%%%%%%%%%%%%%%%%%%%%%%%%%%%%%%%%%%%%%%%%%%%%%%%%%%%%%%%%%%%%%%%%
\subsection{High Level Summary}

    \begin{itemize}

    \item Introduction: definition of the Fuzzy sphere and introduction of the $3$ notions that gives meaning to ``the Fuzzy Sphere converges to $S^2$'': as *-algebras, as representations of $SU(2)$ and as metric spaces.
    
    \item Background definitions: spectral triple definitions, distance, compact quantum metric space.
    
    \item Detailed construction and analysis of spectrum of the canonical Dirac operator on $S^2$ through the approach that will be replicated(?) for the Fuzzy Sphere.
    
    \item Construction of $2$ $SU(2)$ invariant spectral triples on $\mathcal A_N$, the eigenvalues of the Dirac operators, and proposition that shows that these $2$ triples generate the same metric on the state of states of $\mathcal A_N$.
    
    \item Quick overview of other $3$ spectral triples on the Fuzzy Sphere that begin with a chirality operator and then find an anticommuting self-adjoint Dirac-like operator with a plausible spectrum.
    
    \item Introduce a family of pairs of pure states in the state space $\mathcal S(\mathcal A_N)$: (Bloch) $SU(2)$-coherent states on $\mathcal A_N$: $\psi^N_{(\phi, \theta)}$
    
    \item $N = 1$ case: all pure state are coherent states and an explicit formula for the distance is found.
    
    \item Distance between basis vector states associated to $|j, m\rangle$ explicitly found for all $N$.
    
    \item An auxiliary distance $\rho_N$, which will be a lower bound for $d_N$ is introduced: distance along diagonal matrices of $\mathcal A_N$.
    
    \item The distance is $SU(2)$-invariant
    
    \item The distance between $2$ states associated to the same points $p, q \in S^2$ is non-decreasing with $N$.
    
    \item In general:
    \begin{equation}
        \rho_N(\theta - \theta') \leq d_N(\psi^N_{(\phi, \theta)}, \psi^N_{(\phi', \theta')}) \leq d_{geo}((\phi, \theta), (\phi', \theta'))
    \end{equation} and, finally,
    \begin{equation}
        \lim_{N \to \infty} d_N(\psi^N_{(\phi, \theta)}, \psi^N_{(\phi', \theta')}) = d_{geo}(\psi^N_{(\phi, \theta)}, \psi^N_{(\phi', \theta')})
    \end{equation}
    
    \end{itemize}

%%%%%%%%%%%%%%%%%%%%%%%%%%%%%%%%%%%%%%%%%%%%%%%%%%%%%%%%%%%%%%%%%%%%%%%%%%%%%
\subsection{Very Important Facts} % Everything written here is also written somewhere else

    \begin{itemize}
    
    \item \rtext{\Huge Summary: Start from $SU(2)$-equivariance (``equivariant decomposition?'') of the commutatative $\slashed D$ and $\mathcal H = L^2(S^2) \otimes \CC^2$; the desired: ``truncated spectral triple'' comes from truncating at $N = 2j$.}
    
    \item \rtext{Why Fuzzy Spaces?}: To preserve the symmetries and keep the algebra finite dimensional.
    
    
    \item \rtext{Why is $\mathcal A_N = M_{N+1}(\CC)$?}: what Madore did was actually defining $A_N := \bigoplus_{l = 0, 1, \dots}^N V_l$ where $V_j$ is the irrep. of $SO(3)$ of integer spin $l$ (of homogeneous polynomials in $x_1, x_2, x_3$ of degree $l$), since this is a \rtext{truncation of the commutative algebra $L^2(S^2) \cong \bigoplus_{l\in \NN} V_l$ at $N = 2j$} compatible with the action of $SU(2)$, but this truncation is not an algebra; the algebra structure is then imposed to be that of $M_{N+1}(\CC)$, which comes from \rtext{changing the algebra generators $x_k \mapsto \hat x_k = \frac{1}{\sqrt{j(j+1)}}\pi_j(J_k) \in M_{N+1}(\CC)$ ($[J_i, J_j] = i\epsilon J_k$ generate $i\sut = sl(2, \CC)$, i.e. $J_k = \frac{\sigma_k}{2}$)} (i.e. basically changing the $x_k$ by the rotation operator of spinors of spin $j$ w.r.t. this axis); this, is equivalent to changing the vector spaces generators $Y_{lm}$ by some fuzzy spherical harmonics \rtext{$Y_{lm} \mapsto \hat Y_{lm} \in M_{N+1}(\CC)$}; the action of $SU(2)$ that \rtext{makes this map $A_N \to M_{N+1}(\CC)$ an equivalence of representations / $su(2)$-module / ``keep the $SU(2)$ symmetry''} is the adjoint action $Ad \pi_l : G \to End(M_{N+1}(\CC))$, $Ad \pi_l(g)(M) = \pi_l(g) M \pi^*_l(g)$... so might as well say that $\mathcal A_N = M_{N+1}(\CC)$ as a unital $C^*$-algebra AND as a representation of $su(2)$/a $U(su(2))$-module.
    
        \begin{itemize}
            
        \item The precise way We will understand in this paper is as $\mathcal A_N := End(V_j) = \mathcal B(\mathcal V_j)$, and $V_j = \CC^{N+1}$ where the canonical basis correspond to the angular momentum basis $e_m = \set{|j, m\rangle}_{m = -j, \dots, j}$ {\tiny OJO: $\CC^{N+1}$ is not understood as ``$e_m$ = how much of $z_1$ there is''}
        
        \item That ``the $SU(2)$ symmetry'' is kept can be deduced from the equations prior to 3.6, which tell us that $Y_{lm} \mapsto \hat Y_{lm}$, the basis of this algebra, is indeed equivariant.
            
        \end{itemize}
    
    \item \rtext{GOAL}: The fuzzy sphere approximates the: \textbf{1. $U(\alg{su}(2))$-module (i.e. $\alg{su}(2)$ Lie algebra representation) 2. *-algebra 3. metric space} $\mathcal A(S^2) \cong L^2(S^2)$ of complex polynomials in $x_1, x_2, x_3$.
    
    \item \rtext{The notion of convergence of this fuzzy space} to $S^2$ must also involve metrics: $d_N$ in $\mathcal A_n$ and $d_{geo}$ in $S^2$: in this paper the points of $S^2$ are idetified with the corresponding (Block) coherent states of $\mathcal A_N$, and under this identification:
    \begin{equation}
        \lim_{N \to \infty} d_N(p, q) = d_{geo}(p, q)\quad \text{for all }p, q \in S^2
    \end{equation}
    
    %\item \rtext{How is the Dirac operator proposed here?}: start from the $SU(2)$-equivariance and arrive at a neat \textbf{truncation of the classical spectrum}, paying the price of spectral asymmetry.
    
    \item The Bloch $SU(2)$-\rtext{coherent states} are for the group $SU(2)$ what the usual harmonic oscillator coherent states are for the Heisenberg group. In particular, they are minimum uncertainty states.
    
    \item \rtext{What kind of spectral triple do we want?}: \rtext{find an ``\textbf{$SU(2)$-equivariant}'' spectral triple $\supset$ operator $D_N$ on $\mathcal A_N$ whose spectrum is that of $\slashed D: L^2(S^2) \otimes \Sigma_2 \to \slashed D: L^2(S^2) \otimes \Sigma_2$ \textbf{truncated} at $N = 2j$}: ``truncation of the spectral triple''. 
    
    \item \dbtext{Equivariance of a spectral triple} \rtext{ ~ ``isometries'' of the noncommutative manifold} $\Longleftrightarrow$ Hopf algebra: I THINK, TODO the action of the $SU(2)$ group / rotations is an isometry of the Fuzzy Sphere... I can be satisfied for now with a mediocre understanding of the definition of an \rtext{equivariant Dirac element $\mathcal D \in U(\alg g) \otimes U(\alg g)$}: it commutes with the range of the coproduct of $U(\alg g)$
    
    \item \rtext{How is the desired spectral triple obtained here?}: accompanying the truncation of the irrep decomposition $L^2(S^2) \cong \sum_{l \in \NN} V_l$ at $N = 2j$ %and mapping its generators \rtext{$Y_{lm}$ into the fuzzy spherical harmonic  $\hat Y_{lm}$} in an $su(2)$-equivariant way. 
    , knowing that the commutative \rtext{$\slashed D$ comes from the purely algebraic \textit{Dirac element} $\mathcal D \in U(su(2)) \otimes U(su(2))$ which is ``$SU(2)$-equivariant''} {\tiny(it commutes with the coproduct of the Hopf algebra $U(\sut)$)},\rtext{truncate $\slashed D = \oplus_{l \in \NN} (\pi_l \otimes \pi_{1/2})(\mathcal D)$ at $N = 2j$} to define a new Dirac operator $\mathcal D_N: \mathcal A_N \otimes \CC^2 \to \mathcal A_N \otimes \CC^2$ which is ``$SU(2)$-equivariant'' (and the esp. trip. is real, but not even). This \textbf{can be seen to come from creating an auxiliary spectral triple where the Dirac operator is ``a single term'' of $\slashed D$}: $\rtext{D_N = (\pi_j \otimes \pi_{1/2})(\mathcal D)}: V_j \otimes \CC^2 \to V_j \otimes \CC^2$, which turns out to induce precisely the same notion of distance.
    
        \begin{itemize}
        
            \item Use as starting point the view of $S^2$ as the symmetric space $S^3/S^1$ of the compact semisimple Lie group $S^3$.
        
                \begin{itemize}
                    
                \item The theory of this groups tells us that their canonical spectral triple can be seen to come from a purely algebraic element $\mathcal D \in U(\alg g) \otimes U(\alg g)$ which becomes at the end an operator on the sections of the spinor bundle through a couple of steps:
                \begin{align}
                    U(\alg g) \otimes U(\alg g) &\to& U(\alg g) \otimes Cl(\alg g, -Killing) &\to& \mathcal B(L^2(G, SG)) \\
                    \mathcal D  &\mapsto& \mathcal D_S &\mapsto& \slashed D \\
                    \rtext{1 \otimes 1 + 2 \sum_{k = 1}^3 J_k \otimes J_k} &\mapsto& 1 \otimes 1 + \sum_k J_k \otimes \sigma_k &\mapsto& \left( \bigoplus_{l\in \NN} \pi_l \right) \otimes \pi_{1/2}(\mathcal D)
                \end{align}
                
                \end{itemize}
            
            \item A first spectral triple is simply taking $D_N = (\pi_j \otimes \pi_{1/2})(\mathcal D): V_j \otimes \CC^2 \to V_j \otimes \CC^2 = $ to be \textbf{``a term'' of $\slashed D$}.
            
            \item From it create a second spectral triple, which is real, which has $\mathcal D_N = (\textbf{ad}\pi_j \otimes \pi_{1/2})(\mathcal D): \mathcal A_N \otimes \CC^2 \to \mathcal A_N \otimes \CC^2$ where $\mathcal A_N$ is seen as $V_J \otimes V_j^*$
            
        \end{itemize}
    
    \item Points of the $(\phi, \theta) \in S^2$ sphere are approximated by coherent vectors $|\phi, \theta)_N \in V_J \Longleftrightarrow $ states of $\mathcal A_N$ $\psi^N_{\phi, \theta} \in \mathcal S(\mathcal A_N)$. \rtext{This identification of points in $S^2$ with coherent states is $SU(2)$-equivariant}. (This does not seem to be saying that the points \textbf{of} the noncommutative space are these pure states, it seems to be more like these states are \textbf{noncommutative versions of the points of $S^2$}).
    
    \item Apparently, by construction the action on the coherent states of $SU(2)$ $\psi^N_{(\phi, \theta)} \mapsto g_*\psi^N_{(\phi, \theta)}:(a \mapsto \psi^N_{(\phi, \theta)}(a^g))$ ``corresponds to the usual rotation action of $SU(2)$ on $S^2$'' which I understand to mean that this action is the same as the action $\psi^N_{(\phi, \theta)} \mapsto \psi^N_{g(\phi, \theta)}$
    
    \item The distance between any $2$ $SU(2)$-coherent states is $SU(2)$-invariant.
    
    \item In summary, what is $SU(2)/su(2)$-equivariant?:
    
        \begin{itemize}
            
        \item Understanding $A_N$ as $M_{N+1}(\CC)$
        
        \item The Dirac operators
        
        \item The spectral triples
        
        \item The identifications $S^2 \to V_j$ and $S^2 \to S(\mathcal A_N)$. 2 ways it is expressed:
        
            \begin{itemize}
                
            \item Lemma 4.2: $su(2)$
            
            \item $\psi^N_{(\phi, \theta)} \mapsto g_* \psi^N_{(\phi, \theta)}$    
            
            \end{itemize}
        
        \item The distance between coherent states
            
        \end{itemize}
    
    \end{itemize}

%%%%%%%%%%%%%%%%%%%%%%%%%%%%%%%%%%%%%%%%%%%%%%%%%%%%%%%%%%%%%%%%%%%%%%%%%%%%%
\subsection{Important Facts}

    \begin{itemize}

    \item The $\otimes \pi_{1/2}$ is a way to state that we will see what this is acting on as \textbf{an operator on bispinors}(or something similar, since it can't be exactly that as the found spectral triples are not even), e.g. $(\pi_j \otimes \pi_{1/2})(\mathcal D): \CC^{N+1}\otimes \CC^2 \to \CC^{N+1}\CC^2$, $\begin{pmatrix} \psi_1 \in \CC^{N+1} \\ \psi_2 \in \CC^{N+1} \end{pmatrix} \in \CC^{N+1}\otimes \Sigma_2$.
    
    \item The equivariance of the spectral triples come from the commuting properties of the algebraic Dirac operators with the range of the coproduct of $U(\alg g)$
    
    \item These Bloch coherent states are for the group $SU(2)$ what the usual/canonical coherent states are for the Heisenberg group / Harmonic oscillator. In particular, they are minimum uncertainty states.
    
    \end{itemize}

%%%%%%%%%%%%%%%%%%%%%%%%%%%%%%%%%%%%%%%%%%%%%%%%%%%%%%%%%%%%%%%%%%%%%%%%%%%%%
\subsection{Memorize}

    \begin{itemize}

    \item 
    
    \end{itemize}

%%%%%%%%%%%%%%%%%%%%%%%%%%%%%%%%%%%%%%%%%%%%%%%%%%%%%%%%%%%%%%%%%%%%%%%%%%%%%
\subsection{Doubts}

    \begin{itemize}

    \item 
    
    \end{itemize}

%%%%%%%%%%%%%%%%%%%%%%%%%%%%%%%%%%%%%%%%%%%%%%%%%%%%%%%%%%%%%%%%%%%%%%%%%%%%%
\subsection{Detailed summary}

    \begin{itemize}

    \item 
    
    \end{itemize}

%%%%%%%%%%%%%%%%%%%%%%%%%%%%%%%%%%%%%%%%%%%%%%%%%%%%%%%%%%%%%%%%%%%%%%%%%%%%%
\subsection{Notice}

    \begin{itemize}

    \item The NC Quantum Mechanics seems to easily give Dirac operators $\gamma^\mu D_\mu$ since there are momentum operators $\sigma^\mu \hat P^\mu$
    
    \end{itemize}

%%%%%%%%%%%%%%%%%%%%%%%%%%%%%%%%%%%%%%%%%%%%%%%%%%%%%%%%%%%%%%%%%%%%%%%%%%%%% 
\subsection{Yet to understand}

    \begin{itemize}

    \item How are the view of $\mathcal A_N$ as $\bigoplus^N V_j$ and as $V_{N} \otimes V_N^*$ related?
    
    \item Why do we ask Dirac operators to have compact resolvent?
    
    \item Why is it nice for Dirac triples to be real?
    
    \item How is the equivariance of the metric spaces carried into the definition of the spectral triples? I don't see how these equivariance is something that determines the definition of the spectral triples, as it is claimed.
    
    \end{itemize}

%%%%%%%%%%%%%%%%%%%%%%%%%%%%%%%%%%%%%%%%%%%%%%%%%%%%%%%%%%%%%%%%%%%%%%%%%%%%%
%%%%%%%%%%%%%%%%%%%%%%%%%%%%%%%%%%%%%%%%%%%%%%%%%%%%%%%%%%%%%%%%%%%%%%%%%%%%%
\section{Introduction (pg. 1)}

    \begin{itemize}
        
    \item Why Fuzzy Spaces?: To preserve the symmetries and keep the algebra finite dimensional.    
    
    \item GOAL: The fuzzy sphere approximates the: \textbf{1. $U(\alg{su}(2))$-module (i.e. $\alg{su}(2)$ Lie algebra representation) 2. *-algebra 3. metric space} $\mathcal A(S^2) \cong L^2(S^2)$ of complex polynomials in $x_1, x_2, x_3$.
        
    \item Definition of the $*$-algebra and $\alg{su(2)}$-representation $\mathcal A_N$. With respect to these $2$ properties, $\lim_{n \to \infty}$ of these spaces is indeed $\mathcal A(S^2)$, but the third property for real convergence is still missing.
    
        \begin{itemize}
            
        \item Parting from the decomposition of in irreps of $\alg{su}(2)$ of $\mathcal{A}(S^2)$ as $ = \bigoplus_{l =  0, 1, \dots}^\infty V_l$
        
            \begin{itemize}
                
            \item where $V_l$ is homogeneous polynomials of degree $l$
                
            \end{itemize}
        
        \item Introduce cut-off in the energy spectrum: keep only the first $N =: 2j$ representations: it works as a $\sut(2)$ representation but not as an algebra.
        
        \item To make it an algebra, pullback the algebra structure from the right of the following vector space isomorphism... actually, in this article the RHS is taken to BE the $\mathcal A_N$:
        \begin{align}
            \bigoplus_{l =  0, 1, \dots}^{2j} V_l \subset \mathcal{A}(S^2) &\to \mathcal A_N := M_{N+1}(\CC)\\
            x_k &\mapsto \hat x_k := \frac{1}{\sqrt{j(j+1)}}\pi_j(J_k)
        \end{align}
        
            \begin{itemize}
                
            \item $[J_i, J_k] = i \epsilon_{ijk} J_k$ are the standard real generators of $\sut$
            
            \item $\pi_j:\sut\text{ o }SU(2) \to M_{N+1}(\CC)$ is the irrep of spin $j \in \frac{\NN}{2}$
                
            \end{itemize}
        
        \item The *-algebra is made into a $\sut$ representation by pushing forward the representation of the above map, i.e. making it an isomorphism of representations of $\sut$ / $SU(2)$ TODO
        
            \begin{itemize}
                
            \item This action of $SU(2)$ is equivalent to the action of $U(\sut)$ given by:
            
            \end{itemize}
        
        \item The above definitions imply that $g_{ij}\hat x_i \hat x_j = 1 = \hat x_1^2 + \hat x_2^2 + \hat x_3^2$. This means, in particular, that:
        
        \begin{equation}
            [\hat x_k, \hat x_l] = \frac{1}{\sqrt{j(j+1)}} i \epsilon_{klm}\hat x_m
        \end{equation}
        
        \end{itemize}
    
    \item The notion of convergence of this fuzzy space to $S^2$ must also involve metrics: $d_N$ in $\mathcal A_n$ and $d_{geo}$ in $S^2$: in this paper the points of $S^2$ are idetified with the corresponding (Block) coherent states of $\mathcal A_N$, and under this identification:
    \begin{equation}
        \lim_{N \to \infty} d_N(p, q) = d_{geo}(p, q)\quad \text{for all }p, q \in S^2
    \end{equation}    
    
    \end{itemize}
    
\subsection*{Definitions}

    \begin{itemize}

    \item 
    
    \end{itemize}

%%%%%%%%%%%%%%%%%%%%%%%%%%%%%%%%%%%%%%%%%%%%%%%%%%%%%%%%%%%%%%%%%%%%%%%%%%%%%
%%%%%%%%%%%%%%%%%%%%%%%%%%%%%%%%%%%%%%%%%%%%%%%%%%%%%%%%%%%%%%%%%%%%%%%%%%%%%
\section{Preliminaries on noncommutative manifolds (pg. 3)}

\begin{itemize}
        
    \item 
        
    \end{itemize}
    
\subsection*{Definitions}

    \begin{itemize}

    \item 
    
    \end{itemize}

%%%%%%%%%%%%%%%%%%%%%%%%%%%%%%%%%%%%%%%%%%%%%%%%%%%%%%%%%%%%%%%%%%%%%%%%%%%%%
%%%%%%%%%%%%%%%%%%%%%%%%%%%%%%%%%%%%%%%%%%%%%%%%%%%%%%%%%%%%%%%%%%%%%%%%%%%%%
\section{Dirac operators for the fuzzy sphere (pg. 4)}

\begin{itemize}
        
    \item 
        
    \end{itemize}
    
\subsection*{Definitions}

    \begin{itemize}

    \item 
    
    \end{itemize}


%%%%%%%%%%%%%%%%%%%%%%%%%%%%%%%%%%%%%%%%%%%%%%%%%%%%%%%%%%%%%%%%%%%%%%%%%%%%%
\subsection{An abstract Dirac operator (pg. 4.5)}

%%%%%%%%%%%%%%%%%%%%%%%%%%%%%%%%%%%%%%%%%%%%%%%%%%%%%%%%%%%%%%%%%%%%%%%%%%%%%
\subsection{The Dirac operator of $S^2$ (pg. 5.3)}

%%%%%%%%%%%%%%%%%%%%%%%%%%%%%%%%%%%%%%%%%%%%%%%%%%%%%%%%%%%%%%%%%%%%%%%%%%%%%
\subsection{Dirac operators on the fuzzy sphere (pg. 5.6)}

%%%%%%%%%%%%%%%%%%%%%%%%%%%%%%%%%%%%%%%%%%%%%%%%%%%%%%%%%%%%%%%%%%%%%%%%%%%%%
\subsection{Comparisons with the literature (pg. 9)}

%%%%%%%%%%%%%%%%%%%%%%%%%%%%%%%%%%%%%%%%%%%%%%%%%%%%%%%%%%%%%%%%%%%%%%%%%%%%%
%%%%%%%%%%%%%%%%%%%%%%%%%%%%%%%%%%%%%%%%%%%%%%%%%%%%%%%%%%%%%%%%%%%%%%%%%%%%%
\section{Spectral distance between coherent states (pg. 11)}

%%%%%%%%%%%%%%%%%%%%%%%%%%%%%%%%%%%%%%%%%%%%%%%%%%%%%%%%%%%%%%%%%%%%%%%%%%%%%
\subsection{The $N = 1$ case (pg. 12)}

%%%%%%%%%%%%%%%%%%%%%%%%%%%%%%%%%%%%%%%%%%%%%%%%%%%%%%%%%%%%%%%%%%%%%%%%%%%%%
\subsection{Distances between basis vectors (pg. 13)}

%%%%%%%%%%%%%%%%%%%%%%%%%%%%%%%%%%%%%%%%%%%%%%%%%%%%%%%%%%%%%%%%%%%%%%%%%%%%%
\subsection{An auxiliary distance (pg. 14)}

%%%%%%%%%%%%%%%%%%%%%%%%%%%%%%%%%%%%%%%%%%%%%%%%%%%%%%%%%%%%%%%%%%%%%%%%%%%%%
\subsection{$SU(2)$-invariance of the distance (pg. 15)}

%%%%%%%%%%%%%%%%%%%%%%%%%%%%%%%%%%%%%%%%%%%%%%%%%%%%%%%%%%%%%%%%%%%%%%%%%%%%%
\subsection{Dependence on the dimension (pg. 15.5)}

%%%%%%%%%%%%%%%%%%%%%%%%%%%%%%%%%%%%%%%%%%%%%%%%%%%%%%%%%%%%%%%%%%%%%%%%%%%%%
\subsection{Upper and lower bounds and the large $N$ limit (pg. 17-18)}

\end{document}
