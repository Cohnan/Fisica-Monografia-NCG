\section{Further Work}

\begin{frame}{Other Spectral Triples} % % % % % % % % % % % % % % % % % % %

% Definition Moyal Plane

% 3 mentioned in D'Andrea

% Summary I did for Reyes

% End of Fiore2018

Moyal Plane Spectral triples: 
    \begin{itemize}
    
    \item Deformation of $\RR^2$, proposed in \cite{Gayral2004}, studied in \cite{Martinetti2013}, \cite{Cagnache2011}.
    
    \item NCQM, proposed in \cite{Scholtz2013}, studied in \cite{ChaobaDevi2018}.
    
    \end{itemize}

Fuzzy Sphere Spectral triples:
    \begin{itemize}
        
    \item Deformation of canonical s.t. of $S^2$ proposed in \cite{Grosse1995}, studied in \cite{Devi2015}, \cite{ChaobaDevi2018}.
    
    \item Dirac operator as the odd part of a truncated superfield on a supersphere \cite{Grosse1997}. Same distance as here.
    
    \item ``Spin-$j$ Dirac operator'' from spin $j$ representation of the abstract Dirac element \cite{Balachandran2007}, \cite{Balachandran2009}.
    
    \item Start with chirality operator and then find anticommuting self-adjoint operator with plausible spectrum \cite{Carow1997}, \cite{Carow1998}.
        
    \end{itemize}


\end{frame}


\begin{frame}{About New Fuzzy Spheres} % % % % % % % % % % % % % % % % % % %
- So far: \cite{Fiore2018}, \cite{Fiore2019}.

- ``Truncated Dirac Operator'' described in \cite{DAndrea2014} from applying an energy cutoff. Irreducible Dirac operator recovered with this procedure.

- Coherent states, specially in new Fuzzy Spheres: \cite{FioreCoherent2020}, \cite{FioreTheCase2020}.
\end{frame}

\begin{frame}{Possible Table of Contents}
    \tiny
    \begin{enumerate}
    
    \item Introduction: Preliminary Concepts of Noncommutative Geometry and Motivation for NCG and New Fuzzy Spheres: physics ($C^*$-algebras as topological spaces, canonical spectral triple, spectral triples and Connes' distance formula)
    
    \item The Fuzzy Sphere and its Metric Properties (using spectral triples of \cite{DAndrea2013}).
    
        \begin{enumerate}
        \tiny
        
        \item Canonical Spectral triple on $S^2$ (arising from $SU(2)$ equivariant algebraic Dirac element).
        
        \item The Fuzzy Sphere: fuzzy and $SU(2)$-equivariant approximation of $S^2$.
        
        \item Irreducible and Full Spectral Triples
        
        \item $SU(2)$-coherent states: fuzzy approximations of points in $S^2$
        
        \item Distances between coherent states and Convergence of fuzzy sphere to $S^2$ as metric spaces.
        
        \end{enumerate}
    
    \item The New Fuzzy Spheres of Fiore and Pisacane
    
        \begin{enumerate}
        \tiny
        
        \item General Setting
        
        \item Fuzzy Quantum Mechanics in $S^1$
        
            \begin{enumerate}\tiny
                
            \item Definition of the fuzzy space $\{S^1_\Lambda\}$
            
            
            \item Realization of the fuzzy space as representation of $Uso(3)$
            
            \item Convergence to quantum mechanics in $S^1$
            
            \end{enumerate}
        
        \item Fuzzy Quantum Mechanics in $S^2$
            
            \begin{enumerate}\tiny
                
            \item Definition of the fuzzy space $\{S^2_\Lambda\}$
            
            
            \item Realization of the fuzzy space as representation of $Uso(4)$
            
            \item Convergence to quantum mechanics in $S^2$
            
            \end{enumerate}
            
        \end{enumerate}
    
    \end{enumerate}
\end{frame}

% \begin{frame}{Table of Contents (Cont.)}
%     \begin{enumerate}
%     \tiny
    
%     \item Systems of Coherent States in the New Fuzzy Spheres
        
%         \begin{enumerate} \tiny
        
%         \item Basics on Coherent States
           
%         \item Coherent and Localized states on the new fuzzy circle
        
%         \item Coherent and Localized states on the new fuzzy sphere 
            
%         \end{enumerate}
%     \item Study of Distance between Systems of Coherent States in the New Fuzzy Spheres
    
%         \begin{enumerate} \tiny
            
%         \item Proposed Dirac Operator and Spectral Triple
            
%         \end{enumerate}
        
%     \end{enumerate}
      
% \end{frame}