%%%%%%%%%%%%%%%%%%%%%%%%%%%%%%%%%%%%%%%%%%%%%%%%%%%%%%%%%%%%%%%%%%%%%%%%%%%%%%%
%%%%%%%%%%%%%%%%%%%%%%%%%%%%%%%%%%%%%%%%%%%%%%%%%%%%%%%%%%%%%%%%%%%%%%%%%%%%%%%
\section{Two $SU(2)$-equivariant distances/spectral triples on the Fuzzy Sphere}

%%%%%%%%%%%%%%%%%%%%%%%%%%%%%%%%%%%%%%%%%%%%%%%%%%%%%%%%%%%%%%%%%%%%%%%%%%%%%%%
\subsection{Spectral Triples}

\begin{frame}{Canonical Spectral Triple of $S^2$} % % % % % % % % % % % % % % % % % % %

% Inherits from $\RR^3$ the metric and the metric connection, but the spinor space changes

\rtext{Starting point: $SU(2)$-isometries}%\cite{DAndrea2013}
- $S^2$ as the symmetric space $S^3/S^1$ of the compact semisimple Lie group $G = S^3$, $\mathfrak g = su(2)$.
    
- The canonical spectral triple, which is \textbf{$SU(2)$-equivariant} can be seen to come from a purely algebraic element $\lbtext{\mathcal D} \in U(\mathfrak g) \otimes U(\mathfrak g)$:
    \begin{align*}
        U(\mathfrak g) \otimes U(\mathfrak g) &\to& U(\mathfrak g) \otimes Cl(\mathfrak g, -K) &\to& \mathcal B(L^2(G/U, \Sigma G/U)) \\
        1 \otimes 1 + 2 \sum_{k = 1}^3 J_k \otimes J_k &\mapsto& 1 \otimes 1 + \sum_k J_k \otimes \sigma_k &\mapsto& \rtext{\left( \bigoplus_{l\in \bb N} \pi_l \right) \otimes \pi_{1/2}(\mathcal D)}
    \end{align*}
    
%- In Sanchez: $\mu = 1, 2$
% $\nabla_\mu = \partial_\mu - \frac{c_\mu}{q} \sigma_\mu \sigma_{\cut \mu} \cdot$, 
% $\slashed D = -i \frac{q}{2} \sigma^\mu \nabla_\mu

- So $\lbtext{\slashed D} = \begin{pmatrix} 1 + \partial_H & \partial_F \\ \partial_E & 1 - \partial_H\end{pmatrix} = 1 + \partial_F \otimes \sigma_+ + \partial_E \otimes \sigma_- + \partial_H \otimes \sigma_3$ where $\partial_H = -i \partial_\phi$, $\partial_F = e^{i\phi} \left( \partial_\theta + i cot\,\theta \partial_\phi \right)$, $\partial_E = -\partial_F% = e^{-i\phi} \left( \partial_\theta - i cot\,\theta \partial_\phi \right)
$ are the actions of $J_3$, $J^\pm \in i\,su(2)$ on $L^2(S^2)$ respectively. \textit{Eig. vectors}: orth. basis of $\hcal$; \textit{Spectrum} $= \{\pm l\} = \ZZ - 0$ with multiplicities $2l$.%: L^2(S^2) \otimes \CC^2 \to L^2(S^2) \otimes \CC^2 = \oplus_{l \in \bb N} (\pi_l \otimes \pi_{1/2})(\mathcal D)$

% - The eigenvectors (spinor harmonics) $Y^{'}_{lm}, Y^{''}_{lm}$, $l \in \NN + 1/2$, of $D$ make up an orthogonal basis of the spinor fields $\hcal$.
% - The eigenvalues are: of $Y'_{l\cdot}: (l + 1/2) \in \NN$, of $Y{''}_{l\cdot}: -(l + 1/2)$, with multiplicities $2l+1$
\end{frame}

\begin{frame}{The Irreducible Spectral Triple} % % % % % % % % % % % % % % % % % % %
A first spectral triple is simply taking ``one term'' of $\slashed D$:
\begin{multline}
    \lbtext{D_N} 
    := (\pi_j \otimes \pi_{1/2})(\mathcal D): V_j \otimes \CC^2 \to V_j \otimes \CC^2 \\
    = \begin{pmatrix} 1 + \pi_j(H) & \pi_j(F) \\ \pi_j(E) & 1 - \pi_j(H)\end{pmatrix} 
    = 1 + \pi_j(F) \otimes \sigma_+ + \pi_j(E) \otimes \sigma_- + \pi_j(H) \otimes \sigma_3
\end{multline} where $H = J_3$, $F = J_+$, $E = J_-$ are the actions of $J_3$, $J^\pm \in su(2)$ on $V_j$ respectively.

- The spectral triple \rtext{$(\acal_N, \lbtext{H_N} = V_j \otimes \CC^2, D_N)$}
    \begin{itemize}
        
    \item Is $SU(2)$-equivariant
    
    \item Has eigenvalues $j+1$ and $-j$ with multiplicities $2j+2$ and $2j$.
        
    \item Isn't compatible with a grading or a real structure.
    \end{itemize}

\end{frame}

\begin{frame}{The Full Spectral Triple} % % % % % % % % % % % % % % % % % % %
\rtext{$(\acal_N, \lbtext{\hcal_N} := \mathcal A_N \otimes \CC^2 \cong \bigoplus_{l =1}^N H_l, \mathcal D_N \longleftrightarrow \bigoplus_{l =1}^N D_l)$}, where:
\begin{multline}
    \lbtext{\mathcal D_N} := (\textbf{ad}\pi_j \otimes \pi_{1/2})(\mathcal D) = D = \begin{pmatrix} 1 + \text{ad}\pi_j(H) & \text{ad}\pi_j(F) \\ \text{ad}\pi_j(E) & 1 - \text{ad}\pi_j(H)\end{pmatrix} 
    %= \text{ad}\pi_j(E) \otimes \sigma_1 + \text{ad}\pi_j(F) \otimes \sigma_2 + \text{ad}\pi_j(H) \otimes \sigma_3
\end{multline}
where, e.g. $\text{ad}\pi_j(H) = [\pi_J(H), \cdot ]$ is the action of $J_3 \in su(2)$ on $\mathcal A_N$.

    \begin{itemize}
    
    \item It is a real spectral triple.
    
    \item It is $SU(2)$-equivariant
    
    \item Spectrum of $\mathcal D_N$ is the truncation of $\slashed D$ to $\{-N, \dots, N+1\}$.%The eigenvalues of $\mathcal D_N$ are $N+1$ with multiplicity $2N+2$, and $\pm l$ with multiplicity $2l$ for $l = 1, \dots, N$
    
    \item It is not compatible with a grading.
    
    \end{itemize}

\textbf{It is a truncation of the canonical spectral triple}.

- \rtext{\textbf{Theorem}}: the two spectral triples induce the same distance in $\acal_N$. Pf: $[\dcal, a]b \otimes v = \cdots = \sum_k [\pi_j(J_k), a]b \otimes \sigma_k v = [D_N, a] \cdot b \otimes v$
\end{frame}

\begin{frame}{$SU(2)$-invariance of the distance} % % % % % % % % % % % % % % % % % % %
$SU(2) \ni g$ acts on the states: $\lbtext{g_*\omega}(\cdot) := \omega(\cdot^g)$ $\longleftarrow$ it acts on the algebra: $\lbtext{a^g}:= g\circ a \circ g^* \cdot$ $\longleftarrow$ it acts on $V_j$: $\pi_j(g)$.

\textbf{\rtext{Theorem}}: For all $N = 2j \in \bb N$, the distance is $SU(2)$-invariant: 
\begin{align}
    d_N(g_* \omega, \omega') &= d_N(g_* \omega, g_*\omega'), & \text{for all $\omega, \omega' \in \mathcal S(\acal_N)$}.
\end{align}

\textit{Pf}: The theorem follows once we show 
    \rtext{$||[D_N, a^g]|| = ||[D_N, a]||$}:
$d_N(g_* \omega, g_* \omega') = sup_{a \in \acal_N}\{ |\omega(a^g) - \omega'(a^g)|: ||[D_N, a \otimes 1_2]|| \leq 1 \} = sup_{b \in \acal_N}\{ |\omega(b) - \omega'(b)|: ||[D_N, b \otimes 1_2]|| \leq 1 \} = d_N(\omega, \omega')$, where $b = a^g$ sweeps all $\acal_N$.

\textit{Pf}: $[D_N, a^g \otimes 1] = u[D_N, a \otimes 1_2]u^*$ where $u = \pi_j(g) \otimes \pi_{1/2}(g)$ is the induced unitary action of $g$ in $H_N$ $\xleftarrow{}$
1. $a^g \otimes 1_2 = u(a \otimes 1_2)u^*$;
2. The spectral triple is $SU(2)$-equivariant, hence $D_N$ commutes with the $SU(2)$ action.
\end{frame}

%%%%%%%%%%%%%%%%%%%%%%%%%%%%%%%%%%%%%%%%%%%%%%%%%%%%%%%%%%%%%%%%%%%%%%%%%%%%%%%
\subsection{Coherent States}

\begin{frame}{SHO Coherent States} % % % % % % % % % % % % % % % % % % %

\end{frame}

\begin{frame}{(Bloch) $SU(2)$-coherent states on $\acal_N$} % % % % % % % % % % % % % % % % % % %
- The Bloch/$SU(2)$-\rtext{coherent states} are for the group $SU(2)$ what the usual harmonic oscillator coherent states are for the Heisenberg group. In particular, they are minimum uncertainty states. 

%- \rtext{They will be considered fuzzy approximations of the points of $S^2$}.

- \rtext{Points of the $E(\phi, \theta) \in S^2$ sphere are approximated by coherent vectors} $\lbtext{|\phi, \theta)_N} := R_{(\phi, \theta)}|j, -j\rangle \in V_j \Longleftrightarrow $ states of $\mathcal A_N = End(V_j)$ $\lbtext{\psi^N_{(\phi, \theta)}} = (\phi, \theta| \cdot |\phi, \theta)_N \in \mathcal S(\mathcal A_N)$. 

- \rtext{This identification of points in $S^2$ with coherent states %($\psi^N: \acal_N \to L^2(S^2)$ at the algebra level) ???
is $SU(2)$-equivariant}: $g_* \psi^N_{(\phi, \theta)} = \psi^N_{g\cdot (\phi, \theta)}$, \& \rtext{the distance between them is $SU(2)$-invariant}.

%What group used in Fiore?
\end{frame}

%%%%%%%%%%%%%%%%%%%%%%%%%%%%%%%%%%%%%%%%%%%%%%%%%%%%%%%%%%%%%%%%%%%%%%%%%%%%%%
\subsection{Distance between $SU(2)$-coherent states}

\begin{frame}{Generalities} % % % % % % % % % % % % % % % % % % %
From now on: we use the irreducible s.t. and not write the $\pi_j$'s.

- The supremum $d_D(\omega, \omega')$ is always attained in hermitian elements.

- For $a \in \acal_N$, 
\begin{equation} \label{ineqDN}
    ||[H, a]||, ||[E, a]||, ||[F, a]||  \leq ||[D_N, a]||
\end{equation} \label{eqDNdiag}
If $a$ is diagonal hermitian, then
\begin{equation}
    ||[E, a]|| = ||[D_N, a]||.
\end{equation}

- A state may be defined only on hermitian elements of $\acal$, since from there it can be uniquely extended to all $\acal$: $a = \frac{a+a*}{2} + \frac{a - a*}{2} \in Herm. + Antiherm. = Herm. + i\, Herm.$.

\end{frame}

\begin{frame}{The $N = 1$ case} % % % % % % % % % % % % % % % % % % %
% $\pi_{1/2}(J_k) = \sigma_k/2$

- Any hermitian element in $\acal_1 = M_2(\CC)$ can be written as
\begin{align*}
    a &= \begin{pmatrix} a_0 + a_3 & a_1 - i a_2 \\ a_1 + i a_2 & a_0 - a_3  \end{pmatrix}= a_0 + \vec a \cdot \vec \sigma, & \text{for $(a_0, \dots, a_3) \in \RR^4$.}
\end{align*}


- Positivity of states implies that, restricted to hermitian elements, they all are  $\omega_{\vec x}(a) = a_0 + \vec x \cdot \vec a$, for $\vec x \in B^3 \subset \RR^3$.

- $\omega_{\vec x}$ is pure \iff $\vec x = (\sin \theta \cos \phi, \sin \theta, \cos \theta) \in S^2$ \dbtext{\iff} $\omega_x = \psi^1_{(\phi, \theta)}$.

\rtext{\textbf{Theorem}}: For $N = 1$ all pure states are coherent states and 
\begin{equation}
    d_1(\hat p, \hat q) = \frac{1}{2}|\vec p - \vec q|_{\RR^3}
\end{equation}
\textit{Pf}: $|\omega_{\vec x}(b) - \omega_{\vec y}(b)| = |(\vec x - \vec y)\cdot \vec b| \leq |\vec x - \vec y||\vec b|$; $i[D_1, a]$ is hermitian with max. eigenvalue/norm $=2 |\vec a|$ \then $a \in \acal_1$ hermitian with $\vec a$ parallel to $\vec x - \vec y$ st. $2|\vec a| = 1$ saturates the inequality.

\end{frame}

\begin{frame}{$|j, m\ket$ Distances for all $N$} % % % % % % % % % % % % % % % % % % %

\rtext{\textbf{Theorem}}: For any $N$, let $\lbtext{\omega_m} = \langle j, m| \cdot |j, m\rangle \in \mathcal S(\acal_N)$ (SP $\Longleftrightarrow |j, -j\rangle$, NP $\Longleftrightarrow |j, j\rangle$):
\begin{equation}
    d_N(\omega_m, \omega_n) = \sum_{k = m+1}^n \frac{1}{\sqrt{(j+k)(j-k+1)}} = \sum_{k = m+1}^n d_N(\omega_{k-1}, \omega_k)
\end{equation}
%  and so, between the north and south poles
% \begin{equation}
%     d_N(\psi_{(0,0), \psi_{(0, \pi)}}) = \sum_{k = 1}^N \frac{1}{\sqrt{k(N-k+1)}}
% \end{equation}
\textit{Pf}: Recall that $J_\pm|j,m\rangle = \sqrt{(j\mp m)(j\pm m + 1)}|j, m+1 \rangle$ \then $\omega_m(a) - \omega_n(a) = \sum_{k = m+1}^n \langle j, k-1 |a|j, k-1 \rangle - \langle j, k |a| j, k \rangle = \sum_{k = m+1}^n \frac{1}{\sqrt{(j+k)(j-k+1)}} \langle j, k| [E, a] |j, k-1 \rangle$; since $|\langle j, k |[E, a]|j, k-1 \rangle|  \leq ||[E, a]|| \leq ||[D_N, a]|| \leq 1$, we get the upper bound. 

Define the diagonal hermitian operator $\hat a |j, m\rangle := - \left( \sum_{k = -j+1}^m  \right)|j, m\rangle$, \then $[E, \hat a] |j, k \rangle = |j, k+1\rangle$ ($k < j$), and so $\hat a$ saturates the inequality.
\end{frame}

\begin{frame}{General Study} % % % % % % % % % % % % % % % % % % %

Auxiliary distance: lower bound: distance along diagonal matrices of $\acal_N$

The distance is $SU(2)$-invariant

The distance between $2$ states associated to the points $p, q \in S^2$ is non-decreasing with $N$/
- In general:
    The distance between increasing fuzzy approximations of $p, q \in S^2$ is non-decreasing with $N$ and
    \begin{equation}
        \lim_{N \to \infty} d_N(\psi^N_{(\phi, \theta)}, \psi^N_{(\phi', \theta')}) = d_{geo}(E(\phi, \theta), E(\phi', \theta'))
    \end{equation}
In general:
    \begin{equation}
        \rho_N(\theta - \theta') \leq d_N(\psi^N_{(\phi, \theta)}, \psi^N_{(\phi', \theta')}) \leq d_{geo}((\phi, \theta), (\phi', \theta'))
    \end{equation} and, finally,
    \begin{equation}
        \lim_{N \to \infty} d_N(\psi^N_{(\phi, \theta)}, \psi^N_{(\phi', \theta')}) = d_{geo}(\psi^N_{(\phi, \theta)}, \psi^N_{(\phi', \theta')})
    \end{equation}
\end{frame}

% \begin{frame}{Main Points} % % % % % % % % % % % % % % % % % % %
% Very similar procedure to calculate distances followed in Chakraborty Moyal Plane and Chakraborty Fuzzy Sphere.
% \end{frame}