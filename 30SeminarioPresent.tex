\documentclass{beamer}

%%% Packages
\usepackage{amssymb, amsmath, amsthm}
\usepackage[english]{babel}
\usepackage[utf8]{inputenc}
\usepackage{biblatex} \addbibresource{bibliografia.bib}

%% Other Packages
\usepackage[T1]{fontenc}
\usepackage[normalem]{ulem} % For strikethrough with command \sout
\usepackage{slashed} % For slashed in Dirac operator

%%%%%% Theorem Environments
\theoremstyle{definition}

%\newtheorem{theorem}{Theorem}[section]  %numbered according to section environment, so in section to it restarts as 2.1 
\newtheorem{proposition}{Proposition}[section]  %numbered according to section environment, so in section to it restarts as 2.1 
%\newtheorem{lemma}[theorem]{section}     %numbering shared with theorem 
%\newtheorem{corollary}{Corollary}[theorem]


\theoremstyle{remark}

\newtheorem*{remark}{Remark}

%%%%% Color text
%\usepackage[]{xcolor}

\newcommand{\ytext}[1]{\textcolor{yellow}{#1}}
\newcommand{\otext}[1]{\textcolor{orange}{#1}}
\newcommand{\rtext}[1]{\textcolor{red}{#1}}
\newcommand{\lbtext}[1]{\textcolor{cyan}{#1}}
\newcommand{\dbtext}[1]{\textcolor{blue}{#1}}
\newcommand{\ptext}[1]{\textcolor{Plum}{#1}}
\newcommand{\lgtext}[1]{\textcolor{LimeGreen}{#1}}
\newcommand{\dgtext}[1]{\textcolor{OliveGreen}{#1}}

%%%%% Custom Commands
\newcommand{\set}[1]{\{#1\}}
\newcommand{\inv}{{-1}}
\newcommand{\conj}[1]{\overline{#1}}

\newcommand{\twov}[2]{\begin{pmatrix} #1 \\ #2 \end{pmatrix}}
\newcommand{\threev}[3]{\begin{pmatrix} #1 \\ #2 \\ #3 \end{pmatrix}}

\newcommand{\then}{\ensuremath{\longrightarrow}}
\newcommand{\Then}{\ensuremath{\Longrightarrow}}
\renewcommand{\iff}{\ensuremath{\longleftrightarrow}}
\newcommand{\Iff}{\ensuremath{\Longleftrightarrow}}
\newcommand{\suff}{\ensuremath{\longleftarrow}}
\newcommand{\Suff}{\ensuremath{\Longleftarrow}}


\newcommand{\bra}{\langle}
\newcommand{\ket}{\rangle}
\newcommand{\bbra}{\langle\langle}
\newcommand{\kket}{\rangle\rangle}


\newcommand{\bb}[1]{\mathbb #1}
\newcommand{\ZZ}{\mathbb Z}
\newcommand{\RR}{\mathbb R}
\newcommand{\CC}{\mathbb C}
\newcommand{\HH}{\mathbb H}
\newcommand{\TT}{\mathbb T}

\newcommand{\defn}[1]{\lbtext{#1}}
\newcommand{\Defn}[1]{\dbtext{#1}}

\newcommand{\hcal}{\mathcal H}
\newcommand{\acal}{\mathcal A}
\newcommand{\bcal}{\mathcal B}
\newcommand{\dcal}{\mathcal D}
\newcommand{\cut}[1]{\overline{#1}}

%%%%%
% Choose how your presentation looks.
%
% For more themes, color themes and font themes, see:
% http://deic.uab.es/~iblanes/beamer_gallery/index_by_theme.html
%
\mode<presentation>
{
  \usetheme{Darmstadt}      % or try Darmstadt, Madrid, Warsaw, ...
  \usecolortheme{beaver}%seagull} % or try albatross, beaver, crane, ...
  \usefonttheme{default}  % or try serif, structurebold, ...
  \setbeamertemplate{navigation symbols}{}
  \setbeamertemplate{caption}[numbered]
} 

% Show number of slides at the bottom
\addtobeamertemplate{navigation symbols}{}{%
    \usebeamerfont{footline}%
    \usebeamercolor[fg]{footline}%
    \hspace{1em}%
    \insertframenumber/\inserttotalframenumber
}

%%%%%% Show TOC before a new Section
\AtBeginSection[]
{
    \begin{frame}[noframenumbering]{Table of Contents}
        \tableofcontents[currentsection]
    \end{frame}
}

% Show TOC before a new SubSection
\AtBeginSubsection[]
{
    \begin{frame}[noframenumbering]{Table of Contents}
        \tableofcontents[currentsection,currentsubsection]
    \end{frame}
}
% Reduce size of TOC
\AtBeginDocument{
  %\addtocontents{toc}{\tiny}
  %\addtocontents{subsection in toc}{\tiny}
}

%\setbeamerfont{subsection in toc}{size=\tiny}

%%%%%% Make paragraphs start with no indentation and leave spaces between paragraphs
\setlength{\parindent}{0em}
\setlength{\parskip}{1em}

%%%%%% Document Information
\title[l]{Towards the Calculation of Distances in the New Noncommutative Spheres of Fiore and Pisacane}
\author{Sebastian Puerto}
\institute{Universidad de los Andes\\ Seminario de Física-Matemática}
\date{October 19, 2020}

%%%%%%%%%%%%%%%%%%%%%%%%%%%%%%%%%%%%%%%%%%%%%%%%%%%%%%%%%%%%%%%%%%%%%%%%%%%%%%%
%%%%%%%%%%%%%%%%%%%%%%%%%%%%%%%%%%%%%%%%%%%%%%%%%%%%%%%%%%%%%%%%%%%%%%%%%%%%%%%
%%%%%%%%%%%%%%%%%%%%%%%%%%%%%%%%%%%%%%%%%%%%%%%%%%%%%%%%%%%%%%%%%%%%%%%%%%%%%%%
%%%%%%%%%%%%%%%%%%%%%%%%%%%%%%%%%%%%%%%%%%%%%%%%%%%%%%%%%%%%%%%%%%%%%%%%%%%%%%%
\begin{document}

\begin{frame}[noframenumbering]
  \titlepage
\end{frame}

\begin{frame}{Objective} % % % % % % % % % % % % % % % % % % %
Understand the study of distances in noncommutative spaces through examples.

Research the distance between states in the new fuzzy spheres introduced in \cite{Fiore2018}.
\end{frame}

% Uncomment these lines for an automatically generated outline.
\begin{frame}[noframenumbering]{Outline}
  \tableofcontents
\end{frame}

%%%%%%%%%%%%%%%%%%%%%%%%%%%%%%%%%%%%%%%%%%%%%%%%%%%%%%%%%%%%%%%%%%%%%%%%%%%%%%%
%%%%%%%%%%%%%%%%%%%%%%%%%%%%%%%%%%%%%%%%%%%%%%%%%%%%%%%%%%%%%%%%%%%%%%%%%%%%%%%
%%%%%%%%%%%%%%%%%%%%%%%%%%%%%%%%%%%%%%%%%%%%%%%%%%%%%%%%%%%%%%%%%%%%%%%%%%%%%%%
\section{Background}

\begin{frame}{Noncommutative Geometry} % % % % % % % % % % % % % % % % % % %
    
    \textbf{Unital Spectral Triple} $(\acal , \hcal, D)$: $\hcal$ is a complex separable Hilbert space; $\acal$ is a complex associative involutive unital $C^*$-algebra with a faithful unital $*$-representation $\acal \hookrightarrow \mathcal B(\hcal)$; a self-adjoint (unbounded) operator $D: \hcal \to \hcal$ with compact resolvent, such that $[D, a] \in \mathcal B(\hcal)$ $\forall a \in \acal$.
    
    \textbf{Canonical Spectral Triple}: Leg $(M, g)$ be an $n$-dimensional compact oriented Riemannian manifold which admits a spin structure% $(Spin_n \to Spin\,M \to M, \eta: Spin\,M \to SO\,M)$
    , and let $\Sigma M$ % = Spin\, M \times_\rho \Sigma_n$
    be the associated spinor bundle. Then $(\acal = C^\infty(M), \hcal = L^2(M, \Sigma M), \slashed D = -i \gamma^\mu \nabla^g_\mu)$%, where $\Sigma_n$ is the $2^{\lfloor n/2 \rfloor}$-dim. fermionic Fock space.
    
    - \textbf{State of $C^*$-algebra} $\acal$ is an $\omega : \acal \to \CC$ linear, positive, of unit norm. \textbf{Pure}: can not be written as a convex combination of other states. Commutative: pure states $\Longleftrightarrow$ characters $\hat x$ $\Longleftrightarrow$ points $x$.
    
    - \textbf{Distance}: $d_D(\omega, \omega') := sup_{a \in \acal} \left\{|\omega(a) - \omega'(a)| \, | \, \rtext{||[D, a]||_{op} \leq 1} \right\}$
    
\end{frame}

\begin{frame}{Canonical Spectral Triple: $d(x, y) = d_D(\hat x, \hat y)$}
    - $[\slashed D, f]\psi = -i [(\partial_\mu f) \gamma^\mu] \psi = -i(df)\cdot \psi$
    
    - $||[\slashed D, f]||^2 = sup_{x \in M} ||g_x^{-1}(df, d \cut f)||^2 =: ||grad(f) \in \Gamma(TM)||_\infty^2$

    - $d(x, y) := inf\{length(\gamma): \gamma:[0,1] \to M; \gamma(0) = x, \gamma(1) = y\}$
    
    - Good upper bound for $d_D(\hat x, \hat y)$: $d(x, y)$: for all $\gamma$, $f(y) - f(x) = \int_0^1 \frac{d}{dt}[f(\gamma(t))]dt = \int_0^1 df_{\gamma(t)} (\dot \gamma(t)) = \int_0^1 g_{\gamma(t)}(grad(f), \dot \gamma) dt$, 
    so
    $|f(x) - f(y)| \leq \int_0^1 |grad(f)| |\dot \gamma(t)| dt \leq ||grad(f)||_
    \infty \, length(\gamma) = ||[\slashed D, f]|| length(\gamma)$. 
    Thus:
    $d_D(\hat x, \hat y) = sup\{|f(x) - f(y)|: f \in C(M), ||[\slashed D, f]||\leq 1\} \leq inf \, length(\gamma) \equiv d(x, y)$
    
    - Saturation of the upper bound: let $a_x \in C(M)$, $a_x(y) := d(x, y)$, %its continuity is simply the triangle inequality
    $||[\slashed D, a_x]|| = ||grad(a_x)||_\infty = 1$ and $|a_x(y) - a_x(x)| = |d(x,y)| = d(x,y)$ saturates the inequality, meaning that $ d_D(\hat x, \hat y) = d(x, y)$.
    
    
    
    %Compact Quantum Metric Space
\end{frame}


\begin{frame}{The Fuzzy Sphere} % % % % % % % % % % % % % % % % % % %
    \textbf{Fuzzy Space}: %($C^*$? or simply $*$?) 
    family of noncommutative $\acal_n$ parametrized by $n \in \bb N$ with increasing dimension and such that that approximate the commutative algebra $\acal$. Why? \textit{Keep continuous symmetries}.
        % \begin{itemize}
        % \item Why? To preserve the (continuous) symmetries of the space while keeping the algebra finite dimensional.
        % \end{itemize}
    
    \textbf{Fuzzy Sphere}: Notice that $\acal \cong \bigoplus_{l \in \bb N} V_l \cong L^2(S^2)$, where $V_l$ is the spin $l$ representation of $SO(3)$: homogeneous $l$-degree polynomials in $x^1, x^2, x^3$, with basis $\{Y^l_m\}_{|m| \leq l}$. 
    For $N = 2j \in \bb N$, 
    $\lbtext{\acal_N}= \bigoplus_{l = 0}^N V_l$ as $SU(2)$ representation. 
    As algebra: replacing $x^i \mapsto \frac{1}{\sqrt{j(j+1)}} \pi_{j}(J_i)$, $[J_i, J_k] = i \epsilon_{ijk} J_k$, $\lbtext{\acal_N} := End(V_j) = M_{N+1}(\CC)$, understanding $V_j= \CC^{N+1} = span\{|j, m\ket\}_{|m| \leq j}$ as irrep. of $SU(2)$; this follows from  \then \rtext{$[x^i, x^j] = \frac{1}{\sqrt{j(j+1)}} i \epsilon_{ijk} x_k$}, $\sum x_1^2 + x_2^2 + x_3^2 = 1$.
    
    With these Dirac Spectral Triples\cite{DAndrea2013} \rtext{approximates $S^2$ as: \textbf{1.} $C^*$-algebra $\acal$ acting on the spinors $\hcal$; \textbf{2.} Representation of $SU(2)$ (diffeomorphisms); \textbf{3.}  Metric space on which $SU(2)$ acts by isometries.}
    
    %. Under the adjoint action, which makes sense since R J_3 R^{-1} is rotation under rotated axis
    
    % This allows to define fuzzy spherical harmonics (changing x's by new x's) which make up a basis, good action under SU(2)
    
    
\end{frame}


%%%%%%%%%%%%%%%%%%%%%%%%%%%%%%%%%%%%%%%%%%%%%%%%%%%%%%%%%%%%%%%%%%%%%%%%%%%%%%%
%%%%%%%%%%%%%%%%%%%%%%%%%%%%%%%%%%%%%%%%%%%%%%%%%%%%%%%%%%%%%%%%%%%%%%%%%%%%%%%
%%%%%%%%%%%%%%%%%%%%%%%%%%%%%%%%%%%%%%%%%%%%%%%%%%%%%%%%%%%%%%%%%%%%%%%%%%%%%%%%
\section{Introduccion}

\begin{frame}{Titulo} % % % % % % % % % % % % % % % % % % %
    
\end{frame}


%%%%%%%%%%%%%%%%%%%%%%%%%%%%%%%%%%%%%%%%%%%%%%%%%%%%%%%%%%%%%%%%%%%%%%%%%%%%%%%
%%%%%%%%%%%%%%%%%%%%%%%%%%%%%%%%%%%%%%%%%%%%%%%%%%%%%%%%%%%%%%%%%%%%%%%%%%%%%%%
%%%%%%%%%%%%%%%%%%%%%%%%%%%%%%%%%%%%%%%%%%%%%%%%%%%%%%%%%%%%%%%%%%%%%%%%%%%%%%%
\section{Two $SU(2)$-equivariant distances/spectral triples on the Fuzzy Sphere}

\begin{frame}{Canonical Spectral Triple of $S^2$} % % % % % % % % % % % % % % % % % % %

% Inherits from $\RR^3$ the metric and the metric connection, but the spinor space changes

\rtext{Starting point: $SU(2)$-isometries}%\cite{DAndrea2013}
- $S^2$ as the symmetric space $S^3/S^1$ of the compact semisimple Lie group $G = S^3$, $\mathfrak g = su(2)$.
    
- The canonical spectral triple, which is \textbf{$SU(2)$-equivariant} can be seen to come from a purely algebraic element $\lbtext{\mathcal D} \in U(\mathfrak g) \otimes U(\mathfrak g)$:
    \begin{align*}
        U(\mathfrak g) \otimes U(\mathfrak g) &\to& U(\mathfrak g) \otimes Cl(\mathfrak g, -K) &\to& \mathcal B(L^2(G/U, \Sigma G/U)) \\
        1 \otimes 1 + 2 \sum_{k = 1}^3 J_k \otimes J_k &\mapsto& 1 \otimes 1 + \sum_k J_k \otimes \sigma_k &\mapsto& \rtext{\left( \bigoplus_{l\in \bb N} \pi_l \right) \otimes \pi_{1/2}(\mathcal D)}
    \end{align*}
    
%- In Sanchez: $\mu = 1, 2$
% $\nabla_\mu = \partial_\mu - \frac{c_\mu}{q} \sigma_\mu \sigma_{\cut \mu} \cdot$, 
% $\slashed D = -i \frac{q}{2} \sigma^\mu \nabla_\mu

- So $\lbtext{\slashed D} = \begin{pmatrix} 1 + \partial_H & \partial_F \\ \partial_E & 1 - \partial_H\end{pmatrix} = 1 + \partial_F \otimes \sigma_+ + \partial_E \otimes \sigma_- + \partial_H \otimes \sigma_3$ where $\partial_H = -i \partial_\phi$, $\partial_F = e^{i\phi} \left( \partial_\theta + i cot\,\theta \partial_\phi \right)$, $\partial_E = -\partial_F% = e^{-i\phi} \left( \partial_\theta - i cot\,\theta \partial_\phi \right)
$ are the actions of $J_3$, $J^\pm \in i\,su(2)$ on $L^2(S^2)$ respectively. \textit{Eig. vectors}: orth. basis of $\hcal$; \textit{Spectrum} $= \{\pm l\} = \ZZ - 0$ with multiplicities $2l$.%: L^2(S^2) \otimes \CC^2 \to L^2(S^2) \otimes \CC^2 = \oplus_{l \in \bb N} (\pi_l \otimes \pi_{1/2})(\mathcal D)$

% - The eigenvectors (spinor harmonics) $Y^{'}_{lm}, Y^{''}_{lm}$, $l \in \NN + 1/2$, of $D$ make up an orthogonal basis of the spinor fields $\hcal$.
% - The eigenvalues are: of $Y'_{l\cdot}: (l + 1/2) \in \NN$, of $Y{''}_{l\cdot}: -(l + 1/2)$, with multiplicities $2l+1$
\end{frame}

\begin{frame}{The Irreducible Spectral Triple} % % % % % % % % % % % % % % % % % % %
A first spectral triple is simply taking ``one term'' of $\slashed D$:
\begin{multline}
    \lbtext{D_N} 
    := (\pi_j \otimes \pi_{1/2})(\mathcal D): V_j \otimes \CC^2 \to V_j \otimes \CC^2 \\
    = \begin{pmatrix} 1 + \pi_j(H) & \pi_j(F) \\ \pi_j(E) & 1 - \pi_j(H)\end{pmatrix} 
    = 1 + \pi_j(F) \otimes \sigma_+ + \pi_j(E) \otimes \sigma_- + \pi_j(H) \otimes \sigma_3
\end{multline} where $H = J_3$, $F = J_+$, $E = J_-$ are the actions of $J_3$, $J^\pm \in su(2)$ on $V_j$ respectively.

- The spectral triple \rtext{$(\acal_N, \lbtext{H_N} = V_j \otimes \CC^2, D_N)$}
    \begin{itemize}
        
    \item Is $SU(2)$-equivariant
    
    \item Has eigenvalues $j+1$ and $-j$ with multiplicities $2j+2$ and $2j$.
        
    \item Isn't compatible with a grading or a real structure.
    \end{itemize}

\end{frame}

\begin{frame}{The Full Spectral Triple} % % % % % % % % % % % % % % % % % % %
\rtext{$(\acal_N, \lbtext{\hcal_N} := \mathcal A_N \otimes \CC^2 \cong \bigoplus_{l =1}^N H_l, \mathcal D_N \longleftrightarrow \bigoplus_{l =1}^N D_l)$}, where:
\begin{multline}
    \lbtext{\mathcal D_N} := (\textbf{ad}\pi_j \otimes \pi_{1/2})(\mathcal D) = D = \begin{pmatrix} 1 + \text{ad}\pi_j(H) & \text{ad}\pi_j(F) \\ \text{ad}\pi_j(E) & 1 - \text{ad}\pi_j(H)\end{pmatrix} 
    %= \text{ad}\pi_j(E) \otimes \sigma_1 + \text{ad}\pi_j(F) \otimes \sigma_2 + \text{ad}\pi_j(H) \otimes \sigma_3
\end{multline}
where, e.g. $\text{ad}\pi_j(H) = [\pi_J(H), \cdot ]$ is the action of $J_3 \in su(2)$ on $\mathcal A_N$.

    \begin{itemize}
    
    \item It is a real spectral triple.
    
    \item It is $SU(2)$-equivariant
    
    \item Spectrum of $\mathcal D_N$ is the truncation of $\slashed D$ to $\{-N, \dots, N+1\}$.%The eigenvalues of $\mathcal D_N$ are $N+1$ with multiplicity $2N+2$, and $\pm l$ with multiplicity $2l$ for $l = 1, \dots, N$
    
    \item It is not compatible with a grading.
    
    \end{itemize}

\textbf{It is a truncation of the canonical spectral triple}.

- \rtext{\textbf{Theorem}}: the two spectral triples induce the same distance in $\acal_N$. Pf: $[\dcal, a]b \otimes v = \cdots = \sum_k [\pi_j(J_k), a]b \otimes \sigma_k v = [D_N, a] \cdot b \otimes v$
\end{frame}

\begin{frame}{(Bloch) $SU(2)$-coherent states on $\acal_N$} % % % % % % % % % % % % % % % % % % %
- The Bloch/$SU(2)$-\rtext{coherent states} are for the group $SU(2)$ what the usual harmonic oscillator coherent states are for the Heisenberg group. In particular, they are minimum uncertainty states. 

%- \rtext{They will be considered fuzzy approximations of the points of $S^2$}.

- \rtext{Points of the $E(\phi, \theta) \in S^2$ sphere are approximated by coherent vectors} $\lbtext{|\phi, \theta)_N} := R_{(\phi, \theta)}|j, -j\rangle \in V_j \Longleftrightarrow $ states of $\mathcal A_N = End(V_j)$ $\lbtext{\psi^N_{\phi, \theta}} = (\phi, \theta| \cdot |\phi, \theta)_N \in \mathcal S(\mathcal A_N)$. 

- This identification of points in $S^2$ with coherent states %($\psi^N: \acal_N \to L^2(S^2)$ at the algebra level) ???
is $SU(2)$-equivariant \& the distance between them will be $SU(2)$-invariant.

%What group used in Fiore?
\end{frame}

\begin{frame}{Distance between $SU(2)$-coherent states (1/2)} % % % % % % % % % % % % % % % % % % %
From now on: use the irreducible s.t. and not write the $\pi_j$'s.

\textit{General lemma}: For $a \in \acal_N$, \begin{align}
    ||[H, a]|| &\leq ||[D_N, a]|| & ||[E, a]|| &\leq ||[D_N, a]|| & ||[F, a]|| &\leq ||[D_N, a]|| .
\end{align} If $a$ is diagonal hermitian, then
\begin{equation}
    ||[E, a]|| = ||[D_N, a]||.
\end{equation}


- For $N = 1$: all pure states are coherent states and 
\begin{equation}
    d_1(\hat p, \hat q) = \frac{1}{2}|\vec p - \vec q|_{\RR^3}
\end{equation}

\end{frame}

\begin{frame}{$|j, m\ket$ Distances for all $N$} % % % % % % % % % % % % % % % % % % %

- For any $N$, letting $\omega_m = |j, m\rangle \langle j, m|$ (SP $\Longleftrightarrow |j, -j\rangle$, NP $\Longleftrightarrow |j, j\rangle$):
\begin{equation}
    d_N(\omega_m, \omega_n) = \sum_{k = m+1}^n \frac{1}{\sqrt{(j+k)(j-k+1)}} = \sum_{k = m+1}^n d_N(\omega_{k-1}, \omega_k)
\end{equation}
%  and so, between the north and south poles
% \begin{equation}
%     d_N(\psi_{(0,0), \psi_{(0, \pi)}}) = \sum_{k = 1}^N \frac{1}{\sqrt{k(N-k+1)}}
% \end{equation}

\end{frame}

\begin{frame}{General Study} % % % % % % % % % % % % % % % % % % %

Auxiliary distance: lower bound: distance along diagonal matrices of $\acal_N$

The distance is $SU(2)$-invariant

The distance between $2$ states associated to the points $p, q \in S^2$ is non-decreasing with $N$/
- In general:
    The distance between increasing fuzzy approximations of $p, q \in S^2$ is non-decreasing with $N$ and
    \begin{equation}
        \lim_{N \to \infty} d_N(\psi^N_{(\phi, \theta)}, \psi^N_{(\phi', \theta')}) = d_{geo}(E(\phi, \theta), E(\phi', \theta'))
    \end{equation}
In general:
    \begin{equation}
        \rho_N(\theta - \theta') \leq d_N(\psi^N_{(\phi, \theta)}, \psi^N_{(\phi', \theta')}) \leq d_{geo}((\phi, \theta), (\phi', \theta'))
    \end{equation} and, finally,
    \begin{equation}
        \lim_{N \to \infty} d_N(\psi^N_{(\phi, \theta)}, \psi^N_{(\phi', \theta')}) = d_{geo}(\psi^N_{(\phi, \theta)}, \psi^N_{(\phi', \theta')})
    \end{equation}
\end{frame}

% \begin{frame}{Main Points} % % % % % % % % % % % % % % % % % % %
% Very similar procedure to calculate distances followed in Chakraborty Moyal Plane and Chakraborty Fuzzy Sphere.
% \end{frame}

%%%%%%%%%%%%%%%%%%%%%%%%%%%%%%%%%%%%%%%%%%%%%%%%%%%%%%%%%%%%%%%%%%%%%%%%%%%%%%%
%%%%%%%%%%%%%%%%%%%%%%%%%%%%%%%%%%%%%%%%%%%%%%%%%%%%%%%%%%%%%%%%%%%%%%%%%%%%%%%
%%%%%%%%%%%%%%%%%%%%%%%%%%%%%%%%%%%%%%%%%%%%%%%%%%%%%%%%%%%%%%%%%%%%%%%%%%%%%%%
%%%%%%%%%%%%%%%%%%%%%%%%%%%%%%%%%%%%%%%%%%%%%%%%%%%%%%%%%%%%%%%%%%%%%%%%%%%%%%%
\section{New Fuzzy Spheres}

\begin{frame}{General Setting} % % % % % % % % % % % % % % % % % % %
    
    - $H = - \frac{1}{2} \Delta + V(r)$ invariant under $O(D)$
    
    - Introducing the cutoff $\cut E$, as the energy where
    \begin{align}
        \label{eqn5}
            V(r) \approx V_0 + 2k(r-1)^2 && \text{for $r$ such that $V(r) \leq  \cut E$}.
    \end{align}
    %in the region $\nu_{\cut E} = \{r \,|\, V(r) \leq \cut E\}$.
    
    - Eigenfunctions of $H$ as product of a spherical harmonic $Y(\phi, \dots)$ (eigenvector of $L^2$) and an e.v. of the radial equation 
    \begin{align}
        \label{eqn9}
        \left[-\partial_r^2 - (D-1) \frac{1}{r} \partial_r + \frac{1}{r^2} j(j+D-2) + V(r)\right] \tilde f(r) = E \tilde f(r).
    \end{align}
    - This last equation can be approximated by a harmonic oscillator equation, since outside the region $\{V(r) \leq \cut E\}$ $\psi$ is negligibly small.
    : $\hcal_{\cut E} \approx $ solutions of \eqref{eqn9}, $\acal_{\cut E} = End(\hcal_{\cut E})$.
        
    
    
    
    
\end{frame}

%%%%%%%%%%%%%%%%%%%%%%%%%%%%%%%%%%%%%%%%%%%%%%%%%%%%%%%%%%%%%%%%%%%%%%%%%%%%%%%
%\subsection{2D}


\begin{frame}{Construction of $\hcal_{\cut E}$} % % % % % % % % % % % % % % % % % % %
    - Equation \eqref{eqn9} has the approximation, where $\rho := \ln r$
    \begin{equation}
        \label{harmonic2D}
        \hat H f(\rho) = e_m f(\rho), \qquad
        \hat H = - \partial_\rho^2 + k_m(\rho - \tilde \rho_m)^2,
    \end{equation} $
        k_m := 2(k - E'), \quad
        E' := E - V_0, \quad
        \tilde \rho_m := \frac{E'}{k_m}, \quad
        e_m = \frac{E'^2}{k_m} + E' - m^2
    $
    - The solutions $f_{n,m}$ are known, $n \in \bb N, m \in \ZZ$ \then $e_{m, n}(k) = (2n+1)\sqrt{k_m}$ \then $E'_{m,n}(k)$ satisfies a quartic equation \then $E_{m,n}(k, V_0)$.
    
    - Fixing $V_0 = V_0(k)$ such that $E_{0, 0} = 0$ \then $V_0(k) = %-\sqrt{2k} + 2 - \frac{7}{2}\frac{1}{\sqrt{2k}} + o(1/k)$ and 
    \sum_{n = -1}^\infty v_n \left( \sqrt{\frac{1}{k}} \right)^n$ \then
    \begin{equation}
        E_{n, m}(k) = m^2 + 2n\sqrt{2k} - 2n + o(1/\sqrt{k})
    \end{equation}
    
\end{frame}

\begin{frame}{Construction of $\hcal_{\cut E}$ (Cont.)}
    - Choosing $\cut E < 2 \sqrt{2k} - 2$, the spectrum of $H$ is a truncation of $L^2$: \textbf{the radial oscillations are ``frozen''}: $\cut{\partial_r} = 0$.
    \begin{multline*}
        \psi_m(\rho, \phi) = f_{0, m}(\rho) e^{im\phi} = N_m e^{im\phi}exp{\left[ -\frac{(\rho - \tilde \rho_m)^2 \sqrt{k_m}}{2} \right]} \\\xrightarrow{k \to \infty} \delta(r-1)e^{i m \phi}
    \end{multline*}
    \begin{equation}
        E = E_m(k) = m^2 + o(1/\sqrt{k})
    \end{equation}
    
    - For $\Lambda := \lfloor \cut E \rfloor$, 
    %\begin{align}
        $\lbtext{\hcal_{\Lambda}}:= \lbtext{\hcal_{\cut E}} := span\{\psi_m\}_{|m| \leq 
    \lfloor \cut E \rfloor} ,
    \lbtext{\acal_\Lambda} := \mathcal B(\hcal_\Lambda)$
    %\end{align}
    
    - %Since $H$ generates the time evolution, a
    An element of $\hcal_\Lambda$ doesn't evolve out of $\hcal_\Lambda$.
    
    - Get a fuzzy space: e.g. choosing $k = \Lambda^2(\Lambda+1)^2$ %make $k$ diverge with $\Lambda$ while $\nu_{\cut E}$ goes to $\{r = 1\}$
    
    - This cutoff entails replacing every observable by $A \mapsto \lbtext{\cut A} = P_{\cut E} A P_{\cut E}$% \dbtext{when?}
\end{frame}

\begin{frame}{Important Observables and their Commutation Relations} % % % % % % % % % % % % % % % % % % %
    - Up to infinite, $1/k^{1/2}\dbtext{?}$ and $1/k^{3/2}$ orders, respectively
    \begin{align}
    \label{projObs2D}
        \cut L \psi_m &= m \psi_m; & 
        \cut H &= \cut L^2; & 
        \cut x^\pm \psi_m = 
            \begin{cases}
                \frac{a}{\sqrt{2}} \sqrt{ 1 + \frac{m(m \pm 1)}{k} } \psi_{m \pm 1} & -\Lambda \leq \pm m \leq \Lambda - 1 \\
                0 & \text{otherwise}
            \end{cases}
    \end{align}
    - And so, up to terms of $1/k^{3/2}$
    \begin{align}
        \label{conmObs2D}
        \cut{x^+}^\dagger &= \cut{x^-}; &
        [\cut L, \cut{x^\pm}] &= \pm \cut{x^\pm}; &
        [\cut{x^+}, \cut{x^-}] &= - \frac{\cut L}{k} + \left[1 + \frac{\Lambda(\Lambda+1)}{k}\right] (\tilde P_{\Lambda} - P_{-\Lambda}).
    \end{align}
     - If \eqref{projObs2D} are used exactly to define elements of $\mathcal B(\hcal_\Lambda) \equiv \acal_\Lambda$ then \eqref{conmObs2D} are also exact, and $\cut{x^\pm}$ generate $\acal_\Lambda$. % \cut{\partia_\pm} are now redundant... but I'm not sure why
    
\end{frame}

\begin{frame}{Realization of $\acal_\Lambda$ through $Uso(3)$} % % % % % % % % % % % % % % % % % % %
    - $O(2)$ acts on $\hcal_\Lambda \subset L^2(\RR^2)$, and so on $\acal_\Lambda$, since $[H, O(2)\cdot ] = 0$. $SU(N) \ni g$ is the group of $*$-automorphisms of $M_N(\CC) \cong \acal_\Lambda$ acting by $a \mapsto g a g^{-1}$; $O(2)$ is then a subgroup. %through the action induced in $\acal_\Lambda$ by its action on $\RR^2$.
        \begin{itemize}
            
        \item Rotation $R_\theta$: $\cut{x^\pm} \mapsto e^{\pm i \theta} \cut{x ^\pm}; \cut L \mapsto \cut L \in \acal_\Lambda$.
        
        \item Reflection: $\cut{x^\pm} \cut{x^\mp}; \cut L \mapsto -\cut L$
        
        \end{itemize}
    
    - \textbf{We can consider $\acal_\Lambda \cong  M_N(\CC) = \pi_\Lambda(Uso(3))$ as a $*$-algebra and representations of $O(2)$}, where $\pi_\Lambda$ is the $N := 2 \Lambda + 1$ dimensional representation.
    \begin{align}
        \cut{x^\pm} &\longleftrightarrow f_\pm (J^0) J^\pm &
        \cut L &\longleftrightarrow J^0
    \end{align}
    where $J^\pm, J^0$ is the Weyl-Cartan basis of $so(3)$, $f_\pm(s) = \frac{1}{\sqrt{2}} \sqrt{\frac{1 + s(s-1)/k}{\Lambda (\Lambda + 1) - s(s-1)}} = f_-(-s)$%: $[J^+, J^-] = J^0; [J^\pm, J^0] = \pm J^\pm$

    - Rotation: by $\pi_\Lambda(e^{i \theta J_0})$; Reflection: by $\pi_\Lambda(e^{i\pi (J^+ + J^-)/\sqrt{2}})$
\end{frame}

\begin{frame}{Convergence} % % % % % % % % % % % % % % % % % % %

    - $\psi_m$ as fuzzy analogues of $e^{i m \phi} \in \hcal$: $O(2)$-covariant embedding $\hcal_\Lambda \hookrightarrow \hcal = L^2(S^1)$, $\psi_m \mapsto e^{im\phi}$; \hfill \then $\hcal_\Lambda \to \hcal$ as $\Lambda \to \infty$ in the sense that $\forall \phi \in \hcal$, $\phi_\Lambda = \sum_{|m| \leq \Lambda} \phi_m e^{im\phi} \to \phi$ in the $L^2$-norm.
    
    - Induces, embedding $\acal_\Lambda \hookrightarrow \acal = \mathcal B(\hcal)$ and limit $\acal_\Lambda \to \acal$ as $\Lambda \to \infty$.
    
    - Fuzzy analogue of $B(S^1)$ of bounded functions on $S^1$ as subalgebra of $\mathcal B(\hcal)$: $C_\Lambda := \left\{ \sum_{h = -2\Lambda}^\Lambda f_h \eta^h \,|\, f_h \in \CC\right\}$ where $\eta^\pm  = \frac{\sqrt{2}}{a}x^\pm$ (so $\eta^\pm \to e^{\pm i \phi}$ as operators).
    
    Choosing $k(\Lambda) \geq 2 \Lambda(\Lambda + 1)(w\Lambda+1)^2$, then $B(S^1) \to \mathcal B(\hcal)$ due to the strong limits: $\hat f_\Lambda \to f\cdot$, $\hat{(fg)}_\Lambda \to fg\cdot $, $\hat f_\Lambda \hat g_\Lambda \to fg\cdot$

\end{frame}

%%%%%%%%%%%%%%%%%%%%%%%%%%%%%%%%%%%%%%%%%%%%%%%%%%%%%%%%%%%%%%%%%%%%%%%%%%%%%%%
%\subsection{3D}

% \begin{frame}{Construction of $\hcal_{\cut E}$} % % % % % % % % % % % % % % % % % % %
    
% \end{frame}

% \begin{frame}{Important Observables and their Commutation Relations} % % % % % % % % % % % % % % % % % % %

% \end{frame}

% \begin{frame}{Realization of $\acal_\Lambda$ through $Uso(3)$} % % % % % % % % % % % % % % % % % % %

% Notice that the action on $\cut{x^\pm}$ is

% This $C^*$-algebra isomorphism is $O(2)$-equivariant.

% \end{frame}

% \begin{frame}{Convergence} % % % % % % % % % % % % % % % % % % %

% \end{frame}

%%%%%%%%%%%%%%%%%%%%%%%%%%%%%%%%%%%%%%%%%%%%%%%%%%%%%%%%%%%%%%%%%%%%%%%%%%%%%%%
% \begin{frame}{Main Points} % % % % % % % % % % % % % % % % % % %
% We are thinking of the algebras noncommutative version of $L^2(S^2)$, but as approximation of the algebra of observables of a quantum system%, which is the NC $\mathcal B(L^2(S^d))$.
    
%         \begin{itemize}
            
%         \item $\mathcal A_\Lambda \to \mathcal B(L^2(S^d))$
        
%         \item $\mathcal C_\Lambda \subset \mathcal A_N \to \text{ and } \hookrightarrow C(S^d)$ space of polynomials on the coordinates $x^i$.
        
%         \item Also, $\mathcal H_\Lambda \to \to \text{ and } \hookrightarrow L^2(S^d)$
            
%         \end{itemize}
    
%     Each of these approximations can be seen to come from: if only energies below a certain cutoff value $\cut{E}_\Lambda$ are accessible for the wavefunctions, and for the accessible energies the potential is nearly harmonic near $r = 1$ and ``sufficiently'' steep THEN the accessible state space $\mathcal H_{\cut{E}}$ can be studied with the projected observables: $A \mapsto \cut{A} := P_{\overline E} A P_{\overline E}$; this, in particular, means that new commutation relations appear for the coordinate functions $\hat x^i$.
    
%     The new dynamical system has $O(d)$ as a symmetry group: $O(d)$ acts on $\hcal_{\cut E}$ and the time evolution of these states is invariant under its action ($\cut H = H|_{\hcal _{\cut E}} = \cut H^{O(d)}$)
    
%     The coordinates generate the whole algebra $\mathcal A_{\cut{E}}$, and they have nontrivial commutation relations.
    
%     $\mathcal A_{\cut{E}}$ can be realized as the algebra of operators of an irrep. $\pi_{\cut{E}}$ of $so(d+2)$. This means that $\mathcal H_{\cut{E}}$ can be though of as an irrep. of $so(d+2)$, which, in particular, means that $\mathcal H_{\cut{E}}$ is a reducible representation of $so(d+1)$, namely, $\mathcal H_{\cut{E}} \cong \bigoplus_{E \leq \cut{E}} V_l$ as representation space of $so(d+1)$.
    
%     Embeddings and limits: a subalgebra $\mathcal C_\Lambda$ of $\mathcal A_{\Lambda}$ which has the $so(d+1)$-module decomposition $\bigoplus_{E \leq \cut{E}} V_l$ does approximate the commutative space.
% \end{frame}

%%%%%%%%%%%%%%%%%%%%%%%%%%%%%%%%%%%%%%%%%%%%%%%%%%%%%%%%%%%%%%%%%%%%%%%%%%%%%%%
%%%%%%%%%%%%%%%%%%%%%%%%%%%%%%%%%%%%%%%%%%%%%%%%%%%%%%%%%%%%%%%%%%%%%%%%%%%%%%%
\section{Further Work}

\begin{frame}{Other Spectral Triples} % % % % % % % % % % % % % % % % % % %

% Definition Moyal Plane

% 3 mentioned in D'Andrea

% Summary I did for Reyes

% End of Fiore2018

Moyal Plane Spectral triples: 
    \begin{itemize}
    
    \item Deformation of $\RR^2$, proposed in \cite{Gayral2004}, studied in \cite{Martinetti2013}, \cite{Cagnache2011}.
    
    \item NCQM, proposed in \cite{Scholtz2013}, studied in \cite{ChaobaDevi2018}.
    
    \end{itemize}

Fuzzy Sphere Spectral triples:
    \begin{itemize}
        
    \item Deformation of canonical s.t. of $S^2$ proposed in \cite{Grosse1995}, studied in \cite{Devi2015}, \cite{ChaobaDevi2018}.
    
    \item Dirac operator as the odd part of a truncated superfield on a supersphere \cite{Grosse1997}. Same distance as here.
    
    \item ``Spin-$j$ Dirac operator'' from spin $j$ representation of the abstract Dirac element \cite{Balachandran2007}, \cite{Balachandran2009}.
    
    \item Start with chirality operator and then find anticommuting self-adjoint operator with plausible spectrum \cite{Carow1997}, \cite{Carow1998}.
        
    \end{itemize}


\end{frame}


\begin{frame}{About New Fuzzy Spheres} % % % % % % % % % % % % % % % % % % %
- So far: \cite{Fiore2018}, \cite{Fiore2019}.

- ``Truncated Dirac Operator'' described in \cite{DAndrea2014} from applying an energy cutoff. Irreducible Dirac operator recovered with this procedure.

- Coherent states, specially in new Fuzzy Spheres: \cite{FioreCoherent2020}, \cite{FioreTheCase2020}.
\end{frame}

\begin{frame}{Possible Table of Contents}
    \tiny
    \begin{enumerate}
    
    \item Introduction: Preliminary Concepts of Noncommutative Geometry and Motivation for NCG and New Fuzzy Spheres: physics ($C^*$-algebras as topological spaces, canonical spectral triple, spectral triples and Connes' distance formula)
    
    \item The Fuzzy Sphere and its Metric Properties (using spectral triples of \cite{DAndrea2013}).
    
        \begin{enumerate}
        \tiny
        
        \item Canonical Spectral triple on $S^2$ (arising from $SU(2)$ equivariant algebraic Dirac element).
        
        \item The Fuzzy Sphere: fuzzy and $SU(2)$-equivariant approximation of $S^2$.
        
        \item Irreducible and Full Spectral Triples
        
        \item $SU(2)$-coherent states: fuzzy approximations of points in $S^2$
        
        \item Distances between coherent states and Convergence of fuzzy sphere to $S^2$ as metric spaces.
        
        \end{enumerate}
    
    \item The New Fuzzy Spheres of Fiore and Pisacane
    
        \begin{enumerate}
        \tiny
        
        \item General Setting
        
        \item Fuzzy Quantum Mechanics in $S^1$
        
            \begin{enumerate}\tiny
                
            \item Definition of the fuzzy space $\{S^1_\Lambda\}$
            
            
            \item Realization of the fuzzy space as representation of $Uso(3)$
            
            \item Convergence to quantum mechanics in $S^1$
            
            \end{enumerate}
        
        \item Fuzzy Quantum Mechanics in $S^2$
            
            \begin{enumerate}\tiny
                
            \item Definition of the fuzzy space $\{S^2_\Lambda\}$
            
            
            \item Realization of the fuzzy space as representation of $Uso(4)$
            
            \item Convergence to quantum mechanics in $S^2$
            
            \end{enumerate}
            
        \end{enumerate}
    
    \end{enumerate}
\end{frame}

% \begin{frame}{Table of Contents (Cont.)}
%     \begin{enumerate}
%     \tiny
    
%     \item Systems of Coherent States in the New Fuzzy Spheres
        
%         \begin{enumerate} \tiny
        
%         \item Basics on Coherent States
           
%         \item Coherent and Localized states on the new fuzzy circle
        
%         \item Coherent and Localized states on the new fuzzy sphere 
            
%         \end{enumerate}
%     \item Study of Distance between Systems of Coherent States in the New Fuzzy Spheres
    
%         \begin{enumerate} \tiny
            
%         \item Proposed Dirac Operator and Spectral Triple
            
%         \end{enumerate}
        
%     \end{enumerate}
      
% \end{frame}

\printbibliography[title=References]

\end{document}
