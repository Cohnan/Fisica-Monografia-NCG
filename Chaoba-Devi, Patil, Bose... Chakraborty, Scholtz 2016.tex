\documentclass{article}
\usepackage[utf8]{inputenc}
\usepackage[margin=1in]{geometry}

%%%%%% To use hyperlinks, including the formula ones
\usepackage{hyperref}
\hypersetup{
    colorlinks=true,
    linkcolor=blue,
    filecolor=magenta,      
    urlcolor=cyan,
}

%%%%%% Make paragraphs start with no indentation and leave spaces between paragraphs
\setlength{\parindent}{0em}
\setlength{\parskip}{1em}

%%%%%% Math stuff
\usepackage{amsmath, amssymb}
\usepackage{amsthm}

%%%%%% Mis Codigos

% TODO notes package
\usepackage{xargs}                  % Use more than one optional parameter in a new command
\usepackage[pdftex,dvipsnames]{xcolor}
\input{tools/todoCode}

% Colored text and boxes with my color conventions for highlighting
\usepackage[dvipsnames]{xcolor}
\input{tools/colorCode}

% Math symbols
\usepackage{xparse}
\input{tools/common_math_symbols}

% Physics symbols (vectors, units)
\usepackage{tikz}
\input{tools/physics_macros}

% Theorem environments
\input{tools/theorem_definitions}

%%%%%%
\title{Chaoba-Devi, Patil, et al. 2018: Revisiting Connes' Finite Spectral Distance on Non-commutative Spaces}
\author{Sebastian Camilo Puerto}
\date{July 2020}

%%%%%%%%%%%%%%%%%%%%%%%%%%%%%%%%%%%%%%%%%%%%%%%%%%%%%%%%%%%%%%%%%%%%%%%%%%%%%
%%%%%%%%%%%%%%%%%%%%%%%%%%%%%%%%%%%%%%%%%%%%%%%%%%%%%%%%%%%%%%%%%%%%%%%%%%%%%
%%%%%%%%%%%%%%%%%%%%%%%%%%%%%%%%%%%%%%%%%%%%%%%%%%%%%%%%%%%%%%%%%%%%%%%%%%%%%
%%%%%%%%%%%%%%%%%%%%%%%%%%%%%%%%%%%%%%%%%%%%%%%%%%%%%%%%%%%%%%%%%%%%%%%%%%%%%
\begin{document}

\maketitle

\tableofcontents

%%%%%%%%%%%%%%%%%%%%%%%%%%%%%%%%%%%%%%%%%%%%%%%%%%%%%%%%%%%%%%%%%%%%%%%%%%%%%
\subsection{High Level Summary}

    \begin{itemize}

    \item 
    
    \end{itemize}

%%%%%%%%%%%%%%%%%%%%%%%%%%%%%%%%%%%%%%%%%%%%%%%%%%%%%%%%%%%%%%%%%%%%%%%%%%%%%
\subsection{Very Important Facts}

    \begin{itemize}
    
    \item The important part of a Noncommutative Quantum Mechanical system is $\mathcal H_q$:
    
        \begin{itemize}
            
        \item The pure \emph{physical states} are elements $|\psi) \in \mathcal H_q$
        
        \item Normal states are density matrices on here, like $|\psi)(\psi| \in \mathcal B(\mathcal H_q)$
        
        \item ``Algebra states'' are states of $\mathcal A = \mathcal H_q$ as algebra, so we are thinking of this, I think, as the NC space.
        
        \item The weak version that we have of the Spectral theorem applies to this space, i.e. \rtext{we can do weak measurements of the position of the (pure) states on $\mathcal H_q$}.
        
        \item After all, it is \rtext{this space that is a (unitary) representation of the (noncommutative version of the) Heisenberg algebra}.
        
        \item BUT In this paper they say at some point that we are ``calculating the distance in $\mathcal H_c$'' (pg. 5)
        
        \item BUT $\mathcal H_q = \mathcal H_c \otimes \mathcal H_c^*$ and ``the quantum state with maximal localization'' is $|z) := |z\rangle\langle z|$, which \dbtext{can be interpreted as a pure physical state in $\mathcal H_q$ but also the density matrix representation of a pure state in $\mathcal H_c$}; furthermore, \dbtext{I understand that the vectors that saturate Heisenberg's relations are the $|z\rangle$'s, I don't know what properties the states $|z)$, if what makes them localized states is ``only'' the PO operators $\pi_z$ associated to them}.
        
        \end{itemize}
    
    \item \rtext{The probability of finding, for the system described by the density matrix $\Omega \in \mathcal B(\mathcal H_q)$, the outcome of position measurement $z = (x_1, x_2)$ is given by the Positive Operator /\emph{``unnormalized projection''} $\pi_z \in \mathcal B(\mathcal H_q)$ by the formula
    \begin{align}
        P(x_1, x_2) = (\psi|\pi_z|\psi) & \text{if } \Omega = |\psi)(\psi|
        P(x_1, x_2) = tr_q(\pi_z \Omega)
    \end{align}
    }
    
        \begin{itemize}
            
        \item Recall that the Spectral theorem says that: for any self-adjoint operator, not just the (tuple of) position operators, we receive a Projection Valued Measure $E_A$: given an outcome/eigenvalue $\lambda$, $E_A(\lambda)$ is a projector to the eigenspace $\lambda$ of $A$, so that the probability of measuring the outcome $\lambda$ for the state \_\_\_\_ has probability:
            \begin{align*}
                \psi \in \mathcal H &&P(\lambda) = \langle \psi | E_A(\lambda) |\psi \rangle \\
                \Omega \in \mathcal B(\mathcal H) &&P(\lambda) = tr(E_A(\lambda) \Omega)\\
                \omega: \mathcal B(\mathcal H) \to \CC &&P(\lambda) = \omega(\cdots?)... \omega(A)?
            \end{align*}
            
        \end{itemize}
    
    \item Associated to a Noncommutative Quantum System there is a natural spectral triple $(\mathcal A = \mathcal H_q, \mathcal H = \mathcal H_c)$.
    
    \item In the fuzzy sphere, the set of \dbtext{generalized coherent states of $SU(2)$} is \emph{topologically} isomorphic to the homogeneous space $SU(2)/U(1) \cong S^2$, but geometrically it reduces to $S^2$ only in the limit $n \to \infty$.
    
    \item They don't use the algorithm to compute the distances in the Moyal plane or the Fuzzy sphere for $n = 1/2$, they follow Martinetti2013, just like in the thesis of Patil. I think the algorithm they provide is used to calculate approximations for the fuzzy sphere $n = 1$.
    
    \end{itemize}

%%%%%%%%%%%%%%%%%%%%%%%%%%%%%%%%%%%%%%%%%%%%%%%%%%%%%%%%%%%%%%%%%%%%%%%%%%%%%
\subsection{Important Facts}

    \begin{itemize}

    \item The Positive Operator Valued measure exists because $|z) \in \mathcal H_q$ provides an over-complete basis of $\mathcal H_q$, so there is a resolution of the identity $1_{q}$ as an integral which is the sum of the Positive operators $\pi_z$ which form a complete basis \dbtext{of where} and therefore provide a POVM.
    
    \end{itemize}

%%%%%%%%%%%%%%%%%%%%%%%%%%%%%%%%%%%%%%%%%%%%%%%%%%%%%%%%%%%%%%%%%%%%%%%%%%%%%
\subsection{Memorize}

    \begin{itemize}

    \item 
    
    \end{itemize}

%%%%%%%%%%%%%%%%%%%%%%%%%%%%%%%%%%%%%%%%%%%%%%%%%%%%%%%%%%%%%%%%%%%%%%%%%%%%%
\subsection{Doubts}

    \begin{itemize}

    \item 
    
    \end{itemize}

%%%%%%%%%%%%%%%%%%%%%%%%%%%%%%%%%%%%%%%%%%%%%%%%%%%%%%%%%%%%%%%%%%%%%%%%%%%%%
\subsection{Detailed summary}

    \begin{itemize}

    \item 
    
    \end{itemize}

%%%%%%%%%%%%%%%%%%%%%%%%%%%%%%%%%%%%%%%%%%%%%%%%%%%%%%%%%%%%%%%%%%%%%%%%%%%%%
\subsection{Notice}

    \begin{itemize}

    \item 
    
    \end{itemize}

%%%%%%%%%%%%%%%%%%%%%%%%%%%%%%%%%%%%%%%%%%%%%%%%%%%%%%%%%%%%%%%%%%%%%%%%%%%%% 
\subsection{Yet to understand}

    \begin{itemize}

    \item 
    
    \end{itemize}

%%%%%%%%%%%%%%%%%%%%%%%%%%%%%%%%%%%%%%%%%%%%%%%%%%%%%%%%%%%%%%%%%%%%%%%%%%%%%
%%%%%%%%%%%%%%%%%%%%%%%%%%%%%%%%%%%%%%%%%%%%%%%%%%%%%%%%%%%%%%%%%%%%%%%%%%%%%
\section{Introduction (pg. )}

%%%%%%%%%%%%%%%%%%%%%%%%%%%%%%%%%%%%%%%%%%%%%%%%%%%%%%%%%%%%%%%%%%%%%%%%%%%%%
%%%%%%%%%%%%%%%%%%%%%%%%%%%%%%%%%%%%%%%%%%%%%%%%%%%%%%%%%%%%%%%%%%%%%%%%%%%%%
\section{Review of Hilbert-Schmidt operatorial formulation and the spectral triples for non-commutative spaces $\RR^2_*$ and $S^2_*$ (pg. 3)}

%%%%%%%%%%%%%%%%%%%%%%%%%%%%%%%%%%%%%%%%%%%%%%%%%%%%%%%%%%%%%%%%%%%%%%%%%%%%%
\subsection{Moyal Plane ($\RR^2_*$) (pg. 3)}

%%%%%%%%%%%%%%%%%%%%%%%%%%%%%%%%%%%%%%%%%%%%%%%%%%%%%%%%%%%%%%%%%%%%%%%%%%%%%
\subsection{Fuzzy Sphere (pg. 4.5)}

\subsubsection{Perelomov's $SU(2)$ coherent states (pg. 5)}%%%%%%%%%%%%%%%%%

%%%%%%%%%%%%%%%%%%%%%%%%%%%%%%%%%%%%%%%%%%%%%%%%%%%%%%%%%%%%%%%%%%%%%%%%%%%%%
\subsection{Spectral Triple: Moyal Plane (pg. 5.5)}

%%%%%%%%%%%%%%%%%%%%%%%%%%%%%%%%%%%%%%%%%%%%%%%%%%%%%%%%%%%%%%%%%%%%%%%%%%%%%
\subsection{Spectral Triple: Fuzzy Sphere (pg. 6)}

%%%%%%%%%%%%%%%%%%%%%%%%%%%%%%%%%%%%%%%%%%%%%%%%%%%%%%%%%%%%%%%%%%%%%%%%%%%%%
\subsection{Spectral distances a la Connes (pg. 6.5)}

%%%%%%%%%%%%%%%%%%%%%%%%%%%%%%%%%%%%%%%%%%%%%%%%%%%%%%%%%%%%%%%%%%%%%%%%%%%%%
%%%%%%%%%%%%%%%%%%%%%%%%%%%%%%%%%%%%%%%%%%%%%%%%%%%%%%%%%%%%%%%%%%%%%%%%%%%%%
\section{Towards an algorithm to compute finite distances (pg. 7.5)}

%%%%%%%%%%%%%%%%%%%%%%%%%%%%%%%%%%%%%%%%%%%%%%%%%%%%%%%%%%%%%%%%%%%%%%%%%%%%%
%%%%%%%%%%%%%%%%%%%%%%%%%%%%%%%%%%%%%%%%%%%%%%%%%%%%%%%%%%%%%%%%%%%%%%%%%%%%%
\section{Distance between finitely separated coherent states in $\RR^2_*$ (pg. 9.75)}

%%%%%%%%%%%%%%%%%%%%%%%%%%%%%%%%%%%%%%%%%%%%%%%%%%%%%%%%%%%%%%%%%%%%%%%%%%%%%
\subsection{Infinitesimal distance and optimal element (pg. 12)}

%%%%%%%%%%%%%%%%%%%%%%%%%%%%%%%%%%%%%%%%%%%%%%%%%%%%%%%%%%%%%%%%%%%%%%%%%%%%%
%%%%%%%%%%%%%%%%%%%%%%%%%%%%%%%%%%%%%%%%%%%%%%%%%%%%%%%%%%%%%%%%%%%%%%%%%%%%%
\section{Connes distance between discrete ``harmonic oscillator'' states (pg. 15)}

%%%%%%%%%%%%%%%%%%%%%%%%%%%%%%%%%%%%%%%%%%%%%%%%%%%%%%%%%%%%%%%%%%%%%%%%%%%%%
\subsection{Distance between infintesimally separated discrete ``harmonic oscillator'' states $|n\rangle$ and $|n+1\rangle$ in the Moyal plane (pg. 15)}

%%%%%%%%%%%%%%%%%%%%%%%%%%%%%%%%%%%%%%%%%%%%%%%%%%%%%%%%%%%%%%%%%%%%%%%%%%%%%
\subsection{Distance between finitely separated discrete ``harmonic oscillator'' states $|n\rangle$ and $|m\rangle$ in the Moyal plane (pg. 16)}

%%%%%%%%%%%%%%%%%%%%%%%%%%%%%%%%%%%%%%%%%%%%%%%%%%%%%%%%%%%%%%%%%%%%%%%%%%%%%
%%%%%%%%%%%%%%%%%%%%%%%%%%%%%%%%%%%%%%%%%%%%%%%%%%%%%%%%%%%%%%%%%%%%%%%%%%%%%
\section{Fuzzy Sphere (pg. 17)}

%%%%%%%%%%%%%%%%%%%%%%%%%%%%%%%%%%%%%%%%%%%%%%%%%%%%%%%%%%%%%%%%%%%%%%%%%%%%%
\subsection{Distance Between Discrete States (pg. 17)}

\subsubsection{Infinitesimal Distance (pg. 17)}%%%%%%%%%%%%%%%%%

\subsubsection{Finite Distance (pg. 18)}%%%%%%%%%%%%%%%%

%%%%%%%%%%%%%%%%%%%%%%%%%%%%%%%%%%%%%%%%%%%%%%%%%%%%%%%%%%%%%%%%%%%%%%%%%%%%%
\subsection{Upper bound of the distance between coherent states (pg. 19)}


%%%%%%%%%%%%%%%%%%%%%%%%%%%%%%%%%%%%%%%%%%%%%%%%%%%%%%%%%%%%%%%%%%%%%%%%%%%%%
\subsection{Ball condition in the eigen-spinor basis (pg. 20)}

%%%%%%%%%%%%%%%%%%%%%%%%%%%%%%%%%%%%%%%%%%%%%%%%%%%%%%%%%%%%%%%%%%%%%%%%%%%%%
\subsection{Lower Bound for Infinitesimally Separated Coherent States (pg. 20.8)}

%%%%%%%%%%%%%%%%%%%%%%%%%%%%%%%%%%%%%%%%%%%%%%%%%%%%%%%%%%%%%%%%%%%%%%%%%%%%%
\subsection{The $n=1/2$ fuzzy sphere (pg. 21.5)}

%%%%%%%%%%%%%%%%%%%%%%%%%%%%%%%%%%%%%%%%%%%%%%%%%%%%%%%%%%%%%%%%%%%%%%%%%%%%%
\subsection{Analogy with $\CC P^1$ modle and mixed states (pg. 23.6)}


%%%%%%%%%%%%%%%%%%%%%%%%%%%%%%%%%%%%%%%%%%%%%%%%%%%%%%%%%%%%%%%%%%%%%%%%%%%%%
\subsection{The $n=1$ fuzzy sphere (pg. 24)}

\subsubsection{Ball condition, general strategy to compute infinimum and general forms of $\Delta \rho$ (pg. 24)}%%%%%%%%%%%%%%%%%%%%%

\subsubsection{An improved but realistic estimate of spectral distance}%%%%%%%%%%%%%%%%%%%%%%%%%%%

\subsubsection{Spectral distance using $\Delta \rho_{\perp}$ (pg. 27)}%%%%%%%%%%%%%%%%%%

\textbf{Infinitesimal distance (pg. 27)}

\textbf{Finite distance (pg. 28.9)}




%%%%%%%%%%%%%%%%%%%%%%%%%%%%%%%%%%%%%%%%%%%%%%%%%%%%%%%%%%%%%%%%%%%%%%%%%%%%%
%%%%%%%%%%%%%%%%%%%%%%%%%%%%%%%%%%%%%%%%%%%%%%%%%%%%%%%%%%%%%%%%%%%%%%%%%%%%%
\section{Conclusions (pg. 30)}

%%%%%%%%%%%%%%%%%%%%%%%%%%%%%%%%%%%%%%%%%%%%%%%%%%%%%%%%%%%%%%%%%%%%%%%%%%%%%
%%%%%%%%%%%%%%%%%%%%%%%%%%%%%%%%%%%%%%%%%%%%%%%%%%%%%%%%%%%%%%%%%%%%%%%%%%%%%
\section{Appendix (pg. 31)}

%%%%%%%%%%%%%%%%%%%%%%%%%%%%%%%%%%%%%%%%%%%%%%%%%%%%%%%%%%%%%%%%%%%%%%%%%%%%%
\subsection{Dirac Operator in Moyal plane, its responde to $ISO(2)$ symmetry and a useful identity (pg. 31)}

The Dirac operator on $\mathcal H = \mathcal H_c \otimes \CC^2$ used through the paper is constructed from the Dirac operator in $\mathcal H_q \otimes \CC^2$:
\begin{align*}
    \mathcal D:= \sigma_i \hat P^i,
\end{align*}
TODO understanding the elements of $\mathcal H_q \otimes \CC^2$ as test functions?\todo{No entiendo}

There are many possible equivalent Dirac operators, and for instance I way we can say something like ``Connes distance, is invariant under rotation of the configuration space'' since we can use rotated position coordinates, which in turn gives us rotated $b$'s, which in turn generates a new Dirac operator such that the generated Lipschitz ball is the same, since $||[D, \pi(a)]||_{op}$ doesn't change.

\begin{align*}
    ||[\mathcal D, \pi(a)]||_{op} = \sqrt{\frac{2}{\theta}}||[b^\dagger, \pi(a)]||_{op} = \sqrt{\frac{2}{\theta}}||[b, \pi(a)]
\end{align*}

%%%%%%%%%%%%%%%%%%%%%%%%%%%%%%%%%%%%%%%%%%%%%%%%%%%%%%%%%%%%%%%%%%%%%%%%%%%%%
\subsection{Fuzzy Sphere (pg. 32)}

\begin{align*}
    \frac{1}{r}||[J_+, a]||_{op} &\leq ||[\mathcal D, \pi(a)]||_{op}\\
    \frac{1}{r}||[J_-, a]||_{op} &\leq ||[\mathcal D, \pi(a)]||_{op}\\
    \frac{1}{r}||[J_3, a]||_{op} &\leq ||[\mathcal D, \pi(a)]||_{op}
\end{align*}

For $a = a^\dagger \in B$, $||[D, \pi(a)]||_{op}$
\begin{align}
    ||[J_+, a]||_{op} &\leq r\\
    ||[J_-, a]||_{op} & \leq r
\end{align}

\end{document}