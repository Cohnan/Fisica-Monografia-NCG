\documentclass{article}
\usepackage[utf8]{inputenc}
\usepackage[margin=1in]{geometry}

%%%%%% To use hyperlinks, including the formula ones
\usepackage{hyperref}
\hypersetup{
    colorlinks=true,
    linkcolor=blue,
    filecolor=magenta,      
    urlcolor=cyan,
}

%%%%%% Make paragraphs start with no indentation and leave spaces between paragraphs
\setlength{\parindent}{0em}
\setlength{\parskip}{1em}

%%%%%% Math stuff
\usepackage{amsmath, amssymb}
\usepackage{amsthm}

%%%%%% Mis Codigos

% TODO notes package
\usepackage{xargs}                  % Use more than one optional parameter in a new command
\usepackage[pdftex,dvipsnames]{xcolor}
\input{tools/todoCode}

% Colored text and boxes with my color conventions for highlighting
\usepackage[dvipsnames]{xcolor}
\input{tools/colorCode}

% Math symbols
\usepackage{xparse}
\input{tools/common_math_symbols}

% Physics symbols (vectors, units)
\usepackage{tikz}
\input{tools/physics_macros}

% Theorem environments
\input{tools/theorem_definitions}

%%%%%%
\title{My Summary on Spectral Triples}
\author{Sebastian Camilo Puerto}
\date{July 2020}

%%%%%%%%%%%%%%%%%%%%%%%%%%%%%%%%%%%%%%%%%%%%%%%%%%%%%%%%%%%%%%%%%%%%%%%%%%%%%
%%%%%%%%%%%%%%%%%%%%%%%%%%%%%%%%%%%%%%%%%%%%%%%%%%%%%%%%%%%%%%%%%%%%%%%%%%%%%
%%%%%%%%%%%%%%%%%%%%%%%%%%%%%%%%%%%%%%%%%%%%%%%%%%%%%%%%%%%%%%%%%%%%%%%%%%%%%
%%%%%%%%%%%%%%%%%%%%%%%%%%%%%%%%%%%%%%%%%%%%%%%%%%%%%%%%%%%%%%%%%%%%%%%%%%%%%
\begin{document}

\maketitle

\tableofcontents

%%%%%%%%%%%%%%%%%%%%%%%%%%%%%%%%%%%%%%%%%%%%%%%%%%%%%%%%%%%%%%%%%%%%%%%%%%%%%
\subsection{High Level Summary}

    \begin{itemize}

    \item Canonical Spectral Triple: $(\mathcal A = C^\infty(M), \mathcal H = \Gamma(\Sigma M), )$
    
        \begin{itemize}
            
        \item \rtext{The manifold is recovered, as a topological space, as the spectrum of the algebra}. The metric structure from (absolute value of) the Dirac operator.
        
        \item The phase part of the Dirac operator, in conjunction with the algebra of functions, gives a K-cycle which encodes index-theoretic information. The local index formula expresses the pairing of the K-group of the manifold with this K-cycle in two ways:
        
            \begin{itemize}
                
            \item \rtext{The 'index' part of the index theorem: The 'analytic/global' side}: involves the usual trace on the Hilbert space and commutators of functions with the phase operator
            
            \item \rtext{The 'characteristic class integration' part of the index theorem: the 'geometric/local' side}: involves the Dixmier trace and commutators with the Dirac operator 
            
                \begin{itemize}
                    
                \item A Dixmier trace is s a non-normal[clarification needed] trace on a space of linear operators on a Hilbert space larger than the space of trace class operators, e.g. an compact self-adjoint operator with eigenvalues $1$, $1/2$, $1/3$, \dots has Dixmier trace equal to $1$. 
                    
                \end{itemize}
            
            \end{itemize}
            
        \end{itemize}
    
    \item Inspired from Atiyah-Singer theorem
    
    \item Also called unbounded $K$-cycles, or unbounded Fredholm-modules.
    
    \item Given a spectral triple one can apply several important operations to it. The most fundamental one is the \emph{polar decomposition} $\mathcal D = F|\mathcal D|$ of $\mathcal D$ into 
    
        \begin{itemize}
            
        \item a self adjoint unitary operator $F$ (\emph{the 'phase' of $D$}): \rtext{pairing with $K$-theory} $\Longrightarrow$ \rtext{``Index theorem''}/ Local Index Formula: relating
        
            \begin{itemize}
                
                \item Global/Analytical/Index information with
                
                \item Local/Geometric/Chern class integration(\emph{Dixmier trace, creo}) information
                
            \end{itemize}   
        and,
        
        \item a densely defined positive operator |D|: \rtext{the 'metric' part}
            
        \end{itemize} 
    
    \end{itemize}

%%%%%%%%%%%%%%%%%%%%%%%%%%%%%%%%%%%%%%%%%%%%%%%%%%%%%%%%%%%%%%%%%%%%%%%%%%%%%
\subsection{Very Important Facts}

    \begin{itemize}

    \item 
    
    \end{itemize}

%%%%%%%%%%%%%%%%%%%%%%%%%%%%%%%%%%%%%%%%%%%%%%%%%%%%%%%%%%%%%%%%%%%%%%%%%%%%%
\subsection{Important Facts}

    \begin{itemize}

    \item 
    
    \end{itemize}

%%%%%%%%%%%%%%%%%%%%%%%%%%%%%%%%%%%%%%%%%%%%%%%%%%%%%%%%%%%%%%%%%%%%%%%%%%%%%
\subsection{Memorize}

    \begin{itemize}

    \item 
    
    \end{itemize}

%%%%%%%%%%%%%%%%%%%%%%%%%%%%%%%%%%%%%%%%%%%%%%%%%%%%%%%%%%%%%%%%%%%%%%%%%%%%%
\subsection{Doubts}

    \begin{itemize}

    \item 
    
    \end{itemize}

%%%%%%%%%%%%%%%%%%%%%%%%%%%%%%%%%%%%%%%%%%%%%%%%%%%%%%%%%%%%%%%%%%%%%%%%%%%%%
\subsection{Detailed summary}

    \begin{itemize}

    \item 
    
    \end{itemize}

%%%%%%%%%%%%%%%%%%%%%%%%%%%%%%%%%%%%%%%%%%%%%%%%%%%%%%%%%%%%%%%%%%%%%%%%%%%%%
\subsection{Notice}

    \begin{itemize}

    \item 
    
    \end{itemize}

%%%%%%%%%%%%%%%%%%%%%%%%%%%%%%%%%%%%%%%%%%%%%%%%%%%%%%%%%%%%%%%%%%%%%%%%%%%%% 
\subsection{Yet to understand}

    \begin{itemize}

    \item 
    
    \end{itemize}

%%%%%%%%%%%%%%%%%%%%%%%%%%%%%%%%%%%%%%%%%%%%%%%%%%%%%%%%%%%%%%%%%%%%%%%%%%%%%
%%%%%%%%%%%%%%%%%%%%%%%%%%%%%%%%%%%%%%%%%%%%%%%%%%%%%%%%%%%%%%%%%%%%%%%%%%%%%
\section{Some Document}



\end{document}
