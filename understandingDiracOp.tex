\documentclass{article}
\usepackage[utf8]{inputenc}
\usepackage{amsmath, amssymb}
\usepackage{slashed}

\usepackage{hyperref}

\title{Entendiendo Dirac relacionado a version operador de Metrica en variedad}
\author{Sebastian Camilo Puerto}
\date{May 2020}

\begin{document}

\maketitle

\section{Dirac Operators and Spectral Geometry - V\'arilly 2006}

\emph{Noncommutative geometry asks}: “What is the geometry of the Quantum World?”.

\subsubsection*{Physical Arisal of Dirac Operator}

Quantum field theory considers aggregates of “particles”, which are of two general species, “bosons” and “fermions”. These are described by solutions of (relativistic) wave equations:
    \begin{itemize}
        \item Bosons: Klein–Gordon equation, $(\square + m^2)\phi(x) = \rho_b(x)$ - “source term”;
        \item Fermions: Dirac equation, $(i\slashed\partial - m)\psi(x) = \rho_f (x)$ - “source term”;
    \end{itemize}

where $\slashed\partial = \sum_\mu \gamma^\mu \partial/\partial_{x^\mu}$. In order that \emph{$\slashed \partial$ be ``a square root of $\square$}'' we need $\{ \gamma^\mu, \gamma^\nu \} = 2 \eta^{\mu \nu}$, so $\gamma^\mu$ must be matrices, and there are $4 \times 4$ matrices satisfying this (but there are many others).

\subsubsection*{Algebraic version of metric: Dirac operator}

Point-like measurements are often ruled out by quantum mechanics; thus we replace points
$x \in M$ by coordinates $f \in C(M)$. \emph{The metric distance on a Riemannian manifold $(M, g)$ can be computed in two ways}:
\begin{align*}
    d_g(p, q) := & inf\{ length(\gamma : [0, 1] \to M) : \gamma(0) = p, \gamma(1) = q \} \\
              = & sup\{ |f(p) - f(q)| : f \in C(M), ||\slashed D, f|| \leq 1 \}, \qquad \longleftarrow    
\end{align*}

where $\slashed D$ is a Dirac operator with positive-definite signature (i.e. all $( \gamma^\mu)^2 = 1$ if it exists, so \textbf{the Dirac operator specifies the metric}. $\slashed D$ is an (unbounded) operator on a Hilbert space $H = L^2(M, S)$ of “square-integrable spinors”, and $C^\infty(M)$ also acts on $H$ by multiplication
operators with $||[\slashed D, f ]|| = ||grad\,f||_\infty$.

\subsubsection*{Metric in NCG}

Noncommutative geometry generalizes $(C^\infty (M), L^2(M, S), \slashed D)$ to a \textbf{spectral triple} of the form $(\mathcal A, H, D)$, where $\mathcal A$ is a “smooth” algebra acting on a Hilbert space $H$, $D$ is an (unbounded) selfadjoint operator on $H$, subject to certain conditions: in particular that $[D, a]$ be a bounded operator for each $a \in \mathcal A$. The tasks of the geometer are then:

\begin{enumerate}
    \item \textbf{To describe (metric) differential geometry in an operator language}.
    \item \textit{To reconstruct (ordinary) geometry in the operator framework}.
    \item To develop new geometries with noncommutative coordinate algebras.
\end{enumerate}

The long-term goal is to geometrize quantum physics at very high energy scales, but we
are still a long way from there.

\subsubsection*{TOC}

The general program of these lectures is as follows.

\begin{itemize}
    \item The classical theory of spinors and Dirac operators in the Riemannian case.
    \item The operational toolkit for noncommutative generalization.
    \item Reconstruction: how to recover differential geometry from the operator framework.
    \item Examples of spectral triples with noncommutative coordinate algebras.
\end{itemize}

Hola

\end{document}