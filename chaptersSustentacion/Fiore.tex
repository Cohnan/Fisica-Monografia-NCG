\section{New Fuzzy Spheres}

\begin{frame}{General Construction*} % % % % % % % % % % % % % % % % % % %
    
    $\bullet$  SEq: $H = - \frac{1}{2} \Delta + V(r)$ invariant under $O(D)$. Potential $V(r) \approx V_0 + 2k(r-1)^2$, $k$ large. SEq. approximated by harmonic oscillator equation \then \rtext{low energy spectrum equals to that of $L^2$}.
    
    $\bullet$ $\lbtext{\hcal_{\cut E}} \approx $ solutions of SEq. of energy $\leq \cut E$, $\lbtext{\acal_{\cut E}} = End(\hcal_{\cut E})$.
    
    $\bullet$ Observable by $A \mapsto \lbtext{\cut A} = P_{\cut E} A P_{\cut E}$. In particular, \rtext{noncommuting coordinates}: $[\cut x^i, \cut x^j] \neq 0$, transform contravariantly.
    
    $\bullet$ Sequences: $\cut E = \Lambda(\Lambda + D - 2)$, $k(\Lambda) \geq \Lambda^2 (\Lambda + 1)^2$.
    
    $\bullet$ These spaces are \rtext{low energy effective quantum theories} on $S^{D-1}$ and \rtext{fuzzy spaces} approximating $S^{D-1}$.
    
\end{frame}

\begin{frame}{$D = 2$ Observables $\to$ QM in $S^1$} % % % % % % % % % % % % % % % % % % %
    
    $\chi^+$ and $\chi^-$ generate the $*$-algebra $\acal_\Lambda = End(\hcal_\Lambda)$. The generators $\chi^\pm$ and $\cut L$ satisfy the relations:
\begin{align}\label{equationOperatorFormulaRelationsChiD2}
    (\chi^\pm)^{2\Lambda+1} &= 0, & 
    (\chi^+)^*&= \chi^-, & 
    \prod_{m = \Lambda}^\Lambda (\cut L - m), &= 0 & 
    \cut L^* &= \cut L,
\end{align}
\begin{align}\label{equationOperatorFormulaCommutatorsChiD2}
    [\cut L, \chi^\pm] &= \pm \chi^\pm,&
    [\chi^+, \chi^-] &= - \frac{2\cut L}{k} + \left[ 1 + \frac{\Lambda(\Lambda+1)}{k}\right](\tilde P_\Lambda - \tilde P_{-\Lambda}),
\end{align}
where the projections $\tilde P_m$ are polynomials in $\cut L$. Furthermore,
$\label{equationOperatorFormulaRChiD2}
    \vec \chi^2 = 1 + \frac{L^2}{k} - \left[ 1 + \frac{\Lambda(\Lambda+1)}{k}\right]\frac{\tilde P_\Lambda - \tilde P_\Lambda}{2}.
$

\end{frame}

\begin{frame}{Alternative Descriptions} % % % % % % % % % % % % % % % % % % %

     $\bullet$ By the isomorphism of the Hilbert spaces $\hcal_\Lambda \subset L^2(\RR^2)$ and $V_\Lambda \equiv \CC^{2\Lambda + 1}$ given by $\psi_m \mapsto |\Lambda, m\rangle$. \Then $C^*$-isomorphism $\acal_\Lambda = End(\hcal_\Lambda)$ with $\hat \pi_\Lambda(U(\mathfrak{so}(3))) = End(V_\Lambda)$ with the operator norm, and this isomorphism is $O(2)$-equivariant. Generators: $\cut{\chi^\pm} = f^\chi_{\pm}(\cut L) L^\pm$, with $f^\chi_\pm$ polynomials.
     
     $\bullet$ Similarly, $\psi_m \mapsto e^{im\phi}$ induces a $O(2)$-equivariant $C^*$-isomorphism between $\acal_\Lambda$ and the operators on $\tilde \hcal_\Lambda := span\{e^{im\phi}\}_{|m| \leq \Lambda} \subset L^2(S^1) $, which annihilate $\tilde \hcal_\Lambda^\perp$.



    
    \end{frame}

\begin{frame}{Convergence} % % % % % % % % % % % % % % % % % % %
    Let's take the last characterization of $\acal_\Lambda$, as subset of $\bcal(L^2(S^1))$.

    
    $\bullet$ \rtext{To QM of $S^1$}: The following operators converge strongly: $\cut L \to L$, $\chi^\pm \to e^{\pm i\phi}$ (ladder op.).
    
    $\bullet$ \rtext{To $S^1$}: $\chi^\pm$ as the fuzzy analogs of the operators $e^{\pm i\phi}\cdot$, so $C_\Lambda := \left\{ \sum_{h = -2\Lambda}^{2\Lambda} f_h x^h \right\} \subset \acal_\Lambda \subset \bcal(\hcal)$ fuzzy analog of $C(S^1)$.  
    
    Then, for all $f, g \in B(S^1)$ the following strong limits hold:
\begin{align}
    \hat f_\Lambda \to f\cdot && \hat{(fg)}_\Lambda \to fg\cdot && \hat f_\Lambda \hat g_\Lambda \to fg \cdot\,,
\end{align}
where the notation $f\cdot$ emphasizes $f$ as an operator on $\hcal$.
\end{frame}

\begin{frame}{Coherent States (CS): Strong Systems} % % % % % % % % % % % % % % % % % % %

    $\bullet$ The discrete basis of $\hcal_\Lambda$ is a \textit{strong SCS} (i.e. induces resolution of $1$). Generated by $G = \{S^n e^{i(aL + b)}\}\cong U(1)\times U(1) \rtimes \ZZ_{2\Lambda + 1}$. $(\Delta L)^2 = 0$, $(\Delta \vec \chi)^2 \geq \frac{1}{2}$. Characterized by set of UR's.
    
    $\bullet$ All $SO(2)$-invariant strong (induce resolution of $1_\hcal$) systems of (norm.) coherent states are of the form $\{\omega^\beta_\alpha\}_{\alpha \in \frac{\RR}{2\pi\ZZ}}$, $\beta \in (\frac{\RR}{2\pi\ZZ})^{2\Lambda+1}$, generated by $\omega^\beta = \sum_{m = -\Lambda}^\Lambda \frac{e^{i\beta_m}}{\sqrt{2\Lambda+1}}$. $(\Delta L)^2 = \frac{\Lambda(\Lambda + 1)}{3}$.
    
    $\bullet$ $O(2)$-invariant subfamilies are generated if $\beta_m = \beta_{-m}$. 
    
    $\bullet$ \rtext{The system of coherent states that minimize the spacial dispersion} ($(\Delta \vec \chi)^2 \overset{\Lambda \geq 2}{\leq } \frac{2}{3(\Lambda + 1)}$) \rtext{within the families $SO(2)$-invariant SSCS}:  $\beta = \vec 0$: $O(2)$-invariant, \rtext{in a $O(2)$-equivariant correspondence with points in $S^1$} via $\omega^0_\alpha \leftrightarrow e^{i \alpha}$.
\end{frame}

\begin{frame}{Minimum $(\Delta \vec \chi)^2 = \bra \vec \chi^2 \ket - \bra \vec \chi \ket ^2$**} % % % % % % % % % % % % % % % % % % %

    $\bullet$ The set $\mathcal W^1$ of unit vectors minimizing $(\Delta \vec \chi)^2$ is $O(2)$-invariant.
    
    $\bullet$ Closed formulas hard because $\bra \vec \chi ^2 \ket = cte + O(\frac{1}{\Lambda^2})$.
    
    $\bullet$ But, perhaps the highest eigenvalue vectors of $\chi^1$ approximate them up to that order.
    
    $\bullet$ Another approximation $O(\frac{1}{\Lambda^2})$ is needed to find the eigenvalues and eigenvectors: coefs. of $\chi^1$ eq. $1$. Their dispersion $(\Delta \vec \chi)^2 < \frac{3.5}{(\Lambda+1)^2} \overset{\Lambda \to \infty}{\longrightarrow} 0$.
    
    $\bullet$ For $\chi^1$ with ordered spectrum $\{\alpha^\Lambda_k\}_{|k|<\Lambda}$, 
    $\alpha_{\Lambda+1}^{\Lambda+1} > \alpha_{\Lambda}^{\Lambda} > \alpha_\Lambda^{\Lambda+1} > \alpha_{\Lambda-1}^{\Lambda} > \dots > \alpha_{-\Lambda}^{\Lambda} > \alpha_{-\Lambda-1}^{\Lambda+1}$.
    
\end{frame}