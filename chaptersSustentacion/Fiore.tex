\section{New Fuzzy Spheres}

\begin{frame}{General Construction} % % % % % % % % % % % % % % % % % % %
    
    $\bullet$  SEq: $H = - \frac{1}{2} \Delta + V(r)$ invariant under $O(D)$. Potential $V(r) \approx V_0 + 2k(r-1)^2$, $k$ large. SEq. approximated by harmonic oscillator eq..
    
    $\bullet$ $\lbtext{\hcal_{\cut E}} \approx $ solutions of SEq. of energy $\leq \cut E$, $\lbtext{\acal_{\cut E}} = End(\hcal_{\cut E})$.
    
    $\bullet$ Observable by $A \mapsto \lbtext{\cut A} = P_{\cut E} A P_{\cut E}$. In particular, \rtext{Noncommuting coordinates}: $[\cut x^i, \cut x^j] \neq 0$, transform contravariantly.
    
    $\bullet$ These spaces are \rtext{low energy effective quantum theories} on $S^{D-1}$ and \rtext{fuzzy spaces} approximating $S^{D-1}$.
    
\end{frame}

\begin{frame}{$D = 2$ Observables} % % % % % % % % % % % % % % % % % % %
    
    $\chi^+$ and $\chi^-$ generate the $*$-algebra $\acal_\Lambda = End(\hcal_\Lambda)$. The generators $\chi^\pm$ and $\cut L$ satisfy \todo{does this characterize algebraically the algebra?} the relations:
\begin{align}\label{equationOperatorFormulaRelationsChiD2}
    (\chi^+)^{2\Lambda+1} &= 0, & 
    (\chi^-)^{2\Lambda+1} &= 0, & 
    (\chi^+)^*&= \chi^-, & 
    \prod_{m = \Lambda}^\Lambda (\cut L - m), &= 0 & 
    \cut L^* &= \cut L,
\end{align}
\begin{align}\label{equationOperatorFormulaCommutatorsChiD2}
    [\cut L, \chi^\pm] &= \pm \chi^\pm,&
    [\chi^+, \chi^-] &= - \frac{2\cut L}{k} + \left[ 1 + \frac{\Lambda(\Lambda+1)}{k}\right](\tilde P_\Lambda - \tilde P_\Lambda),
\end{align}
where the projections $\tilde P_m$ are polynomials in $\cut L$. Furthermore,
\begin{equation}\label{equationOperatorFormulaRChiD2}
    \vec \chi^2 = 1 + \frac{L^2}{k} - \left[ 1 + \frac{\Lambda(\Lambda+1)}{k}\right]\frac{\tilde P_\Lambda - \tilde P_\Lambda}{2}.
\end{equation}

\end{frame}

\begin{frame}{Alternative Descriptions} % % % % % % % % % % % % % % % % % % %

     For all $\Lambda \in \NN$, let $T: \hcal_\Lambda \to V_\Lambda$ be the isomorphism of the Hilbert spaces $\hcal_\Lambda \subset L^2(\RR^2)$ and $V_\Lambda \equiv \CC^{2\Lambda + 1}$ generated by the mapping $\psi_m \mapsto |\Lambda, m\rangle$, $m \in \{-\Lambda, \dots, \Lambda\}$. Then, $T$ induces an isomorphism $\tilde T$ of the $C^*$-algebra $\acal_\Lambda = End(\hcal_)$, with the operator norm, is a $C^*$-algebra isomorphic to $\hat \pi_\Lambda(U(\soth)) = End(V_\Lambda)$ with the operator norm, and this isomorphism is $O(2)$-equivariant. Furthermore, the tuples of operators $(\tilde T(\cut{x^1}), \tilde T(\cut{x^2}))$ and $(\tilde T(\chi^1), \tilde T(\chi^2))$ transform $O(2)$-contravariantly.
     
     \begin{theorem}\label{theoremEquivalent*IsomorphismALgebraSphericaleimphiD2}
For all $\Lambda \in \NN$, let $T: \hcal_\Lambda \to \tilde \hcal_\Lambda$ be defined by $\psi_m \mapsto e^{im\phi}$, for $m \in \{-\Lambda, \dots, \Lambda\}$. $T$ induces an isomorphism between the $C^*$-algebra $\acal_\Lambda$ is isomorphic to $\tilde \acal_\Lambda$ in an $O(2)$-equivariant way
\end{theorem}



    
\end{frame}

\begin{frame}{Convergence} % % % % % % % % % % % % % % % % % % %


    
    \label{theoremConvergesToQMD2}
Let $k(\Lambda)$ be any function satisfying inequality \eqref{inequationInequalityNeededforKasFunctionLambdaD2}. Under the identification $T: \hcal_\Lambda \to \tilde \hcal_\Lambda \subset \hcal$, then $\acal_\Lambda \subset \acal = \bcal (\hcal)$. Then the operators $\cut L $ converges strongly
\footnote{Ignoring domain issues, a sequence of operators $\{T_n\}_{n \in \NN}$ in a Banach space $V$ converges strongly to the operator $T$ with if, for all $v \in V$, $\lim_{n \to \infty} ||T_n v - T v|| \to 0$.}
 to $L$, and both $\chi^\pm$ and $\cut{x^\pm}$ converge strongly to $e^{\pm i\phi}$. If the operators $\chi^\pm$ are defined not as in definition \ref{definitionChiPMChiiD2}, but via the exact action on the basis \eqref{equationActionChiD2WithoutOO}, the same convergence applies.
 
    
    Suppose that $k(\Lambda) \geq 2\Lambda(\Lambda+1)(2\Lambda+1)^2$. Then, for all $f, g \in B(S^1)$ the following strong limits hold:
\begin{align}
    \hat f_\Lambda \to f\cdot && \hat{(fg)}_\Lambda \to fg\cdot && \hat f_\Lambda \hat g_\Lambda \to fg \cdot\,,
\end{align}
where the notation $f\cdot$ emphasizes $f$ as an operator on $\hcal$.
\end{frame}

\begin{frame}{Coherent States (CS): Strong Systems} % % % % % % % % % % % % % % % % % % %

    $\bullet$ The discrete basis is a \textit{strong system of CS} (i.e. induce resolution of $1_{\hcal}$). Generated by $G = \{S^n e^{i(aL + b)}\}\cong U(1)\times U(1) \rtimes \ZZ_{2\Lambda + 1}$. $(\Delta L)^2 = 0$, $(\Delta \vec \chi)^2 = \cdots$. Characterized by set of UR's.
    
    $\bullet$ All $SO(2)$-invariant strong (induce resolution of $1_\hcal$) systems of (norm.) coherent states are of the form $\{\omega^\beta_\alpha\}_{\alpha \in \frac{\RR}{2\pi\ZZ}}$, $\beta \in (\frac{\RR}{2\pi\ZZ})^{2\Lambda+1}$, generated by $\omega^\beta = \sum_{m = -\Lambda}^\Lambda \frac{e^{i\beta_m}}{\sqrt{2\Lambda+1}}$. $(\Delta L)^2 = \frac{\Lambda(\Lambda + 1)}{3}$.
    
    $\bullet$ $O(2)$-invariant subfamilies are generated if $\beta_m = \beta_{-m}$. 
    
    $\bullet$ \rtext{The system of coherent states that minimize the spacial dispersion} ($(\Delta \vec \chi)^2 \overset{\Lambda \geq 2}{\leq } \frac{2}{3(\Lambda + 1)}$) \rtext{within the families $SO(2)$-invariant SSCS}:  $\beta = \vec 0$: $O(2)$-invariant, in a $O(2)$-equivariant correspondence with points in $S^1$ via $\omega^0_\alpha \leftrightarrow e^{i \alpha}$.
\end{frame}

\begin{frame}{Minimum $(\Delta \vec \chi)^2 = \bra \vec \chi^2 \ket - \bra \vec \chi \ket ^2$} % % % % % % % % % % % % % % % % % % %

    $\bullet$ The set $\mathcal W^1$ of unit vectors minimizing $(\Delta \vec \chi)^2$ is $O(2)$-invariant.
    
    $\bullet$ Closed formulas hard because $\bra \vec \chi ^2 \ket = cte + O(\frac{1}{\Lambda^2})$.
    
    $\bullet$ But, perhaps the highest eigenvalue vectors of $\chi^1$ approximate them up to that order.
    
    $\bullet$ Another approximation $O(\frac{1}{\Lambda^2)}$ is needed to find the eigenvalues and eigenvectors: coefs. of $\chi^1$ eq. $1$. Their dispersion $(\Delta \vec \chi)^2 < \frac{3.5}{(\Lambda+1)^2} \overset{\Lambda \to \infty}{\longrightarrow} 0$.
    
    $\bullet$ For $\chi^1$ with ordered spectrum $\{\alpha^\Lambda_k\}$, 
    $\alpha_{\Lambda+1}^{\Lambda+1} > \alpha_{\Lambda}^{\Lambda} > \alpha_\Lambda^{\Lambda+1} > \alpha_{\Lambda-1}^{\Lambda} > \dots > \alpha_{-\Lambda}^{\Lambda} > \alpha_{-\Lambda-1}^{\Lambda+1}$.
    
\end{frame}