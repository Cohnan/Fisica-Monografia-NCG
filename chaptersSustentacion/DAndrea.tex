\section{The Fuzzy Sphere}

\begin{frame}{Definitions} % % % % % % % % % % % % % % % % % % %
    
    - \textbf{Fuzzy Space}: %($C^*$? or simply $*$?) 
    sequence $\acal_n$ parametrized that approximate the commutative algebra $\acal$. Why? \rtext{Keep continuous symmetries}.
    
    - \textbf{Fuzzy Sphere}: Irrep. $\pi_j$ of $SU(2)$ of spin $\frac{N}{2} = j \in \frac{\mathbb N}{2}$: $V_j \equiv \CC^{2j+1} \ni |j, m\ket$. $\lbtext{\acal_N}:= End(V_j) \equiv M_N(\CC)$. $\hat x^i = \frac{1}{\sqrt{j(j+1)}} \pi_{j}(J_i)$, $[J_i, J_k] = i \epsilon_{ijk} J_k$ \then \rtext{$[x^i, x^j] = \frac{1}{\sqrt{j(j+1)}} i \epsilon_{ijk} x_k$}.
    
    - \rtext{It approximates $S^2$ as}: \rtext{1.} $C^*$-algebra $\acal$ acting on the spinors $\hcal$; \rtext{2.} Representation of $SU(2)$, acting by diffeomorphisms; \rtext{3.}  Metric space on which $SU(2)$ acts by isometries.
    
    - \textbf{$SU(2)$-coherent states}: minimum uncertainty states $\lbtext{|\phi, \theta)_N} := R_{(\phi, \theta)}|j, -j\rangle \in V_j$ \iff\ states \lbtext{$\psi^N_{(\phi, \theta)}$} \iff\ points of $S^2$ \rtext{$SU(2)$-equivariantly}, with \rtext{$SU(2)$-invariant distance}.
    
\end{frame}


\begin{frame}{Spectral Triples} % % % % % % % % % % % % % % % % % % %
    
     - Canonical
     
     - Irreducible
     
     - Full
     
     - Theorem
    
\end{frame}


\begin{frame}{Distance Calculation Procedure} % % % % % % % % % % % % % % % % % % %
    
     - General facts
     
     - Perhaps isometry group or desired spectrum
     
     - Docs and ~Slide 17
    
\end{frame}