\documentclass[12pt]{report}

\usepackage[utf8]{inputenc}
\usepackage[spanish, english]{babel}
\usepackage{graphicx}
\usepackage{amssymb, amsmath, amsthm}
\usepackage{bm} % For boldface greek letters
\usepackage{hyperref} % Para poder tener hyperlinks
\usepackage{enumerate}
\usepackage{calligra} % Para dedicacion
\usepackage[backend=bibtex,
%style=abbrv,
%style=numeric,
%style=alphabetic,
%style=reading,
sorting=none,
firstinits=true
]{biblatex} \addbibresource{bibliografia.bib}
\usepackage{csquotes}

\usepackage[normalem]{ulem} % For strikethrough with command \sout
\usepackage{slashed} % For slashed in Dirac operator


\newcommand{\gtext}[1]{\textcolor{gray}{#1}}
%%%%%% Mis Codigos

% TODO notes package
\usepackage{xargs}                  % Use more than one optional parameter in a new command
\usepackage[pdftex,dvipsnames]{xcolor}
\input{tools/todoCode}

% Colored text and boxes with my color conventions for highlighting
\usepackage[dvipsnames]{xcolor}
\input{tools/colorCode}

% Math symbols
\usepackage{xparse}
\input{tools/common_math_symbols}

% Physics symbols (vectors, units)
\usepackage{tikz}
\input{tools/physics_macros}

% Theorem environments
% \input{tools/theorem_definitions}

\usepackage{tikz-cd} %To do Commutative Diagrams

% Extras
\newcommand{\linea}{\rule{10cm}{1mm}}
\newcommand{\lin}{\rule{5cm}{0.5mm}}
\usepackage{ulem}
\renewcommand{\theequation}{\thechapter.\arabic{equation}} % To label Ch 0 equations as 0.1, 0.2, etc

%%%%%%%
\newenvironment{dedication}
  {\clearpage           % we want a new page
   \thispagestyle{empty}% no header and footer
   \vspace*{\stretch{1}}% some space at the top 
   \raggedleft          % flush to the right margin
  }
  {\par % end the paragraph
   \vspace{\stretch{3}} % space at bottom is three times that at the top
   \clearpage           % finish off the page
  }

%%%%%%%
\graphicspath{ {images/} }

%%%%%%%
\theoremstyle{definition}
\newtheorem{example}{Example}[section]
\newtheorem{definition}[example]{Definition}
\newtheorem{theorem}[example]{Theorem}
\newtheorem{proposition}[example]{Proposition}
\newtheorem{lemma}[example]{Lemma}
\newtheorem{corollary}[example]{Corollary}
\newtheorem{notation}[example]{Notation}
\newtheorem{remark}[example]{Remark}

\usepackage{environ}

\NewEnviron{eqnsplit}{%
  \begin{equation}
  \begin{split}
    \BODY
  \end{split}
  \end{equation}
}

\NewEnviron{eqnsplit*}{%
  \begin{equation*}
  \begin{split}
    \BODY
  \end{split}
  \end{equation*}
}

% Compile document without proofs
\usepackage{comment}

%\excludecomment{proof}
%%%%%%%

%\DeclareMathOperator{\HH}{\bb H}
%\DeclareMathOperator{\FF}{\bb F}
% \DeclareMathOperator{\diffeo}{\cong}
% \DeclareMathOperator{\tr}{tr}

% \DeclareMathOperator{\Diff}{\textit{Diff}}
% \DeclareMathOperator{\Hom}{\textit{Hom}}
% \DeclareMathOperator{\End}{\textit{End}}
% \DeclareMathOperator{\Alt}{\textit{Alt}}
% \DeclareMathOperator{\Sym}{\textit{Sym}}


\newcommand{\alg}[1]{\mathfrak{#1}}

\DeclareMathOperator{\comp}{\circ}

\newcommand{\inv}{{-1}}
\newcommand{\conj}[1]{\overline{#1}}

\newcommand{\bra}{\langle}
\newcommand{\ket}{\rangle}
\newcommand{\bbra}{\langle\langle}
\newcommand{\kket}{\rangle\rangle}

\newcommand{\hcal}{\mathcal H}
\newcommand{\acal}{\mathcal A}
\newcommand{\bcal}{\mathcal B}
\newcommand{\dcal}{\mathcal D}

\newcommand{\sut}{\alg{su}(2)}
\newcommand{\soth}{\alg{so}(3)}

\newcommand{\cut}[1]{\overline{#1}}

\newcommand{\then}{\ensuremath{\Rightarrow}}

%%%%%%%
\allowdisplaybreaks
\setlength{\parskip}{0.5em} % PARECE QUE NO ES USUAL DEJAR ESTE ESPACIO

\interfootnotelinepenalty=10000

\title 
{
	{Towards the Study of the Metric Properties of the New Fuzzy Spheres of Fiore and Pisacane}\\
	%{\large Universidad de los Andes}\\
	\vspace{1.5cm}
	{\includegraphics[width = 0.6\textwidth]{logo.png}}	
}
\author{Sebastian Camilo Puerto Galindo\\[1cm]{\small Thesis submitted for the degree of Physicist}\\ {\small  Advisor: Prof. Andres Fernando Reyes-Lega Ph.D.}}
\date{December 10, 2020}

%%%%%%%%%%%%%%%%%%%%%%%%%%%%%%%%%%%%%%%%%%%%%%%%%%%%%%%%%%%%%%%%%%%%%%%%%%%%%%
%%%%%%%%%%%%%%%%%%%%%%%%%%%%%%%%%%%%%%%%%%%%%%%%%%%%%%%%%%%%%%%%%%%%%%%%%%%%%%
%%%%%%%%%%%%%%%%%%%%%%%%%%%%%%%%%%%%%%%%%%%%%%%%%%%%%%%%%%%%%%%%%%%%%%%%%%%%%%
%%%%%%%%%%%%%%%%%%%%%%%%%%%%%%%%%%%%%%%%%%%%%%%%%%%%%%%%%%%%%%%%%%%%%%%%%%%%%%

\begin{document}

\pagenumbering{Roman}

\maketitle

\begin{dedication}
{\LARGE\calligra Hola}
\end{dedication}

\begin{abstract}
Noncommutative spaces are strong candidates for the description of the underlying quantum structure of spacetime, and in this document we can see them arise through the introduction of energy cut-offs in a quantum theory. The purpose of this document is to develop a basic understanding of the geometry of noncommutative spaces that may arise in physically plausible situations such as those derived from the introduction of such cut-offs. We first develop a basic understanding of the geometry of noncommutative spaces, in particular their metric properties, through the study of the traditional fuzzy sphere of Madore\cite{Madore}. We then embark on the research of the geometry of the fuzzy spheres recently proposed by Fiore and Pisacane\cite{Fiore2018} through energy cut-offs by studying their definitions and alternative versions, as well as coherent states on them \cite{FioreCoherent2020, FioreXi2020} that play the role of points on these noncommutative spheres. %To do so, in the present text we first review the traditional fuzzy sphere\cite{Madore} and some spectral triples proposed on them\cite{DAndrea2013}. Then, we start the study of the fuzzy spheres proposed by Fiore and Pisacane through energy cut-offs, and present some families of coherent states that will play the role of the points on these spheres.
\end{abstract}

\begin{otherlanguage}{spanish}
\begin{abstract}
Traducir la de ingles si es aceptada. \todo{}
\end{abstract}
\end{otherlanguage}



\newpage

% \chapter*{Acknowledgements}

% \section*{English}
% Hello
% \begin{otherlanguage}{spanish}
% \section*{Español}
% Hola
% \end{otherlanguage}

%%%%%%%%%%%%%%%%%%%%%%%%%%%%%%%%%%%%%%%%%%%%%%%%%%%%%%%%%%%%%%%%%%%%%%%%%%%%%%
%%%%%%%%%%%%%%%%%%%%%%%%%%%%%%%%%%%%%%%%%%%%%%%%%%%%%%%%%%%%%%%%%%%%%%%%%%%%%%
\tableofcontents

\pagenumbering{arabic}

%%%%%%%%%%%%%%%%%%%%%%%%%%%%%%%%%%%%%%%%%%%%%%%%%%%%%%%%%%%%%%%%%%%%%%%%%%%%%%
%%%%%%%%%%%%%%%%%%%%%%%%%%%%%%%%%%%%%%%%%%%%%%%%%%%%%%%%%%%%%%%%%%%%%%%%%%%%%%
% \setcounter{chapter}{-3}
% \chapter{Yet to Undertand}
% \section{Doubts}

\subsection{Deal Changers}

    \begin{itemize}
    
    \item If we leave the $\cut {x^\pm}$ coordinates as the generators, their commutation relations look nothing like those of $\xi^\pm$ up to the claimed order, since $a^2 = 1 + \frac{\cdot}{\sqrt{k}} + \frac{\cdot}{k} + \cdot$... how can I justify (physically) the introduction of $\xi^\pm$, and their use to construct the theory and the use of THEIR commutation relations to define the better generators of $\acal_{\cut E}$?
    
        \begin{itemize}
            
        \item It seems to be true that $a = a(k)$ is just a number that \textbf{multiplies}, and in that case maybe it is not so bad to divide by it to obtain other observables $\xi^\pm$?
            
        \end{itemize}
    
    \item Is this example actually a fuzzy space? Can we formally say that our algebras approximate a commutative space? (I need to know the precise meaning of a fuzzy space)
    
        \begin{itemize}
            
        \item The way Madore justifies that the fuzzy sphere is an approximation of $S^2$ is a actually very similar: define an injection of a subalgebra (of near diagonal elements?) into the commutative algebra, and see how they expand within? SOmething like that.
            
        \end{itemize}
    
    \end{itemize}

\subsection{Medium Importance}

    \begin{itemize}
        
    \item The odd commutation relations of the coordinates appear for energies \textbf{below} the energy cutoff... but isn't that the opposite of what we want (i.e. that below some energy space works as we understand it, but "above" it the odd noncommutativity appear)?... Perhaps understand this as how NC is natural, that it is actually there since the beginning?
    
    \item Is this example actually a fuzzy space? Can we formally say that our algebras approximate a commutative space? (I need to know the precise meaning of a fuzzy space)
        
    \item Is $\cut O$ a good a good version of $O$ in the projected theory? For example, if $O = x$, position, I would think there is indeed a notion of measuring position in my system, even though my probes don't allow a very precise measurement... is this effective observable really encoded by $\cut x =\cut P x \cut P$? {\tiny What bugs me is that the observable $\cut x$ involves the measurement of $x$, and then projecting to the achievable energies, BUT I can't get the effect of measuring $x$, so \rtext{THIS IS NOT, I think, PRECISE}}    
    
    \end{itemize}


\subsection{Little Details}

    \begin{itemize}
    
    \item At leading order in what can the lowest eigenvalues of $H$ be considered rhose of the SHO approximation of equation $9$?
        
    \end{itemize} \label{chp:understand}



%%%%%%%%%%%%%%%%%%%%%%%%%%%%%%%%%%%%%%%%%%%%%%%%%%%%%%%%%%%%%%%%%%%%%%%%%%%%%%
%%%%%%%%%%%%%%%%%%%%%%%%%%%%%%%%%%%%%%%%%%%%%%%%%%%%%%%%%%%%%%%%%%%%%%%%%%%%%%
%%%%%%%%%%%%%%%%%%%%%%%%%%%%%%%%%%%%%%%%%%%%%%%%%%%%%%%%%%%%%%%%%%%%%%%%%%%%%%
\setcounter{chapter}{0}
\chapter{Introduction}\label{chp:intro}
\input{chapters/introduction}

%%%%%%%%%%%%%%%%%%%%%%%%%%%%%%%%%%%%%%%%%%%%%%%%%%%%%%%%%%%%%%%%%%%%%%%%%%%%%%
%%%%%%%%%%%%%%%%%%%%%%%%%%%%%%%%%%%%%%%%%%%%%%%%%%%%%%%%%%%%%%%%%%%%%%%%%%%%%%
%%%%%%%%%%%%%%%%%%%%%%%%%%%%%%%%%%%%%%%%%%%%%%%%%%%%%%%%%%%%%%%%%%%%%%%%%%%%%%
\chapter{Some Spectral Triples on the Fuzzy Sphere}\label{chp:fuzzysphere}
The $2$-sphere $S^2$ as a metric space is $SU(2)$-invariant. This invariance will be seen to give rise to a series decomposition of the canonical Dirac operator on $S^2$. We will see how truncating this $SU(2)$-decomposition of the Dirac operator will give rise to equivariant spectral triples on the fuzzy sphere which inherit some of the nice properties of the canonical spectral triple and which allow the convergence of the fuzzy sphere to $S^2$ also as a metric space, in a precise sense to make precise in section \ref{FSsec:convergence}.

\dbtext{Physical applications of the Fuzzy sphere.}

%%%%%%%%%%%%%%%%%%%%%%%%%%%%%%%%%%%%%%%%%%%%%%%%%%%%%%%%%%%%%%%%%%%%%%%%%%%%%%
%%%%%%%%%%%%%%%%%%%%%%%%%%%%%%%%%%%%%%%%%%%%%%%%%%%%%%%%%%%%%%%%%%%%%%%%%%%%%%
\section{Canonical Spectral Triple of $S^2$}

-- Arising from $SU(2)$ equivariant algebraic Dirac element, in the context of $G$ equivariant spectral triples.

\linea 

Presentation:

(Inherits from $\RR^3$ the metric and the metric connection, but the spinor space changes)

\rtext{Starting point: $SU(2)$-isometries}%\cite{DAndrea2013}
- $S^2$ as the symmetric space $S^3/S^1$ of the compact semisimple Lie group $G = S^3$, $\mathfrak g = su(2)$.
    
- The canonical spectral triple, which is \textbf{$SU(2)$-equivariant} can be seen to come from a purely algebraic element $\lbtext{\mathcal D} \in U(\mathfrak g) \otimes U(\mathfrak g)$:
    \begin{align*}
        U(\mathfrak g) \otimes U(\mathfrak g) &\to& U(\mathfrak g) \otimes Cl(\mathfrak g, -K) &\to& \mathcal B(L^2(G/U, \Sigma G/U)) \\
        1 \otimes 1 + 2 \sum_{k = 1}^3 J_k \otimes J_k &\mapsto& 1 \otimes 1 + \sum_k J_k \otimes \sigma_k &\mapsto& \rtext{\left( \bigoplus_{l\in \bb N} \pi_l \right) \otimes \pi_{1/2}(\mathcal D)}
    \end{align*}
    
(- In Sanchez: $\mu = 1, 2$
 $\nabla_\mu = \partial_\mu - \frac{c_\mu}{q} \sigma_\mu \sigma_{\cut \mu} \cdot$, 
 $\slashed D = -i \frac{q}{2} \sigma^\mu \nabla_\mu)$)

- So $\lbtext{\slashed D} = \begin{pmatrix} 1 + \partial_H & \partial_F \\ \partial_E & 1 - \partial_H\end{pmatrix} = 1 + \partial_F \otimes \sigma_+ + \partial_E \otimes \sigma_- + \partial_H \otimes \sigma_3$ where $\partial_H = -i \partial_\phi$, $\partial_F = e^{i\phi} \left( \partial_\theta + i cot\,\theta \partial_\phi \right)$, $\partial_E = -\partial_F% = e^{-i\phi} \left( \partial_\theta - i cot\,\theta \partial_\phi \right)
$ are the actions of $J_3$, $J^\pm \in i\,su(2)$ on $L^2(S^2)$ respectively. \textit{Eig. vectors}: orth. basis of $\hcal$; \textit{Spectrum} $= \{\pm l\} = \ZZ - 0$ with multiplicities $2l$. (The tensor with $2x\times 2$ matrices means that we are looking at spinors as column vectors $\psi = \begin{pmatrix} \cdot \\ \cdot \end{pmatrix}$)%: L^2(S^2) \otimes \CC^2 \to L^2(S^2) \otimes \CC^2 = \oplus_{l \in \bb N} (\pi_l \otimes \pi_{1/2})(\mathcal D)$

( - The eigenvectors (spinor harmonics) $Y^{'}_{lm}, Y^{''}_{lm}$, $l \in \NN + 1/2$, of $D$ make up an orthogonal basis of the spinor fields $\hcal$.
 - The eigenvalues are: of $Y'_{l\cdot}: (l + 1/2) \in \NN$, of $Y{''}_{l\cdot}: -(l + 1/2)$, with multiplicities $2l+1$)

\linea

Although there are several ways to describe the canonical spectral triple on \todo{of/on?} $S^2$, the one that will be useful to use to define the spectral triples on the fuzzy sphere will be from understanding $S^2$ as \ptext{compact Riemannian symmetric\footnote{Symmetric space; U} space} $S^2 \cong G/U$ of the compact semisimple Lie group $G = SU(2)$.

Let $G$ be a compact semisimple Lie group $G$ with Lie algebra $\alg g$, where the semisiplicity means that the Killing form $K: \alg g \times \alg g \to \alg g$ of $\alg g$ is negative definite, giving $G$ a natural Riemannian manifold structure. For such a manifold $G$, and its symmetric spaces, the Dirac operator can be seen to arise \ref{} from an algebraic element $\lbtext{\dcal } \in U(\alg g)\times U(\alg g)$, where $\lbtext{U}(g)$ is \lbtext{the universal enveloping algebra} of the Lie algebra $\alg g$, i.e. the largest\todo{in what sense} \todo{complex?} unital, associative algebra containing $\alg g$ where the Lie bracket coincides with the commutator in $U(\alg g)$; it is important to remark that \otext{the representations of $\alg g$ are in a bijective correspondence with the modules over $U(\alg g)$}. In the explicit case where $G = SU(2)$,
\begin{equation}
    \lbtext{\dcal} := 1 \otimes 1 + 2 \sum_{k = 1}^3 J_k \otimes J_k \in U(\sut) \otimes U(\sut)
\end{equation}
where $J_k \in U(\sut), k = 1, \dots, 3$ are the basis of $\sut\otimes \CC$ such that $[J_i, J_j] = i \epsilon_{ijk} J_k$ (i.e. $J_k = \frac{\sigma_k}{2} = \pi_{1/2}(J_k)$). 

\lin

Let $\lbtext{\pi_j}: U(\sut) \to M_{2j+1} = End(\CC^{2j+1})$ be the spin $j \in \frac{\NN}{2}$ representation of $\sut$ where each element of the canonical basis $e_m \in \CC^{2j+1}$, $m = -j, \dots, j-1, j$ satisfies $\pi_j(\vec J^2)(e_m) = j(j+1) e_m$ and $\pi_j(J_3)(e_m) = m e_m$, where $\vec J^2$ is notation for $J_1^2 + J_2^2 + J_3^3 \in U(\sut)$.

Recall that the every continuous / square integrable / smooth function on $S^2$ may be decomposed in terms of spherical harmonics $Y^l_m \in L^2(S^2)$, $l \in \NN$, $m \in \{-l, \dots, l-1, l\}$, and the spaces $\tilde V_l = span\{Y^l_m\}_{|m| \leq l}$ for fixed $l$ are the spaces of homogeneous polynomials on the coordinates $x^1, x^2, x^3$ of $\RR^3$, which are precisely all the irreducible representation spaces of $SO(3)$, and another description of the spin-$l$ representations $V_l$ of $SU(2)$. 
This decomposition of $L^2(S^2)$ is in fact the one induced by the action of $SU(2)$, inherited from the action on $S^2$, and which can be seen to be induced by the natural action of $\sut$ on $S^2$ as vector fields. In particular, 
\begin{align}
    \partial_H &:= - i \partial_\phi & \text{is the action of $\lbtext{H} := J_3$ in $L^2(S^2)$} \\
    \partial_F &:= e^{i\phi} \left( \partial_\theta + i cot\,\theta \partial_\phi \right) &\text{is the action of $\lbtext{F} := J_+ = J_1 + i J_2$ in $L^2(S^2)$}\\
    \partial_E &:= - \partial_E &\text{is the action of $\lbtext{E} := J_- = J_1 - i J_2$ in $L^2(S^2)$}
\end{align} where $\phi$ and $\theta$ are the azimuthal and polar angles in $S^2$, respectively.
Hence, $L^2(S^2) \cong \bigoplus_{l = 0}^\infty V_l$ as representation spaces of $SU(2)$.

The spinor bundle of $S^2$ is trivial, so the space of spinor fields $\hcal = L^2(S^2, \Sigma S^2)$ of $S^2$ is isomorphic (as a vector space) to $L^2(S^2) \otimes \CC^2$, where $\CC^2$ is understood as the \textit{fermionic Fock space} associated to the tangent spaces of $S^2$, i.e. the unique irreducible representation of the Clifford algebra $\CC l_2 = M_2(\CC) = \pi_{1/2}(U(\sut))$. Therefore, \rtext{the spinor bundle of $S^2$ decomposes as representation of $SU(2)$ as $L^2(S^2, \Sigma S^2) \cong \bigoplus_{l = 1}^\infty V_l \otimes V_{1/2}$}.

\lin

This decomposition of the spinor bundle induced by the action of $SU(2)$ correctly suggests that the Dirac operator can be seen as the operator
\begin{equation}
    \lbtext{\slashed D} = \rtext{ \bigoplus_{l\in \bb N} \pi_l  \otimes \pi_{1/2}(\dcal)} = 1 + \partial_F \otimes \sigma_+ + \partial_E \otimes \sigma_- + \partial_H \otimes \sigma_3 = \begin{pmatrix} 1 + \partial_H & \partial_F \\ \partial_E & 1 - \partial_H\end{pmatrix} ,
\end{equation} acting on $L^2(S^2)\otimes \CC^2$, where $\sigma \pm = \sigma_1 \pm i \sigma_2$.

The eigenvectors of $\slashed D$ $Y^{'}_{jm}, Y^{''}_{jm}$ $j \in \NN + 1/2$, $m = -j, \dots, j$, called the \textit{spinor harmonics}, make up an orthogonal basis of the spinor fields $\hcal$. The eigenvalue of the $Y'_{l\cdot}$ is $(j + 1/2) \in \NN$, and of the $Y{''}_{j\cdot}: -(l + 1/2)$, so the eigenvalue $j$ has multiplicity $2j+1$. The spectrum of $\slashed D$ (and the respective multiplicities) will be an indication of how good of an approximation is another Dirac operator in the Fuzzy sphere. 


\linea

(The following was previously right after the paragraph about the representations of $SU(2)$, but may now be unnecesary)

Now, to $\dcal \in U(\alg g) \otimes U(\alg g)$ corresponds an element $\dcal_S$ in the noncommutative Weil algebra\footnote{CCR algebra of $(\alg g, \omega?)$. } through the injection $\alg g \hookrightarrow Cl(g, K)$/$\alg g_\CC \hookrightarrow \CC l(G/U)?$\todo{not sure we want here the real algebra, and I even think we want instead the complex algebra since it is that one which acts on Fock space}\todo{not even sure if we want this CLifford algebra, of the one before, since it is that one which acts on the spinors on $S^2$, and so it is true that $\CC l_2 = M_2(\CC)$}\footnote{Clifford algebra}. It is known that, for a $3$ dimensional real vector space $V$ with positive definite metric with basis $\{v_k\}_{k = 1, \dots, 3}$, the real Clifford algebra $Cl(V, g) \cong gen\{\sigma_k\}_{k = 1, \dots, 3} = gen\{J_k\}_{k = 1, \dots, 3} = M_2(\CC)$ and $\CC l(V) \cong M_2(\CC) \oplus M_2(\CC)$ under the identification $v_k \Longleftrightarrow \sigma_k \in M_2(\CC)$, meaning, for our explicit example, that the algebra $Cl(\sut, -K) = M_2(\CC)$ 
\todo{there is an $i$ which I don't like when saying this last sentence}, and so the injection can be seen to be simply $\pi_{1/2} = Id_2$:
\begin{equation}
    \dcal_S := (Id_2 \otimes \pi_{1/2})(\dcal) = 1 \otimes 1 + \sum_{k = 1}^3 J_k \otimes \sigma_k = \begin{pmatrix} 1 + H & F \\ E & 1 - H \end{pmatrix} \in U(\sut)\otimes M_2(\CC)
\end{equation}
where $H = J_3$, $E = J_1 + iJ_2$ and $F = E^* \in U(\sut)$


Equivariance of $\dcal$: $U(\alg g)$ is a Hopf algebra\footnote{} with coproduct $\nabla: U(\alg g) \to U(\alg g) \times U(\alg g)$ such that $A \mapsto A \otimes 1 + 1 \otimes A$ for $A \in \alg g$.

\lin 



%%%%%%%%%%%%%%%%%%%%%%%%%%%%%%%%%%%%%%%%%%%%%%%%%%%%%%%%%%%%%%%%%%%%%%%%%%%%%%
%%%%%%%%%%%%%%%%%%%%%%%%%%%%%%%%%%%%%%%%%%%%%%%%%%%%%%%%%%%%%%%%%%%%%%%%%%%%%%
\section{The Fuzzy Sphere}

 -- Change $x_i$ in $\RR^3$ by infinitesimal rotation $J_i$, acting on $\CC^{N+1} = span\{|j, m\rangle\}$.

Presentation:

\textbf{Fuzzy Space}: ($C^*$? or simply $*$?\todo{}) 
    family of noncommutative finite dimensional $\acal_n$ parametrized by $n \in \bb N$ with increasing dimension and such that that approximate the commutative algebra $\acal$. Why? \textit{Keep continuous symmetries}.
        % \begin{itemize}
        % \item Why? To preserve the (continuous) symmetries of the space while keeping the algebra finite dimensional.
        % \end{itemize}
    
    \textbf{Fuzzy Sphere}: Notice that $\acal \cong \bigoplus_{l \in \bb N} V_l \cong L^2(S^2)$, where $V_l$ is the spin $l$ representation of $SO(3)$: homogeneous $l$-degree polynomials in $x^1, x^2, x^3$, with basis $\{Y^l_m\}_{|m| \leq l}$. 
    For $N = 2j \in \bb N$, 
    $\lbtext{\acal_N}= \bigoplus_{l = 0}^N V_l$ as $SU(2)$ representation. 
    As algebra: replacing $x^i \mapsto \frac{1}{\sqrt{j(j+1)}} \pi_{j}(J_i)$, $[J_i, J_k] = i \epsilon_{ijk} J_k$, $\lbtext{\acal_N} := End(V_j) = M_{N+1}(\CC)$, understanding $V_j= \CC^{N+1} = span\{|j, m\ket\}_{|m| \leq j}$ as irrep. of $SU(2)$; this follows from  $\longrightarrow$ \rtext{$[x^i, x^j] = \frac{1}{\sqrt{j(j+1)}} i \epsilon_{ijk} x_k$}, $\sum x_1^2 + x_2^2 + x_3^2 = 1$.
    
    With these Dirac Spectral Triples \cite{DAndrea2013} \rtext{approximates $S^2$ as: \textbf{1.} $C^*$-algebra $\acal$ acting on the spinor fields $\hcal$; \textbf{2.} Representation of $SU(2)$; \textbf{3.}  Metric space on which $SU(2)$ acts by isometries.}
    
    %. Under the adjoint action, which makes sense since R J_3 R^{-1} is rotation under rotated axis
    
    % This allows to define fuzzy spherical harmonics (changing x's by new x's) which make up a basis, good action under SU(2)
    
\linea

%%%%%%%%%%%%%%%%%%%%%%%%%%%%%%%%%%%%%%%%%%%%%%%%%%%%%%%%%%%%%%%%%%%%%%%%%%%%%%
%%%%%%%%%%%%%%%%%%%%%%%%%%%%%%%%%%%%%%%%%%%%%%%%%%%%%%%%%%%%%%%%%%%%%%%%%%%%%%
\section{Spectral Triples}

 --  Truncate the $SU(2)$-equivariant \textbf{decomposition} of the canonical spectral triple

(The tensor with $2x\times 2$ matrices means that we are looking at spinors as column vectors $\psi = \begin{pmatrix} \cdot \\ \cdot \end{pmatrix}$)

%%%%%%%%%%%%%%%%%%%%%%%%%%%%%%%%%%%%%%%%%%%%%%%%%%%%%%%%%%%%%%%%%%%%%%%%%%%%%%
%%%%%%%%%%%%%%%%%%%%%%%%%%%%%%%%%%%%%%%%%%%%%%%%%%%%%%%%%%%%%%%%%%%%%%%%%%%%%%
\subsection{The Irreducible Spectral Triple}
 
 - One term of the canonical spectral triple
 
Presentation:

A first spectral triple is simply taking ``one term'' of $\slashed D$:
\begin{multline}
    \lbtext{D_N} 
    := (\pi_j \otimes \pi_{1/2})(\mathcal D): V_j \otimes \CC^2 \to V_j \otimes \CC^2 \\
    = \begin{pmatrix} 1 + \pi_j(H) & \pi_j(F) \\ \pi_j(E) & 1 - \pi_j(H)\end{pmatrix} 
    = 1 + \pi_j(F) \otimes \sigma_+ + \pi_j(E) \otimes \sigma_- + \pi_j(H) \otimes \sigma_3
\end{multline} where $H = J_3$, $F = J_+$, $E = J_-$ are the actions of $J_3$, $J^\pm \in su(2)$ on $V_j$ respectively.

- The spectral triple \rtext{$(\acal_N, \lbtext{H_N} = V_j \otimes \CC^2, D_N)$}
    \begin{itemize}
        
    \item Is $SU(2)$-equivariant
    
    \item Has eigenvalues $j+1$ and $-j$ with multiplicities $2j+2$ and $2j$.
        
    \item Isn't compatible with a grading or a real structure.
    \end{itemize}
    
\linea




%%%%%%%%%%%%%%%%%%%%%%%%%%%%%%%%%%%%%%%%%%%%%%%%%%%%%%%%%%%%%%%%%%%%%%%%%%%%%%
\subsection{The Full Spectral Triple}

- Truncation of the canonical spectral triple. In particular the Dirac operator ``goes'' to the canonical one.

Presentation:

\rtext{$(\acal_N, \lbtext{\hcal_N} := \mathcal A_N \otimes \CC^2 \cong \bigoplus_{l =1}^N H_l, \mathcal D_N \longleftrightarrow \bigoplus_{l =1}^N D_l)$}, where:
\begin{multline}
    \lbtext{\mathcal D_N} := (\textbf{ad}\pi_j \otimes \pi_{1/2})(\mathcal D) = D = \begin{pmatrix} 1 + \text{ad}\pi_j(H) & \text{ad}\pi_j(F) \\ \text{ad}\pi_j(E) & 1 - \text{ad}\pi_j(H)\end{pmatrix} 
    = \text{ad}\pi_j(E) \otimes \sigma_1 + \text{ad}\pi_j(F) \otimes \sigma_2 + \text{ad}\pi_j(H) \otimes \sigma_3
\end{multline}
where, e.g. $\text{ad}\pi_j(H) = [\pi_J(H), \cdot ]$ is the action of $J_3 \in su(2)$ on $\mathcal A_N$.

    \begin{itemize}
    
    \item It is a real spectral triple.
    
    \item It is $SU(2)$-equivariant
    
    \item Spectrum of $\mathcal D_N$ is the truncation of $\slashed D$ to $\{-N, \dots, N+1\}$. The eigenvalues of $\mathcal D_N$ are $N+1$ with multiplicity $2N+2$, and $\pm l$ with multiplicity $2l$ for $l = 1, \dots, N$
    
    \item It is not compatible with a grading.
    
    \end{itemize}

\rtext{It is a truncation of the canonical spectral triple}: The spectrum and the multiplicites 

- \rtext{\textbf{Theorem}}: the two spectral triples induce the same distance in $\acal_N$. Pf: $[\dcal, a]b \otimes v = \cdots = \sum_k [\pi_j(J_k), a]b \otimes \sigma_k v = [D_N, a] \cdot b \otimes v$

\linea

%%%%%%%%%%%%%%%%%%%%%%%%%%%%%%%%%%%%%%%%%%%%%%%%%%%%%%%%%%%%%%%%%%%%%%%%%%%%%%
%%%%%%%%%%%%%%%%%%%%%%%%%%%%%%%%%%%%%%%%%%%%%%%%%%%%%%%%%%%%%%%%%%%%%%%%%%%%%%
\section{Coherent States}

-- Fuzzy approximations of points in S2. In QM they are the closes thing to classical states we have (minimum uncertainty states for some observables).

- \dbtext{Are they pure}? They are vector states, I think that suffices to make them pure.

%%%%%%%%%%%%%%%%%%%%%%%%%%%%%%%%%%%%%%%%%%%%%%%%%%%%%%%%%%%%%%%%%%%%%%%%%%%%%%
\subsection{SHO Coherent States}

3 characterizations. Heisenberg group.
%%%%%%%%%%%%%%%%%%%%%%%%%%%%%%%%%%%%%%%%%%%%%%%%%%%%%%%%%%%%%%%%%%%%%%%%%%%%%%
\subsection{$SU(2)$-Coherent States}

 - Rotations of $|j, -j \rangle$
 
Presentation:

- The Bloch/$SU(2)$-\rtext{coherent states} are for the group $SU(2)$ what the usual harmonic oscillator coherent states are for the Heisenberg group. In particular, they are minimum uncertainty states. 

- \rtext{They will be considered fuzzy approximations of the points of $S^2$}.

- \rtext{Points of the $E(\phi, \theta) \in S^2$ sphere are approximated by coherent vectors} $\lbtext{|\phi, \theta)_N} := R_{(\phi, \theta)}|j, -j\rangle \in V_j \Longleftrightarrow $ states of $\mathcal A_N = End(V_j)$ $\lbtext{\psi^N_{(\phi, \theta)}} = (\phi, \theta| \cdot |\phi, \theta)_N \in \mathcal S(\mathcal A_N)$. 

- \rtext{This identification of points in $S^2$ with coherent states ($\psi^N: \acal_N \to L^2(S^2)$ at the algebra level\todo{})
is $SU(2)$-equivariant}: $g_* \psi^N_{(\phi, \theta)} = \psi^N_{g\cdot (\phi, \theta)}$, \& \rtext{the distance between them is $SU(2)$-invariant}.

What group used in Fiore? $O(D)$

- In Fiore2020 there is a simple introduction to Coherent states: group $G$ acting (irreducibly) on a Hilbert space. AND the example for $G = SU(2)$ is sketched, although there is a mistake for one of the $3$ characterizations, since it isn't true that the coherent states are eigenvectors of the ``annihilation operator'' $J_+$. D'Andrea mentions other sources.

\linea
%%%%%%%%%%%%%%%%%%%%%%%%%%%%%%%%%%%%%%%%%%%%%%%%%%%%%%%%%%%%%%%%%%%%%%%%%%%%%%
%%%%%%%%%%%%%%%%%%%%%%%%%%%%%%%%%%%%%%%%%%%%%%%%%%%%%%%%%%%%%%%%%%%%%%%%%%%%%%
\section{Distance Between Famillies of \dbtext{Pure} States}

Presentation:

\textit{From now on}: we use the irreducible s.t. and not write the $\pi_j$'s.

- The supremum $d_D(\omega, \omega')$ is always attained in hermitian elements.

- For $a \in \acal_N$, 
\begin{equation} \label{ineqDN}
    ||[H, a]||, ||[E, a]||, ||[F, a]||  \leq ||[D_N, a]||;
\end{equation} \label{eqDNdiag}
if $a$ is diagonal hermitian, then
\begin{equation}
    ||[E, a]|| = ||[D_N, a]||.
\end{equation}

- A state may be defined only on hermitian elements of $\acal$, since from there it can be uniquely extended to all $\acal$: $a = \frac{a+a^*}{2} + \frac{a - a^*}{2} \in Herm. + Antiherm. = Herm. + i\, Herm.$

\linea

- \textbf{Procedure}: Understand $||[D, a]||$; find upper limit for $|\omega(a) - \omega(a')|$ dependent on $||[D, a]||$; find hermitian algebra element that saturates the inequality (or sequence that get close) (if inequalities depending on $||[D, a]||$ are found, this element will need to have a maximizing $||[D, a]||$, probably $1$). 

Very similar procedure to calculate distances followed in Chakraborty Moyal Plane and Chakraborty Fuzzy Sphere.

- \textbf{Discrete Basis States}: In $\acal_N = End(V)$, study vector states of important basis, relating $\omega_m(a) - \omega_n(a)$ to $[D_N, a]$.

- \textbf{$G$-invariance of distance}: if $G$ acts by ``isometries'', simplify what needs to be proven.

- \textbf{Auxiliary distance}: study simpler subalgebra distance: lower bound whose behavior is understood and that encases the actual distance, allowing approximate study.

- \textbf{Behaviour with $N$}: Relate algebras and coherent states of subsequent $N$, finding relation between $||[D_{N+1}, \cdot]||$ and $||[D_N, \cdot]||$.

\linea

%%%%%%%%%%%%%%%%%%%%%%%%%%%%%%%%%%%%%%%%%%%%%%%%%%%%%%%%%%%%%%%%%%%%%%%%%%%%%%
\subsection{$SU(2)$-invariance of the distance}

 - If $G$ acts by ``isometries'', simplify what needs to be proven.

Presentation:

$SU(2) \ni g$ acts on the states: $\lbtext{g_*\omega}(\cdot) := \omega(\cdot^g)$ $\longleftarrow$ it acts on the algebra: $\lbtext{a^g}:= g\circ a \circ g^* \cdot$ $\longleftarrow$ it acts on $V_j$: $\pi_j(g)$.

\textbf{\rtext{Theorem}}: For all $N = 2j \in \bb N$, the distance is $SU(2)$-invariant: 
\begin{align}
    d_N(\omega, \omega') &= d_N(g_* \omega, g_*\omega'), & \text{for all $\omega, \omega' \in \mathcal S(\acal_N)$}.
\end{align}

\textit{Pf}: The theorem follows once we show 
    \rtext{$||[D_N, a^g]|| = ||[D_N, a]||$}:
$d_N(g_* \omega, g_* \omega') = sup_{a \in \acal_N}\{ |\omega(a^g) - \omega'(a^g)|: ||[D_N, a \otimes 1_2]|| \leq 1 \} = sup_{b \in \acal_N}\{ |\omega(b) - \omega'(b)|: ||[D_N, b \otimes 1_2]|| \leq 1 \} = d_N(\omega, \omega')$, where $b = a^g$ sweeps all $\acal_N$.

\textit{Pf}: $[D_N, a^g \otimes 1] = u[D_N, a \otimes 1_2]u^*$ where $u = \pi_j(g) \otimes \pi_{1/2}(g)$ is the induced unitary action of $g$ in $H_N$ $\xleftarrow{}$
1. $a^g \otimes 1_2 = u(a \otimes 1_2)u^*$;
2. The spectral triple is $SU(2)$-equivariant, hence $D_N$ commutes with the $SU(2)$ action.

%%%%%%%%%%%%%%%%%%%%%%%%%%%%%%%%%%%%%%%%%%%%%%%%%%%%%%%%%%%%%%%%%%%%%%%%%%%%%%
\subsection{The $N=1$ distance between coherent states}

 - 
 
 Presentation:
 
$\pi_{1/2}(J_k) = \sigma_k/2$

- Any hermitian element in $\acal_1 = M_2(\CC)$ can be written as
\begin{align*}
    a &= \begin{pmatrix} a_0 + a_3 & a_1 - i a_2 \\ a_1 + i a_2 & a_0 - a_3  \end{pmatrix}= a_0 + \vec a \cdot \vec \sigma, & \text{for $(a_0, \dots, a_3) \in \RR^4$.}
\end{align*}


- Positivity of states implies that, restricted to hermitian elements, they all are  $\omega_{\vec x}(a) = a_0 + \vec x \cdot \vec a$, for $\vec x \in B^3 \subset \RR^3$.

- $\omega_{\vec x}$ is pure $\iff$ $\vec x = (\sin \theta \cos \phi, \sin \theta, \cos \theta) \in S^2$ $\dbtext{\iff}$ $\omega_x = \psi^1_{(\phi, \theta)}$.

\rtext{\textbf{Theorem}}: For $N = 1$ all pure states are coherent states and 
\begin{equation}
    d_1(\hat p, \hat q) = \frac{1}{2}|\vec p - \vec q|_{\RR^3}
\end{equation}
\textit{Pf}: $|\omega_{\vec x}(b) - \omega_{\vec y}(b)| = |(\vec x - \vec y)\cdot \vec b| \leq |\vec x - \vec y||\vec b|$; $i[D_1, a]$ is hermitian with max. eigenvalue/norm $=2 |\vec a|$ $\longrightarrow$ $a \in \acal_1$ hermitian with $\vec a$ parallel to $\vec x - \vec y$ st. $2|\vec a| = 1$ saturates the inequality.

%%%%%%%%%%%%%%%%%%%%%%%%%%%%%%%%%%%%%%%%%%%%%%%%%%%%%%%%%%%%%%%%%%%%%%%%%%%%%%
\subsection{Distance Between (Vector) Discrete States $|j,n\rangle$ for Arbitrary $N$}

Presentation:

\rtext{\textbf{Theorem}}: For any $N$, let $\lbtext{\omega_m} = \langle j, m| \cdot |j, m\rangle \in \mathcal S(\acal_N)$ (SP $\Longleftrightarrow |j, -j\rangle$, NP $\Longleftrightarrow |j, j\rangle$):
\begin{equation}
    d_N(\omega_m, \omega_n) = \sum_{k = m+1}^n \frac{1}{\sqrt{(j+k)(j-k+1)}} = \sum_{k = m+1}^n d_N(\omega_{k-1}, \omega_k)
\end{equation}
%  and so, between the north and south poles
% \begin{equation}
%     d_N(\psi_{(0,0), \psi_{(0, \pi)}}) = \sum_{k = 1}^N \frac{1}{\sqrt{k(N-k+1)}}
% \end{equation}
\textit{Pf}: Recall that $J_\pm|j,m\rangle = \sqrt{(j\mp m)(j\pm m + 1)}|j, m+1 \rangle$ $\longrightarrow$ $\omega_m(a) - \omega_n(a) = \sum_{k = m+1}^n \langle j, k-1 |a|j, k-1 \rangle - \langle j, k |a| j, k \rangle = \sum_{k = m+1}^n \frac{1}{\sqrt{(j+k)(j-k+1)}} \langle j, k| [E, a] |j, k-1 \rangle$; since $|\langle j, k |[E, a]|j, k-1 \rangle|  \leq ||[E, a]|| \leq ||[D_N, a]|| \leq 1$, we get the upper bound. 

Define the diagonal hermitian operator $\hat a |j, m\rangle := - \left( \sum_{k = -j+1}^m  \right)|j, m\rangle$, $\longrightarrow$ $[E, \hat a] |j, k \rangle = |j, k+1\rangle$ ($k < j$), and so $\hat a$ saturates the inequality.

\linea
%%%%%%%%%%%%%%%%%%%%%%%%%%%%%%%%%%%%%%%%%%%%%%%%%%%%%%%%%%%%%%%%%%%%%%%%%%%%%%
\subsection{Relating Distinct $N$'s and Upper Bound}

 - Relate algebras and coherent states of subsequent $N$, finding relation between $||[D_{N+1}, \cdot]||$ and $||[D_N, \cdot]||$.
 
Presentation:

\rtext{\textbf{Theorem}}: the distance $d_N(\psi^N_{(\phi, \theta)}, \psi^N_{(\phi', \theta')})$ is non-decreasing with $N$:
\begin{equation}
    d_N(\psi^N_{(\phi, \theta)}, \psi^N_{(\phi', \theta')}) \leq d_{N+1}(\psi^{N+1}_{(\phi, \theta)}, \psi^{N+1}_{(\phi', \theta')})
\end{equation}

 KEY: Relate $N$ with $N+1$
 
\textit{Pf}: $U^\pm_j |\phi, \theta)_{N+1} = |\phi, \theta)_N \otimes |\phi, \theta)_1$, $\xrightarrow{}$ $\eta^+_N: \acal_N \to \acal_{N + 1}$ $\longrightarrow{}$ $\psi^{N+1}_{(\phi, \theta)} \circ \eta^+_N(a) = \psi^N_{(\phi, \theta)}$ \& $||[D_{N + 1}, \eta^+_N(a)]|| \leq ||[D_N, a]||$.

\rtext{\textbf{Theorem}}: For all $N$, 
\begin{equation}
    \frac{1}{2}|E(\phi, \theta)- E(\phi', \theta')| \leq || \leq d_N(\psi^N_{(\phi, \theta)}, \psi^N_{(\phi', \theta')}) \leq d_{geo}(E(\phi, \theta), E(\phi', \theta')).
\end{equation}

\textit{Pf}: \textbf{use SU(2)-invariance of distance} and only prove it between within the great circle $\theta = \frac{\pi}{2}$: $d_N(\psi^N_{(0, \frac{\pi}{2})}, \psi^N_{(\phi, \frac{\pi}{2})}) \leq |\phi|$. 
$|\psi^N_{(\phi, \frac{\pi}{2})} - \psi^N_{(0, \frac{\pi}{2})}| = |i \int_0^\phi \psi^N_{(\alpha, \frac{\pi}{2})}([H, a]) d\alpha | \leq ||[H, a]|| |\phi| \leq |\phi| ||[D_N, a]||$.

(Dibujito: $N=1$ en un color, other increasing $N's$ (equivalent paths), finally commutative distance).

\linea

%%%%%%%%%%%%%%%%%%%%%%%%%%%%%%%%%%%%%%%%%%%%%%%%%%%%%%%%%%%%%%%%%%%%%%%%%%%%%%
\subsection{Auxiliary Distance and Limit}
 - Study simpler subalgebra distance: lower bound whose behavior is understood and that encases the actual distance, allowing approximate study.

Presentation:

Distance along diagonal, hermitian matrices eassier to understand, but useful.

 KEY: find simpler (we understand [D, a] if a is diagonal hermitian), but useful auxiliary distance (gives a lower bound that sandwhiches the actual distance)
 
Let $\lbtext{\bcal_N} \subset \acal_N$ be sugalgebra of diagonal matrices $\xrightarrow{}$ $\psi^N_{(\phi, \theta)} = \psi^N_{(0, \theta)}$, now 
$\lbtext{\rho_N}(\theta) := sup_{a \in \bcal_N, a = a^*}\{|\psi^N_{(0, \theta)} - \psi^N_{(0, 0)}| : ||[D_N, a]|| \leq 1\}$: \textbf{1.} exact formula related to $\omega_m$ states, \textbf{2.} $0 \leq \rho'_N(\theta) \leq 1$ $\xrightarrow{}$ $\rho_N(\theta) \leq \theta$, \textbf{3.} non-decreasing with $N$.


\textbf{Lemma}: For all $N$, 
\begin{align*}
    1.& \rho_N(\theta - \theta') \leq d_N(\psi^N_{(\phi, \theta)}, \psi^N_{(\phi', \theta')}) d_{geo}(E(\phi, \theta), E(\phi', \theta'))  ; \\
    2.& \text{the sequence $\rho_N(\theta)$ converges uniformly to $\theta$ as $N\to \infty$} \leq .
\end{align*}

\textbf{\rtext{Theorem}}:  
\begin{equation}
        \lim_{N \to \infty} d_N(\psi^N_{(\phi, \theta)}, \psi^N_{(\phi', \theta')}) = d_{geo}(E(\phi, \theta), E(\phi', \theta'))
\end{equation}

%%%%%%%%%%%%%%%%%%%%%%%%%%%%%%%%%%%%%%%%%%%%%%%%%%%%%%%%%%%%%%%%%%%%%%%%%%%%%%
%%%%%%%%%%%%%%%%%%%%%%%%%%%%%%%%%%%%%%%%%%%%%%%%%%%%%%%%%%%%%%%%%%%%%%%%%%%%%%
\section{Convergence of fuzzy sphere to $S^2$}\label{FSsec:convergence}

 - This identification of points in $S^2$ with coherent states ($\psi^N: \acal_N \to L^2(S^2)$ at the algebra level) \todo{???}
is $SU(2)$-equivariant: $g_* \psi^N_{(\phi, \theta)} = \psi^N_{g\cdot (\phi, \theta)}$, \& \rtext{the distance between them is $SU(2)$-invariant}.
 
 - as: \textbf{1.} $C^*$-algebra $\acal$ acting on the spinor fields $\hcal$; \textbf{2.} Representation of $SU(2)$: homeomorphisms?\todo{homeomorphisms? Symmetries?}; \textbf{3.}  Metric space on which $SU(2)$ acts by isometries.


%%%%%%%%%%%%%%%%%%%%%%%%%%%%%%%%%%%%%%%%%%%%%%%%%%%%%%%%%%%%%%%%%%%%%%%%%%%%%%
%%%%%%%%%%%%%%%%%%%%%%%%%%%%%%%%%%%%%%%%%%%%%%%%%%%%%%%%%%%%%%%%%%%%%%%%%%%%%%
%%%%%%%%%%%%%%%%%%%%%%%%%%%%%%%%%%%%%%%%%%%%%%%%%%%%%%%%%%%%%%%%%%%%%%%%%%%%%%
\chapter{The New Fuzzy Spheres of Fiore and Pisacane}\label{chp:NewFuzzy}
{ \color{gray}
    
    \cite{FioreTheCase2020} Projecting a quantum theory onto the Hilbert subspace of states with energies below a cutoff $\cut E$ may\todo{why may? Do they simply refer to the possibility of this not being a general result?} \rtext{lead to an effective [quantum] theory with modified observables, including noncommutative space(time)}.
        
    \cite{FioreTheCase2020} Adding a confining potential well $V$ with a very sharp minimum on a submanifold $N$ of the original space(time) $M$ may\todo{perhaps here the may refers to whether that is a correct interpretation of what is being done, since it might not be correct to say that we are doing $QM$ on $N$?} \rtext{induce a dimensional reduction to a \textbf{noncommutative quantum theory} on $N$} [the noncommutative quantum theory part, just as in the theory of Chakraborty and Scholtz, seems to mean that not only do the coordinates of phase space commute, but also that the space coordinates themselves do not commute.]
        
    
    \lin 
    
    \cite{FioreTheCase2020}First example of NC spacetime: Snyner in [6]: hope that \rtext{nontrivial commutation relations could cure ultraviolet divergences in QFT} But shortly after regularization introducing energy cutoff was introduced for QED, although not Poincare covariant.
    
    
    Why does it make sense to have an energy cutoff? At least $2$ reasons:
        
            \begin{enumerate}
                
            \item We might add $\cut E$ as a point where higher energy physics is unknown.
            
            \item Where neither we nor the environment can bring a state to higher energies. This gives an effective description of the system. It \dbtext{leas to a lower (distance? ~)} bound in the accuracy with which our apparatus can measure observables, \dbtext{coming from a maximum transferable energy}.
                
            \end{enumerate}
            
    \cite{FioreTheCase2020}The following important observations may stem from the same energy cutoff procedure: introducing an energy cutoff $\cut E$ in a quantum theory on a commutative space(time) $M$, i.e. projecting the theory on the Hilbert subspace with energy below $\cut E$, directly induces a \rtext{noncommutative deformation of the latter [the Q. theory]} and \rtext{lower bounds for the space(time) locallyzability}. Moreover, adding a confining potential well $V$ with a very sharp minimum on a submanifold $N$ of $M$ may induce a dimensional \rtext{reduction to a noncommutative quantum theory on $N$} \rtext{\textbf{\Huge [Here the NC Quantum Theory is obtained from the enrgy cutoff, and that the resulting NC Theory is a NC approximation of Q. Theory on $N$ comes from the cutoff!!}} ].
            \begin{enumerate}
            
            \item Regularization introducing energy cutoff was introduced for QED, although not Poincare covariant.
            
            \item DFR: concensus that any mergint of QM and GR should lead to a cutoff of local energy concentration (and an associated lower bound on localization of events). DFR made this more precise, and proposed that \rtext{the bound on the localizability of events could follow from appropriate NC coordinates}.
            
            \end{enumerate}
          
    \lin   
    
    $O(D)$-equivariance/covariance?
    
        \begin{itemize}
            
        \item \rtext{My choice:} It is true that the Hamiltonian and the group action commute, which implies that
            
                \begin{itemize}
                    
                \item The action of $G$ makes things fall again in $\mathcal H_{\cut E}$, since it even makes an eigenvalue of $H$ fall again into an eigenvalue of $H$ with the same energy.
    $\Rightarrow$ $O(2)$ acts on $\mathcal H_{\cut E}$ $\Rightarrow$ $O(2)$ acts on $\mathcal A_{\cut E}$ by inner automorphisms, i.e. \rtext{$O(2)$ acts by diffeomorphisms on the NC space}.         
                \item The time evolution of a ``rotated'' vector is simply given at any time as the rotation of the evolution of the orignal vector. Alternatively, $H^g = H$, so the time evolution of the a rotated vector is given by the same Hamiltonian.
                
                \item The two above things imply that the evolution of a vector in $\mathcal H_{\cut E}$ is given by the same $H$, and that \rtext{this evolution is invariant under $O(2)$}
                
                    
                \end{itemize}
                
            \item Why make so much emphasis on ``the commutation relations (satisfied by the algebra generators) are $G$-invariant?'' And what does it even mean? {\tiny It can't be simply that $[A, B]^g = [A^g, B^g]$ since that is trivial to see given that $g$ acts by inner isomorphisms, and so it wouldn't be necessary to say that this is true since the commutation relations and $H$ (and hence) $P_{\cut E}$ are so... \rtext{but I think that's what makes sense since it would mean that the generated algebra is ``the same''}. It can't mean that $[A, B]^g = [A, B]$, since not even for the generators $\cut{x^\pm}$ that is true, since under reflections $\cut L$ does change.
            }  
            
            \item Fiore and Pisacane try to give an explanation about what this covariance means in one of their last papers, perhaps even the last one, but \dbtext{I haven't read it}.
        
        \item For ``\lbtext{covariance}'': $G$ is a symmetry of the theory:
            
            \begin{itemize}
                
            \item $[g\cdot , H] = 0$ implies \otext{$[g\cdot , P_{\cut E}] = 0$}
            
            \item That, in turn, implies that \otext{$\cut{A}^g = \cut{A^g}$}
                
            \end{itemize}
            
        \end{itemize}
    
    \lin
    
    Applications:
    
        \begin{itemize}
            
        \item \cite{FioreTheCase2020} These models might be suggestive for effective models in quantum field theory, quantum gravity or condense matter physics.
            
        \end{itemize}
}

%%%%%%%%%%%%%%%%%%%%%%%%%%%%%%%%%%%%%%%%%%%%%%%%%%%%%%%%%%%%%%%%
%%%%%%%%%%%%%%%%%%%%%%%%%%%%%%%%%%%%%%%%%%%%%%%%%%%%%%%%%%%%%%%%
\section{General Setting}

{
    \color{gray}
    
    - $H = - \frac{1}{2} \Delta + V(r)$ invariant under $O(D)$
    
    - Introducing the cutoff $\cut E$, as the energy where
    \begin{align}
        \label{eqn5}
            V(r) \approx V_0 + 2k(r-1)^2 && \text{for $r$ such that $V(r) \leq  \cut E$}.
    \end{align}
    %in the region $\nu_{\cut E} = \{r \,|\, V(r) \leq \cut E\}$.
    
    - Eigenfunctions of $H$ as product of a spherical harmonic $Y(\phi, \dots)$ (eigenvector of $L^2$) and an eigenfunction of the radial equation 
    \begin{align}
        \label{eqn9}
        \left[-\partial_r^2 - (D-1) \frac{1}{r} \partial_r + \frac{1}{r^2} j(j+D-2) + V(r)\right] \tilde f(r) = E \tilde f(r).
    \end{align}
    - This last equation can be approximated by a harmonic oscillator equation, since outside the region $\{V(r) \leq \cut E\}$ $\psi$ is negligibly small.
    : \rtext{$\hcal_{\cut E} \approx $ solutions of Schrodinger Eq. of energy $\leq \cut E$, $\acal_{\cut E} = End(\hcal_{\cut E})$}.
}

\linea



%%%%%%%%%%%%%%%%%%%%%%%%%%%%%%%%%%%%%%%%%%%%%%%%%%%%%%%%%%%%%%%%
%%%%%%%%%%%%%%%%%%%%%%%%%%%%%%%%%%%%%%%%%%%%%%%%%%%%%%%%%%%%%%%%
\section{Construction of $\acal_{\cut E}$ for $D = 2$}

{
    \color{gray}
    
    - Equation \eqref{eqn9} has the approximation, where $\rho := \ln r$
    \begin{equation}
        \label{harmonic2D}
        \hat H f(\rho) = e_m f(\rho), \qquad
        \hat H = - \partial_\rho^2 + k_m(\rho - \tilde \rho_m)^2,
    \end{equation} $
        k_m := 2(k - E'), \quad
        E' := E - V_0, \quad
        \tilde \rho_m := \frac{E'}{k_m}, \quad
        e_m = \frac{E'^2}{k_m} + E' - m^2
    $
    - The solutions $f_{n,m}$ are known, $n \in \bb N, m \in \ZZ$ \then $e_{m, n}(k) = (2n+1)\sqrt{k_m}$ \then $E'_{m,n}(k)$ satisfies a quartic equation \then $E_{m,n}(k, V_0)$.
    
    - Fixing $V_0 = V_0(k)$ such that $E_{0, 0} = 0$ \then $V_0(k) = -\sqrt{2k} + 2 - \frac{7}{2}\frac{1}{\sqrt{2k}} + o(1/k)$ and 
    $\sum_{n = -1}^\infty v_n \left( \sqrt{\frac{1}{k}} \right)^n \approx -\sqrt{2k} + 2 - o(1/k)$ \then
    \begin{equation}
        \rtext{E_{n, m}(k)} = m^2 + 2n\sqrt{2k} - 2n + o(1/\sqrt{k})
    \end{equation}
    
    - Choosing $\cut E < 2 \sqrt{2k} - 2$, the spectrum of $H$ is a truncation of $L^2$: \textbf{the radial oscillations are ``frozen''}: $\cut{\partial_r} = 0$.
    \begin{multline*}
        \lbtext{\psi_m(\rho, \phi)} := f_{0, m}(\rho) e^{im\phi} = c_m e^{im\phi}exp{\left[ -\frac{(\rho - \tilde \rho_m)^2 \sqrt{k_m}}{2} \right]} \\\xrightarrow{k \to \infty} \delta(r-1)e^{i m \phi}
    \end{multline*}
    \begin{equation}
        E = E_m(k) = m^2 + o(1/\sqrt{k})
    \end{equation}
    
    - For $\lbtext{\Lambda} := \lfloor \cut E \rfloor$, 
    \begin{align}
        \lbtext{\hcal_{\Lambda}}:= \lbtext{\hcal_{\cut E}} := span\{\psi_m\}_{|m| \leq 
    \lfloor \cut E \rfloor} ,
    \lbtext{\acal_\Lambda} := \mathcal B(\hcal_\Lambda)
    \end{align}
    
    - Since $H$ generates the time evolution, a
    An element of $\hcal_\Lambda$ doesn't evolve out of $\hcal_\Lambda$.
    
    - Get a fuzzy space: e.g. choosing $k = \Lambda^2(\Lambda+1)^2$ --make $k$ diverge with $\Lambda$ while $\nu_{\cut E}$ goes to $\{r = 1\}$
    
    - This cutoff entails replacing every observable by $A \mapsto \lbtext{\cut A} = P_{\cut E} A P_{\cut E}$% \dbtext{when?}
}

\linea


%%%%%%%%%%%%%%%%%%%%%%%%%%%%%%%%%%%%%%%%%%%%%%%%%%%%%%%%%%%%%%%%
%%%%%%%%%%%%%%%%%%%%%%%%%%%%%%%%%%%%%%%%%%%%%%%%%%%%%%%%%%%%%%%%
\section{Important Observables and their Commutation Relations}

{
    \color{gray}
    
    - Up to infinite, $1/k^{1/2}\dbtext{?}$ and $1/k^{3/2}$ orders, respectively
    \begin{align}
    \label{projObs2D}
        \rtext{\cut L} \psi_m &= m \psi_m; & 
        \rtext{\cut H} &= \cut L^2; & 
        \rtext{\cut x^\pm} \psi_m = 
            \begin{cases}
                \frac{a}{\sqrt{2}} \sqrt{ 1 + \frac{m(m \pm 1)}{k} } \psi_{m \pm 1} & -\Lambda \leq \pm m \leq \Lambda - 1 \\
                0 & \text{otherwise}
            \end{cases}
    \end{align}
    - And so, up to terms of $1/k^{3/2}$
    \begin{align}
        \label{conmObs2D}
        \cut{x^+}^\dagger &= \cut{x^-}; &
        [\cut L, \cut{x^\pm}] &= \pm \cut{x^\pm}; &
        \rtext{[\cut{x^+}, \cut{x^-}]} &= - \frac{\cut L}{k} + \left[1 + \frac{\Lambda(\Lambda+1)}{k}\right] (\tilde P_{\Lambda} - P_{-\Lambda})a^2.
    \end{align}
     - If \eqref{projObs2D} are used exactly to define elements of $\mathcal B(\hcal_\Lambda) \equiv \acal_\Lambda$ then \eqref{conmObs2D} are also exact, and $\cut{x^\pm}$ generate $\acal_\Lambda$. -- $\cut{\partial_\pm}$ are now redundant... but I'm not sure why
     
}

\linea


%%%%%%%%%%%%%%%%%%%%%%%%%%%%%%%%%%%%%%%%%%%%%%%%%%%%%%%%%%%%%%%%
%%%%%%%%%%%%%%%%%%%%%%%%%%%%%%%%%%%%%%%%%%%%%%%%%%%%%%%%%%%%%%%%
\section{Realization of $\acal_\Lambda$ through $U\soth$}

{
    \color{gray}
    
    - $O(2)$ acts on $\hcal_\Lambda \subset L^2(\RR^2)$, and so \rtext{on $\acal_\Lambda$}, since $[H, O(2)\cdot ] = 0$. through the action induced in $\acal_\Lambda$ by its action on $\RR^2$.
        \begin{itemize}
            
        \item \textit{Rotation} $R_\theta$: $\cut{x^\pm} \mapsto e^{\pm i \theta} \cut{x ^\pm}; \cut L \mapsto \cut L \in \acal_\Lambda$.
        
        \item \textit{Reflection}: $\cut{x^\pm} \mapsto -\cut{x^\mp}; \cut L \mapsto -\cut L$.
        
        \end{itemize}
    
    - \rtext{We can consider \otext{$\acal_\Lambda \cong  M_N(\CC) = \pi_\Lambda(Uso(3))$} \textbf{as a $*$-algebra and representations of $O(2)$}}, where $\pi_\Lambda$ is the $\lbtext{N} := 2 \Lambda + 1$ dimensional representation. $SU(N) \ni g$ is the group of $*$-automorphisms of $M_N(\CC) \cong \acal_\Lambda$ acting by $a \mapsto g a g^{-1}$; $O(2)$ is then a subgroup. Comes from mapping:
    \begin{align}
        \rtext{\cut{x^\pm}} &\longleftrightarrow \rtext{f_\pm (J^0) J^\pm}, &
        \rtext{\cut L }& \rtext{\longleftrightarrow J^0}
    \end{align}
    where $J^\pm, J^0$ is the Weyl-Cartan basis of $so(3)$, $\lbtext{f_\pm(s)} := \frac{1}{\sqrt{2}} \sqrt{\frac{1 + s(s-1)/k}{\Lambda (\Lambda + 1) - s(s-1)}} =: \lbtext{f_-(-s)}$%: $[J^+, J^-] = J^0; [J^\pm, J^0] = \pm J^\pm$

    - $O(2) \subset SO(3)$: \textit{Rotation}: $\pi_\Lambda(e^{i \theta J_0})$; \textit{Refl.}: $\pi_\Lambda(e^{i\pi (J^+ + J^-)/\sqrt{2}})$
}

\linea


%%%%%%%%%%%%%%%%%%%%%%%%%%%%%%%%%%%%%%%%%%%%%%%%%%%%%%%%%%%%%%%%
%%%%%%%%%%%%%%%%%%%%%%%%%%%%%%%%%%%%%%%%%%%%%%%%%%%%%%%%%%%%%%%%
\section{Convergence}

{
    \color{gray}
    
    - \textbf{$\psi_m$ as fuzzy analogues of $e^{i m \phi} \in \hcal$}: $O(2)$-covariant embedding $\hcal_\Lambda \hookrightarrow \otext{\hcal = L^2(S^1)}$, $\psi_m \mapsto e^{im\phi}$; \hfill \\then $\hcal_\Lambda \to \hcal$ as $\Lambda \to \infty$ in the sense that $\forall \phi \in \hcal$, $\lbtext{\phi_\Lambda} := \sum_{|m| \leq \Lambda} \phi_m e^{im\phi} \to \phi$ in the $L^2$-norm.
    
    - Induces, \rtext{embedding $\acal_\Lambda \hookrightarrow \otext{\acal = \mathcal B(\hcal)}$ and limit $\acal_\Lambda \to \acal$} as $\Lambda \to \infty$.
    
    - \textbf{Fuzzy analogue of $B(S^1)$ of (bounded) functions on $S^1$} as subalgebra (act by mult.) of $\mathcal B(\hcal)$: $\lbtext{C_\Lambda} := \left\{ \sum_{h = -2\Lambda}^\Lambda f_h \eta^h \,|\, f_h \in \CC\right\}$ where $\eta^\pm  = \frac{\sqrt{2}}{a}x^\pm$ (so $\eta^\pm \to e^{\pm i \phi}$ as operators).
    
    - Choosing $k(\Lambda) \geq 2 \Lambda(\Lambda + 1)(w\Lambda+1)^2$, then \rtext{$B(S^1) \to \mathcal B(\hcal)$ as operators} due to the strong limits: $\hat f_\Lambda \to f\cdot$, $\hat{(fg)}_\Lambda \to fg\cdot $, $\hat f_\Lambda \hat g_\Lambda \to fg\cdot$, where $\lbtext{\hat f_\Lambda} := \sum_{h = -2\Lambda}^{2\Lambda} f_h \eta^h$ is ``truncation'' of $f \in B(S^1)$.
    
}

\linea



%%%%%%%%%%%%%%%%%%%%%%%%%%%%%%%%%%%%%%%%%%%%%%%%%%%%%%%%%%%%%%%%%%%%%%%%%%%%%%
%%%%%%%%%%%%%%%%%%%%%%%%%%%%%%%%%%%%%%%%%%%%%%%%%%%%%%%%%%%%%%%%%%%%%%%%%%%%%%
%%%%%%%%%%%%%%%%%%%%%%%%%%%%%%%%%%%%%%%%%%%%%%%%%%%%%%%%%%%%%%%%%%%%%%%%%%%%%%
\chapter{Systems of Coherent States on the New Fuzzy Spheres}\label{chp:CHNew}

{ \color{gray}

\lin


\cite{FioreTheCase2020}
-- Coherent states are, apparently, crucial if we want these spaces to have applications

-- The weak system of coherent states $\mathcal W^d$ is made of the states minimizing $(\Delta \vec x)^2$: they are related to the diagonalization of the coordinates $x_i$; they are $O(D)$-invariant.

-- A class of $O(D)$-invariant strong SCS are introduced, and in particular an interesting one is $\mathcal S^d$ which minimizes, within this class of CS, $(\Delta \vec x)^2$

-- The diagonalization of the position observables seems to be improved on their last paper

\lin



}


% %%%%%%%%%%%%%%%%%%%%%%%%%%%%%%%%%%%%%%%%%%%%%%%%%%%%%%%%%%%%%%%%%%%%%%%%%%%%%%
% %%%%%%%%%%%%%%%%%%%%%%%%%%%%%%%%%%%%%%%%%%%%%%%%%%%%%%%%%%%%%%%%%%%%%%%%%%%%%%
% %%%%%%%%%%%%%%%%%%%%%%%%%%%%%%%%%%%%%%%%%%%%%%%%%%%%%%%%%%%%%%%%%%%%%%%%%%%%%%
% \chapter{Spectral Triples on the New Fuzzy Spheres}\label{chp:NewDistance}
% \input{chapters/NewDistance}

%%%%%%%%%%%%%%%%%%%%%%%%%%%%%%%%%%%%%%%%%%%%%%%%%%%%%%%%%%%%%%%%%%%%%%%%%%%%%%
%%%%%%%%%%%%%%%%%%%%%%%%%%%%%%%%%%%%%%%%%%%%%%%%%%%%%%%%%%%%%%%%%%%%%%%%%%%%%%
\chapter{Final Remarks and Further Work}
\input{chapters/remarks} \label{chp:remarks}

The noncommutative algebras associated to the fuzzy spheres studied in this document may all be seen as square matrix algebras. Their finite dimension and the fact that we understand completely these algebras facilitate the study of the metric properties of the associated noncommutative spaces. This made fairly straighforward in chapter \ref{chp:fuzzysphere} the complete study of the distances between vector discrete basis states of the Madore fuzzy sphere, although this was not possible between the spin-, or $SU(2)$-, coherent states for matrix sizes beyond $2 \times 2$. However, an approximate study was possible by relating the $SU(2)$-coherent states of contiguous dimensions, revealing the relation between subsequent elements of the sequence conforming the fuzzy sphere, as well as its behavior in relation to the commutative space $S^2$ that it approximates. 

The sequence of algebras comprising the fuzzy circle of Fiore and Pisacane studied in chapter \ref{chp:NewFuzzy} can be understood, both as a $C^*$-algebra and a representation space of $O(2)$, as the subsequence of matrix algebras of the fuzzy sphere of odd column and row number. This fuzzy space converges to the circle $S^1$ and is formed as a sequence of low energy effective quantum theories for the quantum mechanics on $S^1$ by introducing energy cut-offs, which naturally induces noncommuting position coordinates.
Various systems of coherent states were proposed and partially studied in chapter \ref{chp:CHNew}, and a special emphasis was made on those with minimal position uncertainty in order to find states that may be associated to the points of the commutative space $S^1$.
The next step to follow is to propose Dirac operators on the algebras conforming the fuzzy circle of Fiore and Pisacane. Although a complete characterization of the real spectral triples on semisimple finite dimensional algebras was achieved by Paschke and Sitarz in  \cite{PaschkeSitarz98}, we need both a spectral triple on each element of the sequence of the fuzzy circle, but also a sequence that approximates the spectral triple of the compact simple groups $S^1$\todo{apparently people talk about ``$2$-spinors'' on $S^1$ and there are two distinct Dirac operators in that case}; it remains to be seen if an approach resembling the one followed in chapter \ref{chp:fuzzysphere} based on the decomposition of the spinor bundle $\hcal \cong L^2(S^1)$ as a representation space of $O(2)$, the desired symmetry group, is possible. A related alternative method may come from the procedure proposed by D'Andrea et at. in \cite{DAndrea2014} to introduce spectral triples on a fuzzy space based on projections of the elements of the canonical spectral triple of the limit space as high momentum cut-offs. Yet another possible source of a Dirac triple may come from the formalism of Noncommutative Quantum Mechanics of Scholtz and Chakraborty \cite{}. Once a spectral triple on the fuzzy circle has been established, the study of the resulting metric properties may be possible to make following the general procedure outlined in \cite{ChaobaDevi2018}, although a more in depth analysis will depend on the nature of the Dirac operators previously proposed. 

Additional to a fuzzy circle, Fiore and Pisacane have researched in \cite{Fiore2018, FioreXi2020, FioreCoherent2020} alternative descriptions and coherent states on the fuzzy sphere that results from the application of the energy cut-off procedure of section 3.1\ref{} when $D = 3$, instead of $D=2$, where the symmetry group is instead $O(4)$. The introduction of spectral triples on this fuzzy space has yet to be done, but the procedure done on the fuzzy circle may illustrate a path towards this objective, as well as that for the general $ D \in \ZZ_{\geq 2}$ case.


%%%%%%%%%%%%%%%%%%%%%%%%%%%%%%%%%%%%%%%%%%%%%%%%%%%%%%%%%%%%%%%%%%%%%%%%%%%%%%
%%%%%%%%%%%%%%%%%%%%%%%%%%%%%%%%%%%%%%%%%%%%%%%%%%%%%%%%%%%%%%%%%%%%%%%%%%%%%%
\printbibliography

\end{document}
