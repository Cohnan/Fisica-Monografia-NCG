\documentclass{beamer}

%%% Packages
\usepackage{amssymb, amsmath, amsthm}
\usepackage[english]{babel}
\usepackage[utf8]{inputenc}
\usepackage{biblatex} \addbibresource{bibliografia.bib}

%% Other Packages
\usepackage[T1]{fontenc}
\usepackage{bm} % For boldface greek letters
\usepackage[normalem]{ulem} % For strikethrough with command \sout
\usepackage{slashed} % For slashed in Dirac operator

%%%%%% Theorem Environments
\theoremstyle{definition}

%\newtheorem{theorem}{Theorem}[section]  %numbered according to section environment, so in section to it restarts as 2.1 
\newtheorem{proposition}{Proposition}[section]  %numbered according to section environment, so in section to it restarts as 2.1 
%\newtheorem{lemma}[theorem]{section}     %numbering shared with theorem 
%\newtheorem{corollary}{Corollary}[theorem]


\theoremstyle{remark}

\newtheorem*{remark}{Remark}

%%%%% Color text
%\usepackage[]{xcolor}

\newcommand{\ytext}[1]{\textcolor{yellow}{#1}}
\newcommand{\otext}[1]{\textcolor{orange}{#1}}
\newcommand{\rtext}[1]{\textcolor{red}{#1}}
\newcommand{\lbtext}[1]{\textcolor{cyan}{#1}}
\newcommand{\dbtext}[1]{\textcolor{blue}{#1}}
\newcommand{\ptext}[1]{\textcolor{Plum}{#1}}
\newcommand{\lgtext}[1]{\textcolor{LimeGreen}{#1}}
\newcommand{\dgtext}[1]{\textcolor{OliveGreen}{#1}}

%%%%% Custom Commands
\newcommand{\set}[1]{\{#1\}}
\newcommand{\inv}{{-1}}
\newcommand{\conj}[1]{\overline{#1}}

\newcommand{\twov}[2]{\begin{pmatrix} #1 \\ #2 \end{pmatrix}}
\newcommand{\threev}[3]{\begin{pmatrix} #1 \\ #2 \\ #3 \end{pmatrix}}

\newcommand{\then}{\ensuremath{\longrightarrow}}
\newcommand{\Then}{\ensuremath{\Longrightarrow}}
\renewcommand{\iff}{\ensuremath{\longleftrightarrow}}
\newcommand{\Iff}{\ensuremath{\Longleftrightarrow}}
\newcommand{\suff}{\ensuremath{\longleftarrow}}
\newcommand{\Suff}{\ensuremath{\Longleftarrow}}


\newcommand{\bra}{\langle}
\newcommand{\ket}{\rangle}
\newcommand{\bbra}{\langle\langle}
\newcommand{\kket}{\rangle\rangle}


\newcommand{\bb}[1]{\mathbb #1}
\newcommand{\NN}{\mathbb N}
\newcommand{\ZZ}{\mathbb Z}
\newcommand{\RR}{\mathbb R}
\newcommand{\CC}{\mathbb C}
\newcommand{\HH}{\mathbb H}
\newcommand{\TT}{\mathbb T}

\newcommand{\defn}[1]{\lbtext{#1}}
\newcommand{\Defn}[1]{\dbtext{#1}}

\newcommand{\hcal}{\mathcal H}
\newcommand{\acal}{\mathcal A}
\newcommand{\bcal}{\mathcal B}
\newcommand{\dcal}{\mathcal D}
\newcommand{\cut}[1]{\overline{#1}}

%%%%%
% Choose how your presentation looks.
%
% For more themes, color themes and font themes, see:
% http://deic.uab.es/~iblanes/beamer_gallery/index_by_theme.html
%
\mode<presentation>
{
  \usetheme{Darmstadt}      % or try Darmstadt, Madrid, Warsaw, ...
  \usecolortheme{beaver}%seagull} % or try albatross, beaver, crane, ...
  \usefonttheme{default}  % or try serif, structurebold, ...
  \setbeamertemplate{navigation symbols}{}
  \setbeamertemplate{caption}[numbered]
} 

% Show number of slides at the bottom
\addtobeamertemplate{navigation symbols}{}{%
    \usebeamerfont{footline}%
    \usebeamercolor[fg]{footline}%
    \hspace{1em}%
    \insertframenumber/\inserttotalframenumber
}

%%%%%% Show TOC before a new Section
\AtBeginSection[]
{
    \begin{frame}[noframenumbering]{Table of Contents}
        \tableofcontents[currentsection]
    \end{frame}
}

% Show TOC before a new SubSection
\AtBeginSubsection[]
{
    \begin{frame}[noframenumbering]{Table of Contents}
        \tableofcontents[currentsection,currentsubsection]
    \end{frame}
}
% Reduce size of TOC
\AtBeginDocument{
  %\addtocontents{toc}{\tiny}
  %\addtocontents{subsection in toc}{\tiny}
}

%\setbeamerfont{subsection in toc}{size=\tiny}

%%%%%% Make paragraphs start with no indentation and leave spaces between paragraphs
\setlength{\parindent}{0em}
\setlength{\parskip}{1em}

% \renewcommand\englishhyphenmins{22}
% \usepackage{microtype}

%%%%%% Document Information
\title[l]{Towards the Calculation of Distances in the New Noncommutative Spheres of Fiore and Pisacane}
\author{Sebastian Puerto}
\institute{Universidad de los Andes\\ Advised by Prof. Andrés Reyes Lega Ph.D.}
\date{January 26, 2020}

%%%%%%%%%%%%%%%%%%%%%%%%%%%%%%%%%%%%%%%%%%%%%%%%%%%%%%%%%%%%%%%%%%%%%%%%%%%%%%%
%%%%%%%%%%%%%%%%%%%%%%%%%%%%%%%%%%%%%%%%%%%%%%%%%%%%%%%%%%%%%%%%%%%%%%%%%%%%%%%
%%%%%%%%%%%%%%%%%%%%%%%%%%%%%%%%%%%%%%%%%%%%%%%%%%%%%%%%%%%%%%%%%%%%%%%%%%%%%%%
%%%%%%%%%%%%%%%%%%%%%%%%%%%%%%%%%%%%%%%%%%%%%%%%%%%%%%%%%%%%%%%%%%%%%%%%%%%%%%%
\begin{document}

\begin{frame}[noframenumbering]
  \titlepage
\end{frame}

\begin{frame}{Objective} % % % % % % % % % % % % % % % % % % %
\begin{itemize}

    \item Understand the study of distances in noncommutative spaces through examples.
    
    \item Research the distance between states in the new fuzzy spheres introduced by Fiore and Pisacane in \cite{Fiore2018}.
    
\end{itemize}

\end{frame}

% Uncomment these lines for an automatically generated outline.
\begin{frame}[noframenumbering]{Outline}
  \tableofcontents
\end{frame}

%%%%%%%%%%%%%%%%%%%%%%%%%%%%%%%%%%%%%%%%%%%%%%%%%%%%%%%%%%%%%%%%%%%%%%%%%%%%%%%
%%%%%%%%%%%%%%%%%%%%%%%%%%%%%%%%%%%%%%%%%%%%%%%%%%%%%%%%%%%%%%%%%%%%%%%%%%%%%%%
\section{Background}

\begin{frame} % % % % % % % % % % % % % % % % % % %
    
    % $\bullet$ Commutative NC Geometry
    
    % $\bullet$ Canonical Spectral triple
    
    % $\bullet$ Connes' Spectral Distance and commutative case.
    
    \begin{table}[h]
        \centering
        \begin{tabular}{c|c}
             Loc. Compact Hausdorff Space $X$ & Commutative $C^*$-Algebra $\acal$\\
             \hline
             Pure states($\acal$) & $C_0(X)$ \\
             Compact space & Unital algebra \\
             Continuous proper map & $*$-homomorphism \\
             %Homeomorphism & automorphism \\
             Open set & ideal \\
             Vector field & Derivation \\
             Vector bundle & Projective f.d. module
        \end{tabular}
        
        \label{tab:my_label}
    \end{table}
    
    $\bullet$ \textbf{Canonical Spectral Triple}: $(M, g)$ be $n$-dim. compact oriented Riemannian manifold which admits a spin structure% $(Spin_n \to Spin\,M \to M, \eta: Spin\,M \to SO\,M)$
    , let $\Sigma M$ % = Spin\, M \times_\rho \Sigma_n$
    be the spinor bundle: $(\acal = C^\infty(M), \hcal = L^2(M, \Sigma M), \slashed D = -i \gamma^\mu \nabla^g_\mu)$.
    
    $\bullet$ \textbf{Distance}: $d_D(\omega, \omega') := sup_{a \in \acal} \left\{|\omega(a) - \omega'(a)| \, | \, \rtext{||[D, a]||_{op} \leq 1} \right\}$
    
    $\bullet$ \textbf{Unital Spectral Triple} $(\acal , \hcal, D)$: $\hcal$ complex separable Hilbert space; $\acal$ complex unital $C^*$-algebra with a faithful unital $*$-representation on $\hcal$; $D: \hcal \to \hcal$ self-adjoint (unbounded) operator with compact resolvent, such that $[D, a] \in \mathcal B(\hcal)$ $\forall a \in \acal$.
\end{frame}


%%%%%%%%%%%%%%%%%%%%%%%%%%%%%%%%%%%%%%%%%%%%%%%%%%%%%%%%%%%%%%%%%%%%%%%%%%%%%%%
%%%%%%%%%%%%%%%%%%%%%%%%%%%%%%%%%%%%%%%%%%%%%%%%%%%%%%%%%%%%%%%%%%%%%%%%%%%%%%%
%%%%%%%%%%%%%%%%%%%%%%%%%%%%%%%%%%%%%%%%%%%%%%%%%%%%%%%%%%%%%%%%%%%%%%%%%%%%%%%%
\section{Introduccion}

\begin{frame}{Titulo} % % % % % % % % % % % % % % % % % % %
    
\end{frame}


%%%%%%%%%%%%%%%%%%%%%%%%%%%%%%%%%%%%%%%%%%%%%%%%%%%%%%%%%%%%%%%%%%%%%%%%%%%%%%%
%%%%%%%%%%%%%%%%%%%%%%%%%%%%%%%%%%%%%%%%%%%%%%%%%%%%%%%%%%%%%%%%%%%%%%%%%%%%%%%
\section{The Fuzzy Sphere}

\begin{frame}{Definitions} % % % % % % % % % % % % % % % % % % %
    
    - FS
    
    - It approximates S2 in the sense that
    
    - Coherent states (~ Slide 11)
    
\end{frame}


\begin{frame}{Spectral Triples} % % % % % % % % % % % % % % % % % % %
    
     - Canonical
     
     - Irreducible
     
     - Full
     
     - Theorem
    
\end{frame}


\begin{frame}{Distance Calculation Procedure} % % % % % % % % % % % % % % % % % % %
    
     - General facts
     
     - Perhaps isometry group or desired spectrum
     
     - Docs and ~Slide 17
    
\end{frame}

%%%%%%%%%%%%%%%%%%%%%%%%%%%%%%%%%%%%%%%%%%%%%%%%%%%%%%%%%%%%%%%%%%%%%%%%%%%%%%%
%%%%%%%%%%%%%%%%%%%%%%%%%%%%%%%%%%%%%%%%%%%%%%%%%%%%%%%%%%%%%%%%%%%%%%%%%%%%%%%
\section{New Fuzzy Spheres}

\begin{frame}{General Construction*} % % % % % % % % % % % % % % % % % % %
    
    $\bullet$  SEq: $H = - \frac{1}{2} \Delta + V(r)$ invariant under $O(D)$. Potential $V(r) \approx V_0 + 2k(r-1)^2$, $k$ large. SEq. approximated by harmonic oscillator equation \then \rtext{low energy spectrum equals to that of $L^2$}.
    
    $\bullet$ $\lbtext{\hcal_{\cut E}} \approx $ solutions of SEq. of energy $\leq \cut E$, $\lbtext{\acal_{\cut E}} = End(\hcal_{\cut E})$.
    
    $\bullet$ Observable by $A \mapsto \lbtext{\cut A} = P_{\cut E} A P_{\cut E}$. In particular, \rtext{noncommuting coordinates}: $[\cut x^i, \cut x^j] \neq 0$, transform contravariantly.
    
    $\bullet$ Sequences: $\cut E = \Lambda(\Lambda + D - 2)$, $k(\Lambda) \geq \Lambda^2 (\Lambda + 1)^2$.
    
    $\bullet$ These spaces are \rtext{low energy effective quantum theories} on $S^{D-1}$ and \rtext{fuzzy spaces} approximating $S^{D-1}$.
    
\end{frame}

\begin{frame}{$D = 2$ Observables $\to$ QM in $S^1$} % % % % % % % % % % % % % % % % % % %
    
    $\chi^+$ and $\chi^-$ generate the $*$-algebra $\acal_\Lambda = End(\hcal_\Lambda)$. The generators $\chi^\pm$ and $\cut L$ satisfy the relations:
\begin{align}\label{equationOperatorFormulaRelationsChiD2}
    (\chi^\pm)^{2\Lambda+1} &= 0, & 
    (\chi^+)^*&= \chi^-, & 
    \prod_{m = \Lambda}^\Lambda (\cut L - m), &= 0 & 
    \cut L^* &= \cut L,
\end{align}
\begin{align}\label{equationOperatorFormulaCommutatorsChiD2}
    [\cut L, \chi^\pm] &= \pm \chi^\pm,&
    [\chi^+, \chi^-] &= - \frac{2\cut L}{k} + \left[ 1 + \frac{\Lambda(\Lambda+1)}{k}\right](\tilde P_\Lambda - \tilde P_{-\Lambda}),
\end{align}
where the projections $\tilde P_m$ are polynomials in $\cut L$. Furthermore,
$\label{equationOperatorFormulaRChiD2}
    \vec \chi^2 = 1 + \frac{L^2}{k} - \left[ 1 + \frac{\Lambda(\Lambda+1)}{k}\right]\frac{\tilde P_\Lambda - \tilde P_\Lambda}{2}.
$

\end{frame}

\begin{frame}{Alternative Descriptions} % % % % % % % % % % % % % % % % % % %

     $\bullet$ By the isomorphism of the Hilbert spaces $\hcal_\Lambda \subset L^2(\RR^2)$ and $V_\Lambda \equiv \CC^{2\Lambda + 1}$ given by $\psi_m \mapsto |\Lambda, m\rangle$. \Then $C^*$-isomorphism $\acal_\Lambda = End(\hcal_\Lambda)$ with $\hat \pi_\Lambda(U(\mathfrak{so}(3))) = End(V_\Lambda)$ with the operator norm, and this isomorphism is $O(2)$-equivariant. Generators: $\cut{\chi^\pm} = f^\chi_{\pm}(\cut L) L^\pm$, with $f^\chi_\pm$ polynomials.
     
     $\bullet$ Similarly, $\psi_m \mapsto e^{im\phi}$ induces a $O(2)$-equivariant $C^*$-isomorphism between $\acal_\Lambda$ and the operators on $\tilde \hcal_\Lambda := span\{e^{im\phi}\}_{|m| \leq \Lambda} \subset L^2(S^1) $, which annihilate $\tilde \hcal_\Lambda^\perp$.



    
    \end{frame}

\begin{frame}{Convergence} % % % % % % % % % % % % % % % % % % %
    Let's take the last characterization of $\acal_\Lambda$, as subset of $\bcal(L^2(S^1))$.

    
    $\bullet$ \rtext{To QM of $S^1$}: The following operators converge strongly: $\cut L \to L$, $\chi^\pm \to e^{\pm i\phi}$ (ladder op.).
    
    $\bullet$ \rtext{To $S^1$}: $\chi^\pm$ as the fuzzy analogs of the operators $e^{\pm i\phi}\cdot$, so $C_\Lambda := \left\{ \sum_{h = -2\Lambda}^{2\Lambda} f_h x^h \right\} \subset \acal_\Lambda \subset \bcal(\hcal)$ fuzzy analog of $C(S^1)$.  
    
    Then, for all $f, g \in B(S^1)$ the following strong limits hold:
\begin{align}
    \hat f_\Lambda \to f\cdot && \hat{(fg)}_\Lambda \to fg\cdot && \hat f_\Lambda \hat g_\Lambda \to fg \cdot\,,
\end{align}
where the notation $f\cdot$ emphasizes $f$ as an operator on $\hcal$.
\end{frame}

\begin{frame}{Coherent States (CS): Strong Systems} % % % % % % % % % % % % % % % % % % %

    $\bullet$ The discrete basis of $\hcal_\Lambda$ is a \textit{strong SCS} (i.e. induces resolution of $1$). Generated by $G = \{S^n e^{i(aL + b)}\}\cong U(1)\times U(1) \rtimes \ZZ_{2\Lambda + 1}$. $(\Delta L)^2 = 0$, $(\Delta \vec \chi)^2 \geq \frac{1}{2}$. Characterized by set of UR's.
    
    $\bullet$ All $SO(2)$-invariant strong (induce resolution of $1_\hcal$) systems of (norm.) coherent states are of the form $\{\omega^\beta_\alpha\}_{\alpha \in \frac{\RR}{2\pi\ZZ}}$, $\beta \in (\frac{\RR}{2\pi\ZZ})^{2\Lambda+1}$, generated by $\omega^\beta = \sum_{m = -\Lambda}^\Lambda \frac{e^{i\beta_m}}{\sqrt{2\Lambda+1}}$. $(\Delta L)^2 = \frac{\Lambda(\Lambda + 1)}{3}$.
    
    $\bullet$ $O(2)$-invariant subfamilies are generated if $\beta_m = \beta_{-m}$. 
    
    $\bullet$ \rtext{The system of coherent states that minimize the spacial dispersion} ($(\Delta \vec \chi)^2 \overset{\Lambda \geq 2}{\leq } \frac{2}{3(\Lambda + 1)}$) \rtext{within the families $SO(2)$-invariant SSCS}:  $\beta = \vec 0$: $O(2)$-invariant, \rtext{in a $O(2)$-equivariant correspondence with points in $S^1$} via $\omega^0_\alpha \leftrightarrow e^{i \alpha}$.
\end{frame}

\begin{frame}{Minimum $(\Delta \vec \chi)^2 = \bra \vec \chi^2 \ket - \bra \vec \chi \ket ^2$**} % % % % % % % % % % % % % % % % % % %

    $\bullet$ The set $\mathcal W^1$ of unit vectors minimizing $(\Delta \vec \chi)^2$ is $O(2)$-invariant.
    
    $\bullet$ Closed formulas hard because $\bra \vec \chi ^2 \ket = cte + O(\frac{1}{\Lambda^2})$.
    
    $\bullet$ But, perhaps the highest eigenvalue vectors of $\chi^1$ approximate them up to that order.
    
    $\bullet$ Another approximation $O(\frac{1}{\Lambda^2})$ is needed to find the eigenvalues and eigenvectors: coefs. of $\chi^1$ eq. $1$. Their dispersion $(\Delta \vec \chi)^2 < \frac{3.5}{(\Lambda+1)^2} \overset{\Lambda \to \infty}{\longrightarrow} 0$.
    
    $\bullet$ For $\chi^1$ with ordered spectrum $\{\alpha^\Lambda_k\}_{|k|<\Lambda}$, 
    $\alpha_{\Lambda+1}^{\Lambda+1} > \alpha_{\Lambda}^{\Lambda} > \alpha_\Lambda^{\Lambda+1} > \alpha_{\Lambda-1}^{\Lambda} > \dots > \alpha_{-\Lambda}^{\Lambda} > \alpha_{-\Lambda-1}^{\Lambda+1}$.
    
\end{frame}

%%%%%%%%%%%%%%%%%%%%%%%%%%%%%%%%%%%%%%%%%%%%%%%%%%%%%%%%%%%%%%%%%%%%%%%%%%%%%%%
%%%%%%%%%%%%%%%%%%%%%%%%%%%%%%%%%%%%%%%%%%%%%%%%%%%%%%%%%%%%%%%%%%%%%%%%%%%%%%%
\section{Further Work}

\printbibliography[title=References]

\end{document}
