\section{Doubts}

\subsection{Deal Changers}

    \begin{itemize}
    
    \item If we leave the $\cut {x^\pm}$ coordinates as the generators, their commutation relations look nothing like those of $\xi^\pm$ up to the claimed order, since $a^2 = 1 + \frac{\cdot}{\sqrt{k}} + \frac{\cdot}{k} + \cdot$... how can I justify (physically) the introduction of $\xi^\pm$, and their use to construct the theory and the use of THEIR commutation relations to define the better generators of $\acal_{\cut E}$?
    
        \begin{itemize}
            
        \item It seems to be true that $a = a(k)$ is just a number that \textbf{multiplies}, and in that case maybe it is not so bad to divide by it to obtain other observables $\xi^\pm$?
        
        \item However, I want to understand this well since I am saying that \textbf{the ``new position observables in the effective theory are the $\cut x^i$''}. If I say that, then I should study the commutation relations between the ``position coordinates'' $\cut x^i$, which is quite different from that of the $\xi^i$.
        
        \item ALTHOUGH notice that in the Fuzzy spheres we might say that it is even more arbitrary what we call the position algebra elements... but unless a physically sound procedure is implemented in this model to validate the change of the position algebra elements (more than just: ``we want their commutation relations to be Snyder-like''), \textbf{it might be better to not call the $\xi^i$ in $D = 2$ the position observables... although some reasoning is provided}: $a$ is just a normalizing constant, the radius operator defined with this operators does converge to $1$, but not anything more convincing (at least to me) EDIT: There might be good reasons to interpret them like that once we see how they converge to the classical coordinates or to the big Hilbert space coordinates
        
        \item HOWEVER, notice that it is still true that $[\cut x^i, \cut x^j] \neq 0 $ and it goes to $0$ with growing $k$.
        
        \item IMPORTANT: whatever interpretation we give to the $\xi$ operators is of no importance to the algebraic properties that they have, starting with the fact that they generate the algebra. \rtext{Perhaps we can study the new fuzzy sphere $S^1_\Lambda$ generated by the $\xi^i$'s without giving any physical interpretation: I lose (maybe this is even false) the physical motivation, but the algebra is still there }
        
        \item 
        
        \end{itemize}
    
    \item Is this example actually a fuzzy space? Can we formally say that our algebras approximate a commutative space? (I need to know the precise meaning of a fuzzy space)
    
        \begin{itemize}
            
        \item The way Madore justifies that the fuzzy sphere is an approximation of $S^2$ is a actually very similar: define an injection of a subalgebra (of near diagonal elements?) into the commutative algebra, and see how they expand within? SOmething like that.
            
        \end{itemize}
    
    \end{itemize}

\subsection{Medium Importance}

    \begin{itemize}
        
    \item The odd commutation relations of the coordinates appear for energies \textbf{below} the energy cutoff... but isn't that the opposite of what we want (i.e. that below some energy space works as we understand it, but "above" it the odd noncommutativity appear)?... Perhaps understand this as how NC is natural, that it is actually there since the beginning?
    
    \item Is this example actually a fuzzy space? Can we formally say that our algebras approximate a commutative space? (I need to know the precise meaning of a fuzzy space)
        
    \item Is $\cut O$ a good a good version of $O$ in the projected theory? For example, if $O = x$, position, I would think there is indeed a notion of measuring position in my system, even though my probes don't allow a very precise measurement... is this effective observable really encoded by $\cut x =\cut P x \cut P$? {\tiny What bugs me is that the observable $\cut x$ involves the measurement of $x$, and then projecting to the achievable energies, BUT I can't get the effect of measuring $x$, so \rtext{THIS IS NOT, I think, PRECISE}}    
    
    \end{itemize}


\subsection{Little Details}

    \begin{itemize}
    
    \item At leading order in what can the lowest eigenvalues of $H$ be considered rhose of the SHO approximation of equation $9$?
        
    \end{itemize}