\section{Doubts}

\subsection{Deal Changers}

    \begin{itemize}
    
    \item If we leave the $\cut {x^\pm}$ coordinates as the generators, their commutation relations look nothing like those of $\xi^\pm$ up to the claimed order, since $a^2 = 1 + \frac{\cdot}{\sqrt{k}} + \frac{\cdot}{k} + \cdot$... how can I justify (physically) the introduction of $\xi^\pm$, and their use to construct the theory and the use of THEIR commutation relations to define the better generators of $\acal_{\cut E}$?
    
        \begin{itemize}
            
        \item It seems to be true that $a = a(k)$ is just a number that \textbf{multiplies}, and in that case maybe it is not so bad to divide by it to obtain other observables $\xi^\pm$?
            
        \end{itemize}
    
    \item Is this example actually a fuzzy space? Can we formally say that our algebras approximate a commutative space? (I need to know the precise meaning of a fuzzy space)
    
        \begin{itemize}
            
        \item The way Madore justifies that the fuzzy sphere is an approximation of $S^2$ is a actually very similar: define an injection of a subalgebra (of near diagonal elements?) into the commutative algebra, and see how they expand within? SOmething like that.
            
        \end{itemize}
    
    \end{itemize}

\subsection{Medium Importance}

    \begin{itemize}
        
    \item The odd commutation relations of the coordinates appear for energies \textbf{below} the energy cutoff... but isn't that the opposite of what we want (i.e. that below some energy space works as we understand it, but "above" it the odd noncommutativity appear)?... Perhaps understand this as how NC is natural, that it is actually there since the beginning?
    
    \item Is this example actually a fuzzy space? Can we formally say that our algebras approximate a commutative space? (I need to know the precise meaning of a fuzzy space)
        
    \end{itemize}


\subsection{Little Details}

    \begin{itemize}
    
    \item At leading order in what can the lowest eigenvalues of $H$ be considered rhose of the SHO approximation of equation $9$?
        
    \end{itemize}