{ \color{gray}
    
    \cite{FioreTheCase2020} Projecting a quantum theory onto the Hilbert subspace of states with energies below a cutoff $\cut E$ may\todo{why may? Do they simply refer to the possibility of this not being a general result?} \rtext{lead to an effective [quantum] theory with modified observables, including noncommutative space(time)}.
        
    \cite{FioreTheCase2020} Adding a confining potential well $V$ with a very sharp minimum on a submanifold $N$ of the original space(time) $M$ may\todo{perhaps here the may refers to whether that is a correct interpretation of what is being done, since it might not be correct to say that we are doing $QM$ on $N$?} \rtext{induce a dimensional reduction to a \textbf{noncommutative quantum theory} on $N$} [the noncommutative quantum theory part, just as in the theory of Chakraborty and Scholtz, seems to mean that not only do the coordinates of phase space commute, but also that the space coordinates themselves do not commute.]
        
    
    \lin 
    
    \cite{FioreTheCase2020}First example of NC spacetime: Snyner in [6]: hope that \rtext{nontrivial commutation relations could cure ultraviolet divergences in QFT} But shortly after regularization introducing energy cutoff was introduced for QED, although not Poincare covariant.
    
    
    Why does it make sense to have an energy cutoff? At least $2$ reasons:
        
            \begin{enumerate}
                
            \item We might add $\cut E$ as a point where higher energy physics is unknown.
            
            \item Where neither we nor the environment can bring a state to higher energies. This gives an effective description of the system. It \dbtext{leas to a lower (distance? ~)} bound in the accuracy with which our apparatus can measure observables, \dbtext{coming from a maximum transferable energy}.
                
            \end{enumerate}
            
    \cite{FioreTheCase2020}The following important observations may stem from the same energy cutoff procedure: introducing an energy cutoff $\cut E$ in a quantum theory on a commutative space(time) $M$, i.e. projecting the theory on the Hilbert subspace with energy below $\cut E$, directly induces a \rtext{noncommutative deformation of the latter [the Q. theory]} and \rtext{lower bounds for the space(time) locallyzability}. Moreover, adding a confining potential well $V$ with a very sharp minimum on a submanifold $N$ of $M$ may induce a dimensional \rtext{reduction to a noncommutative quantum theory on $N$} \rtext{\textbf{\Huge [Here the NC Quantum Theory is obtained from the enrgy cutoff, and that the resulting NC Theory is a NC approximation of Q. Theory on $N$ comes from the cutoff!!}} ].
            \begin{enumerate}
            
            \item Regularization introducing energy cutoff was introduced for QED, although not Poincare covariant.
            
            \item DFR: concensus that any mergint of QM and GR should lead to a cutoff of local energy concentration (and an associated lower bound on localization of events). DFR made this more precise, and proposed that \rtext{the bound on the localizability of events could follow from appropriate NC coordinates}.
            
            \end{enumerate}
          
    \lin   
    
    $O(D)$-equivariance/covariance?
    
        \begin{itemize}
            
        \item \rtext{My choice:} It is true that the Hamiltonian and the group action commute, which implies that
            
                \begin{itemize}
                    
                \item The action of $G$ makes things fall again in $\mathcal H_{\cut E}$, since it even makes an eigenvalue of $H$ fall again into an eigenvalue of $H$ with the same energy.
    $\Rightarrow$ $O(2)$ acts on $\mathcal H_{\cut E}$ $\Rightarrow$ $O(2)$ acts on $\mathcal A_{\cut E}$ by inner automorphisms, i.e. \rtext{$O(2)$ acts by diffeomorphisms on the NC space}.         
                \item The time evolution of a ``rotated'' vector is simply given at any time as the rotation of the evolution of the orignal vector. Alternatively, $H^g = H$, so the time evolution of the a rotated vector is given by the same Hamiltonian.
                
                \item The two above things imply that the evolution of a vector in $\mathcal H_{\cut E}$ is given by the same $H$, and that \rtext{this evolution is invariant under $O(2)$}
                
                    
                \end{itemize}
                
            \item Why make so much emphasis on ``the commutation relations (satisfied by the algebra generators) are $G$-invariant?'' And what does it even mean? {\tiny It can't be simply that $[A, B]^g = [A^g, B^g]$ since that is trivial to see given that $g$ acts by inner isomorphisms, and so it wouldn't be necessary to say that this is true since the commutation relations and $H$ (and hence) $P_{\cut E}$ are so... \rtext{but I think that's what makes sense since it would mean that the generated algebra is ``the same''}. It can't mean that $[A, B]^g = [A, B]$, since not even for the generators $\cut{x^\pm}$ that is true, since under reflections $\cut L$ does change.
            }  
            
            \item Fiore and Pisacane try to give an explanation about what this covariance means in one of their last papers, perhaps even the last one, but \dbtext{I haven't read it}.
        
        \item For ``\lbtext{covariance}'': $G$ is a symmetry of the theory:
            
            \begin{itemize}
                
            \item $[g\cdot , H] = 0$ implies \otext{$[g\cdot , P_{\cut E}] = 0$}
            
            \item That, in turn, implies that \otext{$\cut{A}^g = \cut{A^g}$}
                
            \end{itemize}
            
        \end{itemize}
    
    \lin
    
    Applications:
    
        \begin{itemize}
            
        \item \cite{FioreTheCase2020} These models might be suggestive for effective models in quantum field theory, quantum gravity or condense matter physics.
            
        \end{itemize}
}

%%%%%%%%%%%%%%%%%%%%%%%%%%%%%%%%%%%%%%%%%%%%%%%%%%%%%%%%%%%%%%%%
%%%%%%%%%%%%%%%%%%%%%%%%%%%%%%%%%%%%%%%%%%%%%%%%%%%%%%%%%%%%%%%%
\section{General Setting}

{
    \color{gray}
    
    - $H = - \frac{1}{2} \Delta + V(r)$ invariant under $O(D)$
    
    - Introducing the cutoff $\cut E$, as the energy where
    \begin{align}
        \label{eqn5}
            V(r) \approx V_0 + 2k(r-1)^2 && \text{for $r$ such that $V(r) \leq  \cut E$}.
    \end{align}
    %in the region $\nu_{\cut E} = \{r \,|\, V(r) \leq \cut E\}$.
    
    - Eigenfunctions of $H$ as product of a spherical harmonic $Y(\phi, \dots)$ (eigenvector of $L^2$) and an eigenfunction of the radial equation 
    \begin{align}
        \label{eqn9}
        \left[-\partial_r^2 - (D-1) \frac{1}{r} \partial_r + \frac{1}{r^2} j(j+D-2) + V(r)\right] \tilde f(r) = E \tilde f(r).
    \end{align}
    - This last equation can be approximated by a harmonic oscillator equation, since outside the region $\{V(r) \leq \cut E\}$ $\psi$ is negligibly small.
    : \rtext{$\hcal_{\cut E} \approx $ solutions of Schrodinger Eq. of energy $\leq \cut E$, $\acal_{\cut E} = End(\hcal_{\cut E})$}.
}

\linea



%%%%%%%%%%%%%%%%%%%%%%%%%%%%%%%%%%%%%%%%%%%%%%%%%%%%%%%%%%%%%%%%
%%%%%%%%%%%%%%%%%%%%%%%%%%%%%%%%%%%%%%%%%%%%%%%%%%%%%%%%%%%%%%%%
\section{Construction of $\acal_{\cut E}$ for $D = 2$}

{
    \color{gray}
    
    - Equation \eqref{eqn9} has the approximation, where $\rho := \ln r$
    \begin{equation}
        \label{harmonic2D}
        \hat H f(\rho) = e_m f(\rho), \qquad
        \hat H = - \partial_\rho^2 + k_m(\rho - \tilde \rho_m)^2,
    \end{equation} $
        k_m := 2(k - E'), \quad
        E' := E - V_0, \quad
        \tilde \rho_m := \frac{E'}{k_m}, \quad
        e_m = \frac{E'^2}{k_m} + E' - m^2
    $
    - The solutions $f_{n,m}$ are known, $n \in \bb N, m \in \ZZ$ \then $e_{m, n}(k) = (2n+1)\sqrt{k_m}$ \then $E'_{m,n}(k)$ satisfies a quartic equation \then $E_{m,n}(k, V_0)$.
    
    - Fixing $V_0 = V_0(k)$ such that $E_{0, 0} = 0$ \then $V_0(k) = -\sqrt{2k} + 2 - \frac{7}{2}\frac{1}{\sqrt{2k}} + o(1/k)$ and 
    $\sum_{n = -1}^\infty v_n \left( \sqrt{\frac{1}{k}} \right)^n \approx -\sqrt{2k} + 2 - o(1/k)$ \then
    \begin{equation}
        \rtext{E_{n, m}(k)} = m^2 + 2n\sqrt{2k} - 2n + o(1/\sqrt{k})
    \end{equation}
    
    - Choosing $\cut E < 2 \sqrt{2k} - 2$, the spectrum of $H$ is a truncation of $L^2$: \textbf{the radial oscillations are ``frozen''}: $\cut{\partial_r} = 0$.
    \begin{multline*}
        \lbtext{\psi_m(\rho, \phi)} := f_{0, m}(\rho) e^{im\phi} = c_m e^{im\phi}exp{\left[ -\frac{(\rho - \tilde \rho_m)^2 \sqrt{k_m}}{2} \right]} \\\xrightarrow{k \to \infty} \delta(r-1)e^{i m \phi}
    \end{multline*}
    \begin{equation}
        E = E_m(k) = m^2 + o(1/\sqrt{k})
    \end{equation}
    
    - For $\lbtext{\Lambda} := \lfloor \cut E \rfloor$, 
    \begin{align}
        \lbtext{\hcal_{\Lambda}}:= \lbtext{\hcal_{\cut E}} := span\{\psi_m\}_{|m| \leq 
    \lfloor \cut E \rfloor} ,
    \lbtext{\acal_\Lambda} := \mathcal B(\hcal_\Lambda)
    \end{align}
    
    - Since $H$ generates the time evolution, a
    An element of $\hcal_\Lambda$ doesn't evolve out of $\hcal_\Lambda$.
    
    - Get a fuzzy space: e.g. choosing $k = \Lambda^2(\Lambda+1)^2$ --make $k$ diverge with $\Lambda$ while $\nu_{\cut E}$ goes to $\{r = 1\}$
    
    - This cutoff entails replacing every observable by $A \mapsto \lbtext{\cut A} = P_{\cut E} A P_{\cut E}$% \dbtext{when?}
}

\linea


%%%%%%%%%%%%%%%%%%%%%%%%%%%%%%%%%%%%%%%%%%%%%%%%%%%%%%%%%%%%%%%%
%%%%%%%%%%%%%%%%%%%%%%%%%%%%%%%%%%%%%%%%%%%%%%%%%%%%%%%%%%%%%%%%
\section{Important Observables and their Commutation Relations}

{
    \color{gray}
    
    - Up to infinite, $1/k^{1/2}\dbtext{?}$ and $1/k^{3/2}$ orders, respectively
    \begin{align}
    \label{projObs2D}
        \rtext{\cut L} \psi_m &= m \psi_m; & 
        \rtext{\cut H} &= \cut L^2; & 
        \rtext{\cut x^\pm} \psi_m = 
            \begin{cases}
                \frac{a}{\sqrt{2}} \sqrt{ 1 + \frac{m(m \pm 1)}{k} } \psi_{m \pm 1} & -\Lambda \leq \pm m \leq \Lambda - 1 \\
                0 & \text{otherwise}
            \end{cases}
    \end{align}
    - And so, up to terms of $1/k^{3/2}$
    \begin{align}
        \label{conmObs2D}
        \cut{x^+}^\dagger &= \cut{x^-}; &
        [\cut L, \cut{x^\pm}] &= \pm \cut{x^\pm}; &
        \rtext{[\cut{x^+}, \cut{x^-}]} &= - \frac{\cut L}{k} + \left[1 + \frac{\Lambda(\Lambda+1)}{k}\right] (\tilde P_{\Lambda} - P_{-\Lambda})a^2.
    \end{align}
     - If \eqref{projObs2D} are used exactly to define elements of $\mathcal B(\hcal_\Lambda) \equiv \acal_\Lambda$ then \eqref{conmObs2D} are also exact, and $\cut{x^\pm}$ generate $\acal_\Lambda$. -- $\cut{\partial_\pm}$ are now redundant... but I'm not sure why
     
}

\linea


%%%%%%%%%%%%%%%%%%%%%%%%%%%%%%%%%%%%%%%%%%%%%%%%%%%%%%%%%%%%%%%%
%%%%%%%%%%%%%%%%%%%%%%%%%%%%%%%%%%%%%%%%%%%%%%%%%%%%%%%%%%%%%%%%
\section{Realization of $\acal_\Lambda$ through $U\soth$}

{
    \color{gray}
    
    - $O(2)$ acts on $\hcal_\Lambda \subset L^2(\RR^2)$, and so \rtext{on $\acal_\Lambda$}, since $[H, O(2)\cdot ] = 0$. through the action induced in $\acal_\Lambda$ by its action on $\RR^2$.
        \begin{itemize}
            
        \item \textit{Rotation} $R_\theta$: $\cut{x^\pm} \mapsto e^{\pm i \theta} \cut{x ^\pm}; \cut L \mapsto \cut L \in \acal_\Lambda$.
        
        \item \textit{Reflection}: $\cut{x^\pm} \mapsto -\cut{x^\mp}; \cut L \mapsto -\cut L$.
        
        \end{itemize}
    
    - \rtext{We can consider \otext{$\acal_\Lambda \cong  M_N(\CC) = \pi_\Lambda(Uso(3))$} \textbf{as a $*$-algebra and representations of $O(2)$}}, where $\pi_\Lambda$ is the $\lbtext{N} := 2 \Lambda + 1$ dimensional representation. $SU(N) \ni g$ is the group of $*$-automorphisms of $M_N(\CC) \cong \acal_\Lambda$ acting by $a \mapsto g a g^{-1}$; $O(2)$ is then a subgroup. Comes from mapping:
    \begin{align}
        \rtext{\cut{x^\pm}} &\longleftrightarrow \rtext{f_\pm (J^0) J^\pm}, &
        \rtext{\cut L }& \rtext{\longleftrightarrow J^0}
    \end{align}
    where $J^\pm, J^0$ is the Weyl-Cartan basis of $so(3)$, $\lbtext{f_\pm(s)} := \frac{1}{\sqrt{2}} \sqrt{\frac{1 + s(s-1)/k}{\Lambda (\Lambda + 1) - s(s-1)}} =: \lbtext{f_-(-s)}$%: $[J^+, J^-] = J^0; [J^\pm, J^0] = \pm J^\pm$

    - $O(2) \subset SO(3)$: \textit{Rotation}: $\pi_\Lambda(e^{i \theta J_0})$; \textit{Refl.}: $\pi_\Lambda(e^{i\pi (J^+ + J^-)/\sqrt{2}})$
}

\linea


%%%%%%%%%%%%%%%%%%%%%%%%%%%%%%%%%%%%%%%%%%%%%%%%%%%%%%%%%%%%%%%%
%%%%%%%%%%%%%%%%%%%%%%%%%%%%%%%%%%%%%%%%%%%%%%%%%%%%%%%%%%%%%%%%
\section{Convergence}

{
    \color{gray}
    
    - \textbf{$\psi_m$ as fuzzy analogues of $e^{i m \phi} \in \hcal$}: $O(2)$-covariant embedding $\hcal_\Lambda \hookrightarrow \otext{\hcal = L^2(S^1)}$, $\psi_m \mapsto e^{im\phi}$; \hfill \\then $\hcal_\Lambda \to \hcal$ as $\Lambda \to \infty$ in the sense that $\forall \phi \in \hcal$, $\lbtext{\phi_\Lambda} := \sum_{|m| \leq \Lambda} \phi_m e^{im\phi} \to \phi$ in the $L^2$-norm.
    
    - Induces, \rtext{embedding $\acal_\Lambda \hookrightarrow \otext{\acal = \mathcal B(\hcal)}$ and limit $\acal_\Lambda \to \acal$} as $\Lambda \to \infty$.
    
    - \textbf{Fuzzy analogue of $B(S^1)$ of (bounded) functions on $S^1$} as subalgebra (act by mult.) of $\mathcal B(\hcal)$: $\lbtext{C_\Lambda} := \left\{ \sum_{h = -2\Lambda}^\Lambda f_h \eta^h \,|\, f_h \in \CC\right\}$ where $\eta^\pm  = \frac{\sqrt{2}}{a}x^\pm$ (so $\eta^\pm \to e^{\pm i \phi}$ as operators).
    
    - Choosing $k(\Lambda) \geq 2 \Lambda(\Lambda + 1)(w\Lambda+1)^2$, then \rtext{$B(S^1) \to \mathcal B(\hcal)$ as operators} due to the strong limits: $\hat f_\Lambda \to f\cdot$, $\hat{(fg)}_\Lambda \to fg\cdot $, $\hat f_\Lambda \hat g_\Lambda \to fg\cdot$, where $\lbtext{\hat f_\Lambda} := \sum_{h = -2\Lambda}^{2\Lambda} f_h \eta^h$ is ``truncation'' of $f \in B(S^1)$.
    
}

\linea

