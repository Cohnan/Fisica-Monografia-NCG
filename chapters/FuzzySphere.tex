The $2$-sphere $S^2$ as a metric space is $SU(2)$-invariant. This invariance will be seen to give rise to a series decomposition of the canonical Dirac operator on $S^2$. We will see how truncating this $SU(2)$-decomposition of the Dirac operator will give rise to equivariant spectral triples on the fuzzy sphere which inherit some of the nice properties of the canonical spectral triple and which allow the convergence of the fuzzy sphere to $S^2$ also as a metric space, in a precise sense to make precise in section \ref{FSsec:convergence}.

\dbtext{Physical applications of the Fuzzy sphere.}

%%%%%%%%%%%%%%%%%%%%%%%%%%%%%%%%%%%%%%%%%%%%%%%%%%%%%%%%%%%%%%%%%%%%%%%%%%%%%%
%%%%%%%%%%%%%%%%%%%%%%%%%%%%%%%%%%%%%%%%%%%%%%%%%%%%%%%%%%%%%%%%%%%%%%%%%%%%%%
\section{Canonical Spectral Triple of $S^2$}

% {\color{gray}-- Arising from $SU(2)$ equivariant algebraic Dirac element, in the context of $G$ equivariant spectral triples.

% \linea 

% Presentation:

% (Inherits from $\RR^3$ the metric and the metric connection, but the spinor space changes)

% \rtext{Starting point: $SU(2)$-isometries}%\cite{DAndrea2013}
% - $S^2$ as the symmetric space $S^3/S^1$ of the compact semisimple Lie group $G = S^3$, $\mathfrak g = su(2)$.
    
% - The canonical spectral triple, which is \textbf{$SU(2)$-equivariant} can be seen to come from a purely algebraic element $\lbtext{\mathcal D} \in U(\mathfrak g) \otimes U(\mathfrak g)$:
%     \begin{align*}
%         U(\mathfrak g) \otimes U(\mathfrak g) &\to& U(\mathfrak g) \otimes Cl(\mathfrak g, -K) &\to& \mathcal B(L^2(G/U, \Sigma G/U)) \\
%         1 \otimes 1 + 2 \sum_{k = 1}^3 J_k \otimes J_k &\mapsto& 1 \otimes 1 + \sum_k J_k \otimes \sigma_k &\mapsto& \rtext{\left( \bigoplus_{l\in \bb N} \pi_l \right) \otimes \pi_{1/2}(\mathcal D)}
%     \end{align*}
    
% (- In Sanchez: $\mu = 1, 2$
%  $\nabla_\mu = \partial_\mu - \frac{c_\mu}{q} \sigma_\mu \sigma_{\cut \mu} \cdot$, 
%  $\slashed D = -i \frac{q}{2} \sigma^\mu \nabla_\mu)$)

% - So $\lbtext{\slashed D} = \begin{pmatrix} 1 + \partial_H & \partial_F \\ \partial_E & 1 - \partial_H\end{pmatrix} = 1 + \partial_F \otimes \sigma_+ + \partial_E \otimes \sigma_- + \partial_H \otimes \sigma_3$ where $\partial_H = -i \partial_\phi$, $\partial_F = e^{i\phi} \left( \partial_\theta + i cot\,\theta \partial_\phi \right)$, $\partial_E = -\partial_F% = e^{-i\phi} \left( \partial_\theta - i cot\,\theta \partial_\phi \right)
% $ are the actions of $J_3$, $J^\pm \in i\,su(2)$ on $L^2(S^2)$ respectively. \textit{Eig. vectors}: orth. basis of $\hcal$; \textit{Spectrum} $= \{\pm l\} = \ZZ - 0$ with multiplicities $2l$. (The tensor with $2x\times 2$ matrices means that we are looking at spinors as column vectors $\psi = \begin{pmatrix} \cdot \\ \cdot \end{pmatrix}$)%: L^2(S^2) \otimes \CC^2 \to L^2(S^2) \otimes \CC^2 = \oplus_{l \in \bb N} (\pi_l \otimes \pi_{1/2})(\mathcal D)$

% ( - The eigenvectors (spinor harmonics) $Y^{'}_{lm}, Y^{''}_{lm}$, $l \in \NN + 1/2$, of $D$ make up an orthogonal basis of the spinor fields $\hcal$.
%  - The eigenvalues are: of $Y'_{l\cdot}: (l + 1/2) \in \NN$, of $Y{''}_{l\cdot}: -(l + 1/2)$, with multiplicities $2l+1$)
% }
\linea

Although there are several ways to describe the canonical spectral triple on \todo{of/on?} $S^2$, the one that will be useful to use to define the spectral triples on the fuzzy sphere will be from understanding $S^2$ as \ptext{compact Riemannian symmetric\footnote{Symmetric space; U} space} $S^2 \cong G/U$ of the compact semisimple Lie group $G = SU(2)$.

Let $G$ be a compact semisimple Lie group $G$ with Lie algebra $\alg g$, where the semisiplicity means that the Killing form $K: \alg g \times \alg g \to \alg g$ of $\alg g$ is negative definite, giving $G$ a natural Riemannian manifold structure. For such a manifold $G$, and its symmetric spaces, the Dirac operator can be seen to arise \ref{} from an algebraic element $\lbtext{\dcal } \in U(\alg g)\times U(\alg g)$, where $\lbtext{U}(g)$ is \lbtext{the universal enveloping algebra} of the Lie algebra $\alg g$, i.e. the largest\todo{in what sense} \todo{complex?} unital, associative algebra containing $\alg g$ where the Lie bracket coincides with the commutator in $U(\alg g)$; it is important to remark that \otext{the representations of $\alg g$ are in a bijective correspondence with the modules over $U(\alg g)$}. In the explicit case where $G = SU(2)$,
\begin{equation}
    \lbtext{\dcal} := 1 \otimes 1 + 2 \sum_{k = 1}^3 J_k \otimes J_k \in U(\sut) \otimes U(\sut)
\end{equation}
where $J_k \in U(\sut), k = 1, \dots, 3$ are the basis of $\sut\otimes \CC$ such that $[J_i, J_j] = i \epsilon_{ijk} J_k$ (i.e. $J_k = \frac{\sigma_k}{2} = \pi_{1/2}(J_k)$). 

\lin

Let $\lbtext{\pi_j}: U(\sut) \to M_{2j+1} = End(\CC^{2j+1})$ be the spin $j \in \frac{\NN}{2}$ representation of $\sut$ where each element of the canonical basis $e_m \in \CC^{2j+1}$, $m = -j, \dots, j-1, j$ satisfies $\pi_j(\vec J^2)(e_m) = j(j+1) e_m$ and $\pi_j(J_3)(e_m) = m e_m$, where $\vec J^2$ is notation for $J_1^2 + J_2^2 + J_3^3 \in U(\sut)$.

Recall that the every continuous / square integrable / smooth function on $S^2$ may be decomposed in terms of spherical harmonics $Y^l_m \in L^2(S^2)$, $l \in \NN$, $m \in \{-l, \dots, l-1, l\}$, and the spaces $\tilde V_l = span\{Y^l_m\}_{|m| \leq l}$ for fixed $l$ are the spaces of homogeneous polynomials on the coordinates $x^1, x^2, x^3$ of $\RR^3$, which are precisely all the irreducible representation spaces of $SO(3)$, and another description of the spin-$l$ representations $V_l$ of $SU(2)$. 
This decomposition of $L^2(S^2)$ is in fact the one induced by the action of $SU(2)$, inherited from the action on $S^2$, and which can be seen to be induced by the natural action of $\sut$ on $S^2$ as vector fields. In particular, 
\begin{align}
    \partial_H &:= - i \partial_\phi & \text{is the action of $\lbtext{H} := J_3$ in $L^2(S^2)$} \\
    \partial_F &:= e^{i\phi} \left( \partial_\theta + i cot\,\theta \partial_\phi \right) &\text{is the action of $\lbtext{F} := J_+ = J_1 + i J_2$ in $L^2(S^2)$}\\
    \partial_E &:= - \partial_E &\text{is the action of $\lbtext{E} := J_- = J_1 - i J_2$ in $L^2(S^2)$}
\end{align} where $\phi$ and $\theta$ are the azimuthal and polar angles in $S^2$, respectively.
Hence, $L^2(S^2) \cong \bigoplus_{l = 0}^\infty V_l$ as representation spaces of $SU(2)$.

The spinor bundle of $S^2$ is trivial, so the space of spinor fields $\hcal = L^2(S^2, \Sigma S^2)$ of $S^2$ is isomorphic (as a vector space) to $L^2(S^2) \otimes \CC^2$, where $\CC^2$ is understood as the \textit{fermionic Fock space} associated to the tangent spaces of $S^2$, i.e. the unique irreducible representation of the Clifford algebra $\CC l_2 = M_2(\CC) = \pi_{1/2}(U(\sut))$. Therefore, \rtext{the spinor bundle of $S^2$ decomposes as representation of $SU(2)$ as $L^2(S^2, \Sigma S^2) \cong \bigoplus_{l = 1}^\infty V_l \otimes V_{1/2}$}; \ptext{we will understand from now on the separable Hilbert space $\lbtext{\hcal} = L^2(S^2)\otimes \CC^2$ as the space of spinor fields on the 2-sphere}.

\lin

This decomposition of the spinor bundle induced by the action of $SU(2)$ correctly suggests that the Dirac operator can be seen as the operator
\begin{equation}
    \lbtext{\slashed D} = \rtext{ \bigoplus_{l\in \bb N} \pi_l  \otimes \pi_{1/2}(\dcal)} = 1 + \partial_F \otimes \sigma_+ + \partial_E \otimes \sigma_- + \partial_H \otimes \sigma_3 = \begin{pmatrix} 1 + \partial_H & \partial_F \\ \partial_E & 1 - \partial_H\end{pmatrix} ,
\end{equation} acting on $L^2(S^2)\otimes \CC^2$, where $\sigma \pm = \sigma_1 \pm i \sigma_2$.

The eigenvectors of $\slashed D$ $Y^{'}_{jm}, Y^{''}_{jm}$ $j \in \NN + 1/2$, $m = -j, \dots, j$, called the \textit{spinor harmonics}, make up an orthogonal basis of the spinor fields $\hcal$. The eigenvalue of the $Y'_{l\cdot}$ is $(j + 1/2) \in \NN$, and of the $Y{''}_{j\cdot}: -(l + 1/2)$, so the eigenvalue $j$ has multiplicity $2j+1$. The spectrum of $\slashed D$ (and the respective multiplicities) will be an indication of how good of an approximation is another Dirac operator in the Fuzzy sphere. 

The following aiding lemma will allow us to study the spectrum of the Dirac operators of the Dirac triples that will be defined on the fuzzy sphere.

\begin{lemma}\label{lemmaSpectrumAlgebraic}
The operator $(\pi_0 \otimes \pi_{1/2})(\dcal^2): V_0 \otimes \CC^2 \to V_0 \otimes \CC^2$ has as unique eigenvalue $1$ with multiplicity $2$. For any $j \in \frac{\ZZ_{\geq 1}}{0}$, the $(\pi_j \otimes \pi_{1/2})(\dcal^2)$ has two eigenvalues: $j^2$ with multiplicity $2j$, and $(j+1)^2$ with multiplicity $2j+2$.
\end{lemma}

\linea

{\color{gray} (The following was previously right after the paragraph about the representations of $SU(2)$, but may now be unnecesary)

Now, to $\dcal \in U(\alg g) \otimes U(\alg g)$ corresponds an element $\dcal_S$ in the noncommutative Weil algebra\footnote{CCR algebra of $(\alg g, \omega?)$. } through the injection $\alg g \hookrightarrow Cl(g, K)$/$\alg g_\CC \hookrightarrow \CC l(G/U)?$\todo{not sure we want here the real algebra, and I even think we want instead the complex algebra since it is that one which acts on Fock space}\todo{not even sure if we want this CLifford algebra, of the one before, since it is that one which acts on the spinors on $S^2$, and so it is true that $\CC l_2 = M_2(\CC)$}\footnote{Clifford algebra}. It is known that, for a $3$ dimensional real vector space $V$ with positive definite metric with basis $\{v_k\}_{k = 1, \dots, 3}$, the real Clifford algebra $Cl(V, g) \cong gen\{\sigma_k\}_{k = 1, \dots, 3} = gen\{J_k\}_{k = 1, \dots, 3} = M_2(\CC)$ and $\CC l(V) \cong M_2(\CC) \oplus M_2(\CC)$ under the identification $v_k \Longleftrightarrow \sigma_k \in M_2(\CC)$, meaning, for our explicit example, that the algebra $Cl(\sut, -K) = M_2(\CC)$ 
\todo{there is an $i$ which I don't like when saying this last sentence}, and so the injection can be seen to be simply $\pi_{1/2} = Id_2$:
\begin{equation}
    \dcal_S := (Id_2 \otimes \pi_{1/2})(\dcal) = 1 \otimes 1 + \sum_{k = 1}^3 J_k \otimes \sigma_k = \begin{pmatrix} 1 + H & F \\ E & 1 - H \end{pmatrix} \in U(\sut)\otimes M_2(\CC)
\end{equation}
where $H = J_3$, $E = J_1 + iJ_2$ and $F = E^* \in U(\sut)$


Equivariance of $\dcal$: $U(\alg g)$ is a Hopf algebra\footnote{} with coproduct $\nabla: U(\alg g) \to U(\alg g) \times U(\alg g)$ such that $A \mapsto A \otimes 1 + 1 \otimes A$ for $A \in \alg g$.}

\lin 



%%%%%%%%%%%%%%%%%%%%%%%%%%%%%%%%%%%%%%%%%%%%%%%%%%%%%%%%%%%%%%%%%%%%%%%%%%%%%%
%%%%%%%%%%%%%%%%%%%%%%%%%%%%%%%%%%%%%%%%%%%%%%%%%%%%%%%%%%%%%%%%%%%%%%%%%%%%%%
\section{The Fuzzy Sphere}

% {\color{gray} -- Change $x_i$ in $\RR^3$ by infinitesimal rotation $J_i$, acting on $\CC^{N+1} = span\{|j, m\rangle\}$.

% Presentation:

% \textbf{Fuzzy Space}: ($C^*$? or simply $*$?\todo{}) 
%     family of noncommutative finite dimensional $\acal_n$ parametrized by $n \in \bb N$ with increasing dimension and such that that approximate the commutative algebra $\acal$. Why? \textit{Keep continuous symmetries}.
%         % \begin{itemize}
%         % \item Why? To preserve the (continuous) symmetries of the space while keeping the algebra finite dimensional.
%         % \end{itemize}
    
%     \textbf{Fuzzy Sphere}: Notice that $\acal = C(S^2) \cong \bigoplus_{l \in \bb N} V_l \cong L^2(S^2)$, where $V_l$ is the spin $l$ representation of $SO(3)$: homogeneous $l$-degree polynomials in $x^1, x^2, x^3$, with basis $\{Y^l_m\}_{|m| \leq l}$. 
%     For $N = 2j \in \bb N$, 
%     $\lbtext{\acal_N}= \bigoplus_{l = 0}^N V_l$ as $SU(2)$ representation. 
%     As algebra: replacing $x^i \mapsto \frac{1}{\sqrt{j(j+1)}} \pi_{j}(J_i)$, $[J_i, J_k] = i \epsilon_{ijk} J_k$, $\lbtext{\acal_N} := End(V_j) = M_{N+1}(\CC)$, understanding $V_j= \CC^{N+1} = span\{|j, m\ket\}_{|m| \leq j}$ as irrep. of $SU(2)$; this follows from  $\longrightarrow$ \rtext{$[x^i, x^j] = \frac{1}{\sqrt{j(j+1)}} i \epsilon_{ijk} x_k$}, $\sum x_1^2 + x_2^2 + x_3^2 = 1$.
    
%     With these Dirac Spectral Triples \cite{DAndrea2013} \rtext{approximates $S^2$ as: \textbf{1.} $C^*$-algebra $\acal$ acting on the spinor fields $\hcal$; \textbf{2.} Representation of $SU(2)$; \textbf{3.}  Metric space on which $SU(2)$ acts by isometries.}
    
%     %. Under the adjoint action, which makes sense since R J_3 R^{-1} is rotation under rotated axis
    
%     % This allows to define fuzzy spherical harmonics (changing x's by new x's) which make up a basis, good action under SU(2)
% }    
\linea

A \lbtext{fuzzy space} is a sequence of finite dimensional $C^*$-algebras $\{\acal_N\}_{N \in \NN}$ with increasing dimension which approximate, in some sense to be determined later/in the specific case \todo{is there a precise definition?}, a (commutative) $C^*$ algebra $\acal$ in the limit $N \to \infty$. When the commutative algebra encodes a topological space or a manifold with certain symmetry group $G$ that acts by homeomorphisms or diffeomorphisms, we require a fuzzy space to implement this symmetry, i.e. that $G$ acts on each $\acal_N$ by $*$-isomorphisms (the algebraic version of homeomorphisms\todo{be certain of this.}) and in a way compatible with the derivations of $\acal$ (algebraic version of diffeomorphisms).

With the full spectral triple defined below, the \textit{fuzzy sphere} will approximate $S^2$ as:
    
    \begin{enumerate}
    
    \item A $C^*$-algebra acting on the Hilbert space of spinors \todo{Madore does this}.
    
    \item $SU(2)$ acts by ``diffeomorphisms'' on the space.
    
    \item A metric space on which $SU(2)$ acts by ``isometries''.
    
    \end{enumerate}

\lin 

To define the fuzzy sphere $\{\acal_N\}_{N \in \NN}$ we start with the decomposition of $\lbtext{\acal} = C(S^2) =  \bigoplus_{l \in \NN} \tilde V_l \cong L^2(S^2)$ \todo{different ``limits''} into irreducible representations of $SO(3)$, where, recall, $\tilde V_l$ is the vector space of homogeneous polynomials on the coordinates $x_1, x_2, x_3$ of $\RR^3$. This suggest the definition of $\lbtext{\acal'_N} := \bigoplus_{l = 0}^N \tilde V_l$, for all $N \in \NN$, as the elements of the sequence conforming the fuzzy sphere. This definition seems appropriate in at least two ways: there is a clear way in which they approximate $\acal$; $SU(2)$ acts naturally on each $\acal_N'$.

However, an obvious problem with this definition is that $\acal_N'$ doesn't have a natural multiplication, so it isn't an algebra. From the case $N = 1$ there is a suggestion on how to define a multiplication\cite{Madore}: requiring that the radical\footnote{} of the algebra be zero, there are only two possibilities, and only one being noncommutative:
\begin{align}
    x_i &\mapsto \hat x_i := \lambda \sigma_i & i = 1, 2, 3;
\end{align}
where $\lambda\in \CC$ is any constant; this means that we are defining the $N = 1$ element of the fuzzy sphere to be $\lbtext{\acal_1} := M_2(\CC)$. Notice that the identification $x_i \mapsto \hat x_i$ is $SU(2)$-covariant since $\frac{\sigma_i}{2} \in U(\sut) \cong U(\alg{so(3)})$ corresponds to the generator of rotations with respect to the $x_i$-axis. The constant $\lambda$ is chosen such that ``radius of the sphere be $1$'', meaning that $\hat x_1^2 + \hat x_2^2 + \hat x_3^2 = 1$, hence $\lambda = \frac{2}{\sqrt{3}} = \frac{1}{\sqrt{\frac{1}{2} (\frac{1}{2} + 1)}}$.

Instead of insisting on defining a foreign multiplication in $\acal'_N$, the \lbtext{fuzzy sphere} is now defined by $\lbtext{\acal_N} :=  \pi_{j}(U(\sut)) \equiv End(V_j)$ which, when the irreducible representation $\pi_j$ of $SU(2)$ is defined as in the previous subsection where $V_j = \CC^{N+1}$, $\acal_N \equiv M_{N+1}(\CC)$ for all $N \in \NN$, $j$ satisfying $N =: 2\lbtext{j}$, which arises from the following generalization of the $N = 1$ case studied above:
\begin{align}
    \tilde V_1 \hookrightarrow \bigoplus_{l = 0}^N \tilde V_l &\to \acal_N\\
    x_i &\mapsto \hat x_i := \frac{1}{j(j+1)} \pi_{j}(J_i) & i = 1, 2, 3;
\end{align} that is, for each $N$, the coordinate $x_i$ is understood as the infinitesimal rotation with respect to the axis $x_i$ operator on the space $V_j = \CC^{N+1}$. The normalization constant guarantees again $\hat x_1^2 + \hat x_2^2 + \hat x_3^2 = 1$, and we now obtain in general the \rtext{nonzero commutation relation ``of the coordinates''}:
\begin{align}
    \rtext{[\hat x^i, \hat x^j] = \frac{1}{\sqrt{j(j+1)}} i \epsilon_{ijk} \hat x_k}, && i, j, k \in \{1, 2, 3\},
\end{align} where sum over repeated indices is understood, and $\epsilon_{ijk}$ is the Levi-Civita tensor.

\lin

This \textit{fuzzy} version of the sphere is, so far, good in two of the three ways stated at the beginning of this subsection:

\subsubsection{$SU(2)$ acts by diffeomorphisms}

From the $SU(2)$-equivariance of the identifications $x_i \to \hat x_i$, we see that $\acal_N \cong \acal'_N \cong \bigoplus_{l=0}^N V_l$ as representation space of $SU(2)$, where the action of $g \in SU(2)$ is the one induced in $\acal_N$ seen as $End(V_j)$, namely by the inner automorphism $B  \mapsto \pi_j(g)B \pi_j(g)^{-1}$, where $\pi_j(g) \in SU(N+1)$ since $\pi_j$ is an unitary representation. That $(\pi_j(g)B \pi_j(g)^{-1})^* = \pi_j(g)B^* \pi_j(g)^{-1}$, where $B^*$ is the adjoint of $B \in M_{N+1}(\CC)$, shows that \rtext{$SU(2)$ acts by $*$-automorphisms on $\acal_N$}.

A diffeomorphism of $S^2$ is equivalent to an automorphism of the algebra $C^\infty(S^2)$ (dense subalgebra of $\acal$) that respects the complex conjugation $*$. In the matrix algebras $M_{N+1}(\CC)$, since they are simple, all $*$-automorphisms are precisely of the form $B \mapsto C B C^{-1}$ for some $C \in SU(N+1)$. Furthermore, all such automorphisms also respect the derivations of $\acal_N$, which are the noncommutative analogues of vector fields. Hence, all $*$-automorphisms of $\acal_N$ are good analogues of diffeomorphisms of $S^2$, and so \rtext{$SU(2)$ \textit{is acting by diffeomorphisms on the fuzzy sphere}}.

\subsubsection{As $C^*$-algebra}

The elements $\hat x_i \in \acal_N$, $i = 1, 2, 3$ generate the algebra; furthermore, each element in $a \in \acal_N$ has a unique expansion $a = \sum_{l = 0}^N \frac{1}{l!} a_{\mu_1 \cdots a_l} \hat x_{a_{\mu_1}}  \cdots \hat x_{a_{l}}$, where $a_{\mu_1 \cdots a_l} \in \CC$ is symmetric and trace free as a tensor. Replacing the coordinates $x_i$ by $\hat x_i$ in the polynomial expansion of the spherical harmonics $Y^l_m$, $l = 0, \dots, N$ and $|m| \leq l$, we obtain the \lbtext{fuzzy harmonics} $\hat Y^l_m \in \acal_N$ that make up a basis of $\acal_N$.

Define the linear, injective mapping
\begin{align}
    \acal_N &\to \acal'_N \subset \acal \\
    \sum_{l = 0}^N \frac{1}{l!} a_{\mu_1 \cdots a_l} \hat x_{a_{\mu_1}}  \cdots \hat x_{a_{l}} &\mapsto \sum_{l = 0}^N \frac{1}{l!} a_{\mu_1 \cdots a_l} x_{a_{\mu_1}}  \cdots x_{a_{l}}.
\end{align}
Then, the sequence of noncommutative $C^*$-algebras approximates the commutative algebra $A = C(S^2)$ in the limit $N \to \infty$ in the sense that the above map gets increasingly closer to being an algebra morphism and \textbf{the algebra $A$ can be considered as the image of the diagonal matrices in $\acal_N$}; the precise statements and their proofs may be found in \cite{Madore}.

\subsubsection{As metric space on which $SU(2)$ acts by isometries}

This is what theorem \ref{theoremReducibleSpectralTripleDiracApproximation} and the remark that follows assert.

%%%%%%%%%%%%%%%%%%%%%%%%%%%%%%%%%%%%%%%%%%%%%%%%%%%%%%%%%%%%%%%%%%%%%%%%%%%%%%
%%%%%%%%%%%%%%%%%%%%%%%%%%%%%%%%%%%%%%%%%%%%%%%%%%%%%%%%%%%%%%%%%%%%%%%%%%%%%%
\section{Spectral Triples}

% {\color{gray} --  Truncate the $SU(2)$-equivariant \textbf{decomposition} of the canonical spectral triple}

The introduction of a spectral triple on each unital algebra $\acal_N$ that conforms the fuzzy sphere will introduce the notion of distance between states of the algebra, opening the door to the study of geometry on this spaces. Furthermore, we would like to introduce a spectral triple that somehow allows us to think of the fuzzy sphere as an approximation of $S^2$ also as metric spaces. 

In this document, this will occur in the following senses:

    \begin{itemize}
    
    \item The spectrum (and the multiplicities of the eigenvalues) of the Dirac operator approximates that of $\slashed D$ as $N$ increases
    
    \item To each point in $S^2$ there is a corresponding pure state on each $\acal_N$, and the distance between the coherent states corresponding to two points in $S^2$ tends to the commutative distance between the points as $N \to \infty$.
    
    \end{itemize}

%%%%%%%%%%%%%%%%%%%%%%%%%%%%%%%%%%%%%%%%%%%%%%%%%%%%%%%%%%%%%%%%%%%%%%%%%%%%%%
\subsection{$SU(2)$-equivariance}

% {\color{gray}
% Presentation:

% $SU(2) \ni g$ acts on the states: $\lbtext{g_*\omega}(\cdot) := \omega(\cdot^g)$ $\longleftarrow$ it acts on the algebra: $\lbtext{a^g}:= g\circ a \circ g^* \cdot$ $\longleftarrow$ it acts on $V_j$: $\pi_j(g)$.

% \textbf{\rtext{Theorem}}: For all $N = 2j \in \bb N$, the distance is $SU(2)$-invariant: 
% \begin{align}
%     d_N(\omega, \omega') &= d_N(g_* \omega, g_*\omega'), & \text{for all $\omega, \omega' \in \mathcal S(\acal_N)$}.
% \end{align}

% \textit{Pf}: The theorem follows once we show 
%     \rtext{$||[D_N, a^g]|| = ||[D_N, a]||$}:
% $d_N(g_* \omega, g_* \omega') = sup_{a \in \acal_N}\{ |\omega(a^g) - \omega'(a^g)|: ||[D_N, a \otimes 1_2]|| \leq 1 \} = sup_{b \in \acal_N}\{ |\omega(b) - \omega'(b)|: ||[D_N, b \otimes 1_2]|| \leq 1 \} = d_N(\omega, \omega')$, where $b = a^g$ sweeps all $\acal_N$.

% \textit{Pf}: $[D_N, a^g \otimes 1] = u[D_N, a \otimes 1_2]u^*$ where $u = \pi_j(g) \otimes \pi_{1/2}(g)$ is the induced unitary action of $g$ in $H_N$ $\xleftarrow{}$
% 1. $a^g \otimes 1_2 = u(a \otimes 1_2)u^*$;
% 2. The spectral triple is $SU(2)$-equivariant, hence $D_N$ commutes with the $SU(2)$ action.
% }

Rotating the 2-sphere leaves invariant not only its topological and differential properties, but it also leaves invariant the metric, $g = d\theta^2 + \sin^2(\theta) d\phi^2$, implying that the distance between any $2$ points is the same as the distance between the rotated points, under any rotation $h \in SO(3)$. Since $SU(2)$ is the double covering of $SO(3)$, there is a $2$-to-$1$ Lie group morphism $SU(2) \to SO(3)$ under which any rotation $h \in SO(3)$ corresponds to two distinct points $\pm \tilde h \in SU(2)$. From this it follows that each element of $SU(2)$ acts as a rotation on $S^2$, so that \rtext{$SU(2)$ is also a group of symmetries of $S^2$ that acts by diffeomorphisms and isometries}. Notice, however, that the most general symmetry group of $S^2$ is the orthogonal group $O(3)$, which includes reflections under planes on $\RR^3$ that pass through the origin.

One important property of the spectral triples that will be defined on the fuzzy sphere is their equivariance under $SU(2)$, or more generally the Hopf algebra $U(\sut)$ as defined in\cite{Sitarz}, giving an analogous notion to the action of these symmetry spaces by isometries. The precise definition of a \textit{spectral triple $(\acal, \hcal, \dcal)$ equivariant under a symmetry space $H$}, $H$ being called \textit{the isometry of the spectral triple}, is fairly involved, but, ignoring some details, reduces to the following:
    \begin{itemize}
    
    \item $\acal$ is a representation of $H$.
    
    \item $\hcal $ is a representation of $H$.
    
    \item The action of $H$ on $\acal$ coincides with the action $\acal$ as subalgebra of operators on $H$, i.e. the adjoint action $A \mapsto g \circ A \circ g^{-1}$. %there is a dense subspace $V$ of $\hcal$ on which the action induced on operators
    
    \item The \textit{Dirac operator is equivariant}, i.e. the operator $[\dcal, h] = 0$ for all $h \in H$.
    
    \end{itemize}
\ref{theoInvDistance}
\textbf{For the spectral triples defined in this section, the $SU(2)$-equivariance reduces to the verification of the last two conditions}, since the first two ones are readily satisfied. The following theorem, which will apply to all the spectral triples defined in this chapter, illustrates how this notion of equivariance induces a notion of isometry on the respective noncommutative spaces.

\begin{theorem} \label{theoGInvariantDistance}
Let $(\acal, \hcal, \dcal)$ be a spectral triple. Let $G$ be a group, let $\hcal$ and $\acal$ be representation spaces of $G$, with $G$ acting unitarily on $H$. Then, if for all $g \in G$, $[D, g\cdot] = 0$ and the action on $A$ coincides / interwines with the induced action on $A$ as subalgebra of operators on $H$, then $SU(2)$ acts by isometries on the state space $\mathcal S(\acal)$. That is, for all $g \in G$:
\begin{align}
    d_\dcal(\omega, \omega') &= d_\dcal(g_* \omega, g_*\omega'), & \text{for all $\omega, \omega' \in \mathcal S(\acal)$};
\end{align}
we have denoted by $g^*\omega$ the action of $G$ on $\mathcal S(\acal)$, $g^* \omega: \acal \to \CC$, $a \mapsto \omega(a^g)$, induced from the action $a \mapsto a^g$ of $G$ on $\acal$.
\end{theorem}

\begin{proof}
The theorem follows once we show that 
\begin{equation}\label{normEqualDiracAction}
    \rtext{||[\dcal, a^g]|| = ||[\dcal, a]||},
\end{equation} since, in that case,
\begin{align*}
d_\dcal(g_* \omega, g_* \omega') &= sup_{a \in \acal_N}\{ |\omega(a^g) - \omega'(a^g)|: ||[\dcal, a \otimes 1_2]|| \leq 1 \} \\
    &= sup_{b \in \acal_N}\{ |\omega(b) - \omega'(b)|: ||[\dcal, b \otimes 1_2]|| \leq 1 \} \\
    &= d_\dcal(\omega, \omega'),
\end{align*}
since $b = a^g$, for fixed $g \in SU(2)$, sweeps all $\acal_N$ if $a$ does.

Now, denoting the action of $G \ni g$ on $\hcal$ by $g \cdot$:
\begin{align*}
    (g \cdot) \circ [\dcal, a]\circ (g\cdot )^* &= (g \cdot) \circ [\dcal, a]\circ (g^{-1}\cdot )\\
        &= (g \cdot) \circ \dcal \circ a \circ (g^{-1}\cdot ) - (g \cdot) \circ a \circ  \dcal \circ (g^{-1}\cdot )\\
        &= \dcal (g \cdot) \circ a\circ (g^{-1}\cdot ) - (g \cdot) \circ a\circ (g^{-1}\cdot ) \dcal & \text{$\dcal$ commutes with $g\cdot$} \\
        &= \dcal \circ a^g - a^g \circ \dcal & \text{actions of $g$ and $\acal$ interwine} \\
        & = [\dcal, a^g].
\end{align*} The desired equality \eqref{normEqualDiracAction} follows from this calculation, and from the unitarity of the operators $(g \cdot)$ and $(g\cdot)^*$.
\end{proof}

%%%%%%%%%%%%%%%%%%%%%%%%%%%%%%%%%%%%%%%%%%%%%%%%%%%%%%%%%%%%%%%%%%%%%%%%%%%%%%
%%%%%%%%%%%%%%%%%%%%%%%%%%%%%%%%%%%%%%%%%%%%%%%%%%%%%%%%%%%%%%%%%%%%%%%%%%%%%%
\subsection{The Irreducible Spectral Triple}
 
% {\color{gray} - One term of the canonical spectral triple
 
% Presentation:

% A first spectral triple is simply taking ``one term'' of $\slashed D$:
% \begin{multline}
%     \lbtext{D_N} 
%     := (\pi_j \otimes \pi_{1/2})(\mathcal D): V_j \otimes \CC^2 \to V_j \otimes \CC^2 \\
%     = \begin{pmatrix} 1 + \pi_j(H) & \pi_j(F) \\ \pi_j(E) & 1 - \pi_j(H)\end{pmatrix} 
%     = 1 + \pi_j(F) \otimes \sigma_+ + \pi_j(E) \otimes \sigma_- + \pi_j(H) \otimes \sigma_3
% \end{multline} where $H = J_3$, $F = J_+$, $E = J_-$ are the actions of $J_3$, $J^\pm \in su(2)$ on $V_j$ respectively.

% - The spectral triple \rtext{$(\acal_N, \lbtext{H_N} = V_j \otimes \CC^2, D_N)$}
%     \begin{itemize}
        
%     \item Is $SU(2)$-equivariant
    
%     \item Has eigenvalues $j+1$ and $-j$ with multiplicities $2j+2$ and $2j$.
        
%     \item Isn't compatible with a grading or a real structure.
%     \end{itemize}
% }    
\linea

This spectral triple, although unsatisfactory as an approximation of the canonical spectral triple of $S^2$ with respect to the first criteria stated at the beginning of the section, will turn out to be very useful for computations.

\begin{definition}
For each $N \in \NN$ and $j = N/2$, define the \lbtext{irreducible spectral triple} as $(\acal_N, \lbtext{H_N} := V_j \otimes \CC^2, D_N = (\pi_j \otimes \pi_{1/2})(\mathcal D))$, where $\acal_N = End(V_j)$ acts naturally on the first factor of $H_N$.
\end{definition}

This spectral triple may be seen to come as a single term of the $SU(2)$-induced expansion of $\slashed D$ \eqref{}, although we are also allowing  half integer spin representations in the current construction, which don't appear in the expansion of $\slashed \dcal$. The Dirac operators can also be written as
\begin{equation}
    D_N = \begin{pmatrix} 1 + \pi_j(H) & \pi_j(F) \\ \pi_j(E) & 1 - \pi_j(H)\end{pmatrix} 
    = 1 + \pi_j(F) \otimes \sigma_+ + \pi_j(E) \otimes \sigma_- + \pi_j(H) \otimes \sigma_3
\end{equation}
where $H = J_3$, $F = J_+$, $E = J_-$ are the respective actions of $J_3$, $J_+, J_- \in U(\sut)$ on $V_j$.

We have said before that this spectral triple is $SU(2)$-equivariant, but for this to make sense in the first place we need to define an action of $SU(2)$ on $\acal_N$ and $H_N$, both of which will be inherited from the untitary representation $\pi_j$ of $SU(2)$ on $V_j$. Since $\acal_N = End(V_j)$, the action of each $SU(2) \ni g$ induces the adjoint action on its operators, $a \mapsto \lbtext{a^g} := \pi_j(g) a \pi_j(g)^*$. Similarly, on $H_N = V_j \otimes V_{1/2}$ we may define the unitary left action of $g \in SU(2)$ by $u_g := \pi_j(g) \otimes \pi_{1/2}(g) \in \mathcal B(H_n)$; notice that, when $j \in \NN$, this representation is one of the terms of the decomposition \eqref{} of the representation of $\hcal = L^2(S^2) \otimes \CC^2$ into representations of $SU(2)$.

\begin{proposition}\label{propIrredSpectralTriple}
For each $N \in \NN$, the associated irreducible spectral triple:
    \begin{enumerate}[(i)]
        
    \item Is $SU(2)$-equivariant
    
    \item Has eigenvalues $j+1$ and $-j$ with multiplicities $2j+2$ and $2j$, where $j = N/2$.
        
    \item Isn't compatible with a grading or a real structure.
    \end{enumerate}
\end{proposition}

\begin{proof}
As stated at the beginning of the section, to proof the $SU(2)$-equivariance follows from $[D_N, u_g] = 0$ and $u_g  a  u_g^* \equiv u_g \circ (a \otimes 1) \circ u_g^* = a^g \otimes 1 \equiv a_g$.

That those are the eigenvalues and their multiplicities follows from finding eigenvectors 
\begin{align}\label{irreducibleDiracFuzzySpinorBasis}
    |j, m \kket_+ := 
    \begin{pmatrix} 
    \sqrt{\frac{j+m+1}{2j+1}} |j, m\rangle \\ 
    \sqrt{\frac{j-m}{2j+1}} |j, m+1 \rangle
    \end{pmatrix} \\
    |j, m\kket_- := 
    \begin{pmatrix} 
    -\sqrt{\frac{j-m}{2j+1}} |j, m\rangle \\ 
    \sqrt{\frac{j+m+1}{2j+1}} |j, m+1 \rangle
    \end{pmatrix}
\end{align}
for $|m| \leq j$ with the mentioned eigenvectors, and deducing that there are no more eigenvectors thanks to the lemma \ref{lemmaSpectrumAlgebraic}.

If the spectral triple had a grading, the Dirac operator would have a symmetric spectrum about $0$, hence this can't be the case.

That a real structure $J_N: H_N \to H_N$ doesn't exist doesn't exist follows from the fact that the commutant of $\acal_N$ is $2$, and so it can't contained $J \acal_N J^{-1}$.

\end{proof}


%%%%%%%%%%%%%%%%%%%%%%%%%%%%%%%%%%%%%%%%%%%%%%%%%%%%%%%%%%%%%%%%%%%%%%%%%%%%%%
\subsection{The Full Spectral Triple}

% {\color{gray}- Truncation of the canonical spectral triple. In particular the Dirac operator ``goes'' to the canonical one.

% Presentation:

% \rtext{$(\acal_N, \lbtext{\hcal_N} := \mathcal A_N \otimes \CC^2 \cong \bigoplus_{l =1}^N H_l, \mathcal D_N \longleftrightarrow \bigoplus_{l =1}^N D_l)$}, where:
% \begin{multline}
%     \lbtext{\mathcal D_N} := (\textbf{ad}\pi_j \otimes \pi_{1/2})(\mathcal D) = D = \begin{pmatrix} 1 + \text{ad}\pi_j(H) & \text{ad}\pi_j(F) \\ \text{ad}\pi_j(E) & 1 - \text{ad}\pi_j(H)\end{pmatrix} 
%     = \text{ad}\pi_j(E) \otimes \sigma_1 + \text{ad}\pi_j(F) \otimes \sigma_2 + \text{ad}\pi_j(H) \otimes \sigma_3
% \end{multline}
% where, e.g. $\text{ad}\pi_j(H) = [\pi_J(H), \cdot ]$ is the action of $J_3 \in su(2)$ on $\mathcal A_N$.

%     \begin{itemize}
    
%     \item It is a real spectral triple.
    
%     \item It is $SU(2)$-equivariant
    
%     \item Spectrum of $\mathcal D_N$ is the truncation of $\slashed D$ to $\{-N, \dots, N+1\}$. The eigenvalues of $\mathcal D_N$ are $N+1$ with multiplicity $2N+2$, and $\pm l$ with multiplicity $2l$ for $l = 1, \dots, N$
    
%     \item It is not compatible with a grading.
    
%     \end{itemize}

% \rtext{It is a truncation of the canonical spectral triple}: The spectrum and the multiplicites 

% - \rtext{\textbf{Theorem}}: the two spectral triples induce the same distance in $\acal_N$. Pf: $[\dcal, a]b \otimes v = \cdots = \sum_k [\pi_j(J_k), a]b \otimes \sigma_k v = [D_N, a] \cdot b \otimes v$
% }

\linea

\begin{definition}
For each $N \in \NN$, $j = N/2$, the \lbtext{full spectral triple} is $(\acal_N, \lbtext{\hcal_N} := \mathcal A_N \otimes \CC^2, \lbtext{\mathcal D_N} := (\text{ad}\pi_j \otimes \pi_{1/2})(\mathcal D) )$, where $a \in \acal_N$ acts on the first factor of $\hcal_N$ via the commutator $[a, \cdot]$.
\end{definition}

Notice that the action of $\acal_N$ on itself as the first factor of $\hcal_N$ is simply the derivative of the adjoint action $B \mapsto \pi_j(g) B \pi_j(g)^{-1}$ of $SU(2) \ni g$ on $\acal_N = End(V_j)$. This action extends to an action $\text{ad}\pi_j \otimes \pi_{1/2}$ on $\hcal_N = \acal_N \otimes V_{1/2}$. Under the $SU(2)$-induced decomposition of $\acal_N$ into irreducible representations, the Hilbert space $\hcal_N$ is isomorphic to $\bigoplus_{l =1}^N H_l$, where the representation $\text{ad}\pi_j$ on one side then corresponds to the representation $\bigoplus_{l = 0}^N \pi_l$ on the other side, in which case the Dirac operator $\dcal_N$ corresponds to:
\begin{equation}\label{reducibleDiracDecomposition}
    \mathcal D_N : \hcal_N \to \hcal_N  \iff \bigoplus_{l =1}^N D_l: \bigoplus_{l =1}^N H_l \to \bigoplus_{l =1}^N H_l.
\end{equation}

\begin{theorem}\label{theoremReducibleSpectralTripleDiracApproximation}
Under the adjoint action of $SU(2)$ on the algebra $\acal_N$ and the action of $SU(2)$ defined in the previous paragraph, the full spectral triples satisfies:
    \begin{enumerate}[(i)]
    
    \item It is $SU(2)$-equivariant.
    
    \item It is a real spectral triple, with real structure $\mathcal J_N: \hcal_N \to \hcal_N$, $a \otimes v \mapsto a^* \otimes \sigma_2 \cut v$.
    
    \item The spectrum of $\mathcal D_N$ is the truncation of $\slashed D$ to $\{-N, \dots, N+1\}$, ie. the eigenvalues of $\mathcal D_N$ are $N+1$ with multiplicity $2N+2$, and $\pm l$ with multiplicity $2l$ for $l = 1, \dots, N$.
    
    \item It is not compatible with a grading.
    
    \end{enumerate}
\end{theorem}

\begin{proof}
It is straightforward to verify that: $\mathcal J_N$ is antilinear and antiunitary, $\mathcal J_N^2 = - 1$, $\mathcal J_N (\acal_N \mathcal\otimes 1) J_N^{-1} \subset (\acal_N \mathcal\otimes 1)' \subset \bcal(\hcal_N)$ (the $\acal '$ denotes the algebra that commutes with the algebra $\acal \subset \bcal(\hcal_N)$), and $\mathcal J_N \dcal_N = \dcal_N \mathcal J_N$; hence $\mathcal J_N$ is a real structure.

From the $SU(2)$-decomposition \eqref{reducibleDiracDecomposition} of the Dirac operator, results (ii) and (iii) follow from the application of proposition \ref{reducibleDiracDecomposition} to each of the terms.

Once again, a grading can't exist since the spectrum of $\dcal_N$ isn't symmetric.
\end{proof}

\rtext{The third statement of the theorems says that we have found a spectral triple with the desired property of being a truncation of the canonical spectral triple of $S^2$}. { \color{gray}
Furthermore, just like the eigenvectors of $\slashed D$ are vectors conformed by pairs of spherical harmonics, the eigenvectors of each $\dcal_N$ are pairs of fuzzy spherical harmonics $\hat Y^l_m$ (not sure if they can be the ones I defined before) for which the traditional spherical harmonics $Y^l_m$ may be considered ``symbols''.
} Hence, this will be the spectral triple which we will assign the forming algebras of the fuzzy sphere, under which the metric properties are to be studied, since it \rtext{may be considered a sequence of spectral triple that approximate the metric of $S^2$ increasingly better with growing $N$}. However, the following theorem allows us to study instead the much simpler irreducible spectral triple.

\begin{theorem}
The irreducible and full spectral triple on $\acal_N$, for each $N \in \NN$, induce the same distance in the state space of $\acal_N$.
\end{theorem}
\begin{proof}
Let $\omega, \omega'$ be states on $\acal_N$. Notice that distinct spectral triples induce different notions of distance only due to the condition $||[D, a]|| \leq 1$, which determines over which elements $a$ of the algebra the supremum of $|\omega(a) - \omega'(a)|$ is evaluated. However, for all $a \in \acal$, $||[\dcal_N, a]|| = ||[D_N, a]||$ as shown by the following calculation:
\begin{align*}
    [\dcal, a]b \otimes v &= \dcal_N(ab \otimes v) - a\left(b \otimes v + \sum_{k = 1}^3 [\pi_j(J_k), b] \otimes \sigma_k v\right)\\
        &= ab \otimes v + \sum_{k = 1}^3[\pi_j(J_k), ab] \otimes \sigma_k v - ab \otimes v - \sum_{k = 1}^3 a[\pi_j(J_k), b]\otimes \sigma_k v\\
        &= \sum_k [\pi_j(J_k), a]b \otimes \sigma_k v\\
        &= [1_{N+1} \otimes 1_2 + \sum_{k = 1}^3 \pi_j(J_k)\otimes \sigma_k, a](b, v)\\
        &= [D_N, a] \cdot b \otimes v.
\end{align*}
This tells us that $[\slashed D_N, a]: \hcal_N \to \hcal_N$ is the operator of left multiplication by the matrix $[D_N, a] \in \acal_N \otimes M_2(\CC)$, implying that its operator norm coincides with the norm of the matrix $[D_N, a]$.
\end{proof}

%%%%%%%%%%%%%%%%%%%%%%%%%%%%%%%%%%%%%%%%%%%%%%%%%%%%%%%%%%%%%%%%%%%%%%%%%%%%%%
%%%%%%%%%%%%%%%%%%%%%%%%%%%%%%%%%%%%%%%%%%%%%%%%%%%%%%%%%%%%%%%%%%%%%%%%%%%%%%
\section{Coherent States}\todo{Complement with intro given in Fiore and Pisacane}
 {\color{gray}
% -- Fuzzy approximations of points in S2. In QM they are the closes thing to classical states we have (minimum uncertainty states for some observables).

- \dbtext{Are they pure}? They are vector states, I think that suffices to make them pure.
}

We now define the Bloch or $SU(2)$-coherent states in each $\acal_N$. Given $N$, each coherent state is labeled by a point in $S^2 = SU(2) / S^1$, allowing us to interpret each coherent state as a fuzzy approximation of a point in $S^2$. Furtheremore, the fuzzyness of this approximation decreases as $N \to \infty$ since theorem \ref{} shows that the distance between coherent states corresponding to two points in $S^2$ tends to the respective distance in $S^2$. 

In quantum mechanics, coherent states were first introduced for the quantum version of the simple harmonic oscillator, as minimum uncertainty states for position and momentum, making them states whose time evolution is the closest to that of the corresponding classical evolution. Since then, many generalizations have emerged, using mainly one of the following properties of the system of coherent states $\{|\alpha\rangle\}_{\CC}$ of the simple harmonic oscillator:
    \begin{enumerate}
        
    \item Each $|\alpha\rangle$ saturates the Heisenberg uncertainty relation
    
    \item Each $|\alpha\rangle$ are eigenstates of the annihilation operator, with eigenvalue $\alpha \in \CC$
    
    \item $\{|\alpha\rangle\}$ is generated by the action of the Heisenberg-Weyl group acting on the vacuum $|0\rangle$
        
    \end{enumerate}
Particularly Perolomov \cite{Perolomov} defines systems of coherent states using $(3)$ for various Lie groups $G$ acting on a Hilbert space with a chosen vacuum vector $|0\rangle$, in which case the coherent states are vectors of the Hilbert space, they are labeled by topological space $G/H$; if $|0\rangle$ satisfies certain conditions, then the properties $(1)$ and $(2)$ are also satisfied.

%%%%%%%%%%%%%%%%%%%%%%%%%%%%%%%%%%%%%%%%%%%%%%%%%%%%%%%%%%%%%%%%%%%%%%%%%%%%%%
\subsection{SHO Coherent States}

{\color{gray} 3 characterizations. Heisenberg group.}
%%%%%%%%%%%%%%%%%%%%%%%%%%%%%%%%%%%%%%%%%%%%%%%%%%%%%%%%%%%%%%%%%%%%%%%%%%%%%%
\subsection{$SU(2)$-Coherent States}

{\color{gray} 
 - Rotations of $|j, -j \rangle$
 
Presentation:

- The Bloch/$SU(2)$-\rtext{coherent states} are for the group $SU(2)$ what the usual harmonic oscillator coherent states are for the Heisenberg group. In particular, they are minimum uncertainty states. 

- \rtext{They will be considered fuzzy approximations of the points of $S^2$}.

- \rtext{Points of the $E(\phi, \theta) \in S^2$ sphere are approximated by coherent vectors} $\lbtext{|\phi, \theta)_N} := R_{(\phi, \theta)}|j, -j\rangle \in V_j \Longleftrightarrow $ states of $\mathcal A_N = End(V_j)$ $\lbtext{\psi^N_{(\phi, \theta)}} = (\phi, \theta| \cdot |\phi, \theta)_N \in \mathcal S(\mathcal A_N)$. 

- \rtext{This identification of points in $S^2$ with coherent states ($\psi^N: \acal_N \to L^2(S^2)$ at the algebra level\todo{})
is $SU(2)$-equivariant}: $g_* \psi^N_{(\phi, \theta)} = \psi^N_{g\cdot (\phi, \theta)}$, \& \rtext{the distance between them is $SU(2)$-invariant}.

What group used in Fiore? $O(D)$

- In Fiore2020 there is a simple introduction to Coherent states: group $G$ acting (irreducibly) on a Hilbert space. AND the example for $G = SU(2)$ is sketched, although there is a mistake for one of the $3$ characterizations, since it isn't true that the coherent states are eigenvectors of the ``annihilation operator'' $J_+$. D'Andrea mentions other sources.

\dbtext{WHY DO WE WANT TO STUDY DISTANCE BETWEEN COHERENT STATES???} Make this very explicit.

}

\linea

%%%%%%%%%%%%%%%%%%%%%%%%%%%%%%%%%%%%%%%%%%%%%%%%%%%%%%%%%%%%%%%%%%%%%%%%%%%%%%
%%%%%%%%%%%%%%%%%%%%%%%%%%%%%%%%%%%%%%%%%%%%%%%%%%%%%%%%%%%%%%%%%%%%%%%%%%%%%%
\section{Distance Between Famillies of \dbtext{Pure} States}

% {\color{gray}
% Presentation:

% \textit{From now on}: we use the irreducible s.t. and not write the $\pi_j$'s.

% - The supremum $d_}(\omega, \omega')$ is always attained in hermitian elements.

% - For $a \in \acal_N$, 
% \begin{equation} \label{ineqDN}
%     ||[H, a]||, ||[E, a]||, ||[F, a]||  \leq ||[D_N, a]||;
% \end{equation} \label{eqDNdiag}
% if $a$ is diagonal hermitian, then
% \begin{equation}
%     ||[E, a]|| = ||[D_N, a]||.
% \end{equation}

% - A state may be defined only on hermitian elements of $\acal$, since from there it can be uniquely extended to all $\acal$: $a = \frac{a+a^*}{2} + \frac{a - a^*}{2} \in Herm. + Antiherm. = Herm. + i\, Herm.$

% \linea

% - \textbf{Procedure}: Understand $||[D, a]||$; find upper limit for $|\omega(a) - \omega(a')|$ dependent on $||[D, a]||$; find hermitian algebra element that saturates the inequality (or sequence that get close) (if inequalities depending on $||[D, a]||$ are found, this element will need to have a maximizing $||[D, a]||$, probably $1$). 

% Very similar procedure to calculate distances followed in Chakraborty Moyal Plane and Chakraborty Fuzzy Sphere.

% - \textbf{Discrete Basis States}: In $\acal_N = End(V)$, study vector states of important basis, relating $\omega_m(a) - \omega_n(a)$ to $[D_N, a]$.

% - \textbf{$G$-invariance of distance}: if $G$ acts by ``isometries'', simplify what needs to be proven.

% - \textbf{Auxiliary distance}: study simpler subalgebra distance: lower bound whose behavior is understood and that encases the actual distance, allowing approximate study.

% - \textbf{Behaviour with $N$}: Relate algebras and coherent states of subsequent $N$, finding relation between $||[D_{N+1}, \cdot]||$ and $||[D_N, \cdot]||$.
% }

\linea

Due to theorem \ref{}, although the full spectral triple seems to be a better approximation of the canonical spectral triple due to the properties of the spectrum of $\dcal_N$, for $N$ any natural number, it suffices to study the distances using the irreducible spectral triple since it induces exactly the same notion of distance on each $\acal_N$. From now on we will stop writing the symbols $\pi_j$, $j = N/2$, denoting the unitary representation of $U(\sut) \ni H = J_3, E = J_-, F = J_+$ on $V_j$ unless confusion may arise; hence we will write the Dirac operator as
\begin{equation}
    D_N = \begin{pmatrix} 1 + H & F \\ E & 1 - H\end{pmatrix}.
\end{equation}
Notice that, under this matrix notation that follows from the tensor product with the Pauli matrices, an algebra element $a \in \acal$ corresponds to the matrix
\begin{equation}\label{aAsMatrixOperator}
    a = \begin{pmatrix} a & 0 \\ 0 & a \end{pmatrix}
\end{equation} as an operator on $H_N$.

\begin{remark}
If $\acal$ is a unital $C^*$-algebra, and $D$ is an associated Dirac operator, the supremum  $d_\dcal(\omega, \omega')$, of $|\omega(a) - \omega'(a)|$ for $a \in \acal$ such that $||[D, a]|| \leq 1$, is always attained in hermitian $(a = a^*)$ elements \ref{Rieffel30,33}, hence Connes' distant formula for a spectral triple $(\acal, \hcal, \dcal)$ can be changed to
\begin{equation} \label{distanceFormulaHermitian}
    d_\dcal(\omega, \omega') = sup_{a \in \acal, \,a = a^*} \{|\omega(a) - \omega'(a)| : ||[\dcal, a\|| \leq 1\}.
\end{equation}
\end{remark}

\begin{remark}
A state may be defined from its action on hermitian elements of the $C^*$-algebra $\acal$, since from there it can be uniquely extended to all of $\acal$, since any $a \in \acal$ can be written as $a = \frac{a+a^*}{2} + i \frac{a - a^*}{2i}$, where $\frac{a+a^*}{2}$ and $\frac{a-a^*}{2i}$ are hermitian.
\end{remark}

The following lemma will also be a very helpful tool for our following  studies, and its proof exemplifies how the norm of the commutator $||[D, a]||$ appearing in the definition of Connes' distance, may be studied.

\begin{lemma}
For any $a \in \acal_N$, the following inequalities about operators on $H_N$ are satisfied:
\begin{equation} \label{ineqDN}
    ||[H, a]||, ||[E, a]||, ||[F, a]||  \leq ||[D_N, a]||;
\end{equation} \label{eqDNdiag}
if $a$ is diagonal and hermitian, then
\begin{equation}
    ||[D_N, a]|| = ||[E, a]|| = ||[F, a]||.
\end{equation}
\end{lemma}

\begin{proof}
Let $a \in \acal_N$ be arbitrary. Then
\begin{align*}
    ||[D_N, a]||^2 
        &= sup_{v \in H_N, |v| = 1} |[D_N, a] | v \rangle|^2 \\
        &= sup_{v \in H_N, |v| = 1} \langle v | [D_N, a]^* [D_N, a] | v \rangle.
\end{align*}

To find a lower bound for the norm of $[D_N, a]$, first notice that 
\begin{align*}
    [D_N, a] &= \begin{pmatrix} [H, a] & [F, a] \\ [E, a] & -[H, a] \end{pmatrix}&
    [D_N, a]^* = \begin{pmatrix} [H, a]^* & [F, a]^* \\ [E, a]^* & -[H, a]^* \end{pmatrix},
\end{align*} hence
\begin{equation*}
    [D_N, a]^* [D_N, a] 
    = \begin{pmatrix}
    [H, a]^*[H, a] + [E, a]^*[E, a] & \cdots \\
    \cdots & [H, a]^*[H, a] + [F, a]^*[F, a]
    \end{pmatrix}.
\end{equation*}
If in the equation for $||[D_N, a]||$ instead of taking the supremum over all $v \in H_N$ of unit norm we take it over all unit vectors of the form $(x, 0)^t$, and $(0, y)^t$, we obtain, respectively, the following lower bounds:
\begin{align}
    ||[D_N, a]||^2 &\geq ||[H, a]||^2 + ||[E, a]||^2 \\
    ||[D_N, a]||^2 &\geq ||[H, a]||^2 + ||[F, a]||^2,
\end{align}
and so the first inequalities follow.

If $a \in \acal_N = M_{N+1}(\CC)$ is diagonal, then $[H, a] = 0$ since $H = \pi_j(J_3)$ is also diagonal, and so
\begin{equation*}
    [D_N, a]^* [D_N, a] 
    = \begin{pmatrix}
    [E, a]^*[E, a] & 0 \\
    0 & [F, a]^*[F, a]
    \end{pmatrix}
\end{equation*}
hence
\begin{equation}
    ||[D_N, a]|| = max\left\{ ||[E, a]||, ||[F, a]|| \right\}.
\end{equation}

Finally, $[D_N, a]^* = -[D_N, a^*]$ and $E = F^*$ imply that $[F, a] = - [E, a^*]^*$, so, if $a$ is hermitian and diagonal, $||[E, a]|| = ||[F, a]||$ and the last part of the statement follows.
\end{proof}

The general procedure to find the distance between two states will be the following: 

    \begin{enumerate}
    
    \item Understand $||[D, a]||$ for an arbitrary $a \in \acal$; 
    
    \item find a small upper limit for $|\omega(a) - \omega(a')|$ dependent on $||[D, a]||$;
    
    \item find a hermitian algebra element that saturates the inequality of the upper limit (or a sequence that has this bound as limit).
    
    \end{enumerate} 

%%%%%%%%%%%%%%%%%%%%%%%%%%%%%%%%%%%%%%%%%%%%%%%%%%%%%%%%%%%%%%%%%%%%%%%%%%%%%%
\subsection{Distance Between (Vector) Discrete States $|j,m\rangle$ for Arbitrary $N$}

% {\color{gray} 
% Presentation:

% \rtext{\textbf{Theorem}}: 
% %  and so, between the north and south poles
% % \begin{equation}
% %     d_N(\psi_{(0,0), \psi_{(0, \pi)}}) = \sum_{k = 1}^N \frac{1}{\sqrt{k(N-k+1)}}
% % \end{equation}
% \textit{Pf}: Recall that $J_\pm|j,m\rangle = \sqrt{(j\mp m)(j\pm m + 1)}|j, m+1 \rangle$ $\longrightarrow$ $\omega_m(a) - \omega_n(a) = \sum_{k = m+1}^n \langle j, k-1 |a|j, k-1 \rangle - \langle j, k |a| j, k \rangle = \sum_{k = m+1}^n \frac{1}{\sqrt{(j+k)(j-k+1)}} \langle j, k| [E, a] |j, k-1 \rangle$; since $|\langle j, k |[E, a]|j, k-1 \rangle|  \leq ||[E, a]|| \leq ||[D_N, a]|| \leq 1$, we get the upper bound. 

% Define the diagonal hermitian operator $\hat a |j, m\rangle := - \left( \sum_{k = -j+1}^m  \right)|j, m\rangle$, $\longrightarrow$ $[E, \hat a] |j, k \rangle = |j, k+1\rangle$ ($k < j$), and so $\hat a$ saturates the inequality.
% }

\linea

Another family of pure states on each $\acal_N = End(V_j)$, for $N = 2j \in \NN$  is the family of vector states $\lbtext{\omega_m} := \langle j, m | \cdot | j, m \rangle$, where $m = -j, \dots, j-1, j$. The study of this distance will be useful to the study of coherent states.

\begin{theorem}
For any $N$:
\begin{equation}
    d_N(\omega_m, \omega_n) = \sum_{k = m+1}^n \frac{1}{\sqrt{(j+k)(j-k+1)}} = \sum_{k = m+1}^n d_N(\omega_{k-1}, \omega_k).
\end{equation}
In particular, the distance between this family of states is additive.
\end{theorem}

\begin{proof}
Recall that $E = J_+ = J_-^* \in U(\sut)$ and that, for $m \pm 1 = -j\pm 1, \dots, j$, $J_\pm|j,m\rangle = \sqrt{(j\mp m)(j\pm m + 1)}|j, m+1 \rangle$. Therefore, 
\begin{align*}
    \langle j, m | [E, a] |j, m - 1 \rangle &= \sqrt{(j+m)(j-m+1)} \langle j, m-1| a | j, m-1\rangle \\
    & -\sqrt{(j-m+1)(j+m)} \langle j, m| a | j, m\rangle
\end{align*}
Hence, 
\begin{align*}
    \omega_m(a) - \omega_n(a) &= \sum_{k = m+1}^n \langle j, k-1 |a|j, k-1 \rangle - \langle j, k |a| j, k \rangle \\
    &= \sum_{k = m+1}^n \frac{1}{\sqrt{(j+k)(j-k+1)}} \langle j, k| [E, a] |j, k-1 \rangle
\end{align*}
Applying the triangle inequality on the norm of the above equation, and using proposition \ref{}$, |\langle j, k |[E, a]|j, k-1 \rangle|  \leq ||[E, a]|| \leq ||[D_N, a]|| \leq 1$, so we get the distance formula in the statement as an upper bound. 

Define the diagonal hermitian operator 
\begin{equation}\label{saturatingDiagonalADiscreteBasis}
    \hat a |j, m\rangle := - \left( \sum_{k = -j+1}^m \frac{1}{\sqrt{(j+k)(j-k+1}}  \right)|j, m\rangle.    
\end{equation} 
For this $\hat a$, $[E, \hat a] |j, m \rangle = |j, m+1\rangle$ for all $m = -j, \dots, j-1$, and so $\hat a$ saturates the inequality.
\end{proof}


%%%%%%%%%%%%%%%%%%%%%%%%%%%%%%%%%%%%%%%%%%%%%%%%%%%%%%%%%%%%%%%%%%%%%%%%%%%%%%
\subsection{The distance between $N=1$ coherent states}

% {\color{gray} 
%  - 
 
%  Presentation:
 
% $\pi_{1/2}(J_k) = \sigma_k/2$

% - Any hermitian element in $\acal_1 = M_2(\CC)$ can be written as
% \begin{align*}
%     a &= \begin{pmatrix} a_0 + a_3 & a_1 - i a_2 \\ a_1 + i a_2 & a_0 - a_3  \end{pmatrix}= a_0 + \vec a \cdot \vec \sigma, & \text{for $(a_0, \dots, a_3) \in \RR^4$.}
% \end{align*}


% - Positivity of states implies that, restricted to hermitian elements, they all are  $\omega_{\vec x}(a) = a_0 + \vec x \cdot \vec a$, for $\vec x \in B^3 \subset \RR^3$.

% - $\omega_{\vec x}$ is pure $\iff$ $\vec x = (\sin \theta \cos \phi, \sin \theta, \cos \theta) \in S^2$ $\dbtext{\iff}$ $\omega_x = \psi^1_{(\phi, \theta)}$.

% \rtext{\textbf{Theorem}}: For $N = 1$ all pure states are coherent states and 
% \begin{equation}
%     d_1(\hat p, \hat q) = \frac{1}{2}|\vec p - \vec q|_{\RR^3}
% \end{equation}
% \textit{Pf}: $|\omega_{\vec x}(b) - \omega_{\vec y}(b)| = |(\vec x - \vec y)\cdot \vec b| \leq |\vec x - \vec y||\vec b|$; $i[D_1, a]$ is hermitian with max. eigenvalue/norm $=2 |\vec a|$ $\longrightarrow$ $a \in \acal_1$ hermitian with $\vec a$ parallel to $\vec x - \vec y$ st. $2|\vec a| = 1$ saturates the inequality.
% }

\linea

Any hermitian element in $\acal_1 = M_2(\CC)$ can be written as
\begin{align*}
    a &= \begin{pmatrix} a_0 + a_3 & a_1 - i a_2 \\ a_1 + i a_2 & a_0 - a_3  \end{pmatrix}= a_0 + \vec a \cdot \vec \sigma, & \text{for $(a_0, \dots, a_3) \in \RR^4$.}
\end{align*}

Since very state $\omega: \acal_1 \to \CC$, in addition to its linearity over this $4$-dimensional vector space, must satisfy that $\omega(Id_2) = 1$ and $\omega(a^*a) \geq 0$, then, when evaluated on hermitian elements $a$ as above, every state must be of the form $\omega_{\vec x}(a) = a_0 + \vec x \cdot \vec a$, for $\vec x \in B^3 \subset \RR^3$.

This form for the states identifies the convex combination $\alpha \omega_{\vec x} + \beta \omega_{\vec y}$, i.e. $\alpha, \beta \in \RR_{\geq 0}$ and $\alpha + \beta = 1$, of states $\omega_{\vec x}$ and $\omega_{\vec y}$, with the convex combination points $\alpha\vec x + \beta\vec y$ in $B^3$. Hence, all pure states on $\acal_1$, i.e. those that can not be written as a convex combination of more than one state, are of the form $\omega_{E(\phi, \theta)}$ for $E(\phi, \theta):= (\sin \theta \cos \phi, \sin \theta, \cos \theta) \in S^2$. Furthermore, $\omega_{E(\phi, \theta)} = \psi^1_{(\phi, \theta)}$.

\begin{proposition}
The set $\{\omega_{E(\phi, \theta)} \,|\, E(\phi, \theta)\}$ of pure states on $\acal_1$ coincides with the set of $SU(2)$-coherent states, under the equality $\omega_{E(\phi, \theta)} = \psi^1_{(\phi, \theta)}$.
\end{proposition}
\begin{proof}
Shouldn't be too complicated once the correct framing is achieved...  AND IT CAN BE A GOOD WAY TO UNDERSTAND THE COHERENT STATES\todo{I would like to do this} $|j, j\rangle = \begin{pmatrix} 1&0 \end{pmatrix}$
\end{proof}

\begin{theorem}\label{N=1CaseDistance}
For $N = 1$ all pure states are coherent states and 
\begin{equation}
    d_1(\hat p, \hat q) = \frac{1}{2}|\vec p - \vec q|_{\RR^3},
\end{equation} where $|\cdot |_{\RR^3}$ denotes the euclidean norm in $\RR^3$.
\end{theorem}

\begin{proof}
As stated at the beginning of this section, we may restrict our search to hermitian elements. Let $a \in \acal_1 = M_2(\CC)$ be hermitian. Then,
\begin{align*}
    |\omega_{\vec x}(b) - \omega_{\vec y}(b)| &= |(\vec x - \vec y)\cdot \vec b| \\
        &\leq |\vec x - \vec y||\vec b|.
\end{align*}

Now, let $a_\pm = a_1 \pm i a_2$, so that $a = a_0 + a_+ \sigma_+ + a_- \sigma_- + a_3 \sigma_3$. Recall that $\pi_{1/2}(J_k) = \frac{\sigma_k}{2}$, $k = 1, 2, 3$, hence
\begin{equation*}
    [D_1, a] = 
    \begin{pmatrix}
    0 & a_+ & -a_+ & 0\\
    -a_- & 0 & 2a_3 & a_+ \\
    a_- & -2a_3 & 0 & -a_+ \\
    0 & -a_- & a_- & 0
    \end{pmatrix}
\end{equation*}
and so $i[D_1, a]$ is hermitian, with characteristic polynomial $\lambda^2(\lambda^2 -4|\vec a|^2)$, so its norm is its maximum eigenvalue. Then 
\begin{equation*}
    ||[D_1, a]|| = 2 |\vec a|.
\end{equation*}

Therefore, if we choose $a \in \acal_1$ hermitian with associated $\vec a$ parallel to $\vec x - \vec y$ such that $2|\vec a| = 1$, this element saturates the inequality.
\end{proof}



%%%%%%%%%%%%%%%%%%%%%%%%%%%%%%%%%%%%%%%%%%%%%%%%%%%%%%%%%%%%%%%%%%%%%%%%%%%%%%
\subsection{Relating Distinct $N$'s and Upper Bound}

% {\color{gray} 
%  - Relate algebras and coherent states of subsequent $N$, finding relation between $||[D_{N+1}, \cdot]||$ and $||[D_N, \cdot]||$.
 
% Presentation:

% \rtext{\textbf{Theorem}}: the distance $d_N(\psi^N_{(\phi, \theta)}, \psi^N_{(\phi', \theta')})$ is non-decreasing with $N$:
% \begin{equation}
%     d_N(\psi^N_{(\phi, \theta)}, \psi^N_{(\phi', \theta')}) \leq d_{N+1}(\psi^{N+1}_{(\phi, \theta)}, \psi^{N+1}_{(\phi', \theta')})
% \end{equation}

%  KEY: Relate $N$ with $N+1$
 
% \textit{Pf}: $U^\pm_j |\phi, \theta)_{N+1} = |\phi, \theta)_N \otimes |\phi, \theta)_1$, $\xrightarrow{}$ $\eta^+_N: \acal_N \to \acal_{N + 1}$ $\longrightarrow{}$ $\psi^{N+1}_{(\phi, \theta)} \circ \eta^+_N(a) = \psi^N_{(\phi, \theta)}$ \& $||[D_{N + 1}, \eta^+_N(a)]|| \leq ||[D_N, a]||$.

% \rtext{\textbf{Theorem}}: For all $N$, 
% \begin{equation}
%     \frac{1}{2}|E(\phi, \theta)- E(\phi', \theta')| \leq || \leq d_N(\psi^N_{(\phi, \theta)}, \psi^N_{(\phi', \theta')}) \leq d_{geo}(E(\phi, \theta), E(\phi', \theta')).
% \end{equation}

% \textit{Pf}: \textbf{use SU(2)-invariance of distance} and only prove it between within the great circle $\theta = \frac{\pi}{2}$: $d_N(\psi^N_{(0, \frac{\pi}{2})}, \psi^N_{(\phi, \frac{\pi}{2})}) \leq |\phi|$. 
% $|\psi^N_{(\phi, \frac{\pi}{2})} - \psi^N_{(0, \frac{\pi}{2})}| = |i \int_0^\phi \psi^N_{(\alpha, \frac{\pi}{2})}([H, a]) d\alpha | \leq ||[H, a]|| |\phi| \leq |\phi| ||[D_N, a]||$.

% (Dibujito: $N=1$ en un color, other increasing $N's$ (equivalent paths), finally commutative distance).
% }

\linea

In this subsection we will relate the coherent states and the algebras of subsequent $N$'s, which will then allow us to find a relation between $||[D_{N+1}, \cdot]||$ and $||[D_N, \cdot]||$ for any $N = 2j \in \NN$.

First define the linear injections, that respect the $SU(2)$ action,
\begin{align*}
    U^+_j : V_{j+\frac{1}{2}} &\to V_j \otimes V_{\frac{1}{2}} 
    & U^-_j : V_{j-\frac{1}{2}} &\to V_j \otimes V_{\frac{1}{2}} \\
    \left|j + \frac{1}{2}, m + \frac{1}{2}\right\rangle &\mapsto |j, m\kket_+ &
    \left|j - \frac{1}{2}, m + \frac{1}{2}\right\rangle &\mapsto |j, m\kket_-
\end{align*} where $|j, m\kket_\pm$ are the fuzzy spinors eigenvectors of $D_N$ defined in \ref{irreducibleDiracFuzzySpinorBasis}, and $m \pm 1 = -j \mp 1, \dots, j$ for $U_j^\pm$. 
%Notice that $H_N = V_j \otimes V_{\frac{1}{2}}$ is the orthogonal direct sum of the ranges of $U_j^+$ and $U_j^-$.
Then, using only the formula for the coherent states \ref{coherentStatesDefinitionRotation}, it can be easily shown that:

\begin{lemma}\label{relationCoherentVectorsDifferentN}
$U_+|\phi, \theta)_{N+1} = |\phi, \theta)_N \otimes |\phi, \theta)_1$ for any $E(\phi, \theta) \in S^2$.
\end{lemma}

Now, define the injective linear maps
\begin{align*}
    \eta^\pm_N &: \acal_N \to \acal_{N+1} \\
    a &\mapsto (U_j^\pm)^*(a \otimes 1_2) U_j^\pm.
\end{align*}

\begin{lemma} \label{relationDifferentNCoherentStates}
For any $a \in \acal_N$,
\begin{equation}
    \psi^{N+1}_{(\phi, \theta)} \circ \eta_N^+(a) = \psi^N_{(\phi, \theta)}.
\end{equation}
\end{lemma}
\begin{proof}
This is a direct consequence of the definitions of $U_J^\pm$ and $\eta_N^+$, since
\begin{align*}
    (\phi, \theta| \eta^+_N(a) |\phi, \theta)_{N+1} &= (\phi, \theta|a|\phi, \theta)_N (\phi, \theta| \phi, \theta), & \text{lemma \ref{relationCoherentVectorsDifferentN}}\\
        &= (\phi, \theta| a | \phi, \theta)_N.
\end{align*}
\end{proof}

Finally, using some more properties of the maps $U_j^\pm$ and $\eta^\pm_N$, together with the previous lemmas, the following can be proven:
\begin{lemma}\label{normCommutatorsDiracDifferentNRelation}
For any $a \in \acal_N$,
\begin{equation}
    ||[D_{N+1}, \eta^\pm(a)]|| \leq ||[D_N, a]||.
\end{equation}
\end{lemma}

From this relation between the norms of the commutators $[D_N, \cdot]$ for consecutive $N$'s, we can now show the following.

\begin{theorem}\label{nondecreasingDistanceRelatingNCoherent}
For any $N \geq 1$, the distance $d_N(\psi^N_{(\phi, \theta)}, \psi^N_{(\phi', \theta')})$ is non-decreasing with $N$, i.e.:
\begin{equation}
    d_N(\psi^N_{(\phi, \theta)}, \psi^N_{(\phi', \theta')}) \leq d_{N+1}(\psi^{N+1}_{(\phi, \theta)}, \psi^{N+1}_{(\phi', \theta')}).
\end{equation}
\end{theorem}
\begin{proof}
\begin{align*}
    d_{N+1}(\psi^{N+1}_{(\phi, \theta)}, \psi^{N+1}_{(\phi', \theta')}) 
        &= sup_{a \in \acal_{N+1}} \{|\psi^{N+1}_{(\phi, \theta)}(a) - \psi^{N+1}_{(\phi', \theta')}(a)| : ||[D_{N+1}, a]|| \leq 1\} \\
        &\geq sup_{a = \eta_N^+(b) \in \acal_{N+1}} \{|\psi^{N+1}_{(\phi, \theta)}(a) - \psi^{N+1}_{(\phi', \theta')}(a)| : ||[D_{N+1}, a]|| \leq 1\}\\
        &= sup_{a \in \acal_{N}} \{|\psi^{N+1}_{(\phi, \theta)}\circ \eta_N^+(a) - \psi^{N+1}_{(\phi', \theta')}\circ \eta_N^+(a)| : ||[D_{N+1},  \eta_N^+(a)]|| \leq 1\} \\
        &= sup_{a \in \acal_{N}} \{|\psi^{N}_{(\phi, \theta)}(a) - \psi^{N}_{(\phi', \theta')}(a)| : ||[D_{N+1}, \eta_N^+(a)]|| \leq 1\}\\
        &\geq sup_{a \in \acal_{N}} \{|\psi^{N}_{(\phi, \theta)}(a) - \psi^{N}_{(\phi', \theta')}(a)| : ||[D_{N}, a]|| \leq 1\} \\
        &= d_N(\psi^{N}_{(\phi, \theta)}, \psi^{N}_{(\phi', \theta')});
\end{align*} the fourth line is an application of lemma \ref{relationDifferentNCoherentStates}, and the fifth line of lemma \ref{normCommutatorsDiracDifferentNRelation}.
\end{proof}

\lin

An interesting upper bound is possible for the increasing sequence with $N$ that is the distance between any two coherent states associated to two points in the sphere.

\begin{theorem}\label{geometricDistanceCommutativeUpperBound}
For all $N$, and for all $E(\phi, \theta), E(\phi', \theta') \in S^2$
\begin{equation}
    \frac{1}{2}|E(\phi, \theta)- E(\phi', \theta')|_{\RR^3}  \leq d_N(\psi^N_{(\phi, \theta)}, \psi^N_{(\phi', \theta')}) \leq d_{S^2}(E(\phi, \theta), E(\phi', \theta')),
\end{equation}
where $d_{S^2}$ is the usual distance within $S^2$, inherited from the metric $g = d\theta^2 + \sin^2 \theta \, d\phi^2$.
\end{theorem}
\begin{proof}
The lower bound is a simple consequence of the $N = 1$ distance found in theorem \ref{N=1CaseDistance} and of the nondecrease of the distance with $N$ found in theorem \ref{nondecreasingDistanceRelatingNCoherent}.

To prove the upper bound, thanks to the $SU(2)$-invariance of the distance between states proven in theorem \ref{theoGInvariantDistance}, we only need to prove it within the great circle $\theta = \frac{\pi}{2}$, i.e. we only need to prove that 
\begin{equation*}
    d_N(\psi^N_{(0, \frac{\pi}{2})}, \psi^N_{(\phi, \frac{\pi}{2})}) \leq |\phi|.
\end{equation*}

It can be shown that $\psi^N_{(\phi, \theta)}([H, a]) = -i \frac{\partial}{\partial \phi} \psi^N_{(\phi, \theta)}(a)$; this is a statement of the equivariance under $SU(2)$ of the identification of $SU(2)$-coherent states and points in $S^2$. Then the following calculation
\begin{align*}
    |\psi^N_{(\phi, \frac{\pi}{2})} - \psi^N_{(0, \frac{\pi}{2})}| 
        &= \left|i \int_0^\phi \psi^N_{(\alpha, \frac{\pi}{2})}([H, a]) d\alpha \right|  \\
        &\leq \left| \int_0^\phi d\alpha \right| \left| \psi^N_{(\alpha, \theta)}([H, a]) \right|\\
        &\leq |\phi|\,||[H, a]|| 
        & |\omega(a)| \leq ||a|| \text{ for any state $\omega: \acal \ni a \to \CC$}\\
        &\leq |\phi|\, ||[D_N, a]||    & \text{Theorem \ref{ineqDN}}
\end{align*}
proves the desired upper bound.
\end{proof}
%%%%%%%%%%%%%%%%%%%%%%%%%%%%%%%%%%%%%%%%%%%%%%%%%%%%%%%%%%%%%%%%%%%%%%%%%%%%%%
\subsection{Auxiliary Distance and Limit}

% {\color{gray} 
%  - Study simpler subalgebra distance: lower bound whose behavior is understood and that encases the actual distance, allowing approximate study.

% Presentation:

% Distance along diagonal, hermitian matrices eassier to understand, but useful.

%  KEY: find simpler (we understand [D, a] if a is diagonal hermitian), but useful auxiliary distance (gives a lower bound that sandwhiches the actual distance)
 
% Let $\lbtext{\bcal_N} \subset \acal_N$ be sugalgebra of diagonal matrices $\xrightarrow{}$ $\psi^N_{(\phi, \theta)} = \psi^N_{(0, \theta)}$, now 
% $\lbtext{\rho_N}(\theta) := sup_{a \in \bcal_N, a = a^*}\{|\psi^N_{(0, \theta)} - \psi^N_{(0, 0)}| : ||[D_N, a]|| \leq 1\}$: \textbf{1.} exact formula related to $\omega_m$ states, \textbf{2.} $0 \leq \rho'_N(\theta) \leq 1$ $\xrightarrow{}$ $\rho_N(\theta) \leq \theta$, \textbf{3.} non-decreasing with $N$.


% \textbf{Lemma}: For all $N$, 
% \begin{align*}
%     1.& \rho_N(\theta - \theta') \leq d_N(\psi^N_{(\phi, \theta)}, \psi^N_{(\phi', \theta')}) d_{geo}(E(\phi, \theta), E(\phi', \theta'))  ; \\
%     2.& \text{the sequence $\rho_N(\theta)$ converges uniformly to $\theta$ as $N\to \infty$} \leq .
% \end{align*}

% \textbf{\rtext{Theorem}}:  
% \begin{equation}
%         \lim_{N \to \infty} d_N(\psi^N_{(\phi, \theta)}, \psi^N_{(\phi', \theta')}) = d_{geo}(E(\phi, \theta), E(\phi', \theta'))
% \end{equation}
% }

\linea

A final result, stating that the distance between any two coherent states associated to two points in the sphere is not only bounded by the euclidean distance between the points, but that in fact this is the limit distance when $N \to \infty$ can be shown. To do this, we use an auxiliary distance by limiting the algebra on which we maximize the norm of the difference of the evaluation of the states that appears in the definition of Connes' distance. By reducing the algebra the supremum over it will necessarily be a lower bound, but an exact formula for the distance will be possible.

Let $\lbtext{\bcal_N} \subset \acal_N$ be subalgebra of diagonal matrices. Notice that for any $a \in \bcal_N$, $\psi^N_{(\phi, \theta)} = \psi^N_{(0, \theta)}$ for all $\phi$; also note that, for $|n|, |m| \leq j$, on the diagonal element $a$ with entries $a_{nm} = c_m \delta_{nm}$, $c_m = \omega_m(a)$, where $\omega_m$ is the vector state of $|j, m\rangle \in V_j$. Define
\begin{equation}
    \lbtext{\rho_N}(\theta) := sup_{a \in \bcal_N, a = a^*}\{|\psi^N_{(0, \theta)} - \psi^N_{(0, 0)}| : ||[D_N, a]|| \leq 1\}.
\end{equation}

\begin{lemma} \label{lemmaFormulaAuxiliaryDistanceAllN} \label{lemmaDerivativeAuxiliaryDistance}
For $N \in \NN$ fixed:
    \begin{enumerate}[(i)]
    
    \item Let $\hat a$ be diagonal hermitian element introduced in \eqref{} to saturate the distance between the $\omega_m$'s
    \begin{align}
        \rho_N(\theta) &= \psi^N_{(0, \theta)}(\hat a) - \psi^N_{(0, 0)}(\hat a) \\
        &   = \sum_{k = 1}^N \binom{2j}{j+m} (\sin \frac{\theta}{2})^{2n} (\cos \frac{\theta}{2})^{2(N-n)}  \sum_{k = 1}^n \frac{1}{\sqrt{k(N - k+1)}}
    \end{align}
    
    \item $0 \leq \rho'_N(\theta) \leq 1$. In particular $\rho_N(\theta - \theta') \leq |\theta - \theta'| = d_{S^2}(E(0, \theta), E(0, \theta'))$.
    
    \item
    \begin{equation}
        \rho_{N}(\theta - \theta') \leq \rho_{N+1}(\theta - \theta')
    \end{equation}
    \end{enumerate}
    
\end{lemma}
\begin{proof}
(i) For a diagonal element $a$,  
\begin{equation*}
    \psi^N_{(0, \theta)}(a) = \sum_{k = 1}^N \binom{2j}{j+m} (\sin \frac{\theta}{2})^{2(j+m)} (\cos \frac{\theta}{2})^{2(j-m)} \omega_m(a),
\end{equation*}
hence $|\psi^N_{(0, \theta)}(a) - \psi^N_{(0, 0)}(a)|$ over the diagonal elements can be bounded by the claimed distance using the triangle inequality and the bound $|| \leq d_N(\omega_m, \omega_{-j}) = \sum_{k = 1}^n \frac{1}{\sqrt{k(N - k+1)}}$. Using $\hat a$ to saturate the inequality, we obtain the desired formula.

(ii) From the first formula of the first part of the lemma \ref{lemmaFormulaAuxiliaryDistanceAllN} we easily deduce that $\rho_N'(\theta) = \frac{d}{d\theta} \psi^N_{(0, \theta)} (\hat a)$. Furthermore, from the $SU(2)$-equivariance of the identification of coherent states and points in $S^2$, on the diagonal elements it follows that 
\begin{equation}
    \rho_N'(\theta) 
        =\frac{d}{d\theta} \psi^N_{(0, \theta)} (\hat a) 
        = \psi^N_{(0, \theta)}([E, \hat a])
        = (0, \theta| [E, \hat a] |0, \theta)_N,
\end{equation} 
where, recall, $[E, \hat a]$ is the ladder operator on $V_j$.

Since for every state $\omega$ is is true that $|\omega(a)| \leq ||a||$ for every $a$, we get the following inequality:
\begin{equation*}
    |\rho_N'(\theta)| = |\psi^N_{(0, \theta)}([E, \hat a])| \leq ||[E, a]|| \leq 1
\end{equation*}
recalling that $[E, \hat a]$ is the ladder operator on $V_j$, of unit norm. From expanding the formula $\rho_N'(\theta) = (0, \theta| [E, \hat a] |0, \theta)_N$ using the definition \ref{defnCoherentStatesRotations}, the lower bound $0$ then follows by noticing that every term in the expansion is positive since $\theta \in [0, \pi]$.

(iii) This follows from adapting the proof of \ref{nondecreasingDistanceRelatingNCoherent}, but where we need to take into account that $\eta^+_N(\bcal_N) \subset \acal_{N+1}$ might not be contained in the diagonal subalgebra $\bcal_{N+1}$, but it is contained in an algebra $\bcal_{N+1}'$ conjugate to $\bcal_{N+1}$ through a unitary operator commuting with the $SU(2)$ action on $\acal_{N+1}$ ($\pi_{j + \frac{1}{2}}$); this equates to rotating the basis vectors of $V_{j + \frac{1}{2}}$ in a way that leaves invariant the $\psi^{N+1}_{(\phi, \theta)}$ unchanged. This allows us to write the first equation of the proof of \ref{nondecreasingDistanceRelatingNCoherent} by changing $\acal_{N+1}$ by $\bcal'_N$, and from then on simply changing $\acal_N$ by $\bcal_N$.
\end{proof}


The following lemma is last piece of our puzzle. \cite{DAndrea2013}.
\begin{lemma}: For all $N$ and $\theta \in [0, \pi]$, the sequence $\rho_N(\theta)$ converges uniformly to $\theta = d_{S^2}(E(0, 0), E(0, \theta))$ as $N\to \infty$. 
\end{lemma}

\begin{proof}
 Let $f_N(\theta) := \theta - \rho_N(\theta)$; since $f_N(0) = 0$ and, by the second part of the lemma \ref{lemmaFormulaAuxiliaryDistanceAllN}, $f_N'(\theta) \geq 0$, so for each $N$ the function $f_N(\theta)$ is nondecreasing and positive, so $|| \theta - f_N(\theta)||_\infty - sup_{\theta \in [0, \pi]} f_N(\theta) = f_N(\pi) = \pi - \rho_N(\pi)$. So, the uniform convergence follows if and only if $\lim_{N \to \infty} \rho_N(\pi) = \pi$. For fixed $\theta$, by the third part of lemma \ref{lemmaFormulaAuxiliaryDistanceAllN}, $\rho_N(\theta)$ is a nondecreasing sequence, and it is bounded above by $\pi$, so the sequence $\rho_N(\theta)$ is convergent and any subsequence has the same limit.
 
 By using the formula for $\rho_N(\pi)$ given in the first part of lemma \ref{lemmaFormulaAuxiliaryDistanceAllN}, 
 \begin{align*}
     \rho_N(\pi) &= 2 \sum_{k = 1}^{\frac{1}{2}(N-1)} \frac{1}{k (N - k + 1)} + \frac{2}{\sqrt{N+1}} \\
     &\geq \int_{1}^{\frac{1}{2}} \frac{dx}{\sqrt{x(N-x+1)}} = 2 &\arcsin \frac{N-1}{N+1},
 \end{align*} since the function $\frac{1}{\sqrt{x(N-x+1)}}$ is positive for $1 \leq x \leq $ and symmetric about $x = \frac{1}{2}(N+1)$ and monotonically decreasing\todo{...} for $1 \leq x \leq \frac{1}{2}(N+1)$. The sequence $2 \arcsin \frac{N-1}{N+1}$ converges monotonically to $\pi$ as $N \to \infty$ if we restrict to the subsequence of odd $N$'s, but this suffices to prove that $\lim_{N \to \infty} \rho_N(\theta) = \pi$.
\end{proof}

From the previous lemma, and joining for completeness some previous results, the theorem follows.
\begin{theorem}
$$
    \rho_N(\theta - \theta') \leq d_N(\psi^N_{(\phi, \theta)}, \psi^N_{(\phi', \theta')}) d_{geo}(E(\phi, \theta), E(\phi', \theta'))
$$

\begin{equation}
        \lim_{N \to \infty} d_N(\psi^N_{(\phi, \theta)}, \psi^N_{(\phi', \theta')}) = d_{S^2}(E(\phi, \theta), E(\phi', \theta'))
\end{equation}
\end{theorem}


%%%%%%%%%%%%%%%%%%%%%%%%%%%%%%%%%%%%%%%%%%%%%%%%%%%%%%%%%%%%%%%%%%%%%%%%%%%%%%
%%%%%%%%%%%%%%%%%%%%%%%%%%%%%%%%%%%%%%%%%%%%%%%%%%%%%%%%%%%%%%%%%%%%%%%%%%%%%%
% \section{Convergence of fuzzy sphere to $S^2$}\label{FSsec:convergence}

% {\color{gray} 
%  - This identification of points in $S^2$ with coherent states ($\psi^N: \acal_N \to L^2(S^2)$ at the algebra level) \todo{???}
% is $SU(2)$-equivariant: $g_* \psi^N_{(\phi, \theta)} = \psi^N_{g\cdot (\phi, \theta)}$, \& \rtext{the distance between them is $SU(2)$-invariant}.
 
%  - as: \textbf{1.} $C^*$-algebra $\acal$ acting on the spinor fields $\hcal$; \textbf{2.} Representation of $SU(2)$: homeomorphisms?\todo{homeomorphisms? Symmetries?}; \textbf{3.}  Metric space on which $SU(2)$ acts by isometries.

% }