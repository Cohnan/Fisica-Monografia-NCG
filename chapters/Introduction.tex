The Preliminaries chapter on NC manifolds in DAndrea2013 seems to be taken mainly from Gracia-Bondia 2001.

%%%%%%%%%%%%%%%%%%%%%%%%%%%%%%%%%%%%%%%%%%%%%%%%%%%%%%%%%%%%%%%%%%%%%%%%%%%%%%
%%%%%%%%%%%%%%%%%%%%%%%%%%%%%%%%%%%%%%%%%%%%%%%%%%%%%%%%%%%%%%%%%%%%%%%%%%%%%%
\section{Physical Motivation}

    \begin{itemize}
        
    \item QM's phase space is the seminal NC space: \rtext{noncommuting coordinates!}
    
        \begin{itemize}
            
        \item Coherent states $\iff$ (in this context?) closest notion to classical things (points of commutative space? classical systems?)
        
        \item 
            
        \end{itemize}
    
    \item Applications of NC in physics:
    
        \begin{itemize}
            
        \item Fuzzy Spheres: condensed matter: study of phase transitions? quantum fields?
        
        \item High energy/small distances Spacetime? 
        
        \item Renormalization
    
        \item In M theory apparently in the 90's 
    
        \item Particle physics: Yang-Mills-Higgs theories obtained through these procedures
        
        \item Modelado de estrellas?
        
        \item MORE
            
        \end{itemize}
   
   \item Why study this? toy model to understand how to do geometry in NC spaces: matrices are well understood, should be easy to do calculations (trivial K-theory?)
   
        \begin{itemize}
            
        \item Cutoff: some renormalization procedures (DFR paper?)
            
        \end{itemize}
    
    \end{itemize}

\linea

%%%%%%%%%%%%%%%%%%%%%%%%%%%%%%%%%%%%%%%%%%%%%%%%%%%%%%%%%%%%%%%%%%%%%%%%%%%%%%
%%%%%%%%%%%%%%%%%%%%%%%%%%%%%%%%%%%%%%%%%%%%%%%%%%%%%%%%%%%%%%%%%%%%%%%%%%%%%%
\section{(Commutative) $C^*$-algebras as topological spaces}

As manifold? What preserves differential structure

($A$ unital) States: (DAndrea 2013) $S(A)$ (norm completion of) set of states of $A$. Is an ``extended (allowing infinity) metric space'' under the Connes' metric / \lbtext{spectral distance}. The supremum is alwasy attaine don self-adjoint elements (noted in Iochum 2001). ... COMPACT QUANTUM METRIC SPACE $(A,  d_{D})$.

To space: cofunctor that assigns algebra of functions

To commutative algebra: cofunctor that assigns its spectrum: set of characters = set of multiplicative linear functionals <-> set of pure states.

Gelfand transform

%%%%%%%%%%%%%%%%%%%%%%%%%%%%%%%%%%%%%%%%%%%%%%%%%%%%%%%%%%%%%%%%%%%%%%%%%%%%%%
%%%%%%%%%%%%%%%%%%%%%%%%%%%%%%%%%%%%%%%%%%%%%%%%%%%%%%%%%%%%%%%%%%%%%%%%%%%%%%
\section{Serre-Swan theorem and Differential Structure?}

Assume $M$ compact (the fuzzy sphere will be a ``compact quantum metric space''). $\acal C^\infty(M)$ is not a $C^*$ algebra, but it is dense in $A = C(M)$: set of ``sufficiently regular elements of $A$''.

Characters of $\acal$ are measures, which extend to characters of $A$.

%%%%%%%%%%%%%%%%%%%%%%%%%%%%%%%%%%%%%%%%%%%%%%%%%%%%%%%%%%%%%%%%%%%%%%%%%%%%%%
%%%%%%%%%%%%%%%%%%%%%%%%%%%%%%%%%%%%%%%%%%%%%%%%%%%%%%%%%%%%%%%%%%%%%%%%%%%%%%
\section{Spectral Triples: Metric $\to$ Geometry}

 - Algebraic formulation of the metric

 - The Spectral Triple adds the final element to do geometry in noncommutative spaces: the metric (in gravitation, this is precisely the gravitational field).
 
 Equivariant spectral triple (for D'Andrea, at least): \textbf{isometries}: $D$ and $\acal$ commute with the group action.

Meaning of the action and \textbf{the spectrum} of a Dirac operator (in our case I think we think of the spectrum (and the multiplicities) to be an indication of how similar my Dirac operator is to the commutative one: the full spectral triple's Dirac operator ``tends'' to it in this sense)

Presentation:
    
    \textbf{Canonical Spectral Triple}: Leg $(M, g)$ be an $n$-dimensional compact oriented Riemannian manifold which admits a spin structure% $(Spin_n \to Spin\,M \to M, \eta: Spin\,M \to SO\,M)$
    , and let $\Sigma M$ % = Spin\, M \times_\rho \Sigma_n$
    be the associated spinor bundle. Then $(\acal = C^\infty(M), \hcal = L^2(M, \Sigma M), \slashed D = -i \gamma^\mu \nabla^g_\mu)$%, where $\Sigma_n$ is the $2^{\lfloor n/2 \rfloor}$-dim. fermionic Fock space.
    
        \quad - The particle physics Dirac equation is $|p| = m$, where $|p| = \sqrt{E^2 - \vec p^2} = \sqrt{p^2} = \sqrt{\square} = \gamma_\mu p_\mu = -i \gamma^\mu \nabla_\mu$ is the Dirac operator
        
        \quad - Where $\{\gamma^\mu\}_\mu$ is a basis of $\RR^n = T^*_x M \hookrightarrow Cl(M, g)$ as operators on the fermionic Fock space, so they satisfy $\{\gamma^\mu, \gamma^\nu\} = 2 g^{\mu \nu}$
        
        \quad A connection $\nabla^\Sigma$ on the spinor bundle satisfies $\nabla^\Sigma = dx^\mu \otimes s_i + \psi^i \gamma^\mu \otimes A_i(E_\mu) = \gamma^\mu \otimes (E_\mu \psi - \psi^i A_i(E_\mu))$   ${E_\mu}_\mu$ is a local frame of $T M$ and $\{s_i\}_i$ is a basis of the spinor space / fermionic Fock space, and $A_\psi(X) = \psi^i A_i(X)$ where $A_i = \nabla^\Sigma s_i \in \Omega_U^1(TM) \otimes \Gamma_U(\Sigma_M)$ and $\Gamma_U(\Sigma M) = C^\infty(U, \Sigma_m)$.
    
    - \textbf{State of $C^*$-algebra} $\acal$ is an $\omega : \acal \to \CC$ linear, positive, of unit norm. \textbf{Pure}: can not be written as a convex combination of other states. Commutative: pure states $\Longleftrightarrow$ characters $\hat x$ $\Longleftrightarrow$ points $x$.
        \quad - Interpretation: expected values
        
        \quad - CM: measures: Dirac measures are points
        
        \quad - Pure states: then, closer notion of point.
        
        
    
    \textbf{Unital Spectral Triple} $(\acal , \hcal, D)$: $\hcal$ is a complex separable Hilbert space; $\acal$ is a complex associative involutive unital $C^*$-algebra with a faithful unital $*$-representation $\acal \hookrightarrow \mathcal B(\hcal)$; a self-adjoint (unbounded) operator $D: \hcal \to \hcal$ with compact resolvent, such that $[D, a] \in \mathcal B(\hcal)$ $\forall a \in \acal$.
    
    - \textbf{Distance}: $d_D(\omega, \omega') := sup_{a \in \acal} \left\{|\omega(a) - \omega'(a)| \, | \, \rtext{||[D, a]||_{op} \leq 1} \right\}$
    

\textbf{Canonical: algebraic formulation of known facts (not in present.)}

 - Conjugation $\iff$ \textbf{Real spectral triple / Spin geometry} $\iff$ Matter/Antimatter? $\iff$ ``admits spin structure'' $\to$ exists a unique Dirac operator (see Varilly formal)
 
 - If only $spin^\CC$ $\iff$ Dixmier-Doudary Check cohomology class $\delta = 0$ $\iff$ no obstruction to the existance of the B-A bimodule ``w/'' Morita equivalence $\alg s$: \textbf{Spinor Bundle}: irred. Clifford module of rank $2^m$, where $m = \lfloor rank(M)/2 \rfloor$
 
 - Grading $\gamma^5$: matter-antimatter distinction


\textbf{Canonical: calculation}:

- $[\slashed D, f]\psi = -i [(\partial_\mu f) \gamma^\mu] \cdot \psi = -i(df)\cdot \psi$
    
    - $||[\slashed D, f]||^2 = sup_{x \in M} ||g_x^{-1}(df, d \cut f)||^2 =: ||grad(f) \in \Gamma(TM)||_\infty^2$ ($grad(f)$ is such that $g(grad(f), X) = X(f)$)

    - $d(x, y) := inf\{length(\gamma): \gamma:[0,1] \to M; \gamma(0) = x, \gamma(1) = y\}$
    
    - Good upper bound for $d_D(\hat x, \hat y)$: $d(x, y)$: for all $\gamma$, $f(y) - f(x) = \int_0^1 \frac{d}{dt}[f(\gamma(t))]dt = \int_0^1 df_{\gamma(t)} (\dot \gamma(t)) = \int_0^1 g_{\gamma(t)}(grad(f), \dot \gamma) dt$, 
    so
    $|f(x) - f(y)| \leq \int_0^1 |grad(f)| |\dot \gamma(t)| dt \leq ||grad(f)||_
    \infty \, length(\gamma) = ||[\slashed D, f]|| length(\gamma)$. 
    Thus:
    $d_D(\hat x, \hat y) = sup\{|f(x) - f(y)|: f \in C(M), ||[\slashed D, f]||\leq 1\} \leq inf \, length(\gamma) \equiv d(x, y)$
    
    - Saturation of the upper bound: let $a_x \in C(M)$, $a_x(y) := d(x, y)$, %its continuity is simply the triangle inequality
    $||[\slashed D, a_x]|| = ||grad(a_x)||_\infty = 1$ and $|a_x(y) - a_x(x)| = |d(x,y)| = d(x,y)$ saturates the inequality, meaning that $ d_D(\hat x, \hat y) = d(x, y)$.
    
\linea